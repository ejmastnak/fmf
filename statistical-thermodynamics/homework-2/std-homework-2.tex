\documentclass[11pt, a4paper]{article}
\usepackage[english]{babel}
\usepackage{mwe}
\usepackage{graphicx}
\usepackage{siunitx} % for ranges and scientific notation
\usepackage{amsmath} % for general math things
\usepackage{amssymb} % for the real and complex number symbols
\usepackage{siunitx} % for scientific notation
\usepackage{bm} % for bold vectors in math mode
\usepackage{mathtools} % for defined equals := symbol
\usepackage{physics} % for differential notation
\newcommand*\diff{\mathop{}\!\mathrm{d}} % for differentials to not be italicized and have a little space
\setlength{\parindent}{0pt} % to stop indenting new paragraphs

\usepackage[hidelinks]{hyperref}

\begin{document}
	\title{Statistical Thermodynamics Second Homework Assignment}
	\author{Elijan Mastnak}
	\date{December 2019}
	\maketitle
	
\textit{One end of a flexible polymer consisting of $ 10^{20} $ monomers is fixed from a ceiling so that the polymer is allowed to hang, and a \SI{4e-10}{g} weight is attached to the other end. The length of a single monomer is $ \SI{2}{nm} $ and the temperature is fixed at $ \SI{300}{K} $. Calculate the average potential energy of the weight. How much heat does the polymer exchange with its surroundings if the mass of the weight is doubled?}
\subsubsection*{Parameters}
\begin{itemize}
	\item $ a = \SI{2e-9}{m} $, the length of a monomer
	\item $ N = 10^{20} $, the total number of monomers
	\item $ T = \SI{300}{K} $, the absolute temperature
	\item $ m = \SI{4e-13}{kg} $, the mass of the weight
\end{itemize}

\subsubsection*{Solution: Part 1}
\textbf{Note}: Because the absolute temperature and pressure are fixed, the monomers form a canonical ensemble that is formally best described in terms of Gibbs free energy $ G $ and enthalpy $ H $. However, I have used the more common quantities Helmholtz free energy $ F $ and internal energy $ E $. 
\newline
\newline
The energy contribution $ E $ of an individual monomer oriented at a polar angle $ \theta $ and the average total energy of the system, respectively, are
\begin{align*}
	E = -m g a \cos \theta 	&& \text{and} && \expval{E} = \expval{ N m g a \cos \theta } = N m g a \expval{\cos \theta} 
\end{align*}
where 
\begin{equation*}
	\expval{\cos \theta } = \int_{\Gamma} \cos \theta \rho (E) \diff \Gamma 
\end{equation*}
The probability $ \rho(E) $ of a microstate with energy $ E $ is
\begin{equation*}
	\rho(E) = e^{- \beta (E - F)} = \frac{e^{-\beta E}}{Z_c}
\end{equation*}

The expression for $  \expval{\cos \theta } $ thus becomes
\begin{align*}
	\expval{\cos \theta } &= \int_{\Gamma} \cos \theta \left(\frac{e^{-\beta E}}{Z_c}\right) \diff \Gamma = \frac{1}{Z_c} \int_{\Gamma} \cos \theta e^{-\beta E} \diff \Gamma \\[1.0ex]
	& =  \frac{1}{Z_c} \int_{\Gamma} \cos \theta e^{\beta mga \cos \theta} \diff \Gamma = \frac{1}{Z_c} \int_{\Gamma} \cos \theta e^{\alpha \cos \theta} \diff \Gamma 
\end{align*}
where we define the constant $ \alpha \equiv \beta m g a $ for shorthand.  The partition function $ Z_c $ is given by:
\begin{align*}
	Z_c = e^{-\beta F} = \int_\Gamma e^{-\beta E} \diff \Gamma = \int_\Gamma e^{ \beta m g a \cos \theta} \diff \Gamma = \int_{\Gamma} e^{\alpha \cos \theta} \diff \Gamma
\end{align*}
The value  $  \expval{\cos \theta } $  is thus
\begin{align*}
	\expval{\cos \theta }  = \frac{\int_{\Gamma} \cos \theta e^{\alpha \cos \theta} \diff \Gamma }{\int_{\Gamma} e^{\alpha \cos \theta} \diff \Gamma}
\end{align*}
In spherical coordinates, in terms of the element of solid angle $ \diff \Omega = \sin \theta \diff \theta \diff \phi $, the expression becomes
\begin{align*}
	\expval{\cos \theta }  = \frac{\int_{\Gamma} \cos \theta e^{\alpha \cos \theta} \diff \Omega }{\int_{\Gamma} e^{\alpha \cos \theta} \diff \Omega}
\end{align*}
Formally, in spherical coordinates, the elements of the phase space are $ r, \theta, $ and $ \phi $; however, only the polar angle $ \theta $ is relevant to our problem since it is the only term contributing to a monomer's energy $ E = - m g a \cos \theta $. With this simplification, the average polar angle becomes
\begin{align*}
	\expval{\cos \theta }  = \frac{\int_{0}^{\pi} \cos \theta e^{\alpha \cos \theta} \sin \theta \diff \theta }{\int_{0}^{\pi} e^{\alpha \cos \theta} \sin \theta \diff \theta}
\end{align*}
With the change of variables 
\begin{align*}
	u = \cos \theta && \diff u = - \sin \theta \diff \theta && u(0) = 1 && u(\pi) = -1
\end{align*}
the integrals become
\begin{align*}
	\expval{\cos \theta } & = \frac{\int_{-1}^{1} u e^{\alpha u} \diff u }{\int_{-1}^{1} e^{\alpha u} \diff u } = \frac{\cosh \alpha - \frac{1}{\alpha } \sinh \alpha }{\sinh \alpha } = \coth \alpha - \frac{1}{\alpha} = L(\alpha)
\end{align*}
where $ L(\alpha) $ denotes the Langevin function $ \coth \alpha - \frac{1}{\alpha} $.
The average energy is thus
\begin{align*}
	\expval{E} &= N m g a \expval{\cos \theta} = (N mga )L(\alpha)
\end{align*}
The value of the dimensionless constant $ \alpha $ is
\begin{align*}
	\alpha &= \beta m g a = \frac{mga}{k_B T} = \frac{(\SI{4e-13}{kg}) (\SI{9.8}{m \cdot s^{-2}}) (\SI{2e-9}{m})}{(\SI{300}{K}) (\SI{1.38e-23}{J \, K^{-1}})}\\ &= 1.89
\end{align*}

The numerical value of the average energy is
\begin{align*}
	\expval{E} &= (N mga )L(\alpha) = (N m g a ) \left(\coth(\alpha) + \frac{1}{\alpha} \right) \\[1.0ex]
	&= \left[(10^{20}) (\SI{4e-13}{kg}) (\SI{9.8}{m \cdot s^{-2}}) (\SI{2e-9}{m} ) \right] \left(\coth(1.89) + \frac{1}{1.89} \right)\\
	&=(\SI{0.784}{J})\left(1.047 + 0.529 \right)\\[1.0ex]
	& = \SI{1.23}{J}
\end{align*}


\subsubsection*{Solution: Part 2}
We find the heat $ Q $ exchanged by the polymer with its surroundings in terms of the entropy change $ \Delta S $.
\begin{align*}
	Q = T \Delta S = T \Delta \left(\frac{\expval{E} - F}{T} \right) = \Delta \big (\expval{E} - F \big )
\end{align*}

The average energy and partition functions were already found to be
\begin{align*}
	&\expval{E} = (N mga )L(\alpha) \\[1.0ex]
	&Z_c = \int_{0}^{\pi} e^{\alpha \cos \theta} \sin (\theta)\diff \theta = \frac{1}{\alpha} (e^{\alpha} - e^{-\alpha})= \frac{2}{\alpha } \sinh \alpha
\end{align*}
The Helmholtz free energy $ F $ is given 
\begin{align*}
	Z_c = e^{-\beta F} \quad \implies \quad F = - \frac{1}{\beta} \ln (Z_c) = -\frac{1}{\beta} \ln(\frac{2}{\alpha} \sinh \alpha)
\end{align*}
We then have
\begin{align*}
	Q = \expval{E}_2 - F_2 - \big[  \expval{E}_1 - F_1  \big] = \expval{E}_2 - \expval{E}_1 + F_1 - F_2
\end{align*}
where
\begin{align*}
	& \expval{E}_1 = (N m_1ga )L(\alpha_1) && \expval{E}_2 = (N m_2ga )L(\alpha_2) \\[1.5ex]
	& F_1 = -\frac{1}{\beta} \ln(\frac{2}{\alpha_1} \sinh \alpha_1) && F_2 = -\frac{1}{\beta} \ln(\frac{2}{\alpha_2} \sinh \alpha_2)
\end{align*}
Let $ m_1 = \SI{4e-13}{kg} $ be the original mass, and let $ m_2 = 2m_1 $ be the doubled mass. Reusing calculations from the previous part, we get the following numerical values:
\begin{align*}
	&\alpha_1 = \beta m_1 g a = 1.89 && \alpha_2 = \beta m_2 g a = 2 \alpha _1 =  3.78\\[1.0ex]
	& N m_1 g a = \SI{0.784}{J} && N m_2 g a = 2 N m_1 g a = \SI{1.568}{J}
\end{align*}
The energy contributions to the exchanged heat are:
\begin{align*}
	& \expval{E}_1 = (N m_1ga )L(\alpha_1) = (\SI{0.784}{J}) \left(\coth(1.89) + \frac{1}{1.89} \right) = \SI{1.23}{J} \\[1.5ex]
	&\expval{E}_2 = (N m_2ga )L(\alpha_2) = (\SI{1.568}{J})\left(\coth(3.78) + \frac{1}{3.78} \right) =  \SI{1.98}{J} \\[1.5ex]
	& F_1 = -\frac{1}{\beta} \ln(\frac{2}{\alpha_1} \sinh \alpha_1) = (\SI{4.14e-21}{J})  \ln(\frac{2}{1.89} \sinh (1.89)) \approx  \SI{0}{J}  \\[1.5ex]
	&G_2 = -\frac{1}{\beta} \ln(\frac{2}{\alpha_2} \sinh \alpha_2) = (\SI{4.14e-21}{J})  \ln(\frac{2}{3.78} \sinh (3.78)) \approx \SI{0}{J}
\end{align*}
Because the factor contributed by the thermodynamic beta is of the order $ 10^{-21} $, the values of $ F_1 $ and $ F_2 $ are negligible compared to $ \expval{E}_1  $ and $ \expval{E}_2 $.

The result for the heat exchanged by the polymer with its surroundings when the weight's mass is doubled is:
\begin{align*}
	Q &= \expval{E}_2 - \expval{E}_1 + F_1 - F_2 \approx \expval{E}_2 - \expval{E}_1\\[1.0ex]	
	&= \SI{1.98}{J} - \SI{1.23}{J}\\[1.0ex]
	&= \SI{0.75}{J}
\end{align*}

\end{document}



