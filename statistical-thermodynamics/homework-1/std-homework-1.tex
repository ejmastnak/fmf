\documentclass[11pt, a4paper]{article}
\usepackage[english]{babel}
\usepackage{mwe}
\usepackage{graphicx}
\usepackage{siunitx} % for ranges and scientific notation
\usepackage{amsmath} % for general math things
\usepackage{amssymb} % for the real and complex number symbols
\usepackage{siunitx} % for scientific notation
\usepackage{bm} % for bold vectors in math mode
\usepackage{mathtools} % for defined equals := symbol
\usepackage{physics} % for differential notation
\usepackage[margin=1.25in]{geometry}
\newcommand*\diff{\mathop{}\!\mathrm{d}} % for differentials to not be italicized and have a little space
\setlength{\parindent}{0pt} % to stop indenting new paragraphs

\usepackage[hidelinks]{hyperref}

\begin{document}
	\title{Statistical Thermodynamics First Homework Assignment}
	\author{Elijan Mastnak}
	\date{November 2019}
	\maketitle
	
\textit{The isothermal electrical susceptibility $ \chi_T $ of an oil is governed by the relationship}
\begin{equation*}
	\frac{\chi}{\chi + 3} \propto 1 + \frac{C}{T}
\end{equation*}
\textit{where $ C = \SI{30}{K} $. In an electric field of $ \SI{e7}{V\m} $ at $ \SI{27}{\degreeCelsius} $, the oil has density $ \SI{800}{kg/m^3} $, $ \chi_T = 2 $, and specific heat capacity at constant polarization $ c_P = \SI{1700}{J/kg K} $. Assuming the volume of the oil is constant, find the difference in specific heat capacities at constant electric field and constant polarization $ c_E - c_P $ and the difference in isothermal and isentropic susceptibilities $ \chi_T - \chi_S $.}

\section*{Finding Difference in Heat Capacities}
We recognize that we are working with a $ (E, P, T) $ system, where electric field magnitude $ E $ is the intensive variable, electric polarization $ P $ is the extensive variable, and $ T $ is temperature. Work is given by the product $ \delta W = V E \diff P $, where the volume factor $ V $ is necessary to give units of joules for work. 

\subsection*{Useful Identities for a $ (E, P, T) $ System}
For such a system, some useful identities are:
\begin{align}
	& \diff F = - S \diff T + V E \diff P \nonumber \\[1.0ex]
	& \diff G = - S \diff T - V P \diff E \nonumber \\[1.0ex]
	& \left(\pdv{S}{P}\right)_T = -V \left(\pdv{E}{T}\right)_P \label{eq:maxwell1}\\[1.5ex]
	& \left(\pdv{S}{E}\right)_T = -V \left(\pdv{P}{T}\right)_E \label{eq:maxwell2} \\[1.5ex]
	& c_E = \frac{T}{m} \left(\pdv{S}{T}\right)_E \label{eq:c_E_def}\\[1.5ex]
	& c_P = \frac{T}{m} \left(\pdv{S}{T}\right)_P \label{eq:c_P_def}
\end{align}

\subsection*{Expression for Difference of Heat Capacities}
For such a system, entropy is given by $  S = S(T, P) = S(T, P(T, E)) $. The partial derivative of entropy with respect to $ T $ at constant $ E $, found with the chain rule, is:
\begin{equation}
	\left(\pdv{S}{T}\right)_E  =  \left(\pdv{S}{T}\right)_P +  \left(\pdv{S}{P}\right)_T  \left(\pdv{P}{T}\right)_E \label{eq:pdv:S_T_E}
\end{equation}


Recognizing the expression for $ c_E $ from Equation \ref{eq:c_E_def} in the left side of Equation \ref{eq:pdv:S_T_E}, we write:
\begin{align*}
	c_E = \frac{T}{m} \left[ \left(\pdv{S}{T}\right)_P +  \left(\pdv{S}{P}\right)_T  \left(\pdv{P}{T}\right)_E \right]
\end{align*}
Referring to Equation \ref{eq:c_P_def}, we identify the first term as
\begin{equation*}
	\left(\pdv{S}{T}\right)_P = \frac{m}{T} c_P
\end{equation*}
Rearranging yields an expression for the desired quantity $ c_E - c_P $.
\begin{equation}
	c_E - c_P = \frac{T}{m} \left(\pdv{S}{P}\right)_T  \left(\pdv{P}{T}\right)_E \label{eq:diff_cE_cP}
\end{equation}
To find the desired difference in heat capacities, we must determine the unknown quantities
\begin{equation*}
	\left(\pdv{S}{P}\right)_T \quad \text{and} \quad \left(\pdv{P}{T}\right)_E
\end{equation*}

\subsection*{Equations of State}
\subsubsection*{General Electric Equation of State}
For a general electric substance:
\begin{equation}
	P = \epsilon_0 \chi E \label{eq:elec_eq_state}
\end{equation}
where $ \chi = \chi(T)$ is the temperature-dependent electric susceptibility. We will use this equation to find two partial derivatives that we will need for our analysis. First, the partial derivative of Equation \ref{eq:elec_eq_state} with respect to $ T $ at constant $ E $ is
\begin{equation}
	\left(\pdv{P}{T}\right)_E = \epsilon_0 E \left(\pdv{\chi}{T}\right)_E \label{eq:pdv_P_T_E}
\end{equation}
Second, the partial derivative of Equation \ref{eq:elec_eq_state} with respect to $ T $ at constant $ P $, found with the product rule, gives:
\begin{align}
	&0 = \epsilon_0 \left[E \left(\pdv{\chi}{T}\right)_P + \chi \left(\pdv{E}{T}\right)_P \right] \qquad \big / : \epsilon_0 \chi \nonumber \\[1.5ex]
	&\left(\pdv{E}{T}\right)_P = - \frac{E}{\chi} \left(\pdv{\chi}{T}\right)_P \label{eq:pdv_E_T_P}
\end{align}

\subsubsection*{Principle Equation of State}
For our case in particular, we are given the equation of state
\begin{equation*}
	\frac{\chi}{\chi + 3} \propto \left( 1 + \frac{C}{T}\right) 
\end{equation*}
Writing $ \chi = \chi(T) $, we proceed as follows:
\begin{align*}
	&\frac{\chi}{\chi + 3} = k\left( 1 + \frac{C}{T}\right) & \big / \ln &\\[1.0ex]
	&\ln \left(\frac{\chi}{\chi + 3}\right) = \ln \left[ k  \left( 1 + \frac{C}{T}\right) \right]\\[1.0ex]
	& \ln \chi - \ln(\chi + 3) = \ln k + \ln \left( 1 + \frac{C}{T}\right) & \big / \left(\pdv{}{T}\right)_E \\[1.0ex]
	& \left(\pdv{\chi}{T}\right)_E \left(\frac{1}{\chi} - \frac{1}{\chi + 3}\right) = -\frac{C}{T^2} \frac{1}{1 + \frac{C}{T}}\\[1.0ex]
	&\left(\pdv{\chi}{T}\right)_E \frac{3}{\chi (\chi + 3)} = \frac{C}{T(T + C)}\\[1.0ex]
\end{align*}
Giving the final result
\begin{equation}
	\left(\pdv{\chi}{T}\right)_E = \frac{C}{3} \frac{ \chi (\chi + 3)}{T(T + C)}\label{eq:pdv_chi_T_E}
\end{equation}
An analogous derivation shows that
\begin{equation}
	\left(\pdv{\chi}{T}\right)_P = \frac{C}{3} \frac{ \chi (\chi + 3)}{T(T + C)} \label{eq:pdv_chi_T_P}
\end{equation}

\subsection*{Finding Unknown Partial Derivatives}
We must find two unknown partial derivatives to solve for the difference in specific heat capacities in Equation \ref{eq:diff_cE_cP}.
\subsubsection*{Finding the First Unknown Derivative}
Referring in turn to Maxwell's first relation (Equation \ref{eq:maxwell1}), Equation \ref{eq:pdv_E_T_P}, and finally Equation \ref{eq:pdv_chi_T_P} we see that:
\begin{align*}
	\left(\pdv{S}{P}\right)_T  & = -V \left(\pdv{E}{T}\right)_P = \frac{m}{\rho} \frac{E}{\chi} \left(\pdv{\chi}{T}\right)_P \\[1.5ex]
	& =  \frac{mE}{\rho \chi} \frac{C}{3} \frac{ \chi (\chi + 3)}{T(T + C)}
\end{align*}
where we have used the relationship $ V = \frac{m}{\rho} $.

\subsubsection*{Finding the Second Unknown Derivative}
Referring in turn to Equation \ref{eq:pdv_P_T_E} and Equation \ref{eq:pdv_chi_T_E} gives:
\begin{align*}
	\left(\pdv{P}{T}\right)_E = \epsilon_0 E \left(\pdv{\chi}{T}\right)_E = \epsilon_0 E \frac{C}{3} \frac{ \chi (\chi + 3)}{T(T + C)}
\end{align*}

\subsection*{Finding Difference of Heat Capacities}
Plugging the two unknown partial derivatives into Equation \ref{eq:diff_cE_cP} gives
\begin{align*}
	 c_E - c_P & = \frac{T}{m} \left(\pdv{S}{P}\right)_T  \left(\pdv{P}{T}\right)_E \\[1.5ex]
	 & = \frac{T}{m} \frac{mE}{\rho \chi} \frac{C}{3} \frac{ \chi (\chi + 3)}{T(T + C)} \epsilon_0 E \frac{C}{3} \frac{ \chi (\chi + 3)}{T(T + C)} \\[1.5ex]
	 & = \epsilon_0 \frac{E^2}{9 \rho} \frac{C^2 \chi (\chi + 3)^2}{T (T + C)^2} \\[1.5ex]
	 & = \SI{8.85e-12}{C\cdot V^{-1}\cdot m^{-1}}  \frac{\left(\SI{e7}{V \cdot  m^{-1}}\right)^2}{9 \cdot \SI{800}{kg m^{-3}}} \frac{(\SI{30}{K})^2 \cdot 2 \cdot (2 + 3)^2}{\SI{300}{K} \cdot (\SI{300}{K} + \SI{30}{K})^2} \\[1.5ex]
	 & = \SI{1.69e-4}{\frac{C \cdot V}{kg\cdot K}} \\[1.5ex]
	 & =  \SI{1.69e-4}{\frac{J}{kg\cdot K}} 
\end{align*}

\section*{Finding Difference in Susceptibilities}
This problem can be solved quite quickly by respecting the fact that for a general $ (X, Y, T) $ thermodynamic system, where $ X $ is the intensive variable and $ Y $ is the extensive variable, the ratio of isothermal and isentropic susceptibilities $ \frac{\chi_T}{\chi_S} $ equals the ratio of intensive and extensive heat capacities $ \frac{c_X}{c_Y} $. In our case:
\begin{equation*}
	\frac{c_E}{c_P} = \frac{\chi_T}{\chi_S}
\end{equation*}
Writing $ \Delta c = c_E - c_P $ and $ \Delta \chi = \chi_T - \chi_S $, we get
\begin{align*}
	&\frac{c_P + \Delta c}{c_P} = \frac{\chi_T}{\chi_T - \Delta \chi}\\[1.0ex]
	& \chi_T - \Delta \chi = \frac{\chi_T c_P}{c_P + \Delta c} \\[1.0ex]
	\Delta \chi &= \chi_T - \chi_T\left(\frac{c_P}{c_P + \Delta c}\right) = \chi_T \left(1 - \frac{c_P}{c_P + \Delta c} \right) \\[1.0ex]
	& = 2 \left(1 - \frac{\SI{1700}{\frac{J}{kg\, K}} }{\SI{1700}{\frac{J}{kg\, K}}  + \SI{1.69e-4}{\frac{J}{kg\, K}}} \right)\\[1.0ex]
	& = 2 \left(1 - \frac{1700}{1700.000169} \right) \\[1.0ex]
	& = \SI{1.99e-7}{}
\end{align*}

\end{document}




