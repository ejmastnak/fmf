\documentclass[11pt, a4paper]{article}
\usepackage{mwe}
\usepackage{amsmath}
\usepackage{amssymb}
\usepackage{mathtools}
\usepackage{graphicx}
\graphicspath{{"figures/"}}
\usepackage[svgnames]{xcolor}
\usepackage{bm} % for bold vectors in math mode
\usepackage{physics} % for differential notation, etc...
\usepackage[separate-uncertainty=true]{siunitx}

\usepackage[margin=3cm]{geometry}
\usepackage{fancyhdr}
\usepackage{truncate}
\usepackage[normalem]{ulem}  % for underline with line wrapping
\usepackage[colorlinks = true, allcolors=blue]{hyperref}

\newcommand{\dmath}[1]{\textcolor{Maroon}{#1}}  % to differentiate math inside questions, where the regular text is italicized. Requires xcolor with svgnames

\setlength{\parindent}{0pt} % to stop indenting new paragraphs
\newcommand{\diff}{\mathop{}\!\mathrm{d}} % differential

\renewcommand{\vec}[1]{\bm{#1}} % for vectors
\renewcommand{\op}[1]{\hat{#1}} % for operators
\newcommand{\mat}[1]{\mathbf{#1}} % for matrices
\newcommand{\uvec}[1]{\hat{\vec{#1}}} % for dotted vector quantity

\newcommand{\p}{\psi}  % time-independent wavefunction
\renewcommand{\r}{\vec{r}}  % position vector

\newcommand{\E}{\mathcal{E}}  % electric field to avoid clash with energy

% begin header configuration
\pagestyle{fancy}

% Header and footer on non-section pages (default style)
\fancyhf{}
\fancyhead[R]{\href{https://github.com/ejmastnak/fmf}{\small{\texttt{github.com/ejmastnak/fmf}}}} 
\fancyhead[L]{\textit{\truncate{0.65\headwidth}{\rightmark}}}
\fancyfoot[C]{\thepage} 
\renewcommand{\headrulewidth}{0.1pt}

% Header and footer on section pages---identical to default style, but must be explicitly included
\fancypagestyle{plain}{
    \fancyhf{}  
    \fancyhead[R]{\href{https://github.com/ejmastnak/fmf}{\small{\texttt{github.com/ejmastnak/fmf}}}}  
    \fancyhead[L]{\textit{\truncate{0.65\headwidth}{\rightmark}}} 
    \fancyfoot[C]{\thepage}  % centered page number in footer
    \renewcommand{\headrulewidth}{0.1pt}
}

\renewcommand{\sectionmark}[1]{%
  \markboth{\sectionname \thesection}
  {\noexpand\firstsubsectiontitle}}
\renewcommand{\sectionmark}[1]{}

\renewcommand{\subsectionmark}[1]{%
  \markright{\thesubsection\ \, #1}\gdef\firstsubsectiontitle{#1}}

\newcommand\firstsubsectiontitle{}

% end header configuration


\begin{document}
\title{Optics Lecture Notes}
\author{Elijan Mastnak}
\date{Summer Semester 2020-2021}
\maketitle

\thispagestyle{empty}  % remove headers from introductory page

\begin{center}
\textbf{About These Notes}
\end{center}
These are my lecture notes from the course \textit{Optika} (Optics), an elective course offered to third-year physics students at the Faculty of Math and Physics in Ljubljana, Slovenia. The exact material herein is specific to the physics program at the University of Ljubljana, but the content is fairly standard for an undergraduate course in wave optics. I am making the notes publicly available in the hope that they might help others learning the same material---the most up-to-date version can be found on \href{https://github.com/ejmastnak/fmf/tree/main/optics/}{\underline{GitHub}}.

\vspace{2mm}
\textit{Navigation}: For easier document navigation, the table of contents is ``clickable'', meaning you can jump directly to a section by clicking the colored section names in the table of contents. Unfortunately, the \textit{clickable links do not work in most online or mobile PDF viewers}; you have to download the file first.

\vspace{2mm}
\textit{On Content}: The material herein is far from original---it comes almost exclusively from lecture notes by Professors Irena Dreven\v{s}ek Olenik and Mojca Vilfan at the University of Ljubljana. I take credit for nothing beyond translating the notes to English and typesetting.

\vspace{2mm}
\textit{Disclaimer:} Mistakes---both trivial typos and legitimate errors---are likely. Keep in mind that these are the notes of an undergraduate student in the process of learning the material himself---take what you read with a grain of salt. If you find mistakes and feel like telling me, by \href{https://github.com/ejmastnak/fmf}{\underline{GitHub}} pull request, \href{mailto:ejmastnak@gmail.com}{\underline{email}} or some other means, I'll be happy to hear from you, even for the most trivial of errors.

\newpage

\pagestyle{empty}  % no header in table of contents

\tableofcontents

\newpage


\pagestyle{fancy}
\section{Review of Geometrical Optics}
Geometrical optics is the study of light, in which the light is modelled as rays that:
\begin{itemize}
    \item propagate in straight-line paths as they travel through a homogeneous medium,
    \item bend, and in particular circumstances may split in two, at the interface between two optically dissimilar media,
    \item follow curved paths in a medium in which the refractive index changes
    \item may be absorbed or reflected.
\end{itemize}
This description is taken from the Wikipedia page on 
 \href{https://en.wikipedia.org/wiki/Geometrical_optics}{\textcolor{blue}{\underline{Geometrical optics}}}.

 \vspace{2mm}
\textbf{Overview of Chapter Material:}
\begin{itemize}
    \item Fermat's principle and the ray equation
    \item The ABCD ray transfer matrices
    \item Fundamental optical instruments (mirrors, lenses, etc...)
\end{itemize}

\subsection{Fermat's Principle}

\subsubsection{Introductory Remarks: Speed of Light and Index of Refraction}

\begin{itemize}

    \item Before beginning a quantitative treatment of Fermat's principle, we first introduce two important concepts: the speed of light and the index of refraction. We denote the speed of light in matter by $ c $ and the speed of light in vacuum by $ c_{0} $. The speed of light in vacuum is a universal constant---to four significant figures, its value is $ c_{0} = \SI{2.997}{\meter \, \second^{-1}} $. The light speeds $ c_{0} $ and $ c $ in vacuum and in matter, respectively, are related by
    \begin{equation*}
        c = \frac{c_{0}}{n},
    \end{equation*}
    where $ n $ is the material's index of refraction. The index of refraction is a property of matter and in general may be dependent on both position $ \r $ in the material and the frequency $ \omega $ of the light passing through, i.e. $ n = n(\r, \omega) $. 

    \item Some representative values of the index of refraction for visible light are given in the table below:
    \begin{center}
        \begin{tabular}{c|c}
            Material & Value of $ n $ \\
            \hline
            Vacuum & 1\\
            Water & 1.3\\
            Glass & 1.4 to 1.9\\
            Diamond & 2.9
        \end{tabular}
    \end{center}
    Keep in mind that the index of refraction is frequency-dependent, and that these values may differ for frequencies outside the visible spectrum.

\end{itemize}

\subsubsection{Fermat's Principle}
\begin{itemize}
    \item Qualitatively, Fermat's principle states:
    \begin{quote}
        Light travels between any two points in space along the path minimizing the travel time between the two points.
    \end{quote}
    More quantitatively, Fermat's principle is formulated as a least action principle, as discussed immediately below.

    \item Consider light travelling through material with index of refraction $ n $. We parameterize the light's path through the material with the arc length parameter $ s $, and divide the path into many infinitesimal elements $ \diff s $. 

    \item The light travels a distance $ \diff s $ in the time $ \diff t $, and the two quantities are related according to
    \begin{equation*}
        \diff t = \frac{\diff s}{c} = \frac{\diff s}{(c_{0}/n)} = \frac{n}{c_{0}}\diff s.
    \end{equation*}
    Fermat's theorem states that the light takes the path minimizing the quantity
    \begin{equation*}
        \int_{(1)}^{(2)} \diff t = \frac{1}{c_{0}}\int_{(1)}^{(2)} n \diff s.
    \end{equation*}
    Since $ c_{0} $ is a constant, Fermat's theorem is equivalent to the requirement
    \begin{equation*}
        S \equiv \int_{(1)}^{(2)}n \diff s = \text{min},
    \end{equation*}
    where we have defined the \textit{optical path length} $ S $, which is just the total distance travelled by the light between the points 1 and 2, scaled by the index of refraction $ n $. When written as a minimizing condition on the optical path length $ S $, Fermat's theorem is sometimes called the optical least action principle, and is analogous to the least action principle in Lagrangian mechanics.
    
\end{itemize}

\subsubsection{Example: Deriving the Law of Refraction}
\begin{itemize}
    \item The law of refraction applies to light incident on an interface between two materials with different indexes of refraction. If the light, originally in material 1 with index of refraction $ n_{1} $, is incident at an angle $ \alpha $ on material 2 with index of refraction $ n_{2} $, then the angle of refraction $ \beta $ is given by
    \begin{equation*}
        n_{1} \sin \alpha = n_{2} \sin \beta \implies \beta = \arcsin \left( \frac{n_{1}}{n_{2}} \sin \alpha \right).
    \end{equation*}
    
%FIG

    \item To derive the law of refraction, we first consider the schematic shown in Figure TODO. Our goal is to find the path between point 1 in material 1 and point 2 in material 2 minimizing the optical path $ S $. Assuming the index of refraction is constant in each region, the optical length $ S $ is
    \begin{equation*}
        S \equiv \int_{(1)}^{(2)} n \diff s = n_{1} \int_{\mathrm{I}} \diff s + n_{2} \int_{\mathrm{II}} \diff s = n_{1} s_{1} + n_{2} s_{2},
    \end{equation*}
    where the integral subscripts I and II refer to the paths between materials 1 and 2, respectively. In terms of the coordinates $ x $ and $ z $, the optical path length reads
    \begin{equation*}
        S = n_{1} \sqrt{x_{1}^{2} + z_{1}^{2}} + n_{2} \sqrt{x_{2}^{2} + z_{2}^{2}} = n_{1} \sqrt{z_{1}^{2} + x_{1}^{2}} + n_{2} \sqrt{z_{2}^{2} + (x - x_{1})^{2}},
    \end{equation*}
    where we have introduced the total vertical distance $ x = x_{1} + x_{2} $.
    
    \item We then differentiate $ S $ with respect to $ x_{1} $ to find the value of $ x_{1} $ minimizing the path length; the derivative reads
    \begin{equation}
        0 \equiv \dv{S}{x_{1}} = \frac{2n_{1} x_{1}}{2 \sqrt{z_{1}^{2} + x_{1}^{2}}} + \frac{2 \cdot (-1) \cdot (x - x_{1})}{2 \sqrt{z_{2}^{2} + (x-x_{1})^{2}}} \label{eq:LawOfRefraction-min}
    \end{equation}
    With reference to Figure TODO, we write the angles of incidence and refraction in terms of the $ x $ and $ z $ coordinates in the form
    \begin{equation*}
        \sin \alpha = \frac{x_{1}}{\sqrt{x_{1}^{2} + z_{1}^{2}}} \qquad \text{and} \qquad \sin \beta = \frac{x_{2}}{\sqrt{x_{2}^{2} + z_{2}^{2}}} = \frac{(x - x_{1})}{\sqrt{(x - x_{1})^{2} + z_{2}^{2}}}.
    \end{equation*}
    In terms of the angles $ \alpha $ and $ \beta $, Equation \ref{eq:LawOfRefraction-min}, after simplifying and rearranging, simplifies considerably to
    \begin{equation*}
        n_{1} \sin \alpha = n_{2} \sin \beta,
    \end{equation*}
    which is the familiar law of refraction. Interpretation: precisely the behavior encoded by the law of refraction satisfies Fermat's principle of least optical action.
    
\end{itemize}
    
\subsection{The Ray Equation}
\begin{itemize}
    \item We now consider the behavior of light travelling between materials in which the index of refraction is not constant; instead, we have $ n = n(\r) = n(x, y, z) $. In this case, light's behavior is determined by the \textit{ray equation}, which reads
    \begin{equation*}
        \grad n = \dv{s} \left( n \dv{\r}{s} \right).
    \end{equation*}
    Our goal in this section is to derive the behavior of light in this non-homogeneous material by combining the Euler-Lagrange equations
    \begin{equation}
        \dv{t}\left( \pdv{L}{\dot{\vec{r}}} \right) - \pdv{L}{\r} = 0 \label{eq:lagrange-euler}
    \end{equation}
    with the principle of least optical action.
    
\end{itemize}

\subsubsection{Derivation of the Ray Equation}
\begin{itemize}
    \item Working in Cartesian coordinates, we first write the optical path length $ S $ as
    \begin{equation*}
        S = \int_{(1)}^{(2)}n(x, y, z)\diff s = \int_{(1)}^{(2)} n(x, y, z)\abs{\dot{\vec{r}}} \diff t = \int_{(1)}^{(2)} n(x, y, z)\sqrt{\dot{x}^{2} + \dot{y}^{2} + \dot{z}^{2}} \diff t,
    \end{equation*}
    where the dot denotes differentiation with respect to time, and we have used the identity
    \begin{equation*}
        \diff s = \abs{\dot{\vec{r}}}\diff t = \sqrt{\dot{x}^{2} + \dot{y}^{2} + \dot{z}^{2}} \diff t.
    \end{equation*}
    We then compare the optical path $ S $ to the general Lagrangian action
    \begin{equation*}
        S = \int_{(1)}^{(2)} L(\r, \dot{\vec{r}}) \diff t,
    \end{equation*}
    which motivates the definition of the \textit{optical Lagrangian} as
    \begin{equation*}
        L(\r, \dot{\vec{r}})  = n(\r) \abs{\dot{\vec{r}}}.
    \end{equation*}
    In Cartesian coordinates, the optical Lagrangian reads
    \begin{equation*}
        L(x, y, z, \dot{x}, \dot{y}, \dot{z}) = n(x, y, z) \sqrt{\dot{x}^{2} + \dot{y}^{2} + \dot{z}^{2}}.
    \end{equation*}
    
    \item We then substitute the just-derived optical Lagrangian into the Euler-Lagrange equation (Eq. \ref{eq:lagrange-euler}). Beginning with the $ x $ coordinate, the E-L equation reads
    \begin{equation*}
        \dv{t} \left( \frac{n \dot{x}}{\sqrt{\dot{x}^{2} + \dot{y}^{2} + \dot{z}^{2}}} \right) = \pdv{n}{x} \sqrt{\dot{x}^{2} + \dot{y}^{2} + \dot{z}^{2}}.
    \end{equation*}
    We then multiply the equation through by $ \dv{t}{s} $ to get
    \begin{equation*}
        \dv{t} \left( \frac{n \dot{x}}{\sqrt{\dot{x}^{2} + \dot{y}^{2} + \dot{z}^{2}}} \right)\dv{t}{s} = \pdv{n}{x} \sqrt{\dot{x}^{2} + \dot{y}^{2} + \dot{z}^{2}} \ \dv{t}{s}.
    \end{equation*}
    
    \item Next, we apply the chain rule on the left hand side, which allows to eliminate the differential $ \diff t $ and then write $ \dot{x} \equiv \dv{x}{t} $; on the right hand side we use the earlier identity $ \diff s = \sqrt{\dot{x}^{2} + \dot{y}^{2} + \dot{z}^{2}} \diff t $, which results in
    \begin{equation*}
        \dv{s} \left( \dv{x}{t} \frac{n}{\sqrt{\dot{x}^{2} + \dot{y}^{2} + \dot{z}^{2}}} \right) = \pdv{n}{x} \dv{s}{s} = \pdv{n}{x}.
    \end{equation*}
    One more application of the identity $ \diff s = \sqrt{\dot{x}^{2} + \dot{y}^{2} + \dot{z}^{2}} \diff t $ on the left hand side simplifies things to
    \begin{equation*}
        \dv{s}\left( n \dv{x}{s} \right) = \pdv{n}{x}.
    \end{equation*}
    
    \item An identical procedure for the coordinates $ y $ and $ z $ produces the analogous results
    \begin{equation*}
        \dv{s}\left( n \dv{y}{s} \right) = \pdv{n}{y} \qquad \text{and} \qquad \dv{s}\left( n \dv{z}{s} \right) = \pdv{n}{z}.
    \end{equation*}
    Combining the equations for $ x $, $ y $ and $ z $ into a single vector equation produces
    \begin{equation*}
        \grad n = \dv{s} \left( n \dv{\r}{s} \right),
    \end{equation*}
    which is the ray equation quoted at the beginning of this subsection.
    
\end{itemize}

\subsubsection{Example: The Ray Equation in Homogeneous Matter}
\begin{itemize}
    \item Consider light travelling through a homogeneous material---in the context of optics, this means a material with constant index of refraction $ n $. In this case $ \grad n = 0 $, and the ray equation simplifies to
    \begin{equation*}
        0 = \dv{s} \left( n \dv{\r}{s} \right) = n \dv{s} \dv{\r}{s} = \dv[2]{\r}{s}.
    \end{equation*}
    
    \item The equation $ \r''(s) = 0 $ is solved by the linear ray
    \begin{equation*}
        \r(s) = \vec{a}_{0} + \vec{a}_{1}s,
    \end{equation*}
    where the vector $ a_{0} $ represents an initial point and the vector $ \vec{a}_{1} $ represents the direction of ray propagation.
    
\end{itemize}


\subsubsection{The Ray Equation and the Paraxial Approximation}
\begin{itemize}
    \item In optics, we often encounter situations in which the direction of light propagation deviates only slightly from some given direction in space. We call this direction the \textit{optical axis}, and usually choose it to align with the $ z $ axis.

    % FIG
    \item For simplicity, we consider a two-dimensional system in which light moves in the $ xz $ plane, shown in Figure TODO, and assume the index of refraction is a single-variable function of $ x $, i.e. $ n = n(x) $.

    Qualitatively, the \textit{paraxial approximation} assumes that the light ray deviates only slightly from the $ z $ axis (the optical axis) as it moves through the $ xz $ plane. Mathematically, the approximation reads
    \begin{equation}
        \dv{x}{z} \ll 1. \label{eq:parax-approx}
    \end{equation}

    \item Next, we introduce an angle $ \theta $ between the tangent to the light ray's path and the optical axis, defined as
    \begin{equation*}
        \theta = \dv{x}{z}.
    \end{equation*}
    In the regime of the paraxial approximation, the angle $ \theta $ obeys
    \begin{equation*}
        \sin \theta \approx \tan \theta \approx \theta.
    \end{equation*}

    \item Using the paraxial approximation (Eq. \ref{eq:parax-approx}), an infinitesimal arc length $ \diff s $ in our two-dimensional system reads
    \begin{equation*}
        \diff s = \sqrt{(\diff x)^{2} + (\diff z)^{2}} = \sqrt{1 + \left( \dv{x}{z} \right)^{2}} \diff z \approx \diff z.
    \end{equation*}
    
    \item In terms of the just-derived paraxial result $ \diff s \approx \diff z $, for a two-dimensional system of the form shown in Figure TODO, the ray equation simplifies to
    \begin{equation*}
        \grad n = \dv{s} \left( n(\r) \dv{\r}{s} \right) \stackrel{\text{parax.}}{\longrightarrow} \dv{x}{s} = \dv{z}\left( n(x) \dv{x}{z} \right) = n(x) \dv[2]{x}{z}.
    \end{equation*}
    Finally, we divide through by $ n(x) $ to write the equation in the final form
    \begin{equation}
        \dv[2]{x}{z} = \frac{1}{n(x)} \dv{n}{x}. \label{eq:ray-equation-parax}
    \end{equation}
    
\end{itemize}

\subsubsection{Example: A Material with a Parabolic Refractive Dependence}
\begin{itemize}
    \item Consider a two-dimensional material in which the index of refraction depends on the vertical distance $ x $ from the optical according to the parabolic relationship
    \begin{equation*}
        n(x) = n_{0} \left( 1 - \frac{\alpha^{2} x^{2}}{2} \right),
    \end{equation*}
    where we assume $ \alpha x \ll 1 $. We now aim to determine the trajectory of a light ray through this material in the regime of the paraxial approximation.

    \item We begin by substituting the index of refraction into the planar paraxial ray equation (Eq. \ref{eq:ray-equation-parax}), evaluate the derivative, and apply the identity $ \alpha x \ll 1 $ to get
    \begin{equation*}
        \dv[2]{x}{z} = \frac{1}{n(x)} \dv{n}{x} = - \frac{n_{0} \alpha^{2} x}{n_{0} \left( 1 - \frac{\alpha^{2} x^{2}}{2} \right)} \approx - \frac{n_{0} \alpha^{2} x}{n_{0}} = - \alpha^{2} x.
    \end{equation*}
    The resulting differential equation, i.e. $ x''(z) = - \alpha^{2} x $, is solved by the ansatz
    \begin{equation*}
        x(z) = A \cos (\alpha z) + B \sin (\alpha z).
    \end{equation*}
    If we assume that at $ z = 0 $ the light ray occurs a distance $ x_{0} $ above the optical axis at an angle $ \theta_{0} $ to the optical axis, i.e.
    \begin{equation*}
        x(0) = x_{0} \qquad \text{and} \qquad \left( \dv{x}{z} \right)_{z = 0} = \theta_{0},
    \end{equation*}
    the solution for the light's trajectory through the material is
    \begin{equation*}
        x(z) = x_{0} \cos (\alpha z) + \frac{\theta_{0}}{\alpha}\sin(\alpha z).
    \end{equation*}
    In other words, in a two-dimensional material where $ n $ has parabolic dependence on $ x $, the light ray traces out a sinusoidal curve as it moves through the material.
    
\end{itemize}



\subsection{The Optical Transfer Matrices}
\begin{itemize}
    \item In this section, we will continue working the two-dimensional, paraxial regime shown in Figure TODO, where the path of a light ray through the $ xz $ plane is described by the distance $ x $ from the optical $ z $ axis and the angle $ \theta $ between the path's tangent and the optical axis. Our goal in this section is to derive a formalism relating a light ray at two different points in space separated by an optical medium or optical element, such as a lens.

% FIG 2.8
    \item We begin by considering the generic light ray shown in Figure TODO. Given the position $ x_{1} $ and direction $ \theta_{1} $ at point 1, it is possible to write the position $ x_{2} $ and direction $ \theta_{2} $ at a later point in the trajectory as the linear combinations
    \begin{align}
        & x_{2} = Ax_{1} + B\theta_{1} \nonumber \\
        & \theta_{2} = Cx_{1} + D \theta_{1}. \label{eq:M-system}
    \end{align}
    In matrix form, we can write the relationship between the values $ x_{1} $ and $ \theta_{1} $ and $ x_{2} $ and $ \theta_{2} $ as
    \begin{equation}
        \begin{bmatrix}
            x_{2}\\
            \theta_{2}
        \end{bmatrix}
        = 
        \begin{bmatrix}
            A & B\\
            C & D
        \end{bmatrix}
        \begin{bmatrix}
            x_{1}\\
            \theta_{1}
        \end{bmatrix}
        \equiv \mat{M}
        \begin{bmatrix}
            x_{1}\\
            \theta_{1}
        \end{bmatrix}, \label{eq:M}
    \end{equation}
    where we have defined the \textit{optical transfer matrix}
    \begin{equation*}
        \mat{M} = 
        \begin{bmatrix}
            A & B\\
            C & D
        \end{bmatrix}.
    \end{equation*}
    In general, a transfer matrix encodes the passage of a light ray between media with different optical properties.

\end{itemize}

\subsubsection{Example: Transfer Matrix for Translation Through a Homogeneous Material} \label{sss:M-homogeneous}
\begin{itemize}
% FIG 2.9
    \item As a first example, we consider a homogeneous material with constant index of refraction $ n $. We assume the light ray has the known coordinates $ x_{1} $ and $ \theta_{1} $ at the point $ z_{1} $, as shown in Figure TODO; our goal is to find the position $ x_{2} $ and direction $ \theta_{2} $ after a translation of length $ L $ along the $ z $ axis to the point $ z_{2} $.

    \item The material has a constant index of refraction, so the direction $ \theta $ is unchanged after the translation, i.e. $ \theta_{1} = \theta_{2} $. Meanwhile, the distance $ x_{2} $ from the $ z $ axis changes as
    \begin{equation*}
        x_{2} = x_{1} + (z_{2} - z_{1}) \theta_{1} = x_{1} + L \theta_{2}.
    \end{equation*}
    To reveal the appropriate transfer matrix, we first write the system of equations
    \begin{align*}
        & x_{2} = 1 \cdot x_{1} + L \cdot \theta_{1}\\
        & \theta_{2} = 0 \cdot x_{1} + 1 \cdot \theta_{1}.
    \end{align*}
    We then compare this system of equations to the general form in Equations \ref{eq:M-system} and \ref{eq:M}, which motivates the definition of the homogeneous translation's transfer matrix
    \begin{equation*}
        \mat{M} = 
        \begin{bmatrix}
            1 & L\\
            0 & 1
        \end{bmatrix}.
    \end{equation*}
    
\end{itemize}

\subsubsection{Example: Transfer Matrix for Passage Between Materials Through a Straight Interface} \label{sss:M-straight-boundary}
\begin{itemize}
    \item Assume light in material 1 is incident at an angle $ \theta_{1} $ on a straight boundary between materials 1 and 2, with indexes of refraction $ n_{1} $ and $ n_{2} $, respectively, as shown in Figure TODO
    %FIG 2.10

    \item We describe the light's position just before and just after passing through the interface with two infinitesimally close points. In other words, the $ x $ coordinates before and after crossing the boundary are equal,
    \begin{equation*}
        x_{1} = x_{2}.
    \end{equation*}
    Meanwhile, because the two materials have different indexes of refraction, the light ray's initial direction $ \theta_{1} $ changes after passing through the interface according to the law of refraction
    \begin{equation*}
        n_{1} \sin \theta_{1} = n_{2} \sin \theta_{2} \quad  \stackrel{\theta \ll 1}{\longrightarrow} \quad n_{1} \theta_{1} = n_{2} \theta_{2},
    \end{equation*}
    where have used the small angle approximation $ \sin \theta \approx \theta $.

    \item Written as a system of equations, the coordinates $ x $ and $ \theta $ on either side of the interface are related by
    \begin{align*}
        & x_{2} = 1\cdot x_{1} + 0 \cdot \theta_{1}\\
        & \theta_{2} = 0 \cdot x_{1} + \frac{n_{1}}{n_{2}} \cdot \theta_{1}.
    \end{align*}
    The corresponding transfer matrix is thus
    \begin{equation*}
        \mat{M} =
        \begin{bmatrix}
            1 & 0\\
            1 & \frac{n_{1}}{n_{2}}
        \end{bmatrix}.
    \end{equation*}
    Note that in the case $ n_{1} = n_{2} $, the matrix reduces to the identity matrix. Interpreted physically, this just represents the logical fact that a light ray stays the same when passing through an interface between optically identical materials.
    
\end{itemize}

\subsubsection{Example: Transfer Matrix for Passage Between Materials Through a Curved Interface} \label{sss:M-curved-boundary}
\begin{itemize}
    % FIG 2.11
    \item Consider two materials with indexes of refraction $ n_{1} $ and $ n_{2} $, separated by a curved boundary of radius $ R $,\footnote{By convention, the radius $ R $ is positive if the boundary is convex with respect to the direction of increasing $ z $, as in Figure TODO, and negative if the boundary is concave with respect to increasing $ z $.} as shown in Figure TODO.

    \item As in the previous example for a straight interface, we choose two infinitesimally close points on either side of the boundary, which results in the relationship
    \begin{equation*}
        x_{1} = x_{2}.
    \end{equation*}

    \item Finding the relationship between the directions $ \theta_{1} $ and $ \theta_{2} $ takes a little more work. We begin by introducing three angles:
    \begin{enumerate}
        \item an angle $ \phi $ between the optical axis and the normal to the boundary,

        \item an angle of incidence $ \alpha = \theta_{1} + \phi $ between the normal to the boundary and the incident ray in material 1, and

        \item an angle of refraction $ \beta = \theta_{2} + \phi $ between the normal to the boundary and the refracted ray in material 2.
    \end{enumerate}
    With reference to the geometry of Figure TODO, we see that the angle $ \phi $ is defined via
    \begin{equation*}
        \sin \phi = \frac{x_{1}}{R} \approx \phi,
    \end{equation*}
    where the last equality assumes $ \phi $ is small, i.e. that the normal to the boundary is close the optical axis. In the paraxial regime, where $ \theta \ll 1 $, the law of refraction reads
    \begin{equation*}
        n_{1} \alpha = n_{2} \beta \implies n_{1} (\theta_{1} + \phi) = n_{2}(\theta_{2} + \phi),
    \end{equation*}
    from which we solved for the angle $ \theta_{2} $ according to
    \begin{equation*}
        \theta_{2} = \frac{n_{1} - n_{2}}{n_{2}} \phi + \frac{n_{1}}{n_{2}} \theta_{1} = \frac{n_{1} - n_{2}}{n_{2}} \frac{x_{1}}{R} + \frac{n_{1}}{n_{2}} \theta_{1},
    \end{equation*}
    where the last line uses the geometric identity $ \phi = (x_{1})/R $.

    \item Using the just-derived expression for $ \theta^{2} $, the system of equations relating $ x $ and $ \theta $ on either side of the boundary is
    \begin{equation*}
        \begin{array}{l c l c l}
            x_{2} & = & 1 \cdot x_{1} & + & 0 \cdot \theta_{1}\\
            \theta_{2} & = & \frac{n_{1} - n_{2}}{n_{2}} \cdot \frac{x_{1}}{R} & + & \frac{n_{1}}{n_{2}} \cdot \theta_{1},
        \end{array}
    \end{equation*}
    and the corresponding transfer matrix is
    \begin{equation*}
        \mat{M} = 
        \begin{bmatrix}
            1 & 0\\
            \frac{n_{1} - n_{2}}{n_{2}} \frac{1}{R} & \frac{n_{1}}{n_{2}}
        \end{bmatrix},
    \end{equation*}
    Note that in the limit $ R \to \infty $, which corresponds geometrically to a straight boundary, the transfer matrix approaches the result in \hyperref[sss:M-straight-boundary]{\underline{Example \ref{sss:M-straight-boundary}}}, which should make sense.

\end{itemize}

\subsubsection{Concluding Remarks on the Transfer Matrix}

\textbf{The Determinant of a Transfer Matrix}
\begin{itemize}

    \item In general, the determinant of a transfer matrix encoding the passage of light between two regions equals the ratio of the indexes of refraction in the two materials. 

    \item As an example, the determinant of the transfer matrix in \hyperref[sss:M-homogeneous]{\underline{Example \ref{sss:M-homogeneous}}}, is
    \begin{equation*}
        \det 
        \begin{bmatrix}
            1 & L\\
            0 & 1
        \end{bmatrix}
        = 1,
    \end{equation*}
    which corresponds to the relationship $ n_{1} = n_{2} $ in that example.

    \item Analogously, the determinants of the transfer matrices in Examples \ref{sss:M-straight-boundary} and \ref{sss:M-curved-boundary}, corresponding to passage between materials with refractive indexes $ n_{1} $ and $ n_{2} $, are
    \begin{equation*}
        \det
        \begin{bmatrix}
            1 & 0\\
            1 & \frac{n_{1}}{n_{2}}
        \end{bmatrix} = 
        \det 
        \begin{bmatrix}
            1 & 0\\
            \frac{n_{1} - n_{2}}{n_{2}} \frac{1}{R} & \frac{n_{1}}{n_{2}}
        \end{bmatrix}
        = \frac{n_{1}}{n_{2}}.
    \end{equation*}

\end{itemize}

\textbf{Transfer Matrices For Multiple Boundaries}
\begin{itemize}

    \item So far, we have considered only the passage of light through a single boundary. Conveniently, the cumulative transfer matrix for passage of light between multiple boundaries is simply the product of the individual transfer matrices. In equation for, the total transfer matrix encoding a light ray's transformation through $ N $ boundaries, each with the individual transfer matrix $ \mat{M}_{i} $, where $ i = 1, \ldots, N $, simply
    \begin{equation*}
        \mat{M}_{\text{total}} = \mat{M}_{N}\mat{M}_{n-1} \cdots \mat{M}_{2} \mat{M}_{1}.
    \end{equation*}
    Note, however, that the order of multiplication is important, since matrix multiplication is not commutative. The transfer matrix for the first boundary, i.e. the boundary the light hits first, is furthest to the right.

    \item In terms of the transfer matrix formalism, the light ray coordinates $ x_{N} $ and $ \theta_{N} $ and $ x_{1} $ and $ \theta_{1} $ on either side of series of $ N $ boundaries are related by
    \begin{equation*}
        \begin{bmatrix}
            x_{n} \\
            \theta_{n}
        \end{bmatrix}
        = \mat{M}_{n} \mat{M}_{n-1} \cdots \mat{M}_{2} \mat{M}_{1} 
        \begin{bmatrix}
            x_{1}\\
            \theta_{1}
        \end{bmatrix}
        = \mat{M}_{\text{total}}
        \begin{bmatrix}
            x_{1}\\
            \theta_{1}
        \end{bmatrix}.
    \end{equation*}

\end{itemize}

\subsection{Lenses: TODO}


\end{document}
