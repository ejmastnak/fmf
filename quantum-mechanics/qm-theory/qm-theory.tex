\documentclass[11pt, a4paper]{article}
\usepackage[T1]{fontenc}
\usepackage{mwe}
\usepackage[margin=3cm]{geometry}
\usepackage{fancyhdr}
\usepackage{truncate}
\usepackage{amsmath}
\usepackage{amssymb}
\usepackage{esint}
\usepackage{bm} % for bold vectors in math mode
\usepackage{physics} % many useful physics commands
\usepackage[separate-uncertainty=true]{siunitx} % for scientific notation and units
\usepackage{xcolor}  % to color hyperref links
\usepackage[colorlinks = true, allcolors=blue]{hyperref}

\setlength{\parindent}{0pt} % to stop indenting new paragraphs
\newcommand{\diff}{\mathop{}\!\mathrm{d}} % differential
\newcommand{\dr}{\diff^{3} \r}  % d^3 r
\newcommand{\ddp}{\diff^{3} \vec{p}}  % d^3 p (\dp is a TeX primitive)


\renewcommand{\div}{\nabla \cdot}
\renewcommand{\curl}{\nabla \cross}
\renewcommand{\grad}{\nabla}
\renewcommand{\laplacian}{\nabla^{2}}

\newcommand{\minus}{\mspace{0.5mu}\scalebox{0.9}{-}}
\newcommand{\plus}{\scalebox{0.95}{+}}

\newcommand{\eqtext}[1]{\qquad \text{#1} \qquad}
\newcommand{\Schro}{Schr\"{o}dinger\xspace}
\newcommand{\Ham}{Hamiltonian\xspace}
\newcommand{\Herm}{Hermitian\xspace}
\newcommand{\Mol}{M\o ller\xspace}

\newcommand{\CG}{Clebsch-Gordan}

\renewcommand{\vec}[1]{\bm{#1}}  % for vectors
\renewcommand{\op}[1]{\hat{#1}}  % for operators
\newcommand{\mat}[1]{\mathbf{#1}}  % for matrices with a tilde
\newcommand{\dvec}[1]{\dot{\vec{#1}}}  % for dotted vector quantity
\newcommand{\tvec}[1]{\tilde{\vec{#1}}}  % for tilde vector quantities
\newcommand{\uvec}[1]{\hat{\vec{#1}}}  % for dotted vector quantity

\renewcommand{\t}[1]{\tilde{#1}}
\newcommand{\F}[1]{\widehat{#1}} % fourier transform


\renewcommand{\H}{\mathcal{H}}  % Hilbert space
\newcommand{\ua}{\uparrow}  % for spin up states
\newcommand{\da}{\downarrow}  % for spin down states
\renewcommand{\r}{\vec{r}}  % position vector
\renewcommand{\L}{\vec{L}}  % angular momentum
\renewcommand{\S}{\vec{S}}  % spin
\newcommand{\J}{\vec{J}}  % angular momentum
\newcommand{\Q}{\vec{Q}}  % parameter vector for adiabatic transitions
\newcommand{\gq}{\grad_{\mspace{-3mu}Q}\mspace{1mu}} 

\renewcommand{\SS}{\mat{S}}  % scattering matrix
\newcommand{\M}{\mat{M}}  % transfer matrix
\newcommand{\MM}{\mathrm{M}}  % transfer matrix

% particle in an electromagnetic field
\newcommand{\A}{\vec{A}}  % magnetic vector potential
\newcommand{\B}{\vec{B}}  % magnetic field
\newcommand{\m}{\vec{\mu}}  % magnetic dipole moment

% various operator quantities
\renewcommand{\O}{\mathcal{O}}  % script operator quantity
\newcommand{\II}{\operatorname{I}}  % non-bold identity operator
\newcommand{\T}{\mathcal{T}}  % time reversal operator
\newcommand{\Par}{\mathcal{P}}  % parity operator
\newcommand{\tev}{e^{-i\frac{H}{\hbar}t}}  % time evolution operator
\newcommand{\tevp}{e^{i\frac{H}{\hbar}t}}  % time evolution operator with positive exponent


\newcommand{\p}{\psi}  % time-independent wavefunction
\renewcommand{\P}{\Psi}  % time-dependent wavefunction


% \big braket-style commands
\newcommand{\evb}[1]{\big \langle {#1} \big \rangle}  % for big expectation values
\newcommand{\bket}[1]{\big | {#1} \big \rangle }
\newcommand{\bbra}[1]{ \big \langle {#1} \big |  }
\newcommand{\bbraket}[2]{\big \langle {#1} \big | {#2} \big \rangle}  % for brakets of fixed size
\newcommand{\bmel}[3]{\big \langle {#1} \big | {#2} \big | {#3} \big \rangle}  % for matrix elements of fixed size
% end

% shorthand bra and ket
\renewcommand{\b}[1]{\bra{#1}}
\newcommand{\bb}[1]{\bbra{#1}}
\renewcommand{\k}[1]{\ket{#1}}
\newcommand{\bk}[1]{\bket{#1}}

% begin header configuration
\fancypagestyle{headerstyle}{
\fancyhf{}  % clear default settings from header and footer; create blank slate
\fancyhead[R]{\href{https://github.com/ejmastnak/fmf}{\small{\texttt{github.com/ejmastnak/fmf}}}}  % link in upper right
\fancyhead[L]{\textit{\truncate{0.65\headwidth}{\rightmark}}}  % subsection name; truncate long names
\fancyfoot[C]{\thepage}  % centered page number in footer
\renewcommand{\headrulewidth}{0.1pt}
}
% end header configuration

\begin{document}
\title{Quantum Mechanics Lecture Notes}
\author{Elijan Mastnak}
\date{2020-21 Winter Semester}
\maketitle

\begin{center}
\textbf{About These Notes}
\end{center}
These are my lecture notes from the course \textit{Kvanta Mehanika} (Quantum Mechanics), a mandatory course for third-year physics students at the Faculty of Math and Physics in Ljubljana, Slovenia. The exact material herein is specific to the physics program at the University of Ljubljana, but the content is fairly standard for an late-undergraduate course in quantum mechanics. I am making the notes publicly available in the hope that they might help others learning the same material.


\vspace{2mm}
\textit{Navigation}: For easier document navigation, the table of contents is ``clickable'', meaning you can jump directly to a section by clicking the colored section names in the table of contents. Unfortunately, the \textit{clickable links do not work in most online or mobile PDF viewers}; you have to download the file first.

\vspace{2mm}
\textit{On Content}: The material herein is far from original---it comes almost exclusively from Professor Anton Ram\v{s}ak's lecture notes on quantum mechanics at the University of Ljubljana. I take credit for nothing beyond translating the notes to English and typesetting.

\vspace{2mm}
\textit{Disclaimer:} Mistakes---both trivial typos and legitimate errors---are likely. Keep in mind that these are the notes of an undergraduate student in the process of learning the material himself---take what you read with a grain of salt. If you find mistakes and feel like telling me, by \href{https://github.com/ejmastnak/fmf}{\underline{GitHub}} pull request, \href{mailto:ejmastnak@gmail.com}{\underline{email}} or some other means, I'll be happy to hear from you, even for the most trivial of errors.

\newpage

\tableofcontents

\newpage

\pagestyle{headerstyle}
\section{Fundamentals of Wave Quantum Mechanics}

\subsection{Understanding the \Schro Equation} 
Idea: find the simplest equation the satisfies the following quantum mechanical properties:
\begin{itemize}
	\item A particle has wave characteristics---a wavelength $ \lambda = 2\pi/k $ and frequency $ \nu = 2\pi/\omega $
	
	\item A particle's energy is proportional to its frequency, i.e. $ E = \hbar \omega $ (e.g. photoelectric effect)
	
	\item Momentum is related to a wave vector via $ \vec{p} = \hbar \vec{k} $ (de Broglie)
	
	\item A free particle has the classical energy $ E = \frac{p^{2}}{2m} $ and thus the dispersion relation $ \omega \propto k^{2} $
\end{itemize}
The \Schro equation satisfies these requirements
\begin{equation*}
	i \hbar \pdv{\Psi(\r, t)}{t} = - \frac{\hbar^{2}}{2m}\laplacian \P(\r, t) + V(\r, t)\Psi(\r, t)
\end{equation*}
The potential terms accounts for a particle having energy $ E = \frac{p^{2}}{2m} + V $ in a potential. 

Note that the \Schro equation and thus the solution $ \P $ are complex---we can decompose $ \P $ into a real and imaginary part via
\begin{equation*}
	\P = \P_{\text{Re}} + i \P_{\text{Im}}
\end{equation*}
Substituting this decomposition into the \Schro equation produces
\begin{equation*}
	i \hbar (\dot{\P}_{\text{Re}} + i \dot{\P}_{\text{Im}})  = - \frac{\hbar^{2}}{2m} (\P_{\text{Re}}'' + i \P_{\text{Im}}'') + V(\P_{\text{Re}} + i \P_{\text{Im}})
\end{equation*}
Writing the real and imaginary parts separately gives the coupled system of real equations
\begin{align*}
	&- \hbar \dot{\P}_{\text{Im}} = - \frac{\hbar^{2}}{2m}\Psi_{\text{Re}}'' + V \P_{\text{Re}}\\
	& - \hbar \dot{\P}_{\text{Re}} = - \frac{\hbar^{2}}{2m}\Psi_{\text{Im}}'' + V \P_{\text{Im}}
\end{align*}
Precisely this coupling leads to the desired oscillation and wavelike behavior of the wavefunction $ \P $, even though the \Schro equation is first degree in time.

\vspace{2mm}
\textbf{On the Diffusion Equation}\\
Note the similarity of the \Schro equation to the diffusion equation
\begin{equation*}
	\pdv{T}{t} = D \pdv[2]{T}{x}
\end{equation*}
Both are first degree in time and second degree in position. The wave-like ansatz $ T(x, t) \propto e^{i(kx - \omega t)} $ solves the diffusion equation with a quadratic dispersion relation
\begin{equation*}
	\omega = -iDk^{2},
\end{equation*}
as desired. However, the energy relation requirement $ E = \frac{p^{2}}{2m} $ holds only for $ D = \frac{i\hbar}{2m} \in \mathbb{C} $, and a complex diffusion constant is non-physical. We thus reject the diffusion equation.

\subsection{Probability Interpretation of the Wavefunction}

\begin{itemize}
	\item The wavefunction encodes the probability of detecting a quantum particle in a region of space. We use the wavefunction to define the probability density
	\begin{equation*}
		\rho (\r, t) = \abs{\P(\r, t)}^{2}
	\end{equation*}
	The probability $ \diff P $ of finding the particle in the region of space $ \diff \r $ is
	\begin{equation*}
		\diff P = \rho(\r, t) \diff \r
	\end{equation*}
	Logically, the probability of finding the particle somewhere in all of space $ V $ is one:
	\begin{equation*}
		\int_{V} \abs{\P(\r, t)}^{2}\dr \equiv 1.
	\end{equation*}
	The above relation is called the normalization condition on the wavefunction.
	
	\item If a wavefunction is normalized at a given point in time, we would assume it is normalized at all other times, too. We show this ``conservation of normalization'' by differentiating probability density with the product rule:
	\begin{align*}
		  \pdv{\rho(\r, t)}{t} = \pdv{\abs{\P(\r, t)}^{2}}{t} &= \pdv{\P^{*}(\r, t)}{t} \P(\r, t) + \pdv{\P(\r, t)}{t} \P^{*}(\r t)\\
	\end{align*}
	We substitute in $ \dot{\P} $ from the \Schro equation to get
	\begin{align*}
		\pdv{\rho(\r, t)}{t} = &\left(- \frac{\hbar i}{2m}\laplacian \P^{*}(\r, t) + \frac{i}{\hbar} V^{*}(\r, t)\Psi^{*}(\r, t)\right)\P(\r, t) \\
		& + \left(\frac{\hbar i}{2m}\laplacian \P(\r, t) - \frac{i}{\hbar} V(\r, t)\Psi(\r, t)\right)\Psi^{*}(\r, t)
	\end{align*}
	where we have allowed the possibility of complex potential $ V(\r, t) \in \mathbb{C} $ when conjugating the \Schro equation. We then use the identity
	\begin{equation*}
		\P \laplacian \P^{*} = \div (\P \grad \P^{*}) - \grad \P \cdot \grad \P^{*}
	\end{equation*}
	to write 
	\begin{equation*}
		\pdv{\rho(\r, t)}{t} + \div \vec{j}(\r, t) = q(\r, t)
	\end{equation*}
	where we have defined the probability current
	\begin{equation*}
		\vec{j}(\r, t) = \frac{\hbar}{2im}\big[\P(\r, t)\grad\P(\r, t) - \P(\r, t)\grad\P^{*}(\r, t)\big]
	\end{equation*}
	and the probability source density
	\begin{equation*}
		q(\r, t) = 2 \Im \big[V(\r, t)\rho(\r, t)\big]
	\end{equation*}
	
	\item \textit{Important}: Note that probability is conserved when $ q(\r, t) = 0$, resulting in the continuity equation
	\begin{equation*}
		\pdv{\rho(\r, t)}{t} + \div \vec{j}(\r, t) = 0
	\end{equation*}
	The source density $ q $ is zero if $ V $ is a real function.
	
\end{itemize}
\textbf{Quantum Tomography: $ \P $ from $ \abs{\P}^{2} $}
\begin{itemize}
	\item If you know a system's probability density $\abs{\P(\r, t)}^{2} $, it is possible to reconstruct the wavefunction $ \P $. This process is called quantum tomography. We consider only the one-dimensional case. 
	
	\item First, we write the wavefunction in the polar form with complex modulus $ \abs{\P} = \sqrt{\rho(x, t)} $ and phase $ S(x, t) $
	\begin{equation*}
		\P(x, t) = \sqrt{\rho(x, t)}e^{\frac{iS(x, t)}{\hbar}}
	\end{equation*}
	We substitute this expression for $ \P $ into the probability current density to get
	\begin{align*}
		j(x, t) &\equiv \frac{\hbar}{2im}\left(\P^{*}(x, t)\pdv{x}\P(x, t) - \P(x, t)\pdv{x}\P^{*}(x, t)\right) \\
		& = \frac{1}{m} \rho(x, t)\pdv{S(x, t)}{x}
	\end{align*}
	
	\item Substituting the above expression for $ j(x, t) $ into the probability continuity equation gives
	\begin{equation*}
		\pdv{\rho(x, t)}{t} + \frac{1}{m}\pdv{x}\left[\rho(x, t) \pdv{S(x, t)}{x}\right] = 0
	\end{equation*}
	We then integrate the equation with respect to $ x $ and rearrange to get
	\begin{equation*}
		\frac{\rho(x, t)}{m} \pdv{S(x, t)}{x} = - \int_{-\infty}^{x} \pdv{\rho(\chi, t)}{t}\diff \chi
	\end{equation*}
	where we have assumed the boundary condition $ \rho(-\infty, t) \to 0 $ for the lower limit of integration and $ \chi $ is a dummy variable for integration. We solve for the wavefunction's phase $ S $ to get
	\begin{equation*}
		S(x, t) = S_{0} - \int_{-\infty}^{x} \left[ \frac{m}{\rho(\xi, t)} \int_{-\infty}^{x} \pdv{\rho(\chi, t)}{t}\diff \chi\right] \diff \xi
	\end{equation*}
	
	\item \textit{Important}: The above expression for $ S(x, t) $ shows that, when finding $ \S(x, t) $ from probability density $ \rho(x, t) $, complex phase is determined only up to a constant phase factor $ e^{iS_{0}} $.
\end{itemize}

\subsection{Stationary States}
\begin{itemize}
	\item ``Standing wavefunctions'' in the \Schro equation, in analogy with standing waves in the wave  equation, occur when the wavefunction can be factored into the product of a position-dependent and time-dependent wavefunction in the form
	\begin{equation*}
		\P(\r, t) = \p(\r)f(t)
	\end{equation*}
	Such solutions $ \P $ are called \textit{stationary states}. 
	
	\item Derivation of stationary states: assume the potential is independent of time, i.e. $ V = V(\r) $. Substitute the ansatz $ \P(\r, t) = \p(\r)f(t) $ into the \Schro equation to get
	\begin{equation*}
		i \hbar \p(\r) \pdv{f(t)}{t} = - \frac{\hbar^{2}}{2m} f(t) \laplacian \p(\r) + V(\r)f(t)\p(\r)
	\end{equation*}
	Next, divide by $ \p(\r)f(t) $ to get
	\begin{equation*}
		\frac{i \hbar}{f(t)} \pdv{f(t)}{t} = - \frac{\hbar^{2}}{2m}\frac{\laplacian \p(\r)}{\p(\r)} + V(\r) 
	\end{equation*}
	Since the left-hand side of the equation depends only on time, and the right-hand side only on position, the equality holds for all $ t $ and $ \r $ only if both sides are constant. We make this requirement explicit by writing
	\begin{equation*}
		\frac{i \hbar}{f(t)} \pdv{f(t)}{t} = - \frac{\hbar^{2}}{2m}\frac{\laplacian \p(\r)}{\p(\r)} + V(\r) \equiv E
	\end{equation*}
	where the constant $ E $ represents the stationary state's energy.
	
	\item We use the position-dependent portion of the separated equation to form the stationary \Schro equation
	\begin{equation*}
		-\frac{\hbar^{2}}{2m}\laplacian \p_{n}(\r) + V(\r) \p_{n}(\r) = E_{n}\p_{n}(\r), \qquad n \in \mathbb{N}.
	\end{equation*}
	Note that this is an eigenvalue equation with for the stationary state eigenfunctions $ \p_{n} $ and energy eigenvalues $ E_{n} $.
	
	Meanwhile, we solve the time-dependent portion of the separated equation to get
	\begin{equation*}
		f(t) = e^{-i\frac{E_{n}}{\hbar}t} \equiv e^{-i\omega_{n}t},
	\end{equation*}
	which represents oscillation in time with at the frequency $ \omega_{n} $, which satisfies the familiar quantum-mechanical relation $ E_{n} = \hbar \omega_{n} $. 
	
	\item The complete set of stationary state eigenfunctions $ \{\p_{n}(\r)\} $ form an orthonormal basis of the wavefunction solution space and satisfy the relation
	\begin{equation*}
		\int \p^{*}_{n}(\r)  \p_{m}(\r) \dr = \delta_{nm},
	\end{equation*}
	where $ \delta_{nm} $ is the Kronecker delta. 
	
	\item It is possible to write any solution $ \P(\r, t) $ \Schro equation in terms of the eigenfunction basis. To do this, we first expand the wavefunction $ \P $'s initial state $ \P(\r, 0) $ in the eigenfunction basis in the form
	\begin{equation*}
		\P(\r, 0) = \sum_{n}c_{n} \p_{n}(\r), \qquad c_{n} = \int \p_{n}^{*}\P(\r, 0)\dr.
	\end{equation*}
	We then write the solution $ \P(\r, t) $ at arbitrary time in the form
	\begin{equation*}
		\P(\r, t) = \sum_{n}c_{n}e^{-i\frac{E_{n}}{\hbar}t}\p_{n}(\r),
	\end{equation*}
	where $ c_{n} $ are the coefficients from the expansion of $ \P(\r, 0) $ in the eigenfunction basis and $ E_{n} $ are the eigenfunction' corresponding energy eigenvalues.
\end{itemize}

\subsection{Differentiability of the First and Second Wavefunction Derivatives}
\begin{itemize}
	\item The wavefunction is assumed to be a continuous quantity. What about its derivative? We integrate the stationary \Schro equation on the interval $ x \in [a, b] $ to get
	\begin{equation*}
		-\frac{\hbar^{2}}{2m}\int_{a}^{b}\p''(x) \diff x + \int_{a}^{b}V(x)\p(x)\diff x = E \int_{a}^{b}\p (x) \diff x
	\end{equation*}
	Evaluating the integral of $ \psi''(x) $ and rearranging gives
	\begin{equation*}
		\psi'(b) - \psi'(a) = \frac{2m}{\hbar^{2}}\int_{a}^{b}V(x) \p(x) \diff x - \frac{2mE}{\hbar^{2}}\int_{a}^{b}\p(x)\diff x
	\end{equation*}
	
	\item We are interested in the limit behavior $ a \to b $. Since $ \p $ is continuous, we have $ \int_{a}^{b}\p(x)\diff x \to 0 $ as $ a \to b $ (from introductory real analysis). As long as $ V(x) $ is continuous, then $ V(x)\p(x) $ is also continuous, implying $ \int_{a}^{b}V(x)\p(x)\diff x \to 0 $ as $ a \to b $. We then have
	\begin{equation*}
		\lim_{a \to b} \big[\psi'(b) - \psi'(a)\big] = \frac{2m}{\hbar^{2}} \cdot 0 - \frac{2mE}{\hbar^{2}} \cdot 0 = 0.
	\end{equation*}
	The resulting equality $ \lim_{a \to b} \big[\psi'(b) - \psi'(a)\big] = 0 $ implies the wavefunction derivative $ \psi' $ is also a continuous function.
	
	\item If the potential takes the form of a delta function, i.e. $ V(x) = \lambda \delta (x) $ where $ \lambda $ is a constant, the wavefunction's first  derivative has a discontinuity of the form
	\begin{equation*}
		\lim_{a \to b} \big[\psi'(b) - \psi'(a)\big] = \frac{2m\lambda}{\hbar^{2}}\psi(a)
	\end{equation*}
	
	\item To analyze the second derivative, we write the \Schro equation in the form
	\begin{equation*}
		\frac{1}{\psi(x)} \dv[2]{\p(x)}{x} = \frac{2m}{\hbar^{2}}\big[V(x) - E\big].
	\end{equation*}
	By observing the sign of $ \psi''(x) $ based on the value of $ E $, we see that $ \p $ is concave where $ E > V $ and convex where $ E < V $. 
	
	\item Points of inflection (zeros of $ \psi'' $) occur at the classically-expected turning points where $ E = V $. The wavefunction must be smooth at the turning points to satisfy the continuity conditions on $ \p $ and $ \p' $. 
\end{itemize}

\subsection{Degeneracy and the Nondegeneracy Theorem}
\begin{itemize}
	\item Consider the one-dimensional stationary \Schro equation 
	\begin{equation*}
		-\frac{\hbar^{2}}{2m}\psi_{n}''(x) + V(x)\p_{n}(x) = E_{n}\p_{n}(x).
	\end{equation*}
	An energy eigenvalue $ E $ is called \textit{degenerate} if their exist multiple linearly independent eigenfunctions, e.g. $ \psi_{1}, \psi_{2} $, with the same energy eigenvalue $ E $. The nondegeneracy theorem states that the energy eigenvalue spectrum $ \{E_{n}\} $ of a one-dimensional system is nondegenerate, as long as the wavefunctions $ \psi_{n} $ vanish at $ \pm \infty $.
	
	\item The stationary \Schro equation for the two eigenfunctions reads
	\begin{align*}
		& -\frac{\hbar^{2}}{2m}\psi_{1}''(x) + \big[V(x) - E\big]\p_{1}(x) = 0\\
		& -\frac{\hbar^{2}}{2m}\psi_{2}''(x) + \big[V(x) - E\big]\p_{2}(x) = 0
	\end{align*}
	We multiply the first equation by $ \p_{1} $, the second by $ \p_{2} $ and subtract the equations to get
	\begin{equation*}
		\p_{1}\dv[2]{\p_{2}}{x} - \p_{2}\dv[2]{\p_{1}}{x} = 0
	\end{equation*}
	
	\textit{Mathematical aside}: the Wronskian determinant of the wavefunctions $ \p_{1} $ and $ \p_{2} $ is
	\begin{equation*}
		W_{12} \equiv \det 
		\begin{pmatrix}
			\p_{1} & \p_{2}\\
			\p_{1}' & \p_{2}'
		\end{pmatrix}
		= \p_{1}\p_{2}' - \p_{2}\p_{1}'.
	\end{equation*}
	
	\item In terms of the Wronskian, the above equation relating $ \p_{1} $, $ \p_{2} $ and their second derivatives reads
	\begin{equation*}
		\dv{x}\left(\p_{1}\dv{\p_{2}}{x} - \p_{2}\dv{\p_{1}}{x}\right) = \dv{W_{12}}{x} = 0,
	\end{equation*}
	which implies the Wronskian is constant with respect to $ x $. 
	
	\item Next, we apply the condition $ \p_{1, 2} \to 0 $ and $ \p'_{1, 2} \to 0 $ as $ \abs{x} \to \infty $, which implies $ W_{12} \to 0 $ as $ \abs{x} \to \infty  $. This implies $ W_{12} = 0 $ for all $ x $, since $ W $ is constant with respect to $ x $. The result $ W_{12} = 0 $ implies
	\begin{equation*}
		\p_{1}\dv{\p_{2}}{x} = \p_{2}\dv{\p_{1}}{x} \implies \frac{1}{\p_{1}} \dv{\p_{1}}{x} - \frac{1}{\p_{2}} \dv{\p_{2}}{x} = \dv{x}(\ln \p_{1} - \ln \p_{2}) = 0.
	\end{equation*}
	Integrating the final equality produces
	\begin{equation*}
		\ln \p_{1} - \ln \p_{2} = \ln \frac{\p_{1}}{\p_{2}} = C \implies \p_{1}(x) = \t{C} \p_{2}(x).
	\end{equation*}
	where $ \t{C} $ is a constant. In other words, $ \p_{1} $ and $ \p_{2} $ are linearly dependent, implying the one-dimensional energy spectrum $ \{E_{n}\} $ is nondegenerate, as long as $ \p_{1,2} $ vanish at infinity.
\end{itemize}

\subsection{Expectation Value}
\begin{itemize}
	\item Assume we know a particle's wavefunction $ \psi(x, t) $ and the associated probability density $ \rho(x, t) = \abs{\P(x, t)}^{2} $. 
	
	The moments of the probability density are called expectation values. The probability density's $ n $-th moment is defined just like the mathematical definition of a probability distribution's moment, i.e.
	\begin{equation*}
		\ev{x^{n}} = \int_{-\infty}^{\infty} x^{n}\rho(x, t) \diff x = \int_{-\infty}^{\infty} \Psi^{*}(x, t)x^{n}\Psi(x,t)\diff x
	\end{equation*}
	In general, all of the probability density's moments may not exist. 
	
	\item In quantum mechanics, we generally restrict ourselves to those wavefunctions in the Schwartz space of rapidly falling functions. This space consists of those $ \psi \in L^{2} $ that are infinitely differential and the fall rapidly as $ \abs{x} \to \infty $, i.e. those $ \psi $ for which there exists finite constant $ M \in \mathbb{R} $ such that
	\begin{equation*}
		x^{n} \abs{\psi(x)}^{m} < M \qquad \text{for all } n, m \in \mathbb{N} \text{ and all } x \in \mathbb{R}
	\end{equation*}
	Physical interpretation for why we require wavefunctions fall rapidly: in physical experiments, we expect the majority of the probability for detecting a particle is concentrated in the neighborhood of the experiment and not at infinity. 
\end{itemize}
\textbf{Example: The Momentum Operator}
\begin{itemize}
	\item We begin by finding the derivative of the position expectation value.
	\begin{align*}
		\dv{\ev{x}}{t} &= \pdv{t}\int_{-\infty}^{\infty} \Psi^{*}(x, t)x\Psi(x,t)\diff x \\
		&= \int_{-\infty}^{\infty} \left(\pdv{\P^{*}(x, t)}{t} x \P(x, t) + \P^{*}(x, t)x\pdv{\P(x, t)}{t}\right)\diff x
	\end{align*}
	Assuming a real potential $ V(x) $, we can express $ \pdv{\P^{*}}{t} $ and $ \pdv{\P}{t} $ in terms of $ \pdv[2]{\P^{*}}{x} $ and $ \pdv[2]{\P}{x} $ using the \Schro equation, substitute these expressions in to the above expression for $ \dv{\ev{x}}{t} $, and simplify like terms to get
	\begin{equation*}
		\dv{\ev{x}}{t} = \frac{\hbar}{2im}\int_{-\infty}^{\infty} \left( \pdv[2]{\P^{*}(x, t)}{x} x \P(x, t) - \pdv[2]{\P(x, t)}{x} x \P^{*}(x, t) \right)
	\end{equation*}
	We then rewrite this expression with a reverse-engineered derivative with respect to $ x $:
	\begin{align*}
		\dv{\ev{x}}{t} = & \frac{\hbar}{2im} \int_{-\infty}^{\infty} \pdv{x}\left(\pdv{\P^{*}}{x}x\P - \abs{\P}^{2} - \P^{*}x\pdv{\P}{x}\right)\diff x\\
		& + \frac{1}{m} \int_{-\infty}^{\infty} \P^{*} \left(-i\hbar \pdv{x}\P\right)\diff x.
	\end{align*}
	For rapidly falling wavefunctions in the Schwartz space, the first integral evaluates to zero. We are left with
	\begin{equation*}
		\dv{\ev{x}}{t} = \frac{1}{m} \int_{-\infty}^{\infty} \P^{*} \left(-i\hbar \pdv{x}\P\right)\diff x
	\end{equation*}
	
	\item The above result for $ \dv{\ev{x}}{t} $, written in the form momentum-like form
	\begin{equation*}
		m \dv{\ev{x}}{t} = \ev{p} = \int_{-\infty}^{\infty} \P^{*} \left(-i\hbar \pdv{x}\P\right)\diff x,
	\end{equation*}
	motivates the introduction of the momentum operator
	\begin{equation*}
		\hat{p} \to - \hbar \pdv{x} \implies \ev{p} = \int_{-\infty}^{\infty} \P \hat{p} \P \diff x
	\end{equation*}
	
	\item \textbf{Notation:} The hat in $ \hat{p} $ explicitly indicates the quantity in question is an operator. By convention, however, we usually write operators without the hat symbol and distinguish between operators and scalar quantities based on context.
	
	\item In three dimensions, the momentum operator generalizes to 
	\begin{equation*}
		\hat{\vec{p}} \to i \hbar \grad \eqtext{and}  \ev{\hat{\vec{p}}} = m \dv{\ev{\hat{\vec{r}}}}{t}
	\end{equation*}
	
	\item The momentum operator (dropping the hat notation) and probability current density are related by
	\begin{equation*}
		\vec{j}(\r, t) = \frac{1}{m}\Re \big[\P^{*}(\r, t) \vec{p} \P(\r, t)\big] \eqtext{and} \ev{\bm{p}}  = m \int_{V} \vec{j}(\r, t) \dr
	\end{equation*}
	We discuss operators formally in the following section.
\end{itemize}


\subsection{Operators}
\begin{itemize}
	\item In quantum mechanics, every measurable quantity---called an \textit{observable}---is assigned a corresponding operator. Some common operators are
	\begin{equation*}
		\hat{x} \to x \mathrm{I} \qquad \hat{\r} \to \r \mat{I} \qquad \hat{V} \to V(\r, t)\mathrm{I},
	\end{equation*}
	where $ \mat{I} $ is the identity operator. We typically leave the identity operator implicit and write e.g. $ \hat{x} \to x $. 
	
	The momentum operator in various forms reads
	\begin{equation*}
		p_{\alpha} = -i\hbar \pdv{}{\alpha} \qquad p_{\alpha} = (-i\hbar)^{n} \pdv[n]{}{\alpha} \qquad \vec{\hat{p}} = \sum_{\alpha = x, y, z} \hat{p}_{\alpha} = - i \hbar \grad \qquad \vec{\hat{p}}^{2} = (-i\hbar)^{2} \laplacian.
	\end{equation*}
	
	\item \textbf{Notation:} In this section I will intermittently write operators with a hat, i.e. $ \hat{p} $. However, I stress again that by convention we usually write operators without the hat symbol and distinguish between operators and scalar quantities based on context. I will typically denote generic operator quantities by either $ \O $ or the capital Latin letters $ A $, $ B $, $ C $, $ \ldots  $.
	
	\item We can define operators as functions. Consider analytic complex function $ f(x) $ with the power series definition
	\begin{equation*}
		f(z) = \sum_{n=0}^{\infty}c_{n}z^{n}
	\end{equation*}
	As long as the function $ f $ is defined as a power series, we can define the function of an operator $ \O $, which is itself an operator, as
	\begin{equation*}
		f(\O) = \sum_{n = 0}^{\infty}c_{n} \O^{n}.	
	\end{equation*}
	For example, the exponential function of an operator $ \O $ is defined as via the exponential function's Taylor series as
	\begin{equation*}
		e^{\O} = \mathrm{I} + \O + \frac{\O^{2}}{2!} + \frac{\O^{3}}{3!} + \cdots + \frac{\O^{n}}{n!} + \cdots 
	\end{equation*}
	
	
	
	\item A common example of an operator constructed from other operators is the Hamiltonian $ H $, defined as
	\begin{equation*}
		H = \frac{p^{2}}{2m} + V.
	\end{equation*}
	We can use the Hamiltonian to concisely write the \Schro equation in operator form:
	\begin{equation*}
		i\hbar \pdv{\P}{t} = H\P.
	\end{equation*}
	Note that the Hamiltonian operator has the same form as the Hamiltonian function from classical mechanics. If we observe the stationary \Schro equation in operator form, i.e.
	\begin{equation*}
		H \p_{n} = E_{n}\p_{n},
	\end{equation*}
	we see the Hamiltonian's eigenvalues are a quantum system's energy eigenvalues $ E_{n} $.
	
	\item Functions of operators give simple results when applied to eigenvalue relations. Consider for example $ \O $ for which we know the eigenvalue relation $ \O \p =  \lambda \p $. In this case the operator function $ f(\O) $ applied to $ \p $ reads
	\begin{equation*}
		f(\O) \psi \equiv \left(\sum_{n = 0}^{\infty}c_{n} \O^{n}\right)\p = \sum_{n=0}^{\infty}c_{n} \left(\O^{n} \p\right) =  \sum_{n=0}^{\infty}c_{n} \lambda^{n} \p = f(\lambda) \p.
	\end{equation*}
	In other words, the operator expression $ f(\O) \psi $ reduces to the scalar expression $ f(\lambda) \p $. 
	
	\item Next, we consider the operator $ \pdv{x} $, which forms the basis of the momentum operator $ p_{x} $. Considering two wavefunctions $ \phi $ and $ \p $ and applying integration by parts, we have
	\begin{equation*}
		\int_{-\infty}^{\infty} \phi^{*} \pdv{\p}{x} \diff x = \phi^{*}\p \big |_{-\infty}^{\infty} - \int_{-\infty}^{\infty} \pdv{\phi^{*}}{x} \p  \diff x.
	\end{equation*}
	If the wavefunctions are well-behaved and vanish at infinity (as is commonly assumed for a wavefunction), the equality reduces to
	\begin{equation*}
		\int_{-\infty}^{\infty} \phi^{*} \pdv{\p}{x} \diff x = -\int_{-\infty}^{\infty} \pdv{\phi^{*}}{x} \p \diff x.
	\end{equation*}
	In other words, the action of the operator $ \pdv{x} $ on one wavefunction (e.g. $ \psi $) in the original integrand gives an asymmetric result in which the operator acts on the opposite wavefunction (e.g. $ \phi $) in the result. Because of the asymmetric minus sign, we say the operator $ \pdv{x} $ is antisymmetric or anti-Hermitian.
	
	Meanwhile, the operator $ \pdv[2]{}{x} $ is symmetric (or Hermitian):
	\begin{equation*}
		\int_{-\infty}^{\infty} \phi^{*} \pdv[2]{\p}{x} \diff x = \cdots = \int_{-\infty}^{\infty} \pdv[2]{\phi^{*}}{x} \p \diff x.
	\end{equation*}
	Note that the minus sign does not appear.
	
	\item The momentum operator $ p \to -i\hbar \pdv{x} $ is Hermitian---even though it contains the anti-Hermitian operator $ \pdv{x} $, the presence of the imaginary unit $ i $ recovers the operator's symmetry. We have
	\begin{equation*}
		\int_{-\infty}^{\infty} \phi^{*} p \psi \diff x = \int_{-\infty}^{\infty} \phi^{*} \left(-i\hbar \pdv{\p}{x}\right) = \int_{-\infty}^{\infty} \left(-i\hbar \pdv{\phi}{x}\right)^{*} \p \diff x = \int_{-\infty}^{\infty} (p \phi)^{*} \psi \diff x
	\end{equation*}
	Similarly, the operators $ x, p^{2}, V $ and $ H $ are all Hermitian\footnote{assuming the potential energy $ V $ is real}, i.e.
	\begin{equation*}
		\int_{-\infty}^{\infty} \phi^{*}\O \psi \diff x = \int_{-\infty}^{\infty} (\O \phi)^{*}\p \diff x
	\end{equation*}
	for $ \O = x, p^{2}, V, H $.
	
\end{itemize}

\subsection{Commutators}
\begin{itemize}
	\item The commutator in quantum mechanics is analogous to the Poisson bracket in classical mechanics. The commutator of two operators $ A $ and $ B $ is defined as
	\begin{equation*}
		[A, B] = AB - BA
	\end{equation*}
	If $ [A, B] = 0 $, the two operators are said to commute, in which case $ AB = BA $. If this is not the case, then $ A $ and $ B $ do not commute.
	
	Note that the commutator of two operators is in general also an operator.
	
	\item We calculate the value of a commutator by having the commutator act on an arbitrary wavefunction. As an example, we consider the commutator of position and momentum, which occurs frequently in quantum mechanics. We find $ [x, p] $ as follows:
	\begin{align*}
		[x, p] \p &\equiv (xp - px)\p = x \left(- i\hbar \pdv{x}\right)\p - \left(- i\hbar \pdv{x}\right)x \p \\
		& = - i\hbar x \p ' + i \hbar x \p ' + i \hbar \p = i \hbar \p
	\end{align*}
	The equality $ [x, p] \p = i \hbar \p $ implies $ [x, p] = i \hbar $.
	
	\item More generally, the operators $ \r $ and $ \vec{p} $ obey canonical commutation relations
	\begin{align*}
		& [r_{\alpha}, r_{\beta}] = 0\\
		& [p_{\alpha}, p_{\beta}] = 0\\
		& [r_{\alpha}, p_{\beta}] = i\hbar \delta_{\alpha \beta}
	\end{align*}
	for $ \alpha, \beta \in \{x, y, z\} $. These are analogous to the canonical Poisson bracket relationships between $ \vec{q} $ and $ \vec{p} $ in classical mechanics.
	
	\item Next, we quote some common commutator identities:
	\begin{align*}
		&[\lambda A, B] = \lambda [A, B], \qquad \lambda \in \mathbb{C} \\
		&[A, B] = - [B, A]\\
		&[A + B, C] = [A, C] + [B, C]\\
		& [AB, C] = A[B, C] + [A, C]B.
	\end{align*}
	
	\item Finally, we quote three more identities. The Jacobi identity is
	\begin{equation*}
		\big[A, [B, C]\big] + \big[B, [C, A]\big] + \big[C, [A, B]\big] = 0,
	\end{equation*}
	The Baker-Campbell-Hausdorff formula gives the solution to the equation $ e^{A}e^{B} = e^{C} $, which is
	\begin{equation*}
		C = A + B + \frac{1}{2}[A, B] + \frac{1}{12}\big[A - B, [A, B]\big] + \cdots.
	\end{equation*}
	Finally, the Baker-Hausdorff lemma is
	\begin{equation*}
		e^{A}Be^{-A} = B + [A, B] + \frac{1}{2!}\big[A, [A, B]\big] + \frac{1}{3!} \big[A, [A, [A, B]]\big] + \cdots
	\end{equation*}
	
\end{itemize}


\subsection{Uncertainty Principle}
\begin{itemize}
	\item Recall the for a probability distribution $ \rho(x, t) = \abs{\P(x, t)}^{2} $, position expectation values are defined as
	\begin{equation*}
		\ev{x^{n}} = \int_{-\infty}^{\infty} \P^{*}(x, t) x^{n} \P(x, t) \diff x.
	\end{equation*}
	With reference to this definition of $ \ev{x^{n}} $, we define the ``width'' of a probability distribution $ \rho $ as
	\begin{equation*}
		\Delta x = \sqrt{\ev{x^{2}} - \ev{x}^{2}}.
	\end{equation*}
	Note the equivalence of the width $ \Delta x $ to the familiar standard deviation of a statistical distribution. 
	
	\item More generally, we define the uncertainty of a quantum mechanical operator $ \O $ as
	\begin{equation*}
		\Delta \O = \sqrt{\ev{\O^{2}} - \ev{\O}^{2}},
	\end{equation*}
	where the expectation values $ \ev{\O^{n}} $ are defined as
	\begin{equation*}
		\ev{\O^{n}} = \int_{V} \P^{*}(\r, t) \O^{n} \P(\r, t) \dr.
	\end{equation*}
	
	\item We now quote an important result: the product of uncertainties of two operators $ A $ and $ B $ obeys the inequality
	\begin{equation*}
		\Delta A \Delta B \geq \frac{1}{2} \big | \evb{[A, B]} \big |.
	\end{equation*} 
	\textbf{TODO:} consider adding derivation from Exercises.
	
	This inequality, using the commutator $ [x, p] = i \hbar $, is responsible for the famous Heisenberg uncertainty principle
	\begin{equation*}
		\Delta x \Delta p \geq \frac{\hbar}{2}.
	\end{equation*}
	This inequality implicitly assumes two independent measurements of $ x $ and $ p $.
	
\end{itemize}

\subsection{Time-Dependent Expectation Values}
\begin{itemize}
	\item The time-dependent expectation value of an operator $ \O $ for a quantum system with the wavefunction $ \P(\r, t) $ is defined as
	\begin{equation*}
		\ev{\O, t} = \int_{V} \P^{*}(\r, t) \O \P(\r, t) \dr.
	\end{equation*}
	
	\item The time derivative of $ \ev{\O, t} $ is
	\begin{equation*}
		\dv{\ev{O, t}}{t} = \int_{V} \left(\pdv{\P^{*}}{t} \O \P + \P^{*}\pdv{\O}{t}\P + \P^{*}\O \pdv{\P}{t}\right) \dr.
	\end{equation*}
	We then use the \Schro equation to express time derivatives of $ \P $ in terms of position derivatives, i.e.
	\begin{equation*}
		\pdv{\P}{t} = \frac{1}{i\hbar}H \P \eqtext{and} \pdv{\P^{*}}{t} = -\frac{1}{i\hbar}(H \P)^{*}.
	\end{equation*}
	Substituting these expressions into the time derivative of $ \ev{\O, t} $ gives
	\begin{align*}
		\dv{\ev{O, t}}{t} &= \ev{\pdv{\O}{t}} + \frac{1}{i \hbar} \int_{V} \big[- (H\P)^{*}\O \P + \P^{*}\O H \P \big] \dr\\
		& = \ev{\pdv{\O}{t}} + \frac{1}{i \hbar} \int_{V} (\P^{*}\O H \P - \P^{*}H \O \P) \dr
	\end{align*}
	where we have used $ (H \P)^{*} = \P^{*}H^{*} $ and applied the Hermitian identity $ H^{*} = H $. 
	
	Finally, we use a commutator to compactly write the above result for time derivative of $ \ev{\O, t} $ in the form
	\begin{equation*}
		\dv{\ev{O, t}}{t} =  \ev{\pdv{\O}{t}} + \frac{1}{i\hbar}\ev{[\O, H]}.
	\end{equation*}
	Note the similarity to an analogous result from classical mechanics for a function $ f(p, q) $ of the canonical coordinates, in terms of Poisson brackets, which reads 
	\begin{equation*}
		\dv{f}{t} = \pdv{f}{t} + \{f, H\}.
	\end{equation*}
\end{itemize}

\subsection{The Ehrenfest Theorem}
\begin{itemize}
	\item The Ehrenfest theorem can be interpreted as an quantum-mechanical analog of Newton's second law. We start the derivation of the Ehrenfest theorem by considering the time-dependent expectation value of the position operator $ x $.  Using the above result 
	\begin{equation*}
		\dv{\ev{O, t}}{t} =  \ev{\pdv{\O}{t}} + \frac{1}{i\hbar}\ev{[\O, H]}
	\end{equation*}
	with $ \O = x $ and the identity $ \pdv{x}{t} = 0 $ produces the relationship
	\begin{align*}
		\dv{\ev{x, t}}{t} &= \frac{1}{i\hbar}\ev{[x, H]} = \frac{1}{i\hbar}\ev{\left[x, \frac{p^{2}}{2m} + V\right]}\\
		& = \frac{1}{2i\hbar m} \ev{[x, p^{2}]} + \frac{1}{i\hbar}\ev{[x, V]}
	\end{align*}
	
	\item We pause for a moment to calculate the two commutators. The first is
	\begin{equation*}
		[x, p^{2}] = p[x, p] + [x, p]p = p(i\hbar) + (i\hbar) p = 2i \hbar p
	\end{equation*}
	The second is simply $ [x, V] = 0 $, since $ x $ and $ V $ commute. 
	
	\item Using the just-derived intermediate results $ [x, p^{2}] = 2i\hbar p $ and $ [x, V(x, t)] = 0 $, the time derivative of $ \ev{x, t} $ is 
	\begin{equation*}
		\dv{\ev{x, t}}{t} = \frac{1}{2i\hbar m} \ev{2 i\hbar p} + \frac{1}{i\hbar}\ev{0} = \frac{1}{m}\ev{p, t},
	\end{equation*}
	in analogy with the classical result $ m \dot{x} = p $. 
	
	\item Next, we find the time derivative of $ \ev{p, t} $. Using the general result for the time derivative of an expectation value and implicitly recognizing $ \pdv{p}{t} = 0 $, we have
	\begin{align*}
		\dv{\ev{p, t}}{t} &= \frac{1}{i\hbar}\ev{[p, H]} = \frac{1}{i\hbar}\ev{\left[p, \frac{p^{2}}{2m} + V\right]}\\
		&= \frac{1}{2i \hbar m}\ev{[p, p^{2}]} + \frac{1}{i\hbar} \ev{[p, V]}
	\end{align*}
	
	\item Again, we pause to calculate the two commutators. The first is simply $ [p, p^{2}] = 0 $, which follows from $ [p, p] = 0 $ and $ [A, BC] = B[A, C] + [A, B]C $. We find the second as follows:
	\begin{align*}
		[p, V]\p &\equiv \left[ \left(-i\hbar \pdv{x}\right) V - V\left(-i\hbar \pdv{x}\right) \right] \p = - i \hbar f \pdv{V}{x} - i \hbar V \pdv{f}{x} + i \hbar V \pdv{f}{x} \\
		& =  - i \hbar \pdv{V}{x}f \implies [p, V] = - i \hbar \pdv{V}{x}
	\end{align*}
	
	\item Using the just derived intermediate results $  [p, p^{2}] = 0 $  and $ [p, V] = - i \hbar \pdv{V}{x} $, the time derivative of $ \ev{p, t} $ is 
	\begin{equation*}
		\dv{\ev{p, t}}{t} = \frac{1}{2i \hbar m}\ev{0} + \frac{1}{i\hbar} \ev{- i \hbar \pdv{V}{x}} = \ev{-\pdv{V}{x}}
	\end{equation*}
	
	\item \textit{Note}: I must confess that we have been guilty of a minor notational inconsistency---formally, we have been working with the $ x $ component of momentum $ p_{x} $, even though we have been writing just $ p $ for conciseness. With unambiguous notation, the above result would read
	\begin{equation*}
		\dv{\ev{p_{x}, t}}{t} = \ev{-\pdv{V}{x}}
	\end{equation*}
	We could then apply an analogous derivation for the coordinates $ y $ and $ z $ to get
	\begin{equation*}
		\dv{\ev{p_{y}, t}}{t} = \ev{-\pdv{V}{y}} \eqtext{and} \dv{\ev{p_{z}, t}}{t} =  \ev{-\pdv{V}{z}}
	\end{equation*}
	Putting the $ x, y $ and $ z $ results together and combining the three position derivatives into the single gradient operator gives the Ehrenfest theorem:
	\begin{equation*}
		\dv{\ev{\vec{p}, t}}{t} = \ev{-\grad V} = \ev{\vec{F}}, \qquad \text{where } \vec{F}(\r) = - \grad V(\r).
	\end{equation*}
	Note the similarity to Newton's second law $ \dot{\vec{p}} = \vec{F} $.
	
	\item Without proof, we quote a similar result relating angular momentum $ \vec{L} $ and torque $ \vec{M} $:
	\begin{equation*}
		\dv{\ev{\vec{L}, t}}{t} = \ev{\vec{M}}, \qquad \text{where } \vec{M}(\r) = \r \cross \vec{F} = - \r \cross \grad V(\r)
	\end{equation*}
	The proof analyzes $ \r $ and $ \p $ in terms of their Cartesian components and rests on the commutator identities
	\begin{equation*}
		[x_{\alpha}, x_{\beta}] = 0, \qquad [p_{\alpha}, p_{\beta}] = 0, \qquad [x_{\alpha}, p_{\beta}] = i \hbar \delta_{\alpha, \beta}.
	\end{equation*}
	
\end{itemize}

\subsection{Virial Theorem}
\begin{itemize}
	\item We derive the virial theorem in quantum mechanics by finding the time derivative of the expectation value $ \ev{\r \cdot \vec{p}, t} $. Again using the general result for the time derivative of an expectation value and recognizing $ \pdv{\r \cdot \vec{p}}{t} = 0 $, we have
	\begin{equation*}
		\dv{\ev{\r \cdot \vec{p}}}{t} = \frac{1}{i \hbar} \evb{[\r \cdot \vec{p}, H]}
	\end{equation*}
 	
 	\item We evaluate the commutator by components, starting with
 	\begin{equation*}
 		\big[ x_{\alpha} p_{\alpha}, H\big] = \left[x_{\alpha}p_{\alpha}, \frac{p_{\alpha}^{2}}{2m} + V\right] = \frac{x_{\alpha}}{2m} [p_{\alpha}, p_{\alpha}^{2}] + [x_{\alpha}, p_{\alpha}^{2}]\frac{p_{\alpha}}{2m} + x_{\alpha}[p_{\alpha}, V] + [x_{\alpha}, V]p_{\alpha}
 	\end{equation*}
 	We use the results $ [p_{\alpha}, p_{\alpha}^{2}] = [x_{\alpha}, V] = 0 $ and expand $ [x_{\alpha}, p_{\alpha}^{2}] $ to get
 	\begin{equation*}
 		\big[ x_{\alpha} p_{\alpha}, H\big] = \frac{p_{\alpha}}{2m}[x_{\alpha}, p_{\alpha}]p_{\alpha} + [x_{\alpha}, p_{\alpha}]\frac{p_{\alpha}^{2}}{2m} + x[p_{\alpha}, V]
 	\end{equation*}
 	Reusing the earlier results $ [x_{\alpha}, p_{\alpha}] = i \hbar $ and $ [p_{\alpha}, V] = - i \hbar \pdv{V}{x_{\alpha}} $ gives
 	\begin{equation*}
 		\big[ x_{\alpha} p_{\alpha}, H\big] = 2i \hbar \frac{p_{\alpha}^{2}}{2m} - i \hbar x_{\alpha} \pdv{V}{x_{\alpha}}.
 	\end{equation*}
 	
 	\item If we substitute the above result into the time derivative of $ \ev{\r \cdot \vec{p}} $, write the components in vector form, and use $ \vec{F} = - \grad V $, we get the virial theorem
 	\begin{equation*}
 		\dv{\ev{\r \cdot \vec{p}}}{t} = 2\frac{\ev{p^{2}}}{2m} + \ev{\r \cdot \vec{F}} = 2 \ev{T} + \ev{\r \cdot \vec{F}}
 	\end{equation*}
 	where we have defined the kinetic energy operator $ T = \frac{p^{2}}{2m} $.
 	
 	\item For a stationary state in which $ \dv{\ev{\r \cdot \vec{p}}}{t} = 0 $, we recover the familiar classical results
 	\begin{equation*}
 		2\ev{T} = - \ev{\r \cdot \vec{F}}.
 	\end{equation*}
\end{itemize}

\newpage
\section{The Formalism of Quantum Mechanics}
\subsection{The Copenhagen Interpretation}
\begin{enumerate}
	\item A quantum system is described by a state vector $ \ket{\psi} $ in a function Hilbert space.
	
	\item Every physically observable quantity is associated with a Hermitian operator
	
	\item The expectation value of an observable with operator $ A $ for a system in the state $ \ket{\psi} $ is $ \mel{\psi}{A}{\psi} $.
	
	\item The time evolution of a state $ \ket{\psi} $ is determined by the \Schro equation
	\begin{equation*}
		i \hbar \dv{t} \ket{\psi} = H \ket{\psi},
	\end{equation*}
	where $ H $ is the Hamiltonian operator.
	
	\item When measuring an observable with operator $ A $, the result of a single measurement is an eigenvalue of $ A $ (e.g. the eigenvalue $ a \in \mathbb{R} $). The probability of this measurement result is $ \abs{\braket{a}{\psi}}^{2} $, where $ \ket{a} $ is $ A $'s eigenstate corresponding to the eigenvalue $ a $. After a measurement, the system's wavefunction ``collapses'' into the state $ \ket{a} $.
\end{enumerate}

\subsection{Dirac Notation: Inner Product and Ket}
For the remainder of this chapter, $ L^{2} $ denotes the Hilbert space of all complex functions $ \psi : \mathbb{R}^{3} \to \mathbb{C} $ for which
\begin{equation*}
	\norm{\psi}_{2} \equiv \int_{V} \abs{\psi}^{2} \dr < \infty
\end{equation*}
\begin{itemize}
	\item The inner product of two vectors $ \phi, \psi \in L^{2} $ is written
	\begin{equation*}
		\braket{\phi}{\psi} \equiv \int_{V}\psi^{*}(\r)\psi(\r)\dr
	\end{equation*}
	Some properties of the inner product include
	\begin{align*}
		& \braket{\lambda \psi + \mu \chi}{\phi} = \lambda^{*} \braket{\psi}{\phi} + \mu^{*} \braket{\chi}{\phi}\\
		&\braket{\phi}{\psi} = \braket{\psi}{\phi}^{*}\\
		& \braket{\psi}{\psi} \geq 0 \eqtext{and} \braket{\psi}{\psi} = 0 \iff \psi \equiv 0 \\
		&\abs{\braket{\phi}{\psi}}^{2} \leq \braket{\phi}{\phi}\braket{\p}{\p}
	\end{align*}
	
	\item In Dirac notation, the wavefunction representing a quantum state is written as a ket, which is interpreted as a vector in the Hilbert space $ L^{2} $. A generic wavefunction $ \psi $ and eigenfunction $ \psi_{n} $ are written
	\begin{equation*}
		\begin{array}{ccccc}
			\psi(\r) \in L^{2} & \to & \ket{\p} & & \\
			\psi_{n}(\r) & \to  & \ket{\psi_{n}} & \to & \ket{n}\\
		\end{array}
	\end{equation*}
	Note that the eigenfunction is conventionally written just with its index, e.g. $ \psi_{2} $ is written $ \ket{2} $.
	
	\item A basis of eigenstates is written $ \{\ket{n}\} $, and orthonormal eigenstates obey $ \braket{m}{n} = \delta_{mn} $. 
		
\end{itemize}

\subsection{Linear and Antilinear Operators}
\begin{itemize}
	\item An operator $ \O $ is linear if for all vectors $ \phi, \psi \in L^{2} $ and all scalars $ \lambda, \mu \in \mathbb{C} $
	\begin{equation*}
		\O(\lambda \phi + \mu \p)  = \lambda A \phi + \mu A \p.
	\end{equation*}
	An operator $ \O $ is antilinear if
	\begin{equation*}
		\O(\lambda \phi + \mu \p)  = \lambda^{*} A \phi + \mu^{*} A \p \eqtext{or} \lambda A = A \lambda^{*}.
	\end{equation*}
	
	\item Because the Hamiltonian (or kinetic energy) operator and potential energy operators are both linear, the \Schro equation is linear. The \Schro equation thus obeys the superposition principle: any linear combination of solutions to the \Schro equation also solves the \Schro equation.
\end{itemize}

\subsection{Dirac Notation: Bra}
\begin{itemize}
	\item Linear functionals are linear operators $ f:L^{2} \to \mathbb{C} $ that map wavefunctions in $ L^{2} $ to scalars in $ \mathbb{C} $. 
	
	\item Riesz representation theorem: for each linear functional $ f:L^{2} \to \mathbb{C} $ there exists a vector $ \ket{\phi_{f}} \in L^{2} $ for which 
	\begin{equation*}
		f\ket{\psi} = \braket{\phi_{f}}{\psi} \equiv \int_{V} \phi_{f}^{*} \psi \dr  \quad \text{for all } \psi \in L^{2}.
	\end{equation*}
	
	\item In other words, we can interpret that action of a linear functional $ f $ on a wavefunction $ \ket{\psi} $ as the expression
	\begin{equation*}
		f\ket{\psi} =  \int_{V} \phi_{f}^{*} \psi \dr
	\end{equation*}
	In terms of the bra term in braket notation, the above reads
	\begin{equation*}
		f \ket{\psi} = \mel{\phi_{f}}{}{\psi} = \braket{\phi_{f}}{\psi}
	\end{equation*}
	where $ \bra{\phi_{f}} $ represents the action of the linear functional $ f $ on $ \psi $. A technicality:  $ \mel{\phi_{f}}{}{\psi}  $ represents the action of a linear functional $ f $ on the vector in $ L^{2} $, and the result is the scalar product $ \braket{\phi_{f}}{\psi} \in \mathbb{C} $. 
	
\end{itemize}

\subsection{Expanding a State in a Basis}
Consider an orthonormal basis $ \{\ket{\psi_{n}}\} \equiv \{\ket{n}\} $ consisting of the eigenstates $ \ket{n} $ of some operator. 
\begin{itemize}
	\item Every such basis $ \{\ket{n}\} $ (of the Hilbert space $ L^{2} $) has a corresponding basis $ \{\bra{n}\} $ of the Hilbert space's dual space of linear functionals. 
	
	\item In Dirac notation, the expansion of a state $ \ket{\p} $ in a basis $ \{\ket{n}\} $ takes the general form
	\begin{equation*}
		\ket{\psi} = \sum_{n}c_{n} \ket{n}
	\end{equation*}
	We find the coefficients $ c_{n} $ by acting on the basis expansion with $ \bra{m} $ and applying the basis' orthonormality identity $ \braket{n}{m} = \delta_{nm} $ to get
	\begin{equation*}
		\braket{m}{n} = \sum_{n}c_{n} \braket{m}{n} = sum_{n}c_{n}\delta_{mn} = c_{m},
	\end{equation*}
	which, switching from $ m $ to $ n $, implies
	\begin{equation*}
		c_{n} = \braket{n}{\psi} \eqtext{and} \ket{\psi} = \sum_{n} \braket{n}{\psi} \ket{n}.
	\end{equation*}
	
	\item Since $ \braket{n}{\psi} $ is a scalar, we can rewrite the above expansion of $ \ket{\psi} $ in the basis $ \{\ket{n}\} $ and apply $ \braket{n}{\psi} = \bra{n}\ket{\psi} $ to get
	\begin{equation*}
		\ket{\psi} = \sum_{n} \braket{n}{\psi} \ket{n} = \sum_{n} \ket{n} \braket{n}{\psi} = \sum_{n} \ket{n} \mel{n}{}{\psi} = \left(\sum_{n} \ket{n} \bra{n}\right) \ket{\psi}
	\end{equation*}
	Comparing the first and last term gives an important identity:
	\begin{equation*}
		\ket{\p} = \bigg(\sum_{n} \ket{n} \bra{n}\bigg) \ket{\psi} \implies \sum_{n} \ket{n} \bra{n} = \operatorname{I}
	\end{equation*}
	where $ \operatorname{I} $ is the identity operator. This is an important identity, so I'll write it again:
	\begin{equation*}
		\operatorname{I} = \sum_{n} \ket{n} \bra{n}
	\end{equation*}
	This holds for any orthonormal basis $ \{\ket{n}\} $.

\end{itemize}

\subsection{Expanding an Operator in a Basis}
Consider an operator $ \O $ and an orthonormal basis $ \{\ket{n}\} $. 
\begin{itemize}
	\item Using the previous identity for the identity operator, we have
	\begin{align*}
		\O \ket{\p} &\equiv (\II \O \II) \ket{\p} = \left(\sum_{m} \ket{m} \bra{m}\right) \O \left(\sum_{n} \ket{n} \bra{n}\right) \ket{\p}\\
		& = \sum_{m}\ket{m}\bra{m} \O \sum_{n} \ket{n} \braket{n}{\p}
	\end{align*}
	
	\item We introduce the \textit{matrix element} $ \O_{mn} $ (more on this later)
	\begin{equation*}
		\O_{mn} = \mel{m}{\O}{n} \in \mathbb{C}
	\end{equation*}
	In terms of this matrix element, we can then write $ \O $ in the basis $ \{\ket{n}\} $ as
	\begin{align*}
		\O \ket{\p} & = \sum_{m}\ket{m}\bra{m} \O \sum_{n} \ket{n} \braket{n}{\p}\\
		& = \sum_{mn} \ket{m}\O_{mn}\bra{n} \ket{\p}
	\end{align*}
	Which gives us the desired expression
	\begin{equation*}
		\O = \sum_{mn} \ket{m}\O_{mn}\bra{n}
	\end{equation*}
	In other words, an operator $ \O $ can be represented in an arbitrary orthonormal basis $ \{\ket{n}\} $ in terms of a matrix $ \O_{mn} $ with matrix elements
	\begin{equation*}
		\O_{mn} = \mel{m}{\O}{n} \equiv \int_{V} \psi_{m}^{*}\O \psi_{n} \dr 
	\end{equation*}
	
	\item More on writing an operator in an orthonormal basis... Consider the concrete operator equation
	\begin{equation*}
		\O \ket{\psi} = \ket{\varphi},
	\end{equation*}
	i.e. $ \O $ acts on the vector $ \ket{\psi} $ to produce $ \ket{\varphi} $. Additionally, let $ \ket{\psi} $ be expanded in the basis $ \{\ket{n} \} $ as
	\begin{equation*}
		\ket{\psi} = \sum_{n}c_{n} \ket{n} = \sum_{n} \braket{n}{\psi} \ket{n} 
	\end{equation*}
	 We write the operator equation $ \O \ket{\psi} = \ket{\varphi}, $ in the basis $ \{\ket{n}\} $ as
	\begin{align*}
		\O \ket{\p} &\equiv \sum_{mn}\ket{m}\O_{mn}\bra{n}  \ket{\psi} = \sum_{mn}\ket{m} \O_{mn}  c_{n}\\
		& = \sum_{m}\left(\sum_{n}\O_{mn}c_{n}\right) \ket{m} \\
		& \equiv \sum_{m}d_{m}\ket{m} \\
		& = \ket{\phi}
	\end{align*}
	In other words, the state $ \ket{\varphi} = \O \ket{\p} $ has the basis expansion
	\begin{equation*}
		\ket{\varphi} = \sum_{m} d_{m}\ket{m}
	\end{equation*}
	Where the the coefficients $ d_{m} $, operator $ \O $, and coefficients $ c_{n} $ of the vector $ \psi $ are related by 
	\begin{equation*}
		d_{m} = \sum_{n}\O_{mn} c_{n}
	\end{equation*}
	In vector form, the action of an operator $ \O $ in a basis $ \{\ket{n}\} $ on a state $ \ket{\p} $ with coefficients $ c_{n} $ to produce a state $ \ket{\varphi} $ with coefficients $ d_{m} $ corresponds to the matrix equation
	\begin{equation*}
		\bm{\O} \vec{c} = \vec{d},
	\end{equation*}
	where the matrix elements $ \O_{mn} $ are given by $ \O_{mn} = \mel{m}{\O}{n} $.
	
	\item An important case occurs when we expand an operator in a basis of its eigenstates. Consider an operator $ \O $ with eigenvalues $ \lambda_{n} $ and eigenstates $ \ket{n} $ obeying the eigenvalue relation
	\begin{equation*}
		\O \ket{n} = \lambda_{n} \ket{n}.
	\end{equation*}
	In this case, if we expand $ \O $ in the basis of the eigenstates $ \{\ket{n}\} $, the operator's matrix $ \bm{\O} $ in the basis $ \{\ket{n}\} $is diagonal, and the matrix elements obey
	\begin{equation*}
		\O_{mn} = \mel{m}{\O}{n} = \lambda_{n} \delta_{mn}.
	\end{equation*}
	
\end{itemize}


\subsection{Hermitian Operators}
\begin{itemize}
	\item An operator $ \O $ is symmetric, also called Hermitian, if for all $ \phi, \psi \in L^{2} $ we have
	\begin{equation*}
		\braket{\phi}{\O \p} = \braket{\O\phi}{\p}.
	\end{equation*}
	The operator $ \O $ is antisymmetric, or anti-Hermitian, if
	\begin{equation*}
		\braket{\phi}{\O \p} = - \braket{\O \phi}{\p}.
	\end{equation*}
	
	\item The expectation values of Hermitian operators are real. We show this by applying $ \braket{\p}{\O \p} = \braket{\O \p}{\p} $ (for \Herm operators), followed by $ \braket{\p}{\O \p} = \braket{\O \p}{\p}^{*} $  (for any operator)
	\begin{equation*}
		\ev{\O} \equiv \braket{\psi}{\O \psi} = \braket{\O \psi}{\psi} = 		\braket{\psi}{\O \psi}^{*} = \ev{\O}^{*}
	\end{equation*}
	The equality $ \ev{\O} = \ev{\O}^{*} $ implies $ \ev{\O} \in \mathbb{R} $. 
	
	\item The expectation value of a squared \Herm operator is positive, i.e.
	\begin{equation*}
		\ev{\O^{2}} = \mel{\p}{\O^{2}}{\p} = \braket{\O \p}{\O \p}\geq 0
	\end{equation*}
	Using the equality $ \ev{\O^{2}} \geq 0 $ to eigenstates of the operator $ \O^{2} $ with the eigenvalue relation $ \O^{2} \ket{\psi_{n}} = \lambda_{n} \ket{\psi_{n}} $ and applying the identity $ \braket{\p_{n}}{\p_{n}} \geq 0 $ produces
	\begin{equation*}
		\mel{\p_{n}}{\O^{2}}{\p_{n}} = \lambda_{n}\braket{\psi_{n}}{\p_{n}} \implies \lambda_{n} \geq 0
	\end{equation*}
	In other words, the square $ \O^{2} $ of a \Herm operator is positive definite.
	
	\item The eigenvalues of a \Herm operator are real. To show this, we start with a generic \Herm operator with the eigenvalues relation $  \O \ket{\psi_{n}} = \lambda_{n} \ket{\psi_{n}} $. We then act on both sides of the equation with $ \bra{\p_{n}} $ and apply the eigenvalue relation to get
	\begin{equation*}
		\O \ket{\p_{n}} = \lambda_{n} \ket{\p_{n}} \implies \mel{\p_{n}}{\O}{\p_{n}} = \lambda_{n} \braket{\p_{n}}{\p_{n}}
	\end{equation*}
	We then apply $ \mel{\p_{n}}{\O}{\p_{n}} \in \mathbb{R} $ (expectation value of a \Herm operator is real) and $ \braket{\p_{n}}{\p_{n}} = 1 \in \mathbb{R} $ (the eigenstate normalization condition) to get $ \lambda_{n} \in \mathbb{R} $. 
	
	\item A \Herm operator's eigenfunctions corresponding to different eigenvalues are orthogonal. Start with
	\begin{equation*}
		\O \ket{1} = \lambda_{1}\ket{1} \eqtext{and} \O \ket{2} = \lambda_{2}\ket{2},
	\end{equation*}
	and act on each equation with $ \bra{2} $ and $ \bra{1} $, respectively, to get
	\begin{equation*}
		\mel{2}{\O}{1} = \lambda_{1}\braket{2}{1} \eqtext{and} \mel{1}{\O}{2} = \lambda_{2}\braket{1}{2}
	\end{equation*}
	Take the complex conjugate of the second equation and apply $ \lambda_{n} = \lambda_{n}^{*} $ for a \Herm operator to get 
	\begin{equation*}
		\mel{1}{\O}{2}^{*} = \lambda_{2}\braket{1}{2}^{*}
	\end{equation*}
	The rest is just playing around with the general identity $ \braket{\p}{\phi} = \braket{\phi}{\p}^{*} $, the \Herm identity $ \braket{1}{\O 2} = \braket{\O 1}{2} $, and the eigenvalue relation to get
	\begin{equation*}
		\lambda_{2}\braket{1}{2}^{*} = \mel{1}{\O}{2}^{*} \equiv \braket{1}{\O 2}^{*} = \braket{O2}{1} = \braket{2}{O1} = \lambda_{1}\braket{2}{1}
	\end{equation*}
	We end up with 
	\begin{equation*}
		\lambda_{2}\braket{1}{2}^{*} = \lambda_{1}\braket{2}{1} \implies \lambda_{2}\braket{2}{1} = \lambda_{1}\braket{2}{1}
	\end{equation*}
	And end up with 
	\begin{equation*}
		(\lambda_{2} - \lambda_{1}) \braket{2}{1} = 0
	\end{equation*}
	Goodness gracious I made that way more convoluted than it needed to be.
\end{itemize}

\subsection{Adjoint Operators and Their Properties}
\begin{itemize}
	\item Consider an operator $ \O $. The operator $ \O $'s adjoint, denoted by $ \O^{\dagger} $, is defined by the relationship
	\begin{equation*}
		\braket{\phi}{\O \p} = \bbraket{\O^{\dagger}\phi}{\p}
	\end{equation*}
	From the general identity $ \braket{\phi}{\psi} = \braket{\psi}{\phi}^{*} $, we also have
	\begin{equation*}
		\braket{\phi}{\O \p} = \bbraket{\O^{\dagger}\phi}{\p} = \bbraket{\p}{\O^{\dagger}\phi}^{*}
	\end{equation*}
	
	\item Consider two operators $ A $ and $ B $ related by $ A = \lambda B $ where $ \lambda \in \mathbb{C} $ is a constant. The operators' adjoint are then related by
	\begin{equation*}
		A^{\dagger} = \lambda^{*}B^{\dagger},
	\end{equation*}
	which follows directly from
	\begin{equation*}
		\bbraket{A^{\dagger}\phi}{\p} = \braket{\phi}{A \p} = \braket{\phi}{\lambda B \p} = \bbraket{\lambda^{*}B^{\dagger}\phi}{\p}
	\end{equation*}
	
	\item Any operator $ \O $ obeys $ \big(\O^{\dagger}\big)^{\dagger} = \O $, which implies the operator  $ \O + \O^{\dagger} $ is \Herm and the operator $ \O - \O^{\dagger} $ is anti-\Herm.  More so, if $ \O $ is \Herm, then $ i\O $ is anti-\Herm.
	
	\item The expectation values of an operator $ \O $ obey the convenient identities
	\begin{equation*}
	\begin{array}{ccccc}
		2 \Re \ev{\O} & \equiv & 2 \Re \mel{\p}{\O}{\p} & = & \mel{\p}{(\O + \O^{\dag})}{\p}\\
		2i \Im \ev{\O} & \equiv & 2i \Im \mel{\p}{\O}{\p} & = & \mel{\p}{(\O - \O^{\dag})}{\p}
	\end{array}
	\end{equation*}
	
	\item The adjoint of an operator defined by $ \O = \ket{m}\bra{n} $ is $ \O^{\dagger} = \ket{n} \bra{m} $, which follows from
	\begin{equation*}
		\braket{\phi}{\O \p} = \braket{\phi}{m}\braket{n}{\p} = \big(\bra{\p} \ket{n} \bra{m} \ket{\phi} \big)^{*}
	\end{equation*}
	Similarly, $ \big(\bra{\p}\O\big)^{\dagger} = \O^{\dagger}\ket{\p} $
	
	\item Two operators $ A $ and $ B $ obey
	\begin{equation*}
		\big(AB\big)^{\dagger} = B^{\dagger}A^{\dagger},
	\end{equation*}
	which follows from
	\begin{equation*}
		\braket{\phi}{AB\p} = \bbraket{A^{\dagger}\phi}{B\p} = \bbraket{B^{\dagger}A^{\dagger}\phi}{\p}.
	\end{equation*}
	The product of two \Herm operators is \Herm if the two operators commute.
	
	\item The projection operator $ P_{n} \equiv \ket{n}\bra{n} $ equals its adjoint, i.e. $ P_{n} = P_{n}^{\dagger} $. More so, $ P_{n} = P_{n}^{2} $, which follows from
	\begin{equation*}
		P_{n}^{2} = \ket{n}\bra{n} \ket{n}\bra{n} = \ket{n}\bra{n} = P_{n}
	\end{equation*}
	and the normalization condition $ \braket{n}{n} = 1 $.
	
	\item Consider an operator $ \O $ written in some generic orthonormal basis $ \{\ket{n}\} $:
	\begin{equation*}
		\O = \sum_{mn} \ket{m}\O_{mn}\bra{n}.
	\end{equation*}
	The adjoint operator $ \O^{\dagger} $ is then written in the basis as
	\begin{equation*}
		\O^{\dagger} = \sum_{mn}\ket{n}\O_{mn}^{*}\bra{m} = \sum_{mn}\ket{m} \O_{nm}^{*}\bra{n}
	\end{equation*}
	The matrix elements of an operator and its adjoint are thus related by
	\begin{equation*}
		\big(\O^{\dagger}\big)mn = \O_{nm}^{*}
	\end{equation*}
\end{itemize}

\subsection{Self-Adjoint Operators}
\begin{itemize}
	\item An operator $ \O $ is self-adjoint if:
	\begin{enumerate}
		\item Both $ \O $ and $ \O^{\dagger} $ are Hermitian, i.e.
		\begin{equation*}
			\braket{\phi}{\O \p} = \braket{\O\phi}{\p} \eqtext{and} \bbraket{\phi}{\O^{\dagger} \p} = \bbraket{\O^{\dagger}\phi}{\p} \ \text{for all } \phi, \p \in L^{2},
		\end{equation*}
		
		\item $ \O $ and $ \O^{\dagger} $ act on the same domain (in our case generally $ L^{2} $).
	\end{enumerate} 
	A self-adjoint operator obeys $ \O = \O^{\dagger} $, which makes sense from the name---a self-adjoint operator $ \O $ equals its adjoint $ \O^{\dagger} $.
	
	\item Every self-adjoint operator is \Herm, but in general not every \Herm operator is self-adjoint. However (without proof), in finite $ N $-dimensional vector spaces $ \mathbb{C}^{N} $ and in the Schwartz space of rapidly falling functions, \Herm and self-adjoint operators are equivalent. 
	
	Since physicists typically work only with quantities in $ \mathbb{R}^{N} $ or functions in the Schwartz space, we tend to incorrectly use the terms Hermitian and self-adjoint interchangeably.

\end{itemize}

\subsection{Unitary Operators}
\begin{itemize}
	\item Unitary operators in quantum mechanics are analogous to orthogonal transformations in classical mechanics. A unitary operator $ U $ obeys the relationship
	\begin{equation*}
		UU^{\dagger} = U^{\dagger}U = \II \implies U^{-1} = U^{\dagger}
	\end{equation*}
	
	\item Unitary operators preserve the inner product. In symbols, for a unitary operator $ U $ and any two functions $ \bket{\t{\phi}} = U\ket{\phi} $ and $ \bket{\t{\p}} = U \ket{\p} $, 
	\begin{equation*}
		\braket{\phi}{\p} = \bbraket{\t{\phi}}{\t{\p}}
	\end{equation*}
	The above follows directly from $  \bbraket{\t{\phi}}{\t{\p}} = \braket{U\phi}{U\p} = \braket{UU^{\dagger}\phi}{\p} = \braket{\phi}{\p} $.
	
	\item For matrix elements, using $ U^{\dagger} = U^{-1} $:
	\begin{equation*}
		\mel{\phi}{\O}{\p} = \bmel{U^{\dagger}\t{\phi}}{\O}{U^{\dagger}\t{\p}} =  \bmel{\t{\phi}}{U\O U^{\dagger}}{\t{\p}} \equiv  \bmel{\t{\phi}}{\t{\O}}{\t{\p}} 
	\end{equation*}
	where we have defined $ \t{\O} = U \O U^{\dagger} $. In other words, the matrix element of $ \O $ corresponding to the wavefunctions $ \ket{\phi} $ and $ \ket{\p} $ equal the matrix elements of the transformed operator $ \t{\O} = U \O U^{\dagger}  $ found with the transformed wavefunctions $  \bket{\t{\phi}} $ and $ \bket{\t{\p}} $.
	
	\item If $ \bket{\t{\p}} = U \ket{\p}$ then $ \bra{\t{p}} = \bra{U \p} = \bra{\p}U^{\dagger} $.
	
	\item Consider an orthonormal basis $ \{\ket{\p_{n}}\} $ and the transformed basis $ \{\bket{\t{\p}_{n}}\} $ = $ \{\ket{U\p_{n}}\} $ where $ U $ is a unitary operator. We then have
	\begin{equation*}
		U = U\II = U \sum_{n}\ket{\p_{n}}\bra{\p_{n}} = \sum_{n}U\ket{\p_{n}}\bra{\p_{n}} =  \sum_{n}\bket{\t{\p}_{n}}\bra{\p_{n}}
	\end{equation*}
	We then use $ \II = \sum_{m}\ket{\p_{m}}\bra{\p_{m}} $ and define the matrix elements $ U_{mn} = \bbraket{\p_{m}}{\t{\p}_{n}} $ to get
	\begin{equation*}
		U =  \sum_{n}\bket{\t{\p}_{n}}\bra{\p_{n}} = \sum_{n}\left(\sum_{m}\ket{\p_{m}}\bra{\p_{m}}\right)\bket{\t{\p}_{n}}\bra{\p_{n}} = \sum_{mn}\ket{\p_{m}}U_{mn}\bra{\p_{n}}
	\end{equation*}
	
	\item The identity operator takes the same form in the original basis $ \{\ket{\p_{n}}\} $ and the transformed basis $ \{\bket{\t{\p}_{n}}\} $:
	\begin{equation*}
		UU^{\dagger} = \sum_{mn} \bket{\t{\p}_{m}} \braket{\p_{m}}{\p_{n}} \bbra{\t{\p}_{n}} = \sum_{n}\bket{\t{\p}_{n}}\bbra{\t{\p}_{n}}
	\end{equation*}
	
	\item In a unitary change of basis $ \{\ket{\p_{n}}\} \to \{\bket{\t{\p}_{n}}\} $, the coefficients transform as
	\begin{equation*}
		\ket{\phi} = \sum_{n}c_{n}\ket{\p_{n}} = \sum_{mn}\bket{\t{\p}_{m}}\bmel{\t{\p}_{m}}{c_{n}}{\p_{n}} = \sum_{n}d_{n}\bket{\tilde{\p}_{n}}
	\end{equation*}
	where the new coefficients are
	\begin{equation*}
		d_{n} = \sum_{m}U_{nm}^{\dagger}c_{m}
	\end{equation*}
	
	\item Unitary transformations preserve eigenvalue equations:
	\begin{align*}
		&\O\ket{\psi_{n}} = \lambda_{n} \ket{\p_{n}} \implies U \O \II \ket{\p_{n}} = U \O U^{\dagger}U \ket{\p_{n}} = \lambda_{n} U\ket{\p_{n}}\\
		&\t{\O} \ket{U\p_{n}} = \lambda_{n}\ket{U\p_{n}}\\
		&\t{\O} = \bket{\t{\p}_{n}} = \lambda_{n} \bket{\t{\p}_{n}}
	\end{align*}
	
	\item If $ K $ is \Herm, then $ U = e^{iK} $ is unitary by the Baker-Campbell-Hausdorff formula, i.e. $ UU^{\dagger} = e^{iK}e^{-iK} = \II $.
	
	\item Every single-parameter unitary operator $ U(s) $, where $ s \in \mathbb{R} $ is a real constant, can be written in the form 
	\begin{equation*}
		U(s) = e^{isK}
	\end{equation*}
    where $ K $ is a self-adjoint operator called the \textit{generator} of the unitary operator $ U $. Often, for infinitesimal parameters $ s \to 0 $, we work in the first-order approximation $ U(s) \approx \II + isK $.
	
\end{itemize}

\textbf{Derivation:}
\begin{itemize}
    \item To derive the expression $ U(s) = e^{isK} $, we first expand an arbitarary unitary operator $ U $ in powers of $ s $ for vanishingly small $ s \to 0 $, which gives
    \begin{equation*}
        U(s) = \II + \dv{U}{s}s + \mathcal{O}(s^{2}) \qquad \text{and} \qquad U^{\dagger}(s) = \II + \dv{U^{\dagger}}{s} + \mathcal{O}(s^{2}).
    \end{equation*}
    We then write $ U $ in terms of the identity operator, i.e.
    \begin{equation*}
        UU^{\dagger} = \II + \left( \dv{U}{s} + \dv{U^{\dagger}}{s} \right)s + \mathcal{O}(s^{2}).
    \end{equation*}
    To satisfy the unitary identity $ UU^{\dagger} \equiv \II $ to first order in $ s $, the quantity in parentheses must equal zero, i.e.
    \begin{equation*}
        \dv{U^{\dagger}}{s} = \left( \dv{U}{s} \right)^{\dagger} = -\dv{U}{s}.
    \end{equation*}
    Using the above result, we can write $ \dv{U}{s} $ in the form
    \begin{equation*}
        \dv{U}{s} = i K,
    \end{equation*}
    where $ K $ is a \Herm operator.

    \item Next, we divide the parameter $ s $ into $ N \to \infty $ equal subintervals and apply the operator $ U(\tfrac{s}{N}) $ $ N $ times, which produces the desired result
    \begin{equation*}
        U(s) = \lim_{N \to \infty} U \left(\frac{s}{N}\right) U\left(\frac{s}{N}\right) \cdots U\left(\frac{s}{N}\right) = \lim_{N \to \infty} \left( \II + i K \frac{s}{N} \right)^{N} = e^{isK}
    \end{equation*}
    
\end{itemize}
    
\textbf{Anti-Unitary Operator}: 
\begin{itemize}
	\item An anti-unitary operator $ U $ obeys the relationship
	\begin{equation*}
		\braket{U\phi}{U\psi} = \braket{\phi}{\psi}^{*}= \braket{\psi}{\phi}
	\end{equation*}
	
	\item Anti-unitary operators are antilinear, i.e.
	\begin{equation*}
		U\big(\lambda \ket{\phi} + \mu \ket{\psi}\big) = \lambda^{*}U\ket{\phi} + \mu^{*}U\ket{\psi}
	\end{equation*}

\end{itemize}

\subsection{Time Evolution} \label{ss:time-ev}
\begin{itemize}
	\item Expanding in basis formed of the energy eigenstates $ \{\ket{\phi_{n}}\} $ of the \Ham operator $ H $ reads
	\begin{equation*}
		\ket{\p(t)} = \sum_{m}\braket{\phi_{n}}{\p(0)}e^{-i\frac{E_{n}}{\hbar}t}\ket{\phi_{n}}
	\end{equation*}
	Using the operator function identity $ f(\O)\psi_{n} = f(\lambda_{n} )\psi_{n} $, we can replace the energy eigenvalues $ E_{n} $ in the last line with the \Ham operator $ H $ to get
	\begin{align*}
		\ket{\p(t)} &= \sum_{m}\braket{\phi_{n}}{\p(0)} e^{-i\frac{E_{n}}{\hbar}t}\ket{\phi_{n}} = \sum_{m} \braket{\phi_{n}}{\p(0)} e^{-i\frac{H}{\hbar}t}\ket{\phi_{n}}
	\end{align*}
	Factoring $ e^{-i\frac{H}{\hbar}t} $ out of the sum gives
	\begin{equation*}
		\ket{\p(t)} = e^{-i\frac{H}{\hbar}t} \mel{\phi_{n}}{\p(0)}{\phi(n)} \equiv U(t) \ket{\p(0)}
	\end{equation*}
	where we have defined the time evolution operator $ U(t) \equiv e^{-i\frac{H}{\hbar}t} $.
	
	\item As the notation $ U(t) $ suggests, the time evolution operator is unitary, with generator $ H $. Because $ U $ is unitary, it preserves the inner product.
	
	\item Applying $ U(t) $ to an infinitesimal time step $ \diff t $ in the evolution of a wavefunction $ \ket{\p} $ gives
	\begin{equation*}
		\ket{\delta \p} = \ket{\psi(t + \diff t)} - \ket{\p(t)} = -i\frac{H}{\hbar}\diff t \ket{\p(t)}
	\end{equation*}
	``Dividing'' by $ \diff t $ and rearranging produces the \Schro equation
	\begin{equation*}
		i \hbar \frac{\ket{\psi(t + \diff t)} - \ket{\p(t)}}{\diff t} = i \hbar \dv{t}\ket{\psi(t)} = H \ket{\psi(t)}.
	\end{equation*}
\end{itemize}

\subsection{Momentum Eigenfunction Representation} 
% see Schwabl preview page 176

\begin{itemize}
    \item Review: for rapidly falling functions $ f(x) $, the function and its Fourier transform $ \F{f} $ are related by
    \begin{equation*}
        f(x) = \int_{-\infty}^{\infty} \F{f}(k)e^{ikx}\diff k \qquad \text{and} \qquad \F{f}(k) = \frac{1}{2\pi}\int_{-\infty}^{\infty}f(x)e^{-ikx}\diff x.
    \end{equation*}
    Next, we note the relationship
    \begin{align*}
        f(x) &= \int_{-\infty}^{\infty}\left( \frac{1}{2\pi} \int_{-\infty}^{\infty}f(x')e^{-ikx'}\diff x; \right)e^{ikx}\diff k \\
        &= \int_{-\infty}^{\infty}\left( \frac{1}{2\pi}\int_{-\infty}^{\infty}e^{ik(x'-x)}\diff k \right) f(x')\diff x'\\
        & = \int_{-\infty}^{\infty}\delta(x' - x)f(x')\diff x',
    \end{align*}
    where we have used the integral definition of the delta function, i.e.
    \begin{equation*}
        \delta(x) = \frac{1}{2\pi}\int_{-\infty}^{\infty}e^{ikx}\diff k.
    \end{equation*}
    Finally, for use later in this section, we quote the following delta function properties:
    \begin{itemize}
        \item $ \delta(ax) = \frac{1}{\abs{a}} \delta(x) $,
        \item $ x\delta(x) = 0 $,
        \item $ x \delta(x - y) = y \delta(x - y) $, \quad and
        \item $ \int_{-\infty}^{\infty}\delta(x - y)\delta(x - z)\diff x  = \delta(y - z)$.
    \end{itemize}
    
\end{itemize}


\subsubsection{Eigenvalues and Eigenstates of the Momentum Operator}
\begin{itemize}

    \item We begin with the \Schro equation for a free particle, which reads
    \begin{equation*}
        i \hbar \dv{t}\ket{\psi(t)} = \frac{\hat{p}^{2}}{2m} \ket{\psi(t)}.
    \end{equation*}
    The eigenstates of the momentum operator $ \hat{p} $ are plane waves, and the eigenvalue equation reads
    \begin{equation*}
        \hat{p}\ket{\varphi_{p}} = p \ket{\varphi_{p}},
    \end{equation*}
    where $ p \in \mathbb{R} $ denotes a momentum eigenvalue. 

    The momentum eigenstates $ \ket{\varphi_{p}} $ are simultaneously eigenstates of the stationary \Schro equation, in which case the eigenvalue equation reads
    \begin{equation*}
        \frac{\hat{p}^{2}}{2m}\ket{\varphi_{p}} = E \ket{\varphi_{p}}, \qquad E = \frac{p^{2}}{2m}.
    \end{equation*}
    
    \item Because the momentum eigenstates $ \ket{\varphi_{p}} $ solve the \Schro equation, they form a convenient basis in which to find the time evolution of an arbitrary state $ \ket{t} $.
    \item We now consider the wave functions corresponding to an eigenstate $ \ket{\varphi_{0}} $ with eigenvalue $ p_{0} $. For such a state, the momentum eigenvalue equation $ \hat{p}\ket{\varphi_{p}} = p \ket{\varphi_{p}} $ and its simple exponential solution are
    \begin{equation*}
        \hat{p}\ket{\varphi_{0}} \to - i \hbar \dv{x} \varphi_{p_{0}}(x) = p_{0}\varphi_{p_{0}}(x) \implies \varphi_{p_{0}}(x) = C e^{i \frac{p_{0}}{\hbar}x}.
    \end{equation*}
    Note that a plane wave is not normalizable in the usual sense, i.e. $ \int_{-\infty}^{\infty} \abs{\psi}^{2}\diff x \equiv 1 $, since the integral would diverge. Instead, we normalize plane wave momentum eigenfunctions in the form
    \begin{equation*}
        \int_{-\infty}^{\infty}\abs{\varphi_{p_{0}}}^{2} \diff x = \int_{-\infty}^{\infty}\varphi^{*}_{p_{0}} \varphi_{p_{0}}\diff x \equiv \delta(p - p_{0}).
    \end{equation*}
    This normalization convention implies a normalization constant of the form $ C = \frac{1}{\sqrt{2\pi \hbar}} $ and is derived from the integral definition of the delta function, together with the identity $ \delta(ax) = \frac{1}{\abs{a}} \delta(x) $.

    \item Since the momentum operator has a continuous spectrum of eigenvalues $ p \in \mathbb{R} $, the corresponding momentum eigenbasis $ \{\varphi_{p}\} $ is infinite-dimensional. As a result (i.e. because the basis is continuous), we expand an arbitrary function $ \psi(x) $ in the momentum eigenbasis using an integral instead of a sum. The expansion in the momentum basis reads
    \begin{equation*}
        \psi(x) = \int_{-\infty}^{\infty}\F{\psi}(p)\varphi_{p}(x)\diff p, \quad \text{where} \quad \F{\psi}(p) = \int_{-\infty}^{\infty}\psi(x)\varphi_{p}^{*} \diff x.
    \end{equation*}
    Note that $ \F{\psi}(p) $ is a Fourier transform of the wave function $ \psi(x) $ from the position domain to the momentum domain. 

    The wavefunction $ \psi $ may be normalized using the Parseval equality, which reads
    \begin{equation*}
        \int_{-\infty}^{\infty}\abs{\psi(x)}^{2} \diff x = \int_{-\infty}^{\infty}\abs{\F{\psi}(p)}^{2}\diff p.
    \end{equation*}
    
    \item The momentum eigenfunctions $ \varphi_{p} $ are related to the delta function by the following completeness relation:
    \begin{equation*}
        \int_{-\infty}^{\infty} \varphi_{p}^{*}(x')\varphi_{p}(x)\diff p = \delta(x' - x),
    \end{equation*}
    where the integral represents a ``sum'' over all possible momentum eigenfunctions.
    
    \item Finally, note that when a particle occurs in an external potential with $ V(x) \neq 0 $, the momentum eigenstates $ \ket{\varphi_{p}} $ are no longer stationary states of the stationary \Schro equation. However, even for $ V(x) \neq 0 $, the momentum eigenstates still form a valid basis in which to expand an arbitrary wave function $ \psi(x) $.
    
    
    \item To summarize the results so far, it is possible to describe a quantum particle in terms of a position-domain wave function $ \psi(x) $ or, equivalently, in terms of the momentum-domain wave function $ \F{\psi}(p) $, which is a Fourier transform of $ \psi(x) $ to the $ p $ domain. 

    The action of the momentum operator $ \hat{p} $ on the function $ \psi(x) $ in the position domain is equivalent to multiplication of $ \F{\psi}{p} $ by the eigenvalue $ p $ in the momentum domain, i.e.
    \begin{equation*}
        \hat{p} \psi(x) = - i \hbar \dv{x} \psi(x) = \int_{-\infty}^{\infty}\F{\psi}(p) \left( -i \hbar \dv{x} \varphi_{p}(x) \right)\diff p = \int_{-\infty}^{\infty}\left( p \F{\psi}(p) \right) \varphi_{p}(x)\diff p.
    \end{equation*}
    This relationship generalizes to higher-order derivatives according to
    \begin{equation*}
        (- i \hbar)^{n}\dv[n]{}{x} \psi(x) \iff p^{n} \F{\psi}(p).
    \end{equation*}
    
    \item Analogously, multiplying the position-domain wave function by $ x $ corresponds to differentiation of $ \F{\psi}(p) $, which we show as follows:
    \begin{align*}
        \hat{x}\psi(x) = x \psi(x) &= \int_{-\infty}^{\infty}\F{\psi}(p)x \varphi_{p}(x)\diff p = \int_{-\infty}^{\infty}\F{\psi}(p) \left( -i \hbar \dv{p}\varphi_{p}(x) \right)\diff p\\
        & = \int_{-\infty}^{\infty}\left( i \hbar \dv{p}\F{\psi}(p) \right)\varphi_{p}(x)\diff p.
    \end{align*}
    The result is
    \begin{equation*}
        x^{n}\psi(x) \iff (i \hbar)^{n} \dv[n]{}{p}\F{\psi}(p).
    \end{equation*}
    This result also implies the relationship
    \begin{equation*}
        V(x) \psi(x) \iff V \left( i \hbar \dv{p} \right) \F{\psi}(p).
    \end{equation*}
    
\end{itemize}

\subsubsection{Eigenvalues and Eigenstates of the Position Operator}
\begin{itemize}
    \item We now consider the position operator $ \hat{x} $, for which the eigenvalue relation reads
    \begin{equation*}
        \hat{x}\psi_{0}(x) = x_{0} \psi_{0}(x),
    \end{equation*}
    where $ x_{0} \in \mathbb{R} $ is the position eigenvalue. Transferring to momentum space and using the fact that multiplication by $ x $ in $ x $ space corresponds to applying $ i \hbar \dv{p} $ in $ p $ space produces
    \begin{align*}
        \hat{x} \psi_{0}(x) &\to x \int_{-\infty}^{\infty}\F{\psi}_{0}(p)\varphi_{p}(x)\diff p = \int_{-\infty}^{\infty} \left( i \hbar \dv{p}\F{\psi}_{0}(p) \right)\varphi_{p}(x) \diff p \\ & = x_{0} \int_{-\infty}^{\infty}\F{\psi}_{0}(p)\varphi_{p}(x)\diff p 
    \end{align*}
    Comparaing the last equality implies the position eigenfunctions $ \psi_{0} $ obey the relationship
    \begin{equation*}
        i \hbar \dv{p} \F{\psi}_{0}(p) = x_{0} \F{\psi}_{0}(p) \implies \F{\psi}_{0}(p) = \frac{1}{\sqrt{2\pi \hbar}} e^{i \frac{p}{\hbar}x_{0}} = \varphi_{p}^{*}(x_{0}).
    \end{equation*}
    Since the position eigenfunction's Fourier transform $ \F{\psi_{0}}(p) $ is a plane wave, the position eigenfunctions themselves must be the inverse Fourier transform of a plance wave, which is precisely a delta function:
    \begin{equation*}
        \psi_{0}(x) = \int_{-\infty}^{\infty}\varphi_{p}^{*}(x_{0}) \varphi_{p}(x) \diff p = \delta(x - x_{0}).
    \end{equation*}
    Physically, we can interpret $ \psi_{0}(x) $ as a localized function centered at the eigenvalue $ x = x_{0} $. 

    In terms of the delta function, the position eigenvalue equation $ \hat{x}\psi_{0}(x) = x_{0} \psi_{0}(x) $ reads
    \begin{equation*}
        \hat{x} \delta(x - x_{0}) = x_{0} \delta(x - x_{0}).
    \end{equation*}
    
\end{itemize}

\subsubsection{Example: Particle in a Gravitational Potential}
\textbf{TODO}

\subsection{The Probability Amplitudes $ \braket{p}{\psi} $ and $ \braket{x}{\psi} $}
\begin{itemize}
    \item Dirac braket notation allows to represent a system's quantum state $ \ket{p} $ as an abstract vector, separate from the coordinate representations $ \psi(x) $ or $ \F{p}(p) $ in position and momentum space.

    This is useful---we can denote a system's state with the abstract vector $ \ket{\psi} $, and only later decide which representation---$ x $ or $ p $---suits the problem better.

    \item In braket notation, we denote an operator's eigenstates with the eigenvalue, for example we write the momentum eigenstates as $ \ket{p} $, where $ \ket{p} $ is the momentum eigenstates with eigenvalue $ p $. In this notation, we expand an arbitrary state encoded by the wavefunction $ \psi $ in the form
    \begin{equation*}
        \ket{\psi} = \int_{-\infty}^{\infty}\F{\psi}(p)\ket{p}\diff p.
    \end{equation*}
    We say we have written $ \ket{\psi} $ in the momentum eigenbasis $ \{\ket{p}\} $. 

    Next, we multiply the equation $ \ket{\psi} = \int_{-\infty}^{\infty}\F{\psi}(p)\ket{p}\diff p $ by $ \bra{p'} $ and apply the orthogonal identity $ \braket{p}{p'} = \delta(p - p') $ to get the important result
    \begin{equation*}
        \F{\psi}(p) = \braket{p}{\psi}.
    \end{equation*}
    
    \item Analogously, we can expand an arbitrary state $ \ket{\psi} $ in the position eigenbasis via
    \begin{equation*}
        \ket{\psi} = \int_{-\infty}^{\infty}\psi(x) \ket{x_{0}} \diff x,
    \end{equation*}
    where $ \ket{x_{0}} $ denotes a position eigenstate with eigenvalue $ x_{0} $, which corresponds to the function $ \delta(x - x_{0}) $ in the $ x $ representation. 

    \item We then multiply the equation through by $ \bra{x_{0}} $, which produces
    \begin{equation*}
        \braket{x_{0}}{\psi} = \int_{-\infty}^{\infty}\F{\psi}(p)\braket{x_{0}}{p}\diff p = \int_{-\infty}^{\infty}\F{\psi}(p) \varphi_{p}(x_{0})\diff p = \psi(x_{0}),
    \end{equation*}
    where we have substituted in $ \braket{x_{0}}{p} = \varphi_{p}(x) $, which follows from
    \begin{equation*}
        \braket{x_{0}}{p} = \int_{-\infty}^{\infty}\delta(x - x_{0}) \varphi_{p}(x)\diff x = \varphi_{p}(x_{0}).
    \end{equation*}
    The relationship $ \braket{x_{0}}{\psi} = \psi(x_{0}) $ leads us to the general relationship
    \begin{equation*}
        \psi(x) = \braket{x}{\psi},
    \end{equation*}
    which corresponds to the above position eigenbasis expansion
    \begin{equation*}
        \ket{\psi} = \int_{-\infty}^{\infty}\psi(x) \ket{x_{0}} \diff x,
    \end{equation*}
    
    \item The position eigenstates are orthonormalized in terms of the delta function via $ \braket{x_{0}}{x} = \delta(x - x_{0}) $, which implies
    \begin{equation*}
        \ket{x_{0}} = \int_{-\infty}^{\infty}\delta(x - x_{0})\ket{x}\diff x.
    \end{equation*}
    Next, we multiply the above equation through by $ \bra{x_{1}} $, which produces
    \begin{equation*}
        \int_{-\infty}^{\infty}\delta(x - x_{0})\delta(x - x_{1}) \diff x = \delta(x_{0} - x_{1}).
    \end{equation*}
    
    \item In terms of the position and momentum eigenstates, the identity operator is written
    \begin{equation*}
        \II = \int_{-\infty}^{\infty}\ket{p}\bra{p}\diff p = \int_{-\infty}^{\infty}\ket{x}\bra{x}\diff x.
    \end{equation*}
    
    
\end{itemize}

\subsection{Mutually Commuting Operators}
\begin{itemize}
    \item Consider a one-dimensional plane wave $ \ket{p} $ with momentum $ p \in \mathbb{R} $. This state is double degenerate, since two linear independent states, $ \ket{\pm \abs{p}} $, have the same energy $ E = \frac{p^{2}}{2m} $. 

    As a result, the state of a plane wave is not uniquely determined with its energy alone. We have to include a second quantity, such as momentum, to uniquely characterize a plane wave. In this case, the uniquely determined state would be written $ \ket{E, p} $. In this case, the energy operator $ H $ (the \Ham) and the momentum operator $ \hat{p} $ are mutually commuting operators, i.e. $ [H, p] = 0 $. 

    \item More generally, if the eigenvalue spectrum of some operator $ A $ is degenerate (in the above example this was $ \hat{p} $), there exists some operator $ B $ ($ H $ in the above example), such that $ [A, B] = 0 $. More so, since $ A $ and $ B $ commute, they have the same basis. The reverse is also true---if two operators have the same basis, then the two operators commute.

    \item In fact, any number of operators sharing the same basis are mutually commuting, and their eigenvalues serve as quantum numbers that uniquely determine a system's quantum state. For example, we uniquely characterized a plane wave as $ \ket{p, E} $, using the eigenvalues of the mutually commuting operators $ \hat{p} $ and $ H $. More generally, we could describe a state in terms of the mutually commuting operators $ A $, $ B $, $ C, \ldots $ with eigenvalues $ a $, $ b $ and $ c, \ldots $ in the basis $ \{\ket{a, b, c, \ldots}\} $.

    \item We now show that two commuting operators $ A $ and $ B $ have the same eigenvectors. We begin with the commutation and eigenvalue relations $ [A, B] = 0 \implies AB = BA $ and $ A \ket{a} = a \ket{a} $. We then multiply the eigenvalue equation by $ B $ and apply $ AB = BA $ to get
    \begin{equation*}
        BA \ket{a} = A \big( B \ket{a}\mspace{-3mu}\big) = a \big( B \ket{a} \mspace{-3mu}\big).
    \end{equation*}
    In other words, $ B \ket{a} $ is an eigenstate of the operator $ A $. This leaves two options:
    \begin{enumerate}
        \item A non-degenerate case, in which the eigenvalue $ a $ corresponds to only one eigenvector $ \ket{a} $, which thus differs from $ B \ket{a} $ by only a constant factor, implying the eigenvalue equation $ B \ket{a} = b \ket{a} $.  In other words, $ \ket{a} $ is an eigenvector of both $ A $ and $ B $, or

        \item A degenerate case---we will consider double degeneracy as a concrete example---in which case there exist two linearly independent vectors $ \ket{a_{1}} $ and $ \ket{a_{2}} $ with the same eigenvalue $ a $. We consider this case in the following section.
    \end{enumerate}
\end{itemize}

\textbf{The Degenerate Case}

\begin{itemize}
    
   \item In the degenerate case, every linear combination of the form
    \begin{equation*}
        \ket{\psi} = c_{1} \ket{a_{1}} + c_{2} \ket{a_{2}}
    \end{equation*}
    is an eigenvector of $ A $ with eigenvalue $ a $, but not necessarily an eigenvector of $ B $. However, it is possible to create a linear combination of the form
    \begin{equation*}
        B \big( c_{1} \ket{a_{1}} + c_{2} \ket{a_{2}}\big) = b \big( c_{1} \ket{a_{1}} + c_{2} \ket{a_{2}} \big)
    \end{equation*}
    that is an eigenvector of $ B $. Multiplying the above equation through by $ \ket{a_{1}} $ and $ \ket{a_{2}} $ and applying the (assumed, without loss of generality) orthonormality of $ \ket{a_{1}} $ and $ \ket{a_{2}} $ produces the system of equations
    \begin{align*}
        & B_{11} c_{1} + B_{12} c_{2} = bc_{1}\\
        & B_{21} c_{2} + B_{22} c_{2} = bc_{2},
    \end{align*}
    where $ B_{ij} = \mel{a_{i}}{B}{a_{j}} $. 

    \item Next, we write the above system of equations in the matrix form
    \begin{equation*}
        \begin{pmatrix}
            B_{11} & B_{12}\\
            B_{21} & B_{22}
        \end{pmatrix}
        \begin{pmatrix}
            c_{1}\\
            c_{2}
        \end{pmatrix}
        = b
        \begin{pmatrix}
            c_{1}\\
            c_{2}
        \end{pmatrix} \iff \mat{B} \vec{c} = b \vec{c}
    \end{equation*}
    and impose the condition
    \begin{equation*}
        \det 
        \begin{pmatrix}
            B_{11} - b & B_{12}\\
            B_{21} & B_{22} - b
        \end{pmatrix} 
        = 0,
    \end{equation*}
    which is equivalent to requiring the system has a nontrivial solution. If the two roots $ b_{1} $ and $ b_{2} $ of the above characteristic polynomial are different, their associated eigenvectors are linearly independent, in which case $ B $ removes the degeneracy from the operator $ A $.

    It is also possible that the polynomial's two roots are equal, i.e. $ b_{1} = b_{2} \equiv b $. In this case, we must continue introducing commuting operators until the system's degeneracy is removed. Intepreted physically, more operators serve as more observable quantities needed to uniquely determine the system's state, just like we needed to introduce the energy operator $ H $ in addition to the momentum operator $ \hat{p} $ to uniquely determine a plane wave.
    
\end{itemize}

\textbf{A Few More Notes}
\begin{itemize}
    \item If two operators $ A $ and $ B $ with eigenvalues $ a_{n} $ and $ b_{n} $ share the same basis $ \{\varphi_{n}\} $, then the operators commute. This follows from
    \begin{equation*}
        [A, B] \ket{\varphi_{n}} = (AB - BA)\ket{\varphi_{n}} = (a_{n}b_{n} - b_{n}a_{n})\ket{\varphi_{n}} = 0,
    \end{equation*}
    which implies $ [A, B] \ket{\psi} = 0 $ for an arbitrary wavefunction $ \ket{\psi} $, which in turn implies $ [A, B] = 0 $.

    \item Keep in mind that two operators sharing a single eigenvector does not imply the operators commute---they must share an entire eigenbasis. As an example, the angular momentum operators $ L_{x} $ and $ L_{y} $ (discussed further in a future chapter) share the spherical harmonic ground state $ Y_{0}^{0}(\theta, \phi) $, but the two operators do not commute.

    \item Consider an operator $ C $ of the form $ C = A + B $, where $ C $ has eigenvalues $ \{c_{l}\} $ and $ A $ and $ B $ have eigenvalues $ \{a_{m}\} $ and $ \{b_{n}\} $. 

    If $ A $ and $ B $ commute, then $ C $'s eigenvalues are the sum of $ A $ and $ B $'s eigenvalues, i.e.
    \begin{equation*}
        c_{l} = a_{n} + b_{m}.
    \end{equation*}
    This relationship does not hold if $ A $ and $ B $ do not commute.
    
\end{itemize}

\newpage
\section{Examples of Quantum Systems}

\subsection{Quantum Harmonic Oscillator}
\begin{itemize}
	\item In one dimension, the quantum harmonic oscillator's Hamiltonian reads
	\begin{equation*}
		H = \frac{p^{2}}{2m} + \frac{1}{2}kx^{2} = - -\frac{\hbar^{2}}{2m}\dv[2]{}{x} + \frac{1}{2}m \omega^{2}x^{2}, \qquad \omega = \sqrt{\frac{k}{m}}.
	\end{equation*}
	The standard formalism for analyzing the harmonic oscillator follows.
	
	\item We introduce characteristic energy $ \hbar \omega $ and length $ \xi = \sqrt{\frac{\hbar}{m\omega}}$  and write the Hamiltonian as a difference of perfect squares:
	\begin{equation*}
		H = \frac{\hbar \omega}{2}\left(\frac{x^{2}}{\xi^{2}} - \xi^{2} \dv[2]{}{x}\right).
	\end{equation*}
	Keeping in mind that $ x $ and $ \dv{x} $ don't commute, we factor the above into
	\begin{equation*}
		H = \frac{\hbar \omega}{4}\left[\left(\frac{x}{\xi} + \xi \dv{x}\right)\left(\frac{x}{\xi} - \xi \dv{x}\right) + \left(\frac{x}{\xi} - \xi \dv{x}\right)\left(\frac{x}{\xi} + \xi \dv{x}\right)\right].
	\end{equation*}
	
	\item Next, we introduce the annihilation and creation operators, denoted by $ a $  and $ a^{\dagger} $ respectively, and defined by
	\begin{equation*}
		a = \frac{1}{\sqrt{2}}\left(\frac{x}{\xi} + \xi \dv{x}\right) \eqtext{and} a^{\dagger} = \frac{1}{\sqrt{2}}\left(\frac{x}{\xi} - \xi \dv{x}\right). 
	\end{equation*} 
	We recover $ x $ and $ \dv{x} $ from $ a $ and $ a^{\dagger} $ with
	\begin{equation*}
		x = \frac{\xi}{\sqrt{2}}\big(a + a^{\dagger}\big) \eqtext{and} \dv{x} = \frac{1}{\sqrt{2}\xi}\big(a - a^{\dagger}\big). 
	\end{equation*}
	Additionally, we can write the Hamiltonian as
	\begin{equation*}
		H = \frac{\hbar\omega}{2}\big(a a^{\dagger} + a^{\dagger}a\big)
	\end{equation*}
	
	\item Next, we quote to commutation relation
	\begin{equation*}
		\big[a, a^{\dagger}\big] = 1,
	\end{equation*}
	which is proven with a direct application of $ [x, p] = i \hbar $. The relationship allows use to write the Hamiltonian in the form
	\begin{equation*}
		H = \hbar \omega \left(a^{\dagger}a + \frac{1}{2}\right).
	\end{equation*}
	
\end{itemize}

\subsubsection{Eigenvalues and Eigenfunctions}
% See preview page 59 of Schwabl
\begin{itemize}
	\item The next standard step is introducing the counting operator $ \hat{n} \equiv a^{\dagger}a $
	\begin{equation*}
		\hat{n}\k{\phi_{n}} = n \ket{\phi_{n}},
	\end{equation*}
	whose eigenvalues are the index $ n $ of the eigenfunction $ \ket{\phi_{n}} $. We search for real $ n $ corresponding to normalizable eigenstates. First, we show $ n \geq 0 $, which follows from
	\begin{equation*}
		\mel{\phi_{n}}{\hat{n}}{\phi_{n}} = \mel{\phi_{n}}{a^{\dagger}a}{\phi_{n}} = \braket{a \phi_{n}}{a \phi_{n}} = n \braket{\phi_{n}}{\phi_{n}} \geq 0.
	\end{equation*}
	
	\item First, we confirm $ n = 0 $ is a valid solution of counting operator's eigenvalue equation. This comes down to (why only $ a $?) solving the equation
	\begin{equation*}
		a \ket{\phi_{0}} = 0.
	\end{equation*}
	In the coordinate operator and wavefunction representation, the equation reads
	\begin{equation*}
		\frac{1}{\sqrt{2}}\left(\frac{x}{\xi} + \xi \dv{x}\right)\phi_{0}(x) = 0 \eqtext{or} \xi \dv{x} \phi_{0}(x) = - \frac{x}{\xi}\phi_{0}(x)
	\end{equation*}
	The solution is the Gaussian function
	\begin{equation*}
		\phi_{0}(x) = \frac{1}{\sqrt{\sqrt{\pi}\xi}}e^{-\frac{1}{2}\frac{x^{2}}{\xi^{2}}} \equiv \braket{x}{\phi_{0}}.
	\end{equation*}
	The state $ \k{\phi_{0}} $ is the oscillator's ground state, with energy $ E_{0} = \frac{1}{2}\hbar \omega $. We can find all other solutions from the ground state solution. 

	\item First, we derive the commutator relation
	\begin{equation*}
		\big[\hat{n}, a^{\dagger}\big] = \big[a^{\dagger} a, a^{\dagger}\big] = a^{\dagger}\big[a, a^{\dagger}\big] + \big[a^{\dagger}, a^{\dagger}\big]a = a^{\dagger}
	\end{equation*}
	This relationship shows that $ a^{\dagger} $ acts on a state with eigenvalue $ n $ to create a state with eigenvalue $ n + 1 $. We show this with
	\begin{align*}
		\hat{n}a^{\dagger}\k{\phi_{n}} &\equiv a^{\dagger} a a^{\dagger}\k{\phi_{n}} = a^{\dagger}\big(a^{\dagger}a + 1\big)\ket{\phi_{n}} = \big(a^{\dagger}\hat{n} + a^{\dagger}\big)\k{\phi_{n}}\\
		& = a^{\dagger}n \k{\phi_{n}} + a^{\dagger}\k{\phi_{n}} = (n + 1)a^{\dagger}\ket{\phi_{n}}.
	\end{align*}
	Because the counting operator $ \hat{n} $ acts on the state $ a^{\dagger}\ket{\phi_{n}} $ to produce an eigenvalue $ (n+1) $, $ a^{\dagger} $ must have the effect of raising $ \k{\phi_{n}} $'s index by one. In symbols:
	\begin{equation*}
		a^{\dagger} \k{\phi_{n}} = c_{n}\ket{\phi_{n + 1}}.
	\end{equation*}
	
	\item We find the constant $ c_{n} $ from the assumption that the original state $ \ket{\phi_{n}} $ is normalized, i.e. $ \braket{\phi_{n}}{\phi_{n}} = 1 $. The relevant calculation reads
	\begin{align*}
		\bbraket{c_{n}^{*}\phi_{n+1}}{c_{n}\phi_{n+1}} &= \bbraket{a^{\dagger}\phi_{n}}{a^{\dagger}\phi_{n}} = \bbraket{\phi_{n}}{aa^{\dagger}\phi_{n}} = \bbraket{\phi_{n}}{(a^{\dagger}a + 1)\phi_{n}} \\
		&= \braket{\phi_{n}}{(n + 1)\phi_{n}},
	\end{align*}
	which implies $ \abs{c_{n}}^{2} = (n+1) $. Up to a constant phase factor of magnitude one, we define $ c_{n} = \sqrt{n+1} $. The action of $ a^{\dagger} $ is then fully summarized with
	\begin{equation*}
		a^{\dagger} = \k{\phi_{n}} \sqrt{n+1}\ket{\phi_{n+1}} \eqtext{or} \ket{\phi_{n + 1}} = \frac{a^{\dagger}}{\sqrt{n+1}}\k{\phi_{n}}. 
	\end{equation*}
	If we start with $ \ket{\phi_{n}} = \ket{\phi_{0}} $, the latter expression produces to the recursive relation
	\begin{equation*}
		\ket{\phi_{n}} = \frac{a^{\dagger}}{\sqrt{n}}\ket{\phi_{n-1}} = \frac{\big(a^{\dagger}\big)^{n}}{\sqrt{n!}}\ket{\phi_{0}}.
	\end{equation*}
	
	\item While the creation operator $ a^{\dagger} $ raises the index of a harmonic oscillator's eigenstate, the annihilation operator $ a $ lowers a eigenstate's index. The derivation follows the same pattern as above for $ a^{\dagger} $: we use the commutator relation
	\begin{equation*}
		\big[\hat{n}, a\big] = a^{\dagger}\big[a, a\big] + \big[a^{\dagger}, a\big]a = - a
	\end{equation*}
	to show that
	\begin{equation*}
		\hat{n}a \ket{\phi_{n}} = (a \hat{n} - a)\ket{\phi_{n}} = (n-1)a\ket{\phi_{n}}.
	\end{equation*}
	Because the counting operator acts on the state $ a\k{\phi_{n}} $ to produce an eigenvalue $ (n-1) $, $ a $ must have the effect of lowering $ \ket{\phi_{n}} $'s index by one, i.e.
	\begin{equation*}
		a\ket{\phi_{n}} = d_{n}\ket{\phi_{n-1}}
	\end{equation*}
	
	\item As for $ a^{\dagger} $, we find the constants $ d_{n} $ under that assumption that the original state $ \ket{\phi_{n}} $ is normalized, i.e. $ \braket{\phi_{n}}{\phi_{n}} = 1 $. The relevant calculation reads
 	\begin{equation*}
 		\bbraket{d_{n}^{*}\phi_{n-1}}{d_{n}\phi_{n-1}} = \bbraket{a\phi_{n}}{a\phi_{n}} = \bbraket{\phi_{n}}{a^{\dagger}a\phi_{n}} = \bbraket{\phi_{n}}{n\phi_{n}} = n \bbraket{\phi_{n}}{\phi_{n}}
 	\end{equation*}
 	which implies $ \abs{d_{n}}^{2} = 2 $. Up to a constant phase factor of magnitude one, we define $ d_{n} = \sqrt{n} $. The action of $ a $ is then fully summarized with
 	\begin{equation*}
 		a\k{\phi_{n}} = \sqrt{n}\ket{\phi_{n-1}} \eqtext{or} \ket{\phi_{n - 1}} = \frac{a}{\sqrt{n}}\k{\phi_{n}}. 
 	\end{equation*}
 	The latter expression results in the recursion relations
 	\begin{equation*}
 		\ket{\phi_{n}} = \frac{a}{\sqrt{n+1}}\ket{\phi_{n+1}} \eqtext{and} \ket{\phi_{0}} = \frac{a^{n}}{\sqrt{n!}}\ket{\phi_{n}}.
 	\end{equation*}
 	
 	\item The recursive relations involving $ a^{\dagger} $  and $ a $ solve the harmonic oscillator problem. The results are
 	\begin{equation*}
 		H\ket{n} = E_{n}\ket{n} \qquad E_{n} = \left(n + \frac{1}{2}\right)\hbar \omega \qquad \braket{m}{n} = \delta_{mn}.
 	\end{equation*}
	 
\end{itemize}

\textbf{Some Discussion of the Solution}
\begin{itemize}
	\item In one dimension, the harmonic oscillator's energy eigenvalues $ E_{n} $ are nondegenerate.\footnote{This does not hold in higher dimensions} We prove nondegeneracy by contradiction: assume in addition to $ \k{\phi_{n}} $ there exists another linearly independent eigenstates $ \bk{\t{\phi}_{n}} $ with the same energy $ E_{n} $. From the recursion relation
	\begin{equation*}
		\ket{\phi_{n}} = \frac{a^{n}}{\sqrt{n!}}\ket{\phi_{0}},
	\end{equation*}
	the state $ \bk{\t{\phi}_{n}} $ must obey $ a^{n}\bk{\t{\phi}_{n}} \propto \bk{\t{\phi}_{0}} $. However, the harmonic oscillator's ground state is non-degenerate, since the earlier ground state equation 
	\begin{equation*}
		\xi \dv{x} \phi_{0}(x) = - \frac{x}{\xi}\phi_{0}(x)
	\end{equation*}
	has only one normalized solution:
	\begin{equation*}
		\braket{x}{\phi_{0}} = \frac{1}{\sqrt{\sqrt{\pi}\xi}}e^{-\frac{1}{2}\frac{x^{2}}{\xi^{2}}}.
	\end{equation*}
	Because the oscillator's ground state is nondegenerate and all higher states are proportional to the ground state via $  a^{n}\bk{\phi_{n}} \propto \bk{\phi_{0}}  $, all higher states are also nondegenerate.
	
	\item The harmonic oscillator's energy eigenvalues have only integer indexes $ n \in \mathbb{N} $. We prove this by contradiction: assume there exists an energy eigenstate $ \k{\phi_{\lambda}} $ with index $ \lambda = n + \nu $ where $ \nu \in (0, 1) $. Applying the counting operator to $ \k{\phi_{\lambda}} $ produces
	\begin{equation*}
		\hat{n}\ket{\phi_{\lambda}} = \lambda \ket{\phi_{\lambda}} = (n + \nu)\ket{\phi_{\lambda}}
	\end{equation*}
	Repeatedly applying the annihilation operator $ a $ to the state $ \ket{\phi_{\lambda}} $ and using the recursion relation $ \ket{\phi_{n}} = \frac{a^{n}}{\sqrt{n!}}\ket{\phi_{0}} $ would eventually lead to a state with the index $ \lambda \in (-1, 0) $, i.e. a negative index. This contradicts the earlier result from the beginning of the ``Eigenvalues and Eigenfunctions'' section, which showed that harmonic oscillator's indexes are non-negative, i.e. $ n \geq 0 $. 
\end{itemize}

\subsubsection{Eigenfunctions in the Coordinate Representation}
\begin{itemize}
	\item In the coordinate representation, the harmonic oscillators eigenfunctions are found with the generating formula
	\begin{equation*}
		\braket{x}{\phi_{n}} = \phi_{n}(x) = \frac{1}{\sqrt{2^{n}n!}}\left(\frac{x}{\xi} - \xi \dv{x}\right)^{n}\phi_{0}(x).
	\end{equation*}
	The ground state eigenfunction with $ n = 0 $ is even, and the excited state eigenfunctions with $ n = 1, 2, \ldots $ alternate between even and odd according to the parity of the index $ n $. 
	
	Perhaps more intuitively, the eigenfunctions are just the product of a Hermite polynomial and the fundamental Gaussian solution $ \phi_{0}(x) $. In this form, the eigenfunctions are written
	\begin{align*}
		\phi_{n} = C_{n} H_{n}\left(\frac{x}{\xi}\right)e^{-\frac{1}{2}\frac{x^{2}}{\xi^{2}}},
	\end{align*}
	where $ H_{n} $ is the $ n $th Hermite polynomial and
	\begin{equation*}
		\xi = \sqrt{\frac{\hbar}{m \omega}} \eqtext{and} C_{n} = \frac{1}{\sqrt{2^{n}n!\xi \sqrt{n}}}
	\end{equation*}
	
	\item The characteristic width of each eigenfunction increases with the index $ n $; the width $ \sigma_{x} $ of the $ n $th state obeys
	\begin{equation*}
		\frac{\sigma_{x_{n}}^{2}}{\xi^{2}} = \frac{1}{2}\bmel{n}{\big(a^{\dagger}\big)^{2} + a^{\dagger}a + aa^{\dagger} + a^{2}}{n} = n + \frac{1}{2}
	\end{equation*}
	In the $ p $-space representation, the $ n $th eigenfunctions characteristic width is
	\begin{equation*}
		\sigma^{2} \sigma_{p_{n}}^{2} = - \frac{\hbar^{2}}{2}\bmel{n}{\big(a^{\dagger}\big)^{2} - a^{\dagger}a - aa^{\dagger} + a^{2}}{n} = \hbar^{2}\left(n + \frac{1}{2}\right)
	\end{equation*}
	The product $ \sigma_{x_{n}}\sigma_{p_{n}} $ is thus
	\begin{equation*}
		\sigma_{x_{n}}\sigma_{p_{n}} = \hbar\left(n + \frac{1}{2}\right)
	\end{equation*}
	Note that in the ground state $ \sigma_{x_{n}}\sigma_{p_{n}} = \hbar/2 $, in agreement with the Heisenberg uncertainty principle.
	
	
\end{itemize}

\subsubsection{The Harmonic Oscillator in Three Dimensions}
\begin{itemize}
	\item We consider the three-dimensional case with an anisotropic potential. The \Ham reads
	\begin{equation*}
		H(\r) = \frac{p^{2}}{2m} + \frac{1}{2}\sum_{ij} \omega_{ij}x_{i}x_{j}
	\end{equation*}
	Just like in classical mechanics in the field of small oscillations modeled by harmonic oscillators, we can transform to normal coordinates and conjugate momenta, in which case the \Ham transforms to the diagonal form
	\begin{equation*}
		H = \frac{p^{2}}{2m} + \frac{1}{2}\sum_{i=1}^{3}k_{i}x_{i}^{2}, \qquad k_{i} = m\omega_{i}^{2}
	\end{equation*}
	
	\item As in the one-dimensional case, we introduce annihilation and creation operators, this time for each index $ i $. The Hamiltonian becomes
	\begin{equation*}
		H = \sum_{i = 1}^{3}\hbar \omega_{i}\left(a^{\dagger}_{i}a_{i} + \frac{1}{2}\right), \qquad \big[a_{i}, a_{j}^{\dagger}\big] = \delta_{ij}.
	\end{equation*}
	
	\item Because $ H $ is a sum of linearly independent operators $ H_{i} $, the \Ham's eigenstates can be written in the factored form
	\begin{equation*}
		\P_{n_{1}n_{2}n_{3}}(\r) = \prod_{i = 1}^{3}\phi_{n_{i}}(x_{i}) \equiv \braket{\r}{n_{1}n_{2}n_{3}}
	\end{equation*}
	The higher states can be constructed from the ground state according to
	\begin{equation*}
		\ket{n_{1}n_{2}n_{3}} = \prod_{i = 1}^{3}\frac{\big(a_{i}^{\dagger}\big)^{n_{i}}}{\sqrt{n_{i}!}}\ket{000}
	\end{equation*}
\end{itemize}

\subsubsection{Coherent Ground State}
\begin{itemize}
	\item Consider a particle in the ground state of harmonic potential whose initial state at $ t = 0 $ is initially displaced by $ \ev{x} = x_{0} $ from the equilibrium position. The particle's initial wavefunction is thus $ \phi_{0}(x - x_{0}) $, where $ \phi_{0} $ is the harmonic ground state eigenfunction at the equilibrium position.
	
	\item We begin by considering the eigenvalue equation
	\begin{equation*}
		a \k{\phi_{\alpha}} = \alpha\k{\p_{\alpha}}
	\end{equation*}
	where the eigenvalue is the complex number $ \alpha = \abs{\alpha}e^{i\delta} \in \mathbb{C}$. We expand the state $ \k{\p_{\alpha}} $ in the harmonic oscillator's eigenbasis $ \{\ket{n}\} $. Using the recursion relation
	\begin{equation*}
		\ket{n} = \frac{\big(a^{\dagger}\big)^{n}}{\sqrt{n!}}\k{0},
	\end{equation*}
	the expansion of $ \k{\p_{\alpha}} $ in the basis $ \{\k{n} \} $ reads
	\begin{align*}
		\k{\p_{\alpha}} &= \sum_{n}\braket{n}{\p_{\alpha}}\ket{n} = \sum_{n} \frac{1}{\sqrt{n!}} \braket{\big(a^{\dagger}\big)^{n}\phi_{0}}{\p_{\alpha}}\ket{n} \\
		& = \braket{\phi_{0}}{\p_{\alpha}}\sum_{n}\frac{\alpha^{n}}{\sqrt{n!}}\ket{n} = \braket{\phi_{0}}{\p_{\alpha}}\sum_{n}\frac{\big(\alpha a^{\dagger}\big)^{n}}{n!}\ket{0}\\
		& = \braket{\phi_{0}}{\p_{\alpha}}e^{\alpha a^{\dagger}}\ket{0}
	\end{align*} 
	where we the last line uses the Taylor series definition of the exponential function. We determine the constant $ \braket{\phi_{0}}{\p_{\alpha}} $ from the normalization condition $ \braket{\p_{\alpha}}{\p_{\alpha}} \equiv 1 $; the calculation reads
	\begin{equation*}
		1 \equiv \braket{\p_{\alpha}}{\p_{\alpha}} \abs{\braket{\phi_{0}}{\p_{\alpha}}}^{2}\sum_{n}\frac{\abs{\alpha}^{2n}}{n!} \implies \braket{\phi_{0}}{\p_{\alpha}} = e^{-\frac{1}{2}\abs{\alpha}^{2}}
	\end{equation*}
	
	\item The state $ \ket{\p_{\alpha}} $ at $ t = 0 $ is called a called a coherent state. Its time evolution reads
	\begin{equation*}
		\k{\p_{\alpha}(t)} = \exp(-\frac{1}{2}i\omega t - \frac{1}{2}\abs{\alpha}^{2}) \sum_{n} \frac{\big(\alpha e^{-i\omega t}a^{\dagger}\big)^{n}}{n!}\ket{0}
	\end{equation*}
	In terms of the eigenfunctions $ \ket{\phi_{n}(t)} $, the above time evolution is written
	\begin{equation*}
		\k{\p_{\alpha}(t)} = \sum_{n}c_{n}\k{\phi_{n}(t)} \eqtext{where} c_{n} = \frac{\alpha^{n}}{\sqrt{n!}}e^{-\frac{1}{2}\abs{\alpha}^{2}}
	\end{equation*}
	
	\item For the state $ \ket{\p_{\alpha}(t)} $, the probability $ P_{n} $ for occupation of a state with index $ n $ falls exponentially:
	\begin{equation*}
		P_{n} = \abs{c_{n}}^{2} = \frac{\ev{\hat{n}}^{2}}{n!}e^{-\ev{\hat{n}}}
	\end{equation*}
	where $ \ev{\hat{n}} = \mel{\p_{\alpha}}{\hat{n}}{\p_{\alpha}} $. 
	
	\item The position expectation value of the state $ \ket{\p_{\alpha}(t)} $ obeys
	\begin{equation*}
		\ev{x(t)} = x_{0}\cos(\omega t - \delta), \eqtext{where} x_{0} = \sqrt{2}\xi \abs{\alpha},
	\end{equation*}
	which follows from
	\begin{equation*}
		\mel{\ket{\p_{\alpha}(t)}}{x}{\ket{\p_{\alpha}(t)}} = \frac{\xi}{\sqrt{2}} \bmel{\ket{\p_{\alpha}(t)}}{\big(a + a^{\dagger}\big)}{\ket{\p_{\alpha}(t)}} = \frac{\xi}{\sqrt{2}} \big(\alpha e^{-i\omega t} + \alpha^{*}e^{i\omega t}\big).
	\end{equation*}
	Note that $ \ev{x(t)} $ has the time dependence as the analogous classical solution $ x(t) = x_{0}\cos(\omega t - \delta) $.
	
	The real component of the constant $ \alpha = \abs{\alpha}e^{i\delta} $ corresponds to the displacement of the particle (or wavefunction's center)  from equilibrium, while the imaginary part of $ \alpha $ corresponds to the initial velocity $ v_{0} = \frac{\ev{p}}{m}\big|_{t = 0} $.
	
	\item Next, we note the energy expectation value is
	\begin{equation*}
		\ev{E} = \hbar \omega\left(\abs{\alpha}^{2} +  \frac{1}{2}\right) = \hbar \omega \left(\frac{m\omega x_{0}^{2}}{2\hbar} + \frac{1}{2}\right)
	\end{equation*}
	In the classical limit $ \hbar \to 0 $, the energy reduces to the classical value $ E = \frac{1}{2}m\omega^{2}x_{0}^{2} $. 
	
	\item Finally---characteristic for a coherent state---the probability density $ \rho(t) $ oscillates back and forth in the harmonic potential while preserving the shape of the initial Gaussian distribution, i.e.
	\begin{equation*}
		\rho(x, t) = \abs{\psi_{\alpha}(t)}^{2} = \frac{1}{\sqrt{2\pi}\sigma_{x_{0}}}\exp(-\frac{(x - \ev{x})^{2}}{2 \sigma_{x_{0}}^{2}})
	\end{equation*}
	
\end{itemize}

\subsection{Operators in Matrix Form}
\begin{itemize}
	\item Finally, as an exercise, we write the operators $ a^{\dagger} $, $ x $ and $ p $ in the harmonic oscillator eigenbasis $ \{\ket{n} \} $. 
	
	First, $ a^{\dagger} $. Using the equation $ a^{\dagger}\ket{n} = \sqrt{n+1}\ket{n+1} $, the matrix elements are 
	\begin{equation*}
		a^{\dagger}_{mn} = \bmel{m}{a^{\dagger}}{n} = \mel{n+1}{\sqrt{n+1}}{n}\delta_{m, n+1}
	\end{equation*}
	In matrix form $ a^{\dagger} $ reads
	\begin{equation}
	a^{\dagger} =
	\begin{pmatrix}
	0 & 0 & 0 & \cdots\\
	1 & 0 & 0 & \cdots\\
	0 & \sqrt{2} & 0 & \cdots\\
	\vdots & \vdots & \ddots & \vdots
	\end{pmatrix}
	\end{equation}
	Note that $ a^{\dagger} $ is asymmetric and thus non-\Herm. 
	
	\item Similarly, the expressions for $ x $ and $ p $ are
	\begin{equation*}
		x = \sqrt{\frac{\hbar}{2m\omega}} 
		\begin{pmatrix}
		0 & 1 & 0 & \cdots\\
		1 & 0 & \sqrt{2} & \cdots\\
		0 & \sqrt{2} & 0 & \ddots\\
		\vdots & \vdots & \ddots & \ddots
		\end{pmatrix}
		\eqtext{and}
		p = \sqrt{\frac{m\hbar \omega}{2}} 
		\begin{pmatrix}
		0 & i & 0 & \cdots\\
		-i & 0 & i\sqrt{2} & \cdots\\
		0 & -i\sqrt{2} & 0 & \ddots\\
		\vdots & \vdots & \ddots & \ddots
		\end{pmatrix}
	\end{equation*}
	As expected, both $ x $ and $ p $ have \Herm matrices.
\end{itemize}

\subsection{Gaussian Wave Packet}
\begin{itemize}
	\item Consider a free particle with an generic initial wavefunction expanded in the momentum (plane wave) basis, i.e.
	\begin{equation*}
		\k{\p(0)} = \int \tilde{\p}(p)\ket{p}\diff p.
	\end{equation*}
	Because of the plane wave dispersion relation $ E = \frac{p^{2}}{2m} $, plane waves have a different phase velocity for each $ p $. This varying phase velocity for different $ p $ causes the wavefunction do deform from the initial state in its time evolution $ \ket{\p, (t)} $.
	
	\item We analyze this deformation process in the concrete case when the initial state is a Gaussian wave packet. In the momentum representation, the wavefunction is
	\begin{equation*}
		\tilde{\p}(p) = C\exp(-\frac{(p-p_{0})^{2}}{4\sigma_{p}^{2}}) \eqtext{where} C = \frac{1}{\sqrt{\sqrt{2\pi}\sigma_{p}}}
	\end{equation*}
	The relevant constants are expectation values:
	\begin{align*}
		& \ev{p} = \int p \abs{\tilde{\p}(p)}^{2}\diff p = p_{0}\\
		& \ev{p^{2}} = \int p^{2}\abs{\t{\p}(p)}^{2}\diff p = p_{0}^{2} + \sigma_{p}^{2}\\
		&\Delta p^{2} \equiv \sigma_{p}^{2} = \ev{p^{2}} - \ev{p}^{2} \\
		&\ev{E} = \frac{p_{0}^{2}+\sigma_{p}^{2}}{2m}
	\end{align*}
	
	\item In the $ x $ representation, the wavefunction is the characteristic function of the momentum representation $ \t{\p}(p) $:
	\begin{align*}
		\p(x, 0) &= \braket{x}{\p(0)} = \int \t{\p}(p)\braket{x}{p}\diff p = \frac{C}{\sqrt{2\pi \hbar}}\int \exp(-\frac{(p-p_{0})^{2}}{4\sigma_{p}^{2}} + i \frac{px}{\hbar})\diff p\\
		& = \frac{C}{\sqrt{2\pi \hbar}} \int \exp\left\{ -\frac{1}{4\sigma_{p}^{2}} \left[p - \left(p_{0} + 2i\frac{\sigma_{p}^{2}}{\hbar}x\right) \right]^{2} - \frac{\sigma_{p}^{2}x^{2}}{\hbar^{2}} + i \frac{p_{0}x}{\hbar} \right\}\diff p\\
		& =  \frac{1}{\sqrt{\sqrt{2\pi}\sigma_{p}}} \exp(-\frac{x^{2}}{4\sigma_{x}^{2}} + i\frac{p_{0}x}{\hbar})
	\end{align*}
	where the last line uses $ \sigma_{x} = \frac{\hbar}{2\sigma_{p}} $ and 
	\begin{equation*}
		\int_{-\infty}^{\infty} e^{-zx^{2}}\diff x = \sqrt{\frac{\pi}{z}} \quad \text{for } \Re z > 0.
	\end{equation*}
	
	\item Next, the wavefunction's time evolution for $ t > 0 $ is 
	\begin{align*}
		\psi(x, t) &= \frac{C}{\sqrt{2\pi \hbar}}\int \exp(-\frac{(p-p_{0})^{2}}{4\sigma_{p}^{2}} + i \frac{p^{2}}{2m\hbar}t)\diff p\\
		& =  \frac{1}{\sqrt{\sqrt{2\pi}\tilde{\sigma}(t)}} \exp\left[-\frac{x^{2}}{4\sigma_{x}\tilde{\sigma}(t)} + i\left(\frac{p_{0}}{\hbar}x - \frac{p_{0}^{2}}{2m\hbar}t\right)\right]
	\end{align*}
	where
	\begin{equation*}
		\tilde{\sigma}(t) = \sigma_{x}\left(1 + i \frac{\hbar t}{2m\sigma_{x}^{2}}\right)
	\end{equation*}
	
	\item The corresponding wavefunction (without derivation) is
	\begin{equation*}
		\rho(x, t) = \frac{1}{\sqrt{2\pi}\sigma(t)}\exp\left(-\frac{(x- \ev{x})^{2}}{2\sigma(t)^{2}}\right)
	\end{equation*}
	where
	\begin{equation*}
		\sigma(t) = \abs{\tilde{\sigma}(t)} \eqtext{and} \sigma(t)^{2} = \sigma_{x}^{2}\left(1 + \frac{\hbar^{2}t^{2}}{4m^{2\sigma_{x}^{4}}}\right)
	\end{equation*}
	For reference, this (supposedly) follows from
	\begin{equation*}
		2 \Re \left[- \frac{x^{2}}{4\sigma_{x}\tilde{\sigma}(t)} + i\left(\frac{p_{0}}{\hbar}x - \frac{p_{0}^{2}}{2m\hbar}t\right)\frac{\sigma_{x}}{\t{\sigma}(t)}\right] = - \frac{\left(x - \frac{p_{0}}{m}t\right)^{2}}{2 \abs{\t{\sigma}(t)}^{2}}
	\end{equation*}
	Recall that $ \t{\sigma}(t) $ is complex.
	
	\item Summary: the solution to the \Schro equation does not preserve the shape of the initial condition (like e.g. the wave equation). The deformation is a consequence of the momentum basis functions $ \ket{p} $ having varying phase velocity. The solution remains a Gaussian wave packet, but its width increases with time.
	
\end{itemize}

\subsection{Phase and Group Velocity}
\textbf{TODO:} Optional material, add as time permits.


\subsection{Time Evolution of the Dirac Delta Function}
\textbf{TODO:} Optional material, add as time permits.

\newpage
\section{Symmetries}


\subsection{Translational Symmetry}
% Relevant for constant (free) or periodic potentials 
In this section we consider only active translations, which correspond to a translation of a wavefunction, as opposed to a translation of the coordinate system or basis vectors. 
\begin{itemize}
	\item In one dimension, a translation of a wavefunction $ \p $ by $ s $ reads
	\begin{equation*}
		\tilde{\p}(x) = \p(x - s)
	\end{equation*}
	We write the translation in terms of a translation operator $ U(s) $ according to
	\begin{equation*}
		U(s)\p = \p(x - s)
	\end{equation*}
	
	\item We find the expression for $ U(s) $ with a Taylor series expansion of $ \p(x - s) $:
	\begin{align*}
		U(s)\p(x) &= \p(x - s) = \p(x) - s \pdv{x}{\p(x)} \pm  \cdots + \frac{(-s)^{n}}{n!}\pdv[n]{\p(x)}{x} + \cdots \\
		& = e^{-s \pdv{x}}\p(x) \\
		& = e^{-is\frac{p}{\hbar}}\p(x)
	\end{align*}
	The translation operator in one dimensions is thus
	\begin{equation*}
		U(s) = e^{-is\frac{p}{\hbar}}
	\end{equation*}
	
	\item In three dimensions, a translation by a distance $ s $ in the direction of the unit vector $ \uvec{n} $ reads
	\begin{equation*}
		\tilde{\p}(\r) = \p(\r - s\uvec{n})
	\end{equation*}
	and the corresponding translation operator is
	\begin{equation*}
		U(s \uvec{n}) = e^{-is \frac{\uvec{n}\cdot \vec{p}}{\hbar}} \eqtext{or} U(\vec{s}) = e^{-i \frac{\vec{s}\cdot \vec{p}}{\hbar}}
	\end{equation*}
	where we have defined the vector displacement $ \vec{s} = s \uvec{n} $. Note that $ \uvec{n}\cdot \vec{p} $, i.e. the projection of momentum in the direction $ \uvec{n} $ is the transformation's generator.
	
	\item Like in classical mechanics, symmetries in quantum mechanics correspond to a conserved quantity---translational symmetry corresponds to conservation of (translational) momentum. 
	
	In free space (for a globally constant potential), momentum is conserved under the condition $ [\vec{p}, H] = 0 $, which occurs when the Hamiltonian is invariant under translation, i.e. when
	\begin{equation*}
		[U(\vec{s}), H] = 0 \ \text{for all } \vec{s} \in \mathbb{R}^{3}
	\end{equation*}
	
	\item In the presence of a periodic potential with period $ \vec{a} $, ie. $ V(\r) = V(\r + n \vec{a}) $ where $ n \in \mathbb{Z} $ is an integer, translational invariance holds for translations of the form $ \vec{s}_{n} = n \vec{a} $. In this case, the wavefunction takes the form
	\begin{equation*}
		\psi_{\vec{k}}(\r) = e^{i\vec{k}\cdot \r}u(\r)
	\end{equation*}
	where $ u(\r + \vec{a}) = u(\r) $ is a periodic function. 
	
\end{itemize}	

\textbf{Example: Position Expectation Value After Translation}
\begin{itemize}	
	\item Example: calculating expectation value of position after a one-dimensional translation of the form $ \tilde{\p} = U(s)\p(x) = \p(x - s) $.
	
	First, we define the translated position operator
	\begin{equation*}
		\tilde{x} = U^{\dagger}xU = e^{i\frac{sp}{\hbar}} x e^{-i\frac{sp}{\hbar}}
	\end{equation*}
	and use the Baker-Hausdorff lemma to write
	\begin{equation*}
		\tilde{x} = e^{i\frac{sp}{\hbar}} x e^{-i\frac{sp}{\hbar}} = x + \left[ i \frac{sp}{\hbar}, x\right] + \frac{1}{2!}\left[i\frac{sp}{\hbar}, \left[ i \frac{sp}{\hbar}, x\right]\right] + \cdots 
	\end{equation*}
	The first commutator evaluates to
	\begin{equation*}
		\frac{i}{\hbar}[sp, x] = \frac{i}{\hbar} s[p, x] + \frac{i}{\hbar}[s, x]p = \frac{i}{\hbar}(-i\hbar)s + 0 = s
	\end{equation*}
	The remaining, higher-order commutators evaluate to zero, leaving
	\begin{equation*}
		\tilde{x} = x + s + 0 + \cdots
	\end{equation*}
	
	\item We then find the expectation according to 
	\begin{align*}
		\ev{\tilde{x}} &= \mel{\p(x-s)}{x}{\p(x - s)} \\
		& = \bmel{U(s)\p(x)}{x}{U(s)\p(x)}\\
		& = \bmel{\p(x)}{U^{\dagger}(s)xU(s)}{\p(x)}\\
		& = \bmel{\p(x)}{\t{x}}{\p(x)}\\
		& = \mel{\p(x)}{x + s}{\p(x)}\\
		& = \ev{x} + s
	\end{align*}
	 
\end{itemize}

\subsection{Rotation}
% Relevant for spherically-symmetric potentials
We consider active rotations of a wavefunction $ \psi $ about an axis in the direction of the unit vector $ \uvec{n} $. 
\begin{itemize}
	\item We first consider rotations by an infinitesimal angle $ \diff \phi $, for which the rotated wavefunction $ \t{\p} $ is
	\begin{equation*}
		\t{\p}(\r) = \p(\r - \diff \r) \ \text{where } \diff \r = \diff \phi (\uvec{n} \cross \r)
	\end{equation*}
	We find the expression for the rotation operator with a first-order Taylor expansion
	\begin{align*}
		\t{\p}(\r) &= \p(\r - \diff \r) = \p(\r) - \frac{i}{\hbar}\big[(\uvec{n}\cross \r) \cdot \vec{p}\big]\p(\r)\diff \phi + \mathcal{O} (\diff \phi^{2})\\
		& = \left[\mat{I} - \frac{i}{\hbar}\big[\uvec{n}\cdot (\r \cross \vec{p})\big]\diff \phi\right] \p(\r)  + \mathcal{O} (\diff \phi^{2})\\
		& = \left[\mat{I} - \frac{i}{\hbar}(\uvec{n}\cdot \vec{L})\diff \phi\right] \p(\r) + \mathcal{O} (\diff \phi^{2})
	\end{align*}
	where $ \mat{I} $ is the identity operator and $ \vec{L} = \r \cross \vec{p} $ is the angular momentum operator. 
	
	\item We then construct a rotation by the macroscopic angle $ \phi $ from a product of $ N \to \infty $ infinitesimal rotations by $ \diff \phi = \frac{\phi}{N} $ according to 
	\begin{equation*}
		\tilde{\p}(\r) = \lim_{N \to \infty} \left(\mat{I} - \frac{i}{\hbar}(\uvec{n}\cdot \vec{L})  \frac{\phi}{N} \right)^{N}\p(\r) \equiv \exp(- \frac{i}{\hbar}(\uvec{n}\cdot \vec{L})\phi) \p(\r)
	\end{equation*}
	The rotation operator for an angle $ \phi $ about the axis $ \uvec{n} $ is thus
	\begin{equation*}
		U(\phi \uvec{n}) = U(\vec{\phi}) = \exp(- \frac{i}{\hbar}(\vec{\phi}\cdot \vec{L})) 
	\end{equation*}
	where we have defined the ``vector angle'' $ \vec{\phi} = \phi \uvec{n} $. The generator of the rotation operator is $ \uvec{n} \cdot \vec{L} $, the component of angular momentum along the rotation axis $ \uvec{n} $. 
	
	\item Rotational symmetry corresponds to conservation of angular momentum. A system's angular momentum is conserved if the system's Hamiltonian commutes with the angular momentum operator, i.e. $ [\vec{L}, H] = 0 $, which occurs when the \Ham is invariant under rotation, i.e.
	\begin{equation*}
		[U(\vec{\phi}), H] = 0 \ \text{for all rotations } \vec{\phi}
	\end{equation*}
	This form of conservation occurs for spherically symmetric potentials of the form $ V(\r) = V(r) $ where $ r = \abs{r} $. 
\end{itemize}

\subsection{Parity}
% Relevant for even potentials
\begin{itemize}
	\item Space inversion is encoded by the parity operator $ \Par $, which maps $ \r $ to $ -\r $ in the form $ \Par:\p(\r) \mapsto \p(-\r) $.
	
	\item The parity operator is \Herm, which we prove with
	\begin{equation*}
		\mel{\phi(\r)}{\Par}{\p(\r)} = \braket{\phi(\r)}{\p(-\r)} = \braket{\phi(-\r)}{\p(\r)} = \braket{\Par \phi(\r)}{\p(\r)}
	\end{equation*}
	The parity operator is also unitary, i.e. $ \Par \Par = \II \implies \Par = \Par^{-1} $.

	\item The parity operator changes the sign of the gradient (or derivative) operator:
	\begin{equation*}
		\Par \grad \p = - \grad \Par \psi \implies \Par \grad = - \grad \Par
	\end{equation*}
	The relationship $ \Par \grad = - \grad \Par $ implies
	\begin{equation*}
		\Par \grad^{n} = (-1)^{n}\grad^{n} \Par \eqtext{and} \Par \dv[2]{}{x} = \dv[2]{}{x} \Par
	\end{equation*}
	The last two identities lead to
	\begin{equation*}
		\Par \vec{p} = - \vec{p} \Par \eqtext{and} \Par (\r \cross \vec{p}) = \Par \vec{L} = \vec{L} \Par 
	\end{equation*}
	
	\item For an even potential $ V(\r) = V(-\r) $, the parity operator acts on $ V $ as $ \Par V(\r) = V(-\r)\Par = V(\r)\Par $, in which case
	\begin{equation*}
		\Par H\p(\r) = H \Par \p(\r) \implies [\Par, H] = 0
	\end{equation*}
	In this case, if $ \ket{\p(\r)} $ is a stationary state of the \Ham and obeys the stationary \Schro equation
	\begin{equation*}
		H\k{\p(\r)} = E \ket{\p(\r)}
	\end{equation*}
	then $ \ket{\p(-\r)} $ is also a stationary state with the same energy $ E $, i.e.
	\begin{equation*}
		H\k{\p(-\r)} = E \ket{\p(-\r)}
	\end{equation*}
	We can then (again, this applies only to an even potential) combine the stationary state solutions $  \ket{\p(\r)} $ and $ \ket{\p(-\r)} $ to create the odd and even functions $  \ket{\p_{+}(\r)} $ and $ \k{\p_{-}(\r)} $ according to
	\begin{equation*}
		\p_{\pm}(\r) = \frac{1}{\sqrt{2}}\left(\p(\r) \pm \p(-\r)\right)
	\end{equation*}
	In other words, for an even potential, we can always create an even or odd stationary state eigenfunction for each energy eigenvalue $ E $ (assuming $ E $ is nondegenerate).
	
	Note also that both $  \ket{\p_{+}(\r)} $ and $ \k{\p_{-}(\r)} $ are eigenfunctions of the parity operator with eigenvalues $ \pm 1 $, i.e.
	\begin{equation*}
		\Par \p_{+}(\r) =  \p_{+}(\r) \eqtext{and} \Par \p_{-}(\r) =  -1 \cdot \p_{-}(\r)
	\end{equation*}
	
\end{itemize}

\subsection{Time Reversal}
% Relevant for time-independent potentials
\begin{itemize}
	\item The time reversal operator $ T $ maps time $ t $ to $ -t $ in the form $ T: \P(\r, t) \mapsto \P(\r, -t) $.
	
	\item Assume $ \P(\r, t) $ solves the \Schro equation for for some time-independent potential $ V = V(\r) $ and \Ham $ H \neq H(t) $. The \Schro equation reads
	\begin{equation*}
		i \hbar \pdv{\P(\r, t)}{t} = H \P(\r, t)
	\end{equation*}
	We then act on the equation with the time reversal operator to get
	\begin{align*}
		& T\left(i \hbar \pdv{\P(\r, t)}{t}\right) = i \hbar \pdv{\P(\r, -t)}{(-t)} = H \P(\r, -t)  \\
		& \implies i \hbar \pdv{\P(\r, -t)}{t} = -H \P(\r, -t) 
	\end{align*}
	In other words, $ T\P(\r, t) = \P(\r, -t) $ solves the same \Schro for $ H \to - H $.
	
	\item Alternatively, we can define a modified time reversal operator $ \T = KT $ where $ K : \psi \mapsto \psi^{*} $ is the complex conjugation operator. The complex conjugation obeys $ K z = z^{*}K $ for all $ z \in \mathbb{C} $ and equals its inverse, ie. $ K = K^{-1} $. 
	
	Again assuming a real Hamiltonian, we return to the \Schro equation
	\begin{equation*}
		i \hbar \pdv{\P(\r, t)}{t} = H \P(\r, t) 
	\end{equation*}
	and act on the equation with the $ \T $ to get
	\begin{align*}
		& \T\left(i \hbar \pdv{\P^{*}(\r, t)}{t}\right) = -i \hbar \pdv{\P^{*}(\r, -t)}{(-t)} = H \P^{*}(\r, -t) \\
		& \implies i \hbar \pdv{\P^{*}(\r, -t)}{t} = H \P^{*}(\r, -t) 
	\end{align*}
	In other words, $ \P^{*}(\r, -t) $ also solves the \Schro equation for the same Hamiltonian $ H $. 
	
	\item Next, we consider stationary states of the form
	\begin{equation*}
		\P(\r, t) = \p(\r)e^{-i\frac{E}{\hbar}t}
	\end{equation*}
	The modified time reversal operator $ \T $ acts on this state to produce
	\begin{equation*}
		\P^{*}(\r, -t) = \p^{*}(\r)e^{-(-i)\frac{E}{\hbar}(-t)} = \p^{*}(\r)e^{-i\frac{E}{\hbar}t}
	\end{equation*}
	In other words, $ \T $ affects only the position-dependent term $ \p(\r) $, which it conjugates. With this in mind,	$ \T $ acts on the stationary \Schro equation $ H \p(\r) = E\p(\r) $ to produce
	\begin{equation*}
		 H \p^{*}(\r) = E\p^{*}(\r) 
	\end{equation*}
	In other words, but $ \p $ and $ \p^{*} $ solve the stationary \Schro equation for a given energy eigenvalue $ E $. For a non-degenerate spectrum, the conjugate function can be written $ \p^{*} = e^{i\phi}\p(\r) $. Since $ \p $ and $ \p^{*} $ differ only by a constant phase term $ e^{i\delta} $ of magnitude 1, they correspond to physically identical wavefunction, since phase information is lost in any physically observable quantities, which involve the squared modulus of $ \p $.
	
	\item The time reversal operator $ \T $ acts on the momentum operator $ \vec{p} $, angular momentum operator $ \vec{L} $, and \Ham $ H $ (assuming $ H $ is time-independent and real) as
	\begin{equation*}
		\T \vec{p} = - \vec{p} \T \qquad \T \vec{L} = - \vec{L} \T \qquad \T H = H \T
	\end{equation*}
	
	\item Finally, we briefly mention that for particles with spin quantum number $ s = 1/2 $, we require $ \T $ act on the spin operator $ \vec{S} $ according in the same way as for angular momentum, i.e. $ \T \vec{S} = - \vec{S} \T $. For this too hold, we generalize the definition of $ \T $ for spin $ s = 1/2 $ particles to
	\begin{equation*}
		\T = i \sigma_{y} K T \quad \text{where }  \sigma_{y} = 
		\begin{pmatrix}
			0 & - i\\
			i & 0
		\end{pmatrix}
	\end{equation*}
	We will discuss spin and time reversal more thoroughly in a dedicated chapter. 
	% TODO add link
	
	\item Finally, we note that position doesn't change sign under $ \T $ reversal, i.e. $ \T x = x \T $, as opposed to momentum, which obeys $ \T p = - p \T $. These to identities imply
	\begin{equation*}
		\T[x, p] = - [x, p]\T
	\end{equation*}
	For the fundamental commutator relationship $ [x, p] = i\hbar $ to remain invariant under $ \T $ reversal, $ \T $ must obey $ \T i = i \T $, i.e. $ \T $ must be an anti-unitary operator.
	
\end{itemize}

\subsection{Gauge Transformations}
% Corresponds to wavefunction invariance under phase change
\begin{itemize}
	\item We have already noted a few times in this text that multiplying a wavefunction by a phase factor $ e^{i\delta} $ of magnitude one has no physically observable effect on the wavefunction. 
	
	Multiplying a wavefunction by $ e^{i\delta} $ is a case of a so-called global gauge transformation, which is a unitary transformation of the form
	\begin{equation*}
		U(\delta)\ket{\psi} = e^{i\delta}\ket{\p} \equiv \bket{\tilde{\p}}
	\end{equation*}
	If we apply the transformation $ U(\delta) = e^{i\delta} $ to all basis functions $ \{\ket{n}\} $ spanning the Hilbert space of wavefunctions, then all matrix elements of an arbitrary operator $ \O $ remain unchanged, i.e. 
	\begin{equation*}
		\bmel{U(\delta) \phi}{\O}{U(\delta)\p} = \bmel{\t{\phi}}{\O}{\t{\p}} = \bmel{\phi}{\O}{\p} 
	\end{equation*}
	Even more, we can multiply each basis vector $ \ket{n} $ by an individual factor $ e^{i \delta_{n}} $, and all physical observable remain unchanged.
	
	\item Recall that in classical mechanics potential energy is determined up to an additive constant $ V_{0} $, i.e we can make the transformation $ V(\r) \to V(\r) + V_{0} $ without changing a system's equations of motion. This follows from the relationship between force and potential energy $ \vec{F} = - \grad\big[V(\r) + V_{0}\big] = - \grad V(\r) $ is unchanged by $ V_{0} $.
	
	Meanwhile, in quantum mechanics, a the transformation $ V(\r) \to \to V(\r) + V_{0} $ shifts a system's energy eigenvalues by $ V_{0} $, i.e. $ E_{n} \to E_{n} + V_{0} $. In this case, the time evolution operator changes according to
	\begin{equation*}
		e^{-i\frac{E}{\hbar}t}\k{\p} \to e^{-i\frac{E+V_{0}}{\hbar}t}\k{\p} = e^{-i\frac{E}{\hbar}t} e^{-i\frac{V_{0}}{\hbar}t}  \k{\p} 
	\end{equation*}
	We define the corresponding global gauge transformation as
	\begin{equation*}
		U(\delta(t)) \equiv e^{-i\frac{V_{0}}{\hbar}t} \qquad \text{ where } \delta(t) = -\frac{V_{0}}{\hbar}t
	\end{equation*}
	
	\item We can also define a so-called local gauge transformation
	\begin{equation*}
		U(\delta(\r, t)) \ket{\P(\r, t)} = e^{i\delta(\r, t)}\ket{\P(\r, t)} \equiv \bket{\t{\P}(\r, t)}
	\end{equation*}
	This gauge transformation preserves probability density, i.e.
	\begin{equation*}
		\big|\t{\P}(\r, t) \big|^{2} = \abs{\P(\r, t)}^{2}
	\end{equation*}
	We will return to local gauge transformations when discussing a particle in an electromagnetic field. 
\end{itemize}


\newpage
\section{Angular Momentum}
\begin{itemize}
	\item Angular momentum is associated with the angular momentum operator $ \vec{L} $. The operator $ \vec{L} $ is \Herm (so angular momentum is a physically observable quantity) and obeys
	\begin{equation*}
		\L = \r \cross \vec{p} = - \vec{p} \cross \r
	\end{equation*}
	This relationship is proven with the help of
	\begin{align*}
		(\curl \r)\p &= \curl (\r \p) = (\grad \p) \cross \r + \p (\curl \r) = (\grad \p)\cross \r + 0\\
		& = - \r \cross (\grad \p) = - (\r \cross \grad)\p
	\end{align*}
	We could also prove the relationship by components, e.g.
	\begin{equation*}
		L_{z}\p = - i\hbar(y p_{x} - xp_{y}) \p = i\hbar(p_{y}x - p_{x}y)\p
	\end{equation*}
	We would proceed analogously for $ L_{x} = yp_{z} - zp_{y} $ and $ L_{y} = zp_{x} - x p_{z} $. 
	
	\item As a side note, in quantum mechanics the dot and cross products of non-commutative operators do not commute as in classical mechanics. Consider for example
	\begin{align*}
		&\r \cdot \vec{p} = \vec{p} \cdot \r + 3i\hbar \qquad \vec{p} \cross \L = - \L \cross \vec{p} + 2i\hbar \vec{p} \\
		& (\r \cross \L)\cdot (\r \cross \L) = \r^{2}\vec{p}^{2} - (\r \cdot \vec{p})^{2} + i \hbar (\r \cdot \vec{p})
	\end{align*}
	
	\item The components of angular momentum $ L_{x}, L_{y} $ and $ L_{z} $ obey analogous commutator relations to the Poisson bracket relations obeyed by angular momentum in classical mechanics, e.g.
	\begin{equation*}
		[L_{x}, L_{y}] = i\hbar(xp_{y} - yp_{x}) = i\hbar L_{z} \eqtext{and} [L^{2}, L_{\alpha}] = 0, \quad \alpha = x, y, z
	\end{equation*}
	We can prove the first relationship with
	\begin{align*}
		[L_{x}, L_{y}] &= [yp_{z} - zp_{y}, zp_{x} - xp_{z}]\\
		& = [yp_{z}, zp_{x}] - [zp_{y}, zp_{z}] - [yp_{z}, xp_{z}] + [zp_{y}, xp_{z}]
	\end{align*}
	The middle two commutators are zero, since both sides contain identical terms $ z $ and $ p_{z} $, respectively. We expand the remaining commutators using $ [AB, C] = A[B, C] + [A, C]B $ and $ [A, BC] = B[A, C] + [A, B]C $ and apply the canonical commutator relations $ [r_{\alpha}, r_{\beta}] = [p_{\alpha}, p_{\beta}] = 0 $ and $ [r_{\alpha}, p_{\beta}] = i\hbar \delta_{\alpha \beta} $ to get
	\begin{align*}
		[L_{x}, L_{y}] &= yp_{x}[p_{z}, z] + xp_{y}[z, p_{z}] = i\hbar (yp_{x} + xp_{y}) = i\hbar L_{z}
	\end{align*}

	
	\item More generally, angular momentum obeys the commutator relation
	\begin{equation*}
		[L_{\alpha}, \O_{\beta}] = i \hbar \epsilon_{\alpha \beta \gamma}\O_{\gamma}
	\end{equation*}
	where the operator $ \O $ can be any of $ \r $, $\vec{p}$ or $ \L $.
	
\end{itemize}

\subsection{Properties of the Angular Momentum Operator}

\subsubsection{Commutation Relations}
\begin{itemize}
	\item Essentially all properties of the angular momentum operator arise from the commutator relation
	\begin{equation*}
		\big[L_{\alpha}, L_{\beta}\big] = i \hbar \epsilon_{\alpha \beta \gamma}L_{\gamma} \eqtext{or, in vector form,} \L \cross \L = i \hbar \L
	\end{equation*}
	
	\item If an operator $ \O $ commutes with $ \L $, then $ \L $ and $ \O $ have mutual eigenvectors, which can be exploited to simplify quantum mechanical problems.  
	
	In general, any operator $ \O $ that is invariant under rotation (i.e. for which $ U(\vec{\phi})\O = \O U(\vec{\phi})  $) commutes with the angular momentum operator. In symbols,
	\begin{equation*}
		U(\vec{\phi})\O = \O U(\vec{\phi})  \implies [\L, \O] = 0
	\end{equation*}
	where $ U(\phi) $ is the operator encoding rotation about the axis $ \uvec{n} $ by the angle $ \phi $. Applicable operators invariant under rotation include $ \r \cdot \vec{p} $, $ \vec{p}^{2} $, $ \L^{2} $ and rotationally invariant potentials $ V = V(\abs{\r}) $. 
	
	\item \textbf{TODO} we proof the above commutation relation with the series definition of $ U(\vec{\phi}) $. 
	
	\item We often work in terms of the squared angular momentum $ \L^{2} $, which we can write in any of the equivalent forms\footnote{Note that we have previously used $ L^{2} $ in this text to denote the Hilbert space of wavefunctions. The difference between the square of angular momentum and the Hilbert space should be clear from context.}
	\begin{equation*}
		\L^{2} = \L \cdot \L = L^{2} = \sum_{\alpha} L_{\alpha}^{2}
	\end{equation*}
	Because $ \L^{2} $ is invariant under rotations, we have
	\begin{equation*}
		[L_{\alpha}, L^{2}] = 0 \quad \text{for } \alpha \in \{x, y, z\}
	\end{equation*}
	which means that squared angular momentum $ L^{2} $ and its components $ L_{\alpha} $ can share the same eigenvectors and basis.
	
\end{itemize}

\subsubsection{The Ladder Operators}
\begin{itemize}
	\item We often analyze angular momentum problems in terms of the ladder operators $ L_{+} $ and $ L_{-} $, defined by
	\begin{equation*}
		L_{+} \equiv L_{x} + iL_{y} \eqtext{and} L_{-} \equiv L_{x} - i L_{y}
	\end{equation*}
	The ladder operators obey $ L_{\pm} = L_{\mp}^{\dagger} $, i.e. they are each other's \Herm conjugates.
	
	\item The ladder operators commute with the squared angular momentum operator, i.e.
	\begin{equation*}
		[L^{2}, L_{+}] = [L^{2}, L_{-}] = 0
	\end{equation*}
	
	\item Analogy to the quantum harmonic oscillator: the ladder operators $ L_{+} $ and $ L_{-} $ are analogous to the creation and annihilation operators $ a^{\dagger} $  and $ a $, i.e. they ``raise'' and ``lower'' the indexes of angular momentum basis states, just like $ a^{\dagger} $ and $ a $ raise and lower the indexes of the harmonic oscillator's Hamiltonian's basis states. 
	
	The operator $ L_{z} $ is analogous to the number operator $ \hat{n} = a^{\dagger}a $, in that it counts the number of angular momentum quanta in an angular basis state. 
	
	\item First, we derive the important commutation relation $ [L_{z}, L_{\pm}] = \pm \hbar L_{\pm} $ via
	\begin{align*}
		[L_{z}, L_{\pm}] &\equiv [L_{z}, L_{x} \pm i L_{y}] = [L_{z}, L_{x}] \pm i [L_{z}, L_{y}] = i \hbar L_{y} \pm i(-i \hbar L_{x})\\
		& = \hbar [\pm L_{x} + iL_{y}] = \pm \hbar L_{\pm}
	\end{align*}
	where we have used $ [L_{\alpha}, L_{\beta}] = i \hbar \epsilon_{\alpha \beta \gamma} L_{\gamma} $.
	
	\item Second, we derive the equally important commutation relation $ [L_{+}, L_{-}] = 2\hbar L_{z} $. To show this, we start with
	\begin{align*}
 		L_{\pm}L_{\mp} &= (L_{x} \pm i L_{y})(L_{x} \mp i L_{y}) = L_{x}^{2} + L_{y}^{2} \pm i L_{y}L_{x} \mp i L_{x}L_{y} \\
 		& = L^{2} - L_{z}^{2} \pm i L_{y}L_{x} \mp i L_{x}L_{y} 
	\end{align*}
	Next, we use $ [L_{\alpha}, L_{\beta}] = i \hbar \epsilon_{\alpha \beta \gamma} L_{\gamma}  $ to show 
	\begin{equation*}
		\pm i L_{y}L_{x} \mp i L_{x}L_{y} = \pm \hbar L_{z}
	\end{equation*}
	which leaves the two equations (two because of the $ \pm $ terms)
	\begin{equation*}
		L_{\pm}L_{\mp} = L^{2} - L_{z}^{2} \pm \hbar L_{z}
	\end{equation*}
	Finally we subtract the equations with plus and minus to get
	\begin{equation*}
		L_{+}L_{-} - L_{-}{+}  \equiv [L_{+}, L_{-}] = 2 \hbar L_{z}
	\end{equation*}
	
\end{itemize}

\subsection{Eigenvalues and Eigenfunctions of $ L_{z} $ and $ L^{2} $}
We will now use the just-derived relations $ [L_{z}, L_{\pm}] = \pm \hbar L_{\pm} $ and $ [L_{+}, L_{-}] = 2 \hbar L_{z} $ to find the eigenvalues of the operators $ L_{z} $ and $ L^{2} $. 

\subsubsection{Eigenvalues of $ L_{z} $}
\begin{itemize}
	
	\item Let $ \ket{m} $ be an eigenstate of $ L_{z} $, meaning $ \ket{m} $ satisfies the eigenvalue relation
	\begin{equation*}
		L_{z}\ket{m} = m \hbar \ket{m}
	\end{equation*}
	\textit{Note}: although the complete corresponding eigenvalue is $ m\hbar $ with units of angular momentum; we often refer to eigenvalues of $ L_{z} $ solely in terms of the dimensionless index $ m $ and leave the $ \hbar $ implicit.
	
	\item Next, using $ [L_{z}, L_{\pm}] = \pm \hbar L_{\pm} $ in the form $ L_{z}L_{\pm} = L_{\pm}L_{z} \pm \hbar L_{z} $, the operator $ L_{z}L_{\pm} $ acts on $ \ket{m} $ as
	\begin{align*}
		L_{z}L_{\pm} \ket{m} &= (L_{\pm}L_{z} \pm \hbar L_{z})\ket{m} = L_{\pm} L_{z}\ket{m} \pm \hbar L_{z}\ket{m}\\
		& = L_{\pm} \hbar \ket{m} \pm \hbar L_{z}\ket{m}\\
		& = (m \pm 1)\hbar L_{\pm}\ket{m}
	\end{align*}
	Since $ L_{z} $ acts on the state $ L_{\pm}\ket{m} $,  to produce $ (m \pm 1)\hbar L_{\pm}\ket{m} $, i.e. the same state with eigenvalue $ (m \pm 1)\hbar $, it follows that $ \L_{+} $ and $ L_{-} $ raise and lower the index $ m $ of the state $ \ket{m} $ by one, i.e. 
	\begin{equation*}
		L_{\pm} \ket{m} \propto \ket{m+1}
	\end{equation*}
	We will determine the exact relationship in the following sections.
	
	\item Because $ L_{z} $ and $ L^{2} $ commute, the eigenstate $ \ket{m} $ of $ L_{z} $ is also an eigenstate of $ \ket{L^{2}} $. 
	
	We now show that the eigenvalues of $ L^{2} $ cannot be negative:
	\begin{equation*}
		L^{2}\ket{m} = \lambda \ket{m} \implies \mel{m}{L^{2}}{m} = \sum_{\alpha} \braket{L_{\alpha}m}{L_{\alpha}m} = \lambda \braket{m}{m}
	\end{equation*}
	Since the quantities $ \braket{L_{\alpha}m}{L_{\alpha}m} $ and $ \braket{m}{m} $ are non-negative, the equality holds only if $ \lambda \geq 0 $, meaning the eigenvalues of $ L^{2} $ are non-negative.
	
	\item Next, we use the commutator relation $ [L^{2}, L_{\pm}] = 0 \implies L^{2} L_{\pm} = L_{\pm} L^{2}  $ and the eigenvalue relation $ L^{2}\ket{m} = \lambda \ket{m} $ to show
	\begin{equation*}
		L^{2} L_{\pm}\ket{m} = L_{\pm} L^{2} \ket{m} = L_{\pm}\lambda \ket{m} = \lambda L_{\pm} \ket{m}
	\end{equation*}
	In other words, the state $ L_{\pm} \ket{m} $ is also an eigenvalue of $ L^{2} $ with the same eigenvalue $ \lambda $. But we know from the previous bullet that $ L_{\pm} $ raises or lowers the index of $ \ket{m} $ by one, i.e. $ L_{\pm} \ket{m} \propto \ket{m \pm 1} $, which means that (under the action of $ L^{2} $) states $ \ket{m \pm 1} $ have the same eigenvalue as $ \ket{m} $.
	
	\item Next, we introduce the orbital quantum number\footnote{Note that $ l $ is also called the azimuthal quantum number.} $ l > 0 $, which is related to the $ L^{2} $ eigenvalue $ \lambda $ by $ \lambda = l(l+1)\hbar^{2} $. The motivation for this apparently strange parameterization of $ \lambda $ will quickly be clear.
	
	We use $ L_{\pm}L_{\mp} = L^{2} - L_{z}^{2} \pm \hbar L_{z} $ and $ L_{\pm} = L_{\mp}^{\dagger} $ to show
	\begin{equation*}
		\braket{L_{\pm}m}{L_{\pm}m} = \mel{m}{L_{\mp}L_{\pm}}{m} = \mel{m}{(L^{2} - L_{z}^{2} \mp \hbar L_{z})}{m}\\
	\end{equation*}
	Next, we apply the eigenvalue relations for $ L^{2} $ and $ L_{z} $ to get
	\begin{align*}
		\braket{L_{\pm}m}{L_{\pm}m} &= \bmel{m}{\big[l (l+1) - m(m\pm 1)\big]\hbar^{2}}{m}\\
		&\equiv (C_{l,m\pm 1})^{2} \braket{m}{m}
	\end{align*}
	Where we have defined the constant
	\begin{equation*}
		C_{l, m\pm1} = \hbar \sqrt{l (l+1) - m(m\pm 1)} \in \mathbb{R}
	\end{equation*}
	Since both $ \braket{L_{\pm}m}{L_{\pm}m} $ and $ \braket{m}{m} $ are both non-negative, it follows that $ (C_{l,m\pm 1})^{2} \geq 0 $, which is why $ C_{l, m\pm1} $ is real.
	
	\item The identity $  C_{l, m\pm1} \in \mathbb{R} $ is important---it means that for a state with a given orbital quantum number $ l $, we can raise or lower states with $ L_{\pm} $ only as long as $  C_{l, m\pm1} $ remains real, which implies
	\begin{equation*}
		l (l+1) \geq m(m\pm 1) \implies \abs{m} \leq l
	\end{equation*}
	
\end{itemize}

\subsubsection{Eigenvalues of $ L^{2} $}
\begin{itemize}
	\item Using $ \abs{m} \leq l $, we can now find the possible values of the orbital quantum number $ l $ and thus determine $ L^{2} $'s eigenvalues $ \lambda = l (l+1)\hbar^{2} $. 
	
	\item First, we consider a generic $ L^{2} $ eigenstate $ \ket{lm} $ indexed by both $ m $ and $ l $. We start with the maximum permitted value of $ m $, i.e. $ m = l $, and act on the state $ \ket{ll} $ with $ L_{-} $ until we reach the minimum possible value $ m = -l $. This reads
	\begin{equation*}
		\begin{array}{lcl}
			L_{-}\k{ll}     & = & C_{l,l-1}\k{l, l-1}\\
			L^{2}_{-}\k{ll} & = & C_{l,l-2}\k{l, l-2}\\[-0.3em]
			& \vdots &\\[-0.3em]
			L^{k}_{-}\k{ll} & = & C_{l,l-k}\k{l, l-k} =  C_{l,l-k}\k{l, -l}\\
		\end{array}
	\end{equation*}
	Since we reached the state with $ m = -l $ after $ k \in \mathbb{N}$ integral steps, we have
	\begin{equation*}
		l - k = - l \implies 2l = k \implies 2l \in \mathbb{N}
	\end{equation*}
	The possible values of $ l $, (accounting for $ l \geq 0 $), are thus
	\begin{equation*}
		l = 
		\begin{cases}
			0, 1, 2, \ldots & k \text{ even}\\
			\frac{1}{2}, \frac{3}{2}, \frac{5}{2}, \ldots & k \text{ odd}
		\end{cases}
	\end{equation*}
	More so, the unit increments of $ l $ mean the general condition $ \abs{m} \leq l $ can be written in the form
	\begin{equation*}
		m \in \big\{\hspace{-0.3em}-l, -l+1, \ldots, l-1, l\big\}
	\end{equation*}
	
	\item Since $ m $ can take on $ 2l + 1 $ values at a given $ l $, $ L^{2} $'s eigenvalue spectrum has degeneracy $ 2l + 1 $, since at a given $ l $ there are $ 2l + 1 $ linearly independent eigenstates $ \ket{lm} $ with the same eigenvalue $ \lambda = l (l+1)\hbar^{2} $. 
	
	\item The recursive action of $ L_{-} $ on $ \ket{lm} $ also reveals the relationship
	\begin{equation*}
		L_{\pm}\ket{lm} = C_{l, m \pm 1} \ket{l, m \pm 1} = \hbar \sqrt{l (l+1) - m(m\pm 1)} \ket{l, m \pm 1}
	\end{equation*}
	Earlier, we had determined this relationship only to $ L_{\pm} \ket{m} \propto \ket{m \pm 1} $.
	
	\item \textbf{TODO:} Only integer or half-integer values of $ l $ are possible because (similar to the harmonic oscillator) non-integer values of $ l $  would allow us to repeatedly lower indices with $ L_{-} $ until $ C_{l, m-1} $ were a imaginary number, which is prohibited.
\end{itemize}


\subsubsection{Eigenfunctions of $ L_{z} $}
\begin{itemize}
	\item In spherical coordinates, the coordinate representation of the operator $ L_{z} $ reads
	\begin{equation*}
		L_{z} = - i \hbar \pdv{\phi}
	\end{equation*}
	where $ \phi $ is the azimuthal angle. For each $ m $, the eigenvalues equation
	\begin{equation*}
		L_{z} \p_{m} = \left(-i\hbar\pdv{\phi} \right)\p_{m} = m \hbar \p_{m}
	\end{equation*}
	has the unique solution
	\begin{equation*}
		\p_{m} = Ce^{im \phi}
	\end{equation*}
	
	\item We consider only $ \p_{m} $ solving the \Schro equation, which must be continuous. To satisfy continuity, the $ \p_{m} $ must be periodic over $ \phi \in [0, 2\pi] $, i.e.
	\begin{equation*}
		\p_{m}(\phi) = \p_{m}(\phi + 2\pi) \iff 1 = e^{2\pi i m}  \implies m \in \mathbb{Z}
	\end{equation*}
	In other words, only integer values of $ m $ satisfy the \Schro equation and correspond to physical eigenstates of $ L_{z} $.
	
\end{itemize}

\subsubsection{Eigenfunctions of $ L^{2} $}
\begin{itemize}
	\item Without derivation, the eigenfunctions of the angular momentum operator $ L^{2} $, in the coordinate representation, are the spherical harmonics, i.e.
	\begin{equation*}
		\braket{\r}{lm} = Y_{l}^{m}(\theta, \phi) 
	\end{equation*}
	The spherical harmonics arise in the angular solution of the Laplace equation $ \laplacian u(\r) = 0 $, i.e. if we separate $ u(\r) $ into radial and angular component, the solution is
	\begin{equation*}
		\laplacian u(\r) = f(r)Y_{l}^{m}(\theta, \phi) = 0
	\end{equation*}
	where $ Y_{l}^{m}(\theta, \phi)  $ are the spherical harmonics. 
	
	\item In quantum mechanics for $ m \geq 0  $ we often use the definition
	\begin{equation*}
		Y_{l}^{m}(\theta, \phi) = (-1)^{m}\sqrt{\frac{(2l+1)}{4\pi}\frac{(l-m)!}{(l+m)!}} P_{l}^{m}(\cos \theta)e^{im\phi}
	\end{equation*}
	where $ P_{l}^{m} $ are the associated Legendre polynomials.
	
	\item The spherical harmonics obey 
	\begin{equation*}
		Y_{l}^{-m} = (-1)^{m}Y_{l}^{m^{*}}
	\end{equation*}
	
	\item As a concrete example, the first few spherical harmonics for for $ l = 0, 1, 2 $ are
	\[
		\begin{array}{ll}
			Y_{0}^{0} = \frac{1}{\sqrt{4\pi}} &\\
			Y_{1}^{0} = \sqrt{\frac{3}{4\pi}}\cos \theta & Y_{l}^{\pm 1} = \mp \sqrt{\frac{3}{8\pi}}\sin \theta e^{\pm i \phi}\\
			Y_{2}^{0} = \sqrt{\frac{5}{16\pi}}  (3\cos^{2}\theta - 1) & Y_{2}^{\pm 1} = \mp \sqrt{\frac{15}{8\pi}}\sin \theta \cos \theta e^{\pm i \phi}\\
			Y_{2}^{\pm 2} = \sqrt{\frac{5}{32\pi}}\sin^{2}\theta e^{\pm 2i \phi}
		\end{array}
	\]
\end{itemize}

\subsubsection{Matrix Representation of Angular Momentum}
\begin{itemize}
	\item We write a generic state $ \ket{\p} $ in the basis $ \{\ket{lm}\} $ of angular momentum eigenfunctions as
	\begin{equation*}
		\k{\p} = \sum_{l = 0}^{\infty}\sum_{m=-l}^{l}c_{lm}\ket{lm}
	\end{equation*}
	Note that the presence of two quantum numbers $ l $ and $ m $ introduces a double sum.
	
	\item We write a generic operator $ \O $ in the $ \k{lm} $ basis as
	\begin{equation*}
		\O = \sum_{l'lm'm}\k{l'm'}\O_{l'lm'm}\bra{lm}
	\end{equation*}
	
	\iffalse
	
	\item As an example, we find the matrix representation of the ladder operator $ L_{+} $, i.e.
	\begin{equation*}
		L_{+} = \sum_{ll'mm'}\k{l'm'}L_{+_{l'lm'm}}\bra{lm}
	\end{equation*}
	The matrix element $ L_{+_{l'lm'm}} $ reads
	\begin{align*}
		L_{+_{l'lm'm}} &= \mel{l'm'}{L_{+}}{lm}\\
		& = \hbar \b{l, m+1}\sqrt{l (l+1) - m(m+1)}\k{l, m+1}\delta_{l'l}\delta_{m'+1,m}
	\end{align*}
	Because of the $ \delta_{l'l} $ matrix is block diagonal with respect to $ l $, and reads
	\begin{equation*}
	\begingroup
	\setlength\arraycolsep{1.0pt}
	\renewcommand*{\arraystretch}{0}
		L_{+} \to \hbar
		\begin{pmatrix}
		 \t{L}_{+}^{(0)}   & & & & \\
		 & \t{L}_{+}^{(1)}   & & & \\[-0.4em]
		 & & \ddots           & & \\
		 & & & \t{L}_{+}^{(l)}   & \\[-0.4em]
		 & & & & \ddots         & \\
		\end{pmatrix}
	\endgroup
	\end{equation*}
	where $ \t{L}_{+}^{(l)} $ is a $ l(l+1) \cross l(l+1) $ is an off-diagonal matrix of the form: 
	
	Blah stupid matrices in latex forget it lol :D
	
	\fi
	
	\item Finally, as a concrete example, for $ l = 1 $ the matrices for $ L_{x, y, z} $ and $ L^{2} $ read
	\begin{align*}
		& L_{x} = \frac{\hbar}{\sqrt{2}} 
		\begin{pmatrix}
			0 & 1 & 0\\
			1 & 0 & 1\\
			0 & 1 & 0
		\end{pmatrix}
		&&
		L_{y} = \frac{\hbar}{\sqrt{2}} 
		\begin{pmatrix}
			0 & -i & 0 \\
			i & 0 & -i \\
			0 & i & 0
		\end{pmatrix}\\
		& L_{z} = \hbar
		\begin{pmatrix}
			1 & 0 & 0\\
			0 & 0 & 0\\
			0 & 0 & -1
		\end{pmatrix}
		&&
		L^{2} = 2\hbar^{2}
		\begin{pmatrix}
			1 & 0 & 0\\
			0 & 1 & 0\\
			0 & 0 & 1
		\end{pmatrix}
	\end{align*}
	As would be expected, $ L^{2} $ and $ L_{z} $ are diagonal in the $ \ket{lm} $ basis.
	
\end{itemize}


\newpage
\section{Central Potential}
\begin{itemize}
	\item We consider a particle in the time-independent central potential $ V = V(r) $ with \Ham 
	\begin{equation*}
		H = \frac{p^{2}}{2m} + V(r) = -\frac{\hbar^{2}}{2m} \laplacian + V(r)
	\end{equation*}
	The particle's angular momentum $ \L $ and magnitude of angular momentum $ L^{2} $ are conserved, i.e.
	\begin{equation*}
		[\L, H] = [L^{2}, H] = 0
	\end{equation*}
	Second, the we state the relations
	\begin{equation*}
		\r \cdot \L = 0 \eqtext{and} \vec{p} \cdot \L = 0
	\end{equation*}
	These two equations are the quantum mechanical analog of a particle's motion and velocity lying in a two-dimensional plane in central force motion.
	
\end{itemize}

\subsection{The Radial Equation}
\begin{itemize}
	\item We analyze rotationally-symmetric central potential problems in spherical coordinates, where the Laplace operator reads
	\begin{equation*}
		\laplacian = \frac{1}{r^{2}}\pdv{r}r^{2}\pdv{r} + \frac{1}{r^{2}\sin \theta}\pdv{\theta}\sin \theta \pdv{\theta} + \frac{1}{r^{2}\sin^{2}\theta} \pdv[2]{}{\phi}
	\end{equation*}
	Without proof, the Laplace operator's angular component is related to angular momentum $ L^{2} $ via
	\begin{equation*}
		 \frac{1}{r^{2}\sin \theta}\pdv{\theta}\sin \theta \pdv{\theta} + \frac{1}{r^{2}\sin^{2}\theta} \pdv[2]{}{\phi} = -\frac{L^{2}}{\hbar^{2}r^{2}}
	\end{equation*}
	The Laplacian can thus be written
	\begin{equation*}
		\laplacian = \frac{1}{r^{2}}\pdv{r}r^{2}\pdv{r} -\frac{L^{2}}{\hbar^{2}r^{2}} 
	\end{equation*} 
	where the angular component $ \frac{L^{2}}{\hbar^{2}r^{2}} $ corresponds to rotational kinetic energy.
	
	\item We can compose the \Ham into a radial and angular component:
	\begin{equation*}
		H = -\frac{\hbar^{2}}{2m}\laplacian + V(r) = - \frac{h^{2}}{2m} \left(\frac{1}{r^{2}} \pdv{r}r^{2} \pdv{r}\right) + \frac{L^{2}}{2mr^{2}} + V(r)
	\end{equation*}
	Next, to solve the stationary \Schro equation
	\begin{equation*}
		H\P(\r) = E\P(\r),
	\end{equation*}
	we use the ansatz
	\begin{equation*}
		\P(\r) = \p(r)Y_{l}^{m}(\theta, \phi)
	\end{equation*}
	where we have separated $ \P(\r) $ into a radial and angular component. The spherical harmonics $ Y_{l}^{m}(\theta, \phi) $ are a natural choice for the angular component because they are the eigenfunctions of the angular momentum operator $ L^{2} $.
	
	\item Substituting the ansatz $ \P(\r) = \p(r)Y_{l}^{m}(\theta, \phi) $ into the stationary \Schro equation, applying the angular momentum eigenvalue relation
	\begin{equation*}
		L^{2} Y_{l}^{m} = l (l+1)\hbar^{2} Y_{l}^{m}
	\end{equation*}
	and canceling $ Y_{l}^{m} $ from both sides of the equations produces the purely radial problem
	\begin{equation*}
		- \frac{h^{2}}{2m} \left(\frac{1}{r^{2}} \pdv{r}r^{2} \pdv{r}\right)\p(r) + \left(V(r) + \frac{l (l+1)\hbar^{2}}{2mr^{2}}\right)\p(r) = E\p(r)
	\end{equation*}
	Note that all angular dependence is gone---we have completed an important step towards finding the complete eigenfunction $ \P(\r) $.
	
	\item We solve for the radial eigenfunction $ \p(r) $ with ansatz
	\begin{equation*}
		\p(\r) = \frac{u(r)}{r}
	\end{equation*}
	We then substitute this ansatz into radial eigenvalue equation. First, as a intermediate step, we calculate
	\begin{equation*}
		\frac{1}{r^{2}}\pdv{r}r^{2}\pdv{r} \left(\frac{u}{r}\right) = \frac{1}{r^{2}}\pdv{r}r^{2}\left(\frac{u'}{r} - \frac{u}{r^{2}}\right) = \frac{1}{r^{2}}\pdv{r}(ru' - u) = \frac{u''}{r}
	\end{equation*}
	Using this intermediate result, the radial eigenvalue equation in terms of $ u $ reads
	\begin{equation*}
		-\frac{\hbar^{2}}{2m}u''(r) + \left[V(r) + \frac{l (l+1)\hbar^{2}}{2mr^{2}}\right]u(r) = Eu(r)
	\end{equation*}
	
	\item Finally, we define an effective potential 
	\begin{equation*}
		V_{\text{eff}}(r) = V(r) + \frac{l (l+1)\hbar^{2}}{2mr^{2}}
	\end{equation*}
	which includes the potential $ V(r) $ in additional to the ``centrifugal'' term $ \frac{l (l+1)\hbar^{2}}{2mr^{2}} $.
	
	
	In terms of $ V_{\text{veff}} $, the stationary \Schro equation for $ u $ reads
	\begin{equation*}
	-\frac{\hbar^{2}}{2m}u''(r) + V_{\text{eff}}(r)u(r) = Eu(r)
	\end{equation*}
	Note that we have reduced originally three-dimensional problem, involving $ \r = (r, \phi, \theta)  $ to a one-dimensional problem.
	
\end{itemize}

\subsection{General Properties of the Radial Solution}
\subsubsection{Solutions for $ r \to 0 $}
\begin{itemize}
	\item We consider potentials of the form
	\begin{equation*}
		\lim_{r \to 0}r^{2}V(r) = 0
	\end{equation*}
	For such potentials, the potential $ V(r) $ and energy $ E $ in the radial eigenvalue equation for $ u $ are negligible in comparison to the centrifugal component of $ V_{\text{eff}} $, and the eigenvalue equation simplifies to
	\begin{equation*}
		-\frac{\hbar^{2}}{2m}u''(r) + \frac{l (l+1)\hbar^{2}}{2mr^{2}}u(r) = 0 \implies u''(r) = \frac{l (l+1)}{r^{2}}u(r)
	\end{equation*}
	We solve the equation with the ansatz $ u(r) = Cr^{\lambda} $, which produces
	\begin{align*}
		\lambda (\lambda - 1)Cr^{\lambda-2} = \frac{l (l+1)}{r^{2}}Cr^{\lambda} \implies 	\lambda (\lambda - 1) = l (l+1)
	\end{align*}
	
	\item We solve the equation $ \lambda (\lambda - 1) = l (l+1) $ with the quadratic formula:
	\begin{equation*}
		\lambda_{\pm} = \frac{1}{2} \pm \frac{1}{2}\sqrt{1 + 4l (l+1)} = \frac{1}{2} \pm \frac{1}{2}\sqrt{(2l+1)^{2}}  = \frac{1}{2} \pm \left(l + \frac{1}{2}\right)
	\end{equation*}
	The two possible values of $ \lambda $ are
	\begin{equation*}
		\lambda_{+} = l + 1 \eqtext{and} \lambda_{-} = -l
	\end{equation*}
	The general solution to the second-order linear eigenvalue equation is thus the linear combination
	\begin{equation*}
		u(r) = C_{+}r^{\lambda_{+}} + C_{-}r^{\lambda_{-}} = C_{l}r^{l+1} + \frac{D_{l}}{r^{l}}
	\end{equation*}
	
	\item We determine the constants $ C_{l} $ and $ D_{l} $ from boundary and normalization conditions. 
	
	We start all the way back at the normalization condition on $ \P(\r) $, which, when integrating in spherical coordinates, reads
	\begin{equation*}
		1 \equiv \braket{\P}{\P} = \int_{r = 0}^{\infty}\abs{\p(r)}^{2}r^{2}\diff r \int_{\phi = 0}^{2\pi}\int_{\theta = 0}^{\pi}\abs{Y_{l}^{m}(\theta, \phi)}^{2}\sin \theta \diff \theta \diff \phi 
	\end{equation*}
	From the angular momentum chapter, we know the spherical harmonics are normalized. The integral's angular component thus evaluates to one, which implies
	\begin{equation*}
		\int_{ 0}^{\infty}\abs{\p(r)}^{2}r^{2}\diff r = \int_{0}^{\infty}\abs{u(r)}^{2}\diff r \equiv 1
	\end{equation*}
	This normalization condition on $ u $ requires $ D_{l} = 0 $ for $ l > 0 $, since the integral of $ \abs{u(r)}^{2} $ would otherwise diverge at $ 0 $. 
	
	\item \textbf{TODO:} resolve what's going on here.
	
	For $ l = 0 $, we turn to the solution of the Poisson equation for electrostatic potential $ \phi $, which reads
	\begin{equation*}
		\laplacian \phi(\r) = -\frac{\rho(\r)}{\epsilon_{0}}
	\end{equation*}
	For a point charge, with charge density $ \rho(\r) = q \delta^{3}(\r) $, the solution is 
	\begin{equation*}
		\phi(\r) = \frac{q}{4\pi \epsilon_{0}r}
	\end{equation*}
	If we cancel like terms, the Poisson equation reads
	\begin{equation*}
		\laplacian \frac{1}{4\pi \epsilon_{0}r} = \delta^{3}(\r)
	\end{equation*}
	
	\textbf{TODO:} This step here I'm not sure about. The above seems to imply that $ \p(\r) $ solves the equation
	\begin{equation*}
		\laplacian \p(\r) = - 4 \pi D_{0} \delta^{3}(\r)
	\end{equation*}
	But a solution $ \p \sim r^{-1} $ does not solve the \Schro equation, because the potential does not agree with $ \delta^{3}(\r) $ at the origin. 
	
	This means that, to satisfy the \Schro equation, we must have $ D_{l} = 0 $ for all $ l $ (to remove proportionality of $ \psi $ to $ \delta^{3}(\r) $ at the origin). With $ D_{l} = 0 $ we're left with 
	\begin{equation*}
		u(r) = C_{l}r^{l+1} \eqtext{and} \p(r) = C_{l}r^{l}
	\end{equation*}
	
\end{itemize}

\subsubsection{Solutions for $ r \to \infty $}

\textbf{Continuum States with $ E > 0 $}
\begin{itemize}
	\item We consider potentials $ V(r) $ for which 
	\begin{equation*}
		\lim_{r \to \infty} V(r) = 0
	\end{equation*}
	More so, we assume there exists some finite distance $ r_{0} \in \mathbb{R} $ such that the potential $ V(r) $ is negligible for $ r > r_{0} $. 
	
	\item If we also neglect the centrifugal term, the entire effective potential $ V_{\text{eff}} $ vanishes for $ r > r_{0} $. The radial eigenvalue equation reduces to
	\begin{equation*}
		- \frac{\hbar^{2}}{2m}u''(r) = Eu(r)
	\end{equation*}
	The solution to this problem is
	\begin{equation*}
		u(r) = C_{-}e^{-i k r} + C_{+}e^{i k r}, \qquad k = \sqrt{\frac{2mE}{\hbar^{2}}}
	\end{equation*}
	Note that each energy has degeneracy two, since their exist two linearly independent eigenfunctions $ u(r) $ at a given value of $ k $. 
	
	\item If we neglect $ V(r) $ for $ r > r_{0} $ but include the centrifugal term, the radial eigenvalue equation is
	\begin{equation*}
		-\frac{\hbar^{2}}{2m}u''(r) + \frac{l (l+1)\hbar^{2}}{2mr^{2}}u(r) = E u(r)
	\end{equation*}
	In this case, without derivation, the solutions for $ u(r) $ are the spherical Bessel functions $ j_{l} $ and Neumann functions $ n_{l} $,
	\begin{equation*}
		u(r) = rj_{l}(kr) \eqtext{and} u(r) = r n_{l}(kr)
	\end{equation*}
	In the asymptotic limit $ r \to \infty $, the solutions simplify to
	\begin{equation*}
		u(r) \to \sin\big(kr - \tfrac{\pi}{2}l\big) \eqtext{and} u(r) \to - \cos\big(kr - \tfrac{\pi}{2}l\big)
	\end{equation*}
\end{itemize}
\textbf{Bound States with $ E < 0 $}
\begin{itemize}
	\item We now consider potentials $ V(r) $ obeying
	\begin{equation*}
		\lim_{r \to \infty}rV(r) = 0
	\end{equation*}
	In this case, the asymptotic solution to the radial eigenvalue equation for large $ r $ is
	\begin{equation*}
		u(r) \to D_{-}e^{-\kappa r} + D_{+}e^{\kappa r}, \qquad \kappa = \sqrt{\frac{2m\abs{E}}{\hbar^{2}}}
	\end{equation*}
	At the energy eigenvalues $ E = E_{n} $ we have $ D_{+} = 0 $, and the corresponding solutions
	\begin{equation*}
		u_{n}(r) = D_{n}e^{-\kappa r}
	\end{equation*}
	is a bound state. Because the problem is one dimensional, the bound energy eigenvalues are nondegenerate by the nondegeneracy theorem.
	
\end{itemize}

\subsubsection{Discussion of the Solutions for $ r \to \infty $}
\begin{itemize}
	\item We now ask more quantitatively how fast the potential $ V(r) $ must fall as $ r $ approaches infinity to justify the free and bound state ansatzes
	\begin{equation*}
		u_{\text{free}}(r) = C_{-}e^{-i k r} + C_{+}e^{i k r}\eqtext{and} u_{\text{bound}}(r) = D_{-}e^{-\kappa r} + D_{+}e^{\kappa r}
	\end{equation*}
\end{itemize}

\textbf{Bound States with $ E < 0 $}
\begin{itemize}
	\item We first consider the bound state with $ E < 0 $ and write the solution as
	\begin{equation*}
		u(r) = v(r) e^{\pm \kappa r}
	\end{equation*}
	We substitute this expression for $ u(r) $ into the radial eigenvalue equation to get
	\begin{equation*}
		-\frac{\hbar^{2}}{2m}\Big[v''(r) \pm 2\kappa v'(r) + \kappa^{2}v(r)\Big]e^{\pm \kappa r} + V_{\text{eff}}(r)v(r)e^{\pm \kappa r} = E v(r)e^{\pm \kappa r}
	\end{equation*}
	We then cancel $ e^{\pm i \kappa r} $ from the equation, multiply through by $ \frac{2m}{\hbar^{2}} $, and recognize that $ \kappa^{2} = \sqrt{\frac{-2mE}{\hbar^{2}}} $ (recall $ E < 0 $) cancels with $ \frac{2mE}{\hbar^{2}} $ to get
	\begin{equation*}
		v''(r) \pm \kappa v'(r) - \frac{2m}{\hbar^{2}}V_{\text{eff}}(r)v(r) = 0
	\end{equation*}
	
	\item Since $ v(r) $ is just a correction to $ e^{\pm \kappa r} $, we assume $ v(r) $ changes slowly with $ r $ and neglect the second derivative $ v''(r) $. We're left with
	\begin{equation*}
		\pm \kappa v'(r) = \frac{2m}{\hbar^{2}}V_{\text{eff}}(r)v(r) \eqtext{or} \frac{v'(r)}{v(r)} = \pm \frac{m}{\kappa \hbar^{2}}V_{\text{eff}}(r)
	\end{equation*}
	This is a first-order equation with separable variables, which we can integrate, i.e. 
	\begin{equation*}
		\int \frac{\diff v}{v} = \pm \frac{m}{\kappa \hbar^{2}} \int V_{\text{eff}}(r) \diff r
	\end{equation*}
	to get
	\begin{equation*}
		v(r) = v(r_{0}) \exp(\pm \frac{m}{\kappa \hbar^{2}}\int_{r_{0}}^{r}V_{\text{eff}}(\t{r})\diff \t{r})
	\end{equation*}
	where $ r_{0} $ is a ``large'' value of $ r $ where $ V_{\text{eff}} $ decays slowly.
	
	\item From the above expression for $ v(r) $, we see that the bound state ansatz
	\begin{equation*}
		u_{\text{bound}}(r) = D_{-}e^{-\kappa r} + D_{+}e^{\kappa r} = v(r)e^{\pm \kappa r}
	\end{equation*}
	is valid as long as $ v(r) $ converges to a constant value as $ r $ approaches infinity. This holds when the limit
	\begin{equation*}
		\lim_{r \to \infty} \int_{r_{0}}^{r}V(\t{r})\diff \t{r}
	\end{equation*}
	converges, which occurs when 
	\begin{equation*}
		\lim_{r \to \infty}rV(r) = 0
	\end{equation*}
	Note that the centrifugal component of $ V_{\text{eff}} $ falls with $ r^{-2} $ and is not problematic. 
	
	
	To summarize, the bound state ansatz $ u_{\text{bound}}(r) = D_{-}e^{-\kappa r} + D_{+}e^{\kappa r} $ is valid for potentials for which $ r V(r) $ vanishes at infinity.
	
	\item The limiting case at which the bound state condition 
	\begin{equation*}
		\lim_{r \to \infty}rV(r) = 0
	\end{equation*}
	no long holds is potentials of the form $ V(r) = - \lambda/r $, for which we have
	\begin{equation*}
		\lim_{r \to \infty} \int_{r_{0}}^{r}V(\t{r})\diff \t{r} = \lim_{r \to \infty} \left(- \lambda \ln \frac{r}{r_{0}}\right) \to \infty
	\end{equation*}
	To correspond solutions for $ u(r) $ in this limiting case is
	\begin{equation*}
		u(r) \to v(r)e^{\pm \kappa r} = e^{\pm \kappa r} \exp(\mp \nu \ln  \frac{r}{r_{0}}), \qquad \nu = \frac{m \lambda}{\kappa \hbar^{2}}
	\end{equation*}
	Canceling the exponent and logarithm shows the bound state solutions fall as
	\begin{equation*}
		u(r) \sim r^{\nu} e^{- \kappa r}
	\end{equation*}
	
\end{itemize}

\textbf{Free Scattering States with $ E > 0 $}
\begin{itemize}
	\item Finally, we consider the free scattering states with positive energy, for which we assumed the ansatz
	\begin{equation*}
		u_{\text{free}}(r) = C_{-}e^{-i k r} + C_{+}e^{i k r}, \qquad k = \sqrt{\frac{2mE}{\hbar^{2}}}
	\end{equation*}
	
	\item Because the free and bound state ansatzes are so similar, differing only by the presence of the imaginary unit $ i $ in the exponent and the replacement of $ \kappa $ with $ k $, we would follow an analogous procedure to the above analysis of the bound states. To avoid repeating an analogous procedure, we simply quote the result: 
	
	As before, to justify the exponential free state ansatz, the potential $ V(r) $ must obey
	\begin{equation*}
		\lim_{r \to \infty} r V(r) = 0
	\end{equation*}
	The corresponding free state solutions are
	\begin{equation*}
		u(r) \to e^{\pm i k r} \exp(\mp i \nu \ln  \frac{r}{r_{0}}), \qquad \nu = \frac{m \lambda}{k \hbar^{2}}
	\end{equation*}
	and decay asymptotically as
	\begin{equation*}
		u(r) \sim r^{\nu} e^{- ik r}
	\end{equation*}
\end{itemize}

\subsection{The Coulomb Potential}

\subsubsection{Energy Eigenvalues}
\begin{itemize}
	\item We aim to find energy levels of an electron with charge $ e_{0} $ and mass $ m_{\text{e}} $ in a Coulomb potential, for which the radial eigenvalue equation reads
	\begin{equation*}
		-\frac{\hbar^{2}}{2m_{\text{e}}}u''(r) + \left(\frac{l (l+1)\hbar^{2}}{2m_{\text{e}}r^{2}} - \frac{e_{0}^{2}}{4\pi \epsilon_{0}r}\right)u(r) = Eu(r)
	\end{equation*}
	An electron in a Coulomb potential is the basis for solving the problem of the hydrogen atom.
	
	\item We first introduce the dimensionless coordinate $ \rho = \kappa r $ where, as before, $ \kappa = \sqrt{\frac{2mE}{\hbar^{2}}} $. In this case the equation simplifies to 
	\begin{equation*}
		\left(-\dv[2]{}{\rho} + \frac{l (l+1)}{\rho^{2}} - \frac{m_{\text{e}}e_{0}^{2}}{2\pi \epsilon_{0} \kappa \hbar^{2}}\frac{1}{\rho}\right)u(\rho) = - u(\rho)
	\end{equation*}
	Finally, in terms of $ \rho_{0} $, we have
	\begin{equation*}
		u'' - \frac{l \left(l+1\right)}{\rho^{2}}u + \frac{\rho_{0}}{\rho}u - u = 0, \qquad \rho_{0} = \frac{m_{\text{e}}e_{0}^{2}}{2\pi \epsilon_{0}\kappa \hbar^{2}}
	\end{equation*}
	
	\item We proceed with the ansatz
	\begin{equation*}
		u(\rho) = \rho^{l + 1}v(\rho)e^{-\rho},
	\end{equation*}
	which is intended to model the localized behavior of bound states for large $ \rho $. 
	
	\item As an intermediate step, the ansatz's first and second are
	\begin{align*}
		& u' = \rho^{l}e^{-\rho} \big[(l+1-\rho)v + \rho v'\big]\\
		& u'' = \rho^{l}e^{-\rho}\left\{\left[-2l -2 + \rho + \frac{l (l+1)}{\rho}\right]v + 2(l + 1 - \rho)v' + \rho v''\right\}
	\end{align*}
	We substitute then substitute $ u $ and $ u'' $ into the dimensionless radial eigenvalue equation. After some tedious but simple algebra involving combining like terms and dividing through by $ \rho^{l}e^{-\rho} $ we get the equation
	\begin{equation*}
		\rho v'' + 2(l + 1 - \rho)v' + \big[\rho_{0} - 2(l+1)\big]v = 0
	\end{equation*}
	Note that this equation contains only $ v(\rho) $. 
	
	\item We solve the equation for $ v(\rho) $ with the Frobenius method. This involves writing $ v(\rho) $ as a power series, i.e.
	\begin{equation*}
		v(\rho) = \sum_{k = 0}^{\infty} c_{k}\rho^{k}
	\end{equation*}
	The plan is to find $ v $'s first two derivatives, substitute the power series ansatz into the equation for $ v $, cancel like terms, and find a recursion relation for the coefficients. The first two derivatives are
	\begin{align*}
		& v' = \sum_{k = 0}^{\infty}kc_{k}\rho^{k-1} \ \stackrel{k\to k+1}{=} \ \sum_{k=0}^{\infty}(k+1)c_{k+1}\rho^{k}\\
		&v'' = \sum_{k = 0} k(k+1)c_{k+1}\rho^{k-1}
	\end{align*}
	Note the shifting of the index, which is shown explicitly for the $ v' $ and left implicit for $ v'' $.
	
	We substitute the power series ansatz expressions into the equation for $ v $ to get
	\begin{align*}
		\rho\sum_{k = 0}^{\infty}k(k+1)c_{k}\rho^{k-1}  &+ 2(l+1-\rho)\sum_{k=0}^{\infty}(k+1)c_{k+1}\rho^{k} \\
		& + \big[\rho_{0} - 2(l+1)\big]\sum_{k}^{\infty} c_{k}\rho^{k} = 0
	\end{align*}
	We then distribute coefficients and re-index the $ 2(l+1) $ term from $ k + 1 $ to $ k $ to get
	\begin{align*}
		\sum_{k = 0}^{\infty}k(k+1)c_{k}\rho^{k}  &+ 2(l+1)\sum_{k=0}^{\infty}(k+1)c_{k+1}\rho^{k} - 2 \sum_{k=1}kc_{k}\rho^{k} \\
		& + \big[\rho_{0} - 2(l+1)\big]\sum_{k=0}c_{k}\rho^{k} = 0
	\end{align*}
	For the equation to hold, the coefficients of $ \rho^{k} $ at a given $ k $ must be equal, which implies 
	\begin{equation*}
		\big[k(k+1) + 2(l+1)(k+1)\big]c_{k+1} = \big[2k - (\rho_{0} - 2(l+1))\big]c_{k}
	\end{equation*}
	We then rearrange the above equation to get the recursive relation
	\begin{equation*}
		c_{k+1} = \frac{2(k+l+1)-\rho_{0}}{(k+1)(k+2l + 2)}c_{k}
	\end{equation*}
	For large $ k $, i.e. $ k \gg l, \rho_{0} $, the equation reduces to the relationship for the exponential function, i.e.
	\begin{equation*}
		\frac{c_{k}}{c_{k-1}} \to \frac{2}{k} \eqtext{or} c_{k} = \frac{2^{k}}{k!}c_{0}
	\end{equation*}
	where we have re-indexed the first term by one. With the coefficients $ c_{k} $ known (at least for large $ k $) we have
	\begin{equation*}
		v(\rho) = \sum_{k = 0}^{\infty}c_{k}\rho^{k} = c_{0}\sum_{k = 0}^{\infty} \frac{1}{k!}(2\rho)^{k} = c_{0}e^{2\rho}
	\end{equation*}
	
	\item In terms of the large $ k $ solution $ v(\rho) \sim e^{2\rho} $, which corresponds to the asymptotic behavior of $ v(\rho) $ for large $ \rho $, the solution for $ u(\rho) $ is
	\begin{equation*}
		u(\rho) = \rho^{l+1}v(\rho)e^{-\rho} = \rho^{l+1}e^{2\rho}e^{-\rho} =  \rho^{l+1}e^{\rho}
	\end{equation*}
	The relationship $ u(\rho) \sim \rho^{l+1}e^{\rho} $ does not un general converge for large $ \rho $. In fact, $ u(\rho) $ is convergent only if the series ansatz for $ v(\rho) $, i.e.
	\begin{equation*}
		v(\rho) = \sum_{k = 0}^{\infty}c_{k}\rho^{k},
	\end{equation*}
	truncates at a finite $ k_{\text{max}} \geq 0 $. Truncating at $ k_{\text{max}} $ implies $ c_{k} = 0 $ for $ k > k_{\text{max}} $. If we return to the recursive coefficient relation
	\begin{equation*}
		c_{k+1} = \frac{2(k+l+1)-\rho_{0}}{(k+1)(k+2l + 2)}c_{k},
	\end{equation*}
	the condition $ c_{k} = 0 $ for $ k > k_{\text{max}} $ implies
	\begin{equation*}
		2(k_{\text{max}} + l + 1) - \rho_{0} = 0 \implies \rho_{0} = 2(k_{\text{max}} + l + 1) \in \mathbb{N}
	\end{equation*}
	In other words, $ \rho_{0} $ must be an integer to satisfy the convergence of $ v(\rho) $ and thus $ u(\rho) $ for large $ \rho $. 
	
	\item With this integer restriction on $ \rho_{0} $ in mind, we define the principle quantum number
	\begin{equation*}
		n \equiv k_{\text{max}} + l + 1
	\end{equation*}
	which implies $ \rho_{0} = 2n $. In terms of $ \rho_{0} = 2n $ and the earlier equations
	\begin{equation*}
		\rho_{0} = \frac{m_{\text{e}}e_{0}^{2}}{2\pi \epsilon_{0}\kappa \hbar^{2}} \eqtext{and} \kappa = \sqrt{\frac{2m_{\text{e}}E}{\hbar^{2}}},
	\end{equation*}
	the energy eigenvalues of an electron in a Coulomb potential are thus
	\begin{equation*}
		E_{n} = - \frac{m_{\text{e}}}{2\hbar^{2}}\left(\frac{e_{0}^{2}}{4\pi \epsilon_{0}}\right)^{2}\frac{1}{n^{2}} \equiv - \frac{\text{Ry}}{n^{2}}, \quad n = 1, 2, 3, \ldots
	\end{equation*}
	where we have defined the Rydberg energy unit
	\begin{equation*}
		\SI{1}{Ry} = \frac{m_{\text{e}}}{2\hbar^{2}}\left(\frac{e_{0}^{2}}{4\pi \epsilon_{0}}\right)^{2} = \abs{E_{1}} = \SI{13.6}{\electronvolt}
	\end{equation*}
\end{itemize}

\subsubsection{Eigenfunctions}
\begin{itemize}
	\item First, we consider energy eigenvalue degeneracy. For each value of $ E_{n} $, which depends only on the principle quantum number $ n $, their exist $ n $ values of the orbital quantum number $ l = 0, 1, \ldots, n-1 $. 
	
	The complete wavefunction 
	\begin{equation*}
		\P(\r) = \p(r)Y_{l}^{m}(\theta, \phi) 
	\end{equation*}
	is thus $ n $-times degenerate with respect to the radial component $ \p(r) $, since $ n $ values of $ l $ correspond to the same energy $ E_{n} $. 
	
	Additionally, the energy eigenvalues are degenerate with respect to the quantum number $ m $, which corresponds to the projection of angular momentum onto the $ z $ axis. Since $ E_{n} $ does not depend on $ m $ and $ m $ can assume $ 2l + 1 $ values from $ -l $ to $ l $, there are $ 2l + 1 $ states proportional to $ Y_{l}^{m} $ with energy $ E_{n} $ at a given $ l $.  
	
	Considering both the degeneracy with respect to both $ l $ and $ m $, the total degeneracy of a given energy level $ E_{n} $ is
	\begin{equation*}
		\sum_{l = 0}^{n-1}(2l+1) = n^{2}
	\end{equation*}
	In other words, the energy level $ E_{n} $ has degeneracy $ n^{2} $. 
	
	\item Next, we return to the series for $ v(\rho) $, i.e.
	\begin{equation*}
		v(\rho) = \sum_{k = 0}^{k_{\text{max}}} c_{k}\rho^{k}
	\end{equation*}
	where we have made the truncation at $ k_{\text{max}} $ explicit. Since $ k_{\text{max}} = n - l - 1 $, the function $ v(\rho) $ is a polynomial of order $ n - l - 1 $. 
	
	Without derivation, it turns out that $ v(\rho) $ takes the form of an associated Laguerre polynomial, i.e.
	\begin{equation*}
		v(\rho) \propto L_{k_{\text{max}}}^{2l+1}(2\rho) = L_{n - l - 1}^{2l + 1}
	\end{equation*}
	Note that $ v(\rho) $ is indexed by both $ l $ and $ n $.
	
	\item The complete wavefunction is thus thus
	\begin{equation*}
		\P_{nlm}(\r) = \p_{nl}(r)Y_{l}^{m}(\theta, \phi)
	\end{equation*}
	where the radial component is
	\begin{equation*}
		\p_{nl} = \sqrt{\left(\frac{2}{na_{0}}\right)^{3}\frac{(n - l - 1)!}{2n(n + l)!}}\cdot \left(\frac{2r}{na_{0}}\right)^{l}\cdot L_{n - l - 1}^{2l + 1} \left(\frac{2r}{na_{0}}\right)\cdot e^{-\frac{r}{na_{0}}}
	\end{equation*}
	where we have introduced the Bohr radius
	\begin{equation*}
		a_{0} = \frac{4\pi \epsilon_{0}\hbar^{2}}{m_{\text{e}} e_{0}^{2}} = \SI{0.053}{\nano \meter}
	\end{equation*}
	
	\item The wavefunctions $ \P_{nlm} $ are conveniently orthonormal, i.e.
	\begin{equation*}
		\int \P^{*}_{n'l'm'}(\r)\P_{nlm}(\r)\dr = \braket{n'l'm'}{nlm} = \delta_{n'n}\delta_{l'l}\delta_{m'm}
	\end{equation*}
\end{itemize}

\subsubsection{Semi-Classical and Classical Limits}
\begin{itemize}
	\item The angular momentum of the electron around the hydrogen nucleus is quantized according to the Wilson-Sommerfeld quantization condition
	\begin{equation*}
		\frac{1}{2\pi}\oint p \diff q = n \hbar, \qquad n \in \mathbb{Z}
	\end{equation*}
	where $ p $ and $ q $ are a system's momentum and coordinates. 
	
	For an electron on a hypothetical circular orbit of radius $ r $ at speed $ v $ about the nucleus, the integral reads 
	\begin{equation*}
		n \hbar \equiv \frac{1}{2\pi}\oint (m_{\text{e}}v)\diff q = \frac{1}{2\pi}m_{\text{e}} v (2\pi r) = m_{\text{e}} r v = L_{z}
	\end{equation*}
	This produces the angular momentum quantization condition $ L_{z} = n \hbar $. 
	
	Combining the quantization $ n\hbar = L_{z} = m_{\text{e}}rv $ with Newton's law
	\begin{equation*}
		F = m_{\text{e}}a = \frac{m_{\text{e}}v^{2}}{r} = \frac{e_{0}^{2}}{4\pi \epsilon_{0}r^{2}}
	\end{equation*}
	Reproduces the Bohr energy formula
	\begin{equation*}
		E_{n} = - \frac{m_{\text{e}}}{2\hbar^{2}}\left(\frac{e_{0}^{2}}{4\pi \epsilon_{0}}\right)^{2}\frac{1}{n^{2}} \equiv - \frac{\text{Ry}}{n^{2}}, \quad n = 1, 2, 3, \ldots
	\end{equation*}
	
	\item Next, we consider the eigenfunctions $ \ket{nlm} $, which do not in general correspond to a uniform circular orbit of the electron about the nucleus. 
	
	Circular orbits correspond to orbits with large angular momentum for which the quantum numbers obey
	\begin{equation*}
		n \gtrsim l \gg 1 \eqtext{and} l \gtrsim m \gg 1
	\end{equation*}
	For maximum possible angular momentum, i.e. $ l = n - 1 $, the associated Laguerre polynomial is $ L_{n - l - 1}^{2l + 1} = L_{0}^{2l + 1} = 1 $ and the corresponding radial eigenfunction is
	\begin{equation*}
		\p_{n, l}(r) = \p_{n, n-1}(r) = 2^{n}\big[n^{4}(2n-1)!a_{0}^{3}\big]^{-1/2}\left(\frac{r}{na_{0}}\right)^{n-1}e^{-\frac{r}{na_{0}}}
	\end{equation*}
	The expectation values of $ r $ and $ r^{2} $ for this radial function are
	\begin{align*}
		& \ev{r} = \int_{0}^{\infty}r \p_{n, n-1}^{2}(r)r^{2}\diff r = n\left(n+\frac{1}{2}\right)a_{0}\\
		& \ev{r^{2}} = \cdots =  n^{2}(n+1)\left(n + \frac{1}{2}\right)a_{0}^{2}
	\end{align*}
	The corresponding uncertainty in $ r $ is
	\begin{equation*}
		\Delta r = \sqrt{\ev{r^{2}} - \ev{r}^{2}} = \frac{n}{2}\sqrt{2n + 1}a_{0}
	\end{equation*}
	In the limit of large $ n $, the relative uncertainty in radius is
	\begin{equation*}
		\lim_{n \to \infty} \frac{\Delta r}{\ev{r}} = \lim_{n \to \infty} \frac{1}{\sqrt{2n + 1}} = 0
	\end{equation*}
	In other words, the orbit approaches a spherical shell with radius $ \ev{r} $ for large $ n $, in agreement with the classical limit.
	
\end{itemize}

\subsubsection{The Quantum Laplace-Runge-Lenz Vector}
\begin{itemize}
	\item The $ n^{2} $-fold degeneracy of the hydrogen atom's energy eigenvalues $ E_{n} $ has a classical analog: in the Kepler problem, multiple elliptical orbits with different orientations in space correspond to the same orbital energy. 
	
	Recall from classical mechanics that the ellipses differ in their orientation of the semi-major axis in space, which corresponds to the orbit's conserved Laplace-Runge-Lenz (LRL) vector. 
	
	\item We can introduce a quantum LRL vector $ \vec{A} $ of the form
	\begin{equation*}
		\vec{A} = \frac{1}{2} \big[\vec{p} \cross \L + (\vec{p}\cross \L)^{\dagger}\big] - \frac{m_{\text{e}}e_{0}^{2}}{4\pi \epsilon_{0}} \frac{\r}{r}
	\end{equation*}
	The expression is similar to the classical LRL vector, except that is has been ``Hermitized'', i.e. we have generalized the classical expression $ \vec{p}\cross \vec{L} $ to 
	\begin{equation*}
		\frac{1}{2} \big[\vec{p} \cross \L + (\vec{p}\cross \L)^{\dagger}\big] 
	\end{equation*}
	so that $ \vec{A} $ is Hermitian. 
	
	\item The large degeneracy of $ E_{n} $ in the hydrogen atom indicates the presence of conserved quantity in addition to energy and momentum, and this conserved quantity is precisely the LRL vector $ \vec{A} $. 
	
	Namely, the hydrogen atom obeys the following conservation laws:
	\begin{equation*}
		[\L, H] = [L^{2}, H]  = [\vec{A}, H] = 0 \eqtext{and} [\vec{A}, \L] = 0
	\end{equation*}
	The relationship $ [\vec{A}, \L] = 0 $ is analogous to a well-known phenomenon from classical mechanics, namely that the LRL vectors lies in the plane of the elliptical orbit.
	
\end{itemize}

\newpage
\section{Charged Particle in a Magnetic Field}
\begin{itemize}
	\item We analyze a particle of charge $ q $ and mass $ m $ in an electric field in terms of the electric potential $ \phi $ and magnetic potential $ \A $:
	\begin{equation*}
		\vec{B} = \curl \A \eqtext{and} \vec{E} = - \grad \phi - \pdv{\A}{t}
	\end{equation*}
	The particle's Hamiltonian is
	\begin{equation*}
		H = \frac{(\vec{p} - q\A)^{2}}{2m} + q \phi
	\end{equation*}
	
	\item The \Schro equation for a particle in an electromagnetic field reads
	\begin{align*}
		i \hbar \pdv{\P}{t} &= H \P = \frac{(\vec{p} - q\A)^{2}}{2m}\P + q \phi\P = \frac{1}{2m}\big(-i\hbar \grad - q\A\big)^{2}\P + q\phi \P\\
		& = -\frac{\hbar^{2}}{2m}\laplacian \P + \frac{i \hbar q}{2m}(\div \A + \A \cdot \grad)\P + \frac{q^{2}}{2m}\A^{2}\P + q \phi \P
	\end{align*}
	We first make the intermediate calculation
	\begin{align*}
		(\div \A) \P + \A \cdot \grad \P & = \div (\P \A) + \A \cdot \grad \P = \A \cdot \grad \P + \P \div \A + \A \cdot \grad \P \\
		& = 2\A \cdot \grad \P + \P \div \A 
	\end{align*}
	The \Schro equation is thus
	\begin{equation*}
		i \hbar \pdv{\P}{t} = -\frac{\hbar^{2}}{2m}\laplacian \P + i \frac{\hbar q}{m}\A \cdot \grad \P + \left(i\frac{\hbar q}{2m}(\div \A) + \frac{q^{2}}{2m}\A^{2} + q\phi\right)\P
	\end{equation*}
	In the conventional Coulomb gauge $ \div \A = 0 $, the \Schro equation reduces to
	\begin{equation*}
		i \hbar \pdv{\P}{t} = -\frac{\hbar^{2}}{2m}\laplacian \P + i \frac{\hbar q}{m}\A \cdot \grad \P + \left(\frac{q^{2}}{2m}\A^{2} + q\phi\right)\P
	\end{equation*}
\end{itemize}

\subsection{Zeeman Coupling}
\begin{itemize}
	\item The term $ i \frac{\hbar q}{m}\A \cdot \grad \P $ in the \Schro equation for a particle in a electromagnetic field contains the dominant coupling between the particle and the magnetic field. 
	
	We consider a homogeneous magnetic field $ \vec{B} = (0, 0, B) $, generated by the vector potential
	\begin{equation*}
		\A = -\frac{1}{2}\r \cross \vec{B}
	\end{equation*}
	We continue to use the Coulomb gauge $ \div \A = 0 $. 
	
	\item For the above choice of vector potential (corresponding to the homogeneous magnetic field $ \vec{B} = (0, 0, B) $), the dominant magnetic coupling term reads
	\begin{align*}
		i \frac{\hbar q}{m}\A \cdot \grad \P &= - i\frac{\hbar q}{2m}(\r \cross \vec{B}) \cdot \grad \P = i\frac{\hbar q}{2m}(\r \cross \nabla) \cdot \vec{B} \P\\
		& = \frac{q}{2m}\big[\r \cross (i\hbar \nabla)\big] \cdot \vec{B} \P = - \frac{q}{2m}(\r \cross \vec{p}) \cdot \vec{B} \P \\
		& = - \frac{q}{2m}\vec{L} \cdot \B \P
	\end{align*}
	
	\item Next, we define the magnetic dipole moment operator $ \m $ and Bohr magnetic $ \mu_{B} $ according to
	\begin{equation*}
		\m = \frac{q}{2m}\L \eqtext{and} \mu_{B} = \frac{e_{0}\hbar}{2m_{\text{e}}}
	\end{equation*}
	The Bohr magneton is the quantum of magnetic dipole moment. 
	
	\item In terms of magnetic dipole moment, the above coupling term between a particle and an external magnetic field reads
	\begin{equation*}
		i \frac{\hbar q}{m}\A \cdot \grad = - \frac{q}{2m}\vec{L} \cdot \B = - \m \cdot \B
	\end{equation*}
	This coupling term is called the normal Zeeman coupling term, defined as
	\begin{equation*}
		H_{\text{Zeeman}} = - \m \cdot \B
	\end{equation*}
	
	\item The quadratic coupling term $ \frac{q^{2}}{2m}\A^{2} $, again using the vector potential $ \A = -\frac{1}{2}\r \cross \vec{B} $ and homogeneous magnetic field $ \B = (0, 0, B) $, reads
	\begin{align*}
		\frac{q^{2}}{2m} \A \cdot \A &= \frac{q^{2}}{2m}\left[\left(-\frac{1}{2}\r \cross \vec{B}\right) \cdot \left(-\frac{1}{2}\r \cross \vec{B}\right)\right] = \frac{q^{2}}{8m}\big[B^{2}r^{2} - (\B \cdot \r)^{2}\big]\\
		& = \frac{q^{2}}{8m}\Big\{ B^{2}r^{2} - \big[(0, 0, B)\cdot (x, y, z)\big]^{2}\Big\} = \frac{q^{2}}{8m} \big(B^{2}r^{2} - B^{2}z^{2}\big)\\
		& = \frac{q^{2}B^{2}}{8m}(x^{2} + y^{2})
	\end{align*}
	Note that the value of this quadratic coupling term depends on the choice of the gauge for $ \A $, and the above result holds only for the Coulomb gauge.
	
	\item In terms of the above quadratic coupling term and the linear Zeeman coupling term, the \Schro equation for a particle in a homogeneous magnetic field reads
	\begin{equation*}
		H = \frac{p^{2}}{2m} - \m \cdot \B + \frac{q^{2}B^{2}}{8m}(x^{2} + y^{2}) + q\phi
	\end{equation*}
	The quadratic coupling term is often negligible; for example, for an electron with magnetic moment $ \abs{\m} = \m_{B}$, charge $ q = e_{0} $ and distance scale $ x^{2} + y^{2} \sim a_{0} $ in a magnetic field $ B = \SI{1}{\tesla} $, we have
	\begin{equation*}
		\frac{e_{0}B^{2}}{8m_{\text{e}}}a_{0}^{2} \Big / (\mu_{B}B) \sim 10^{-6}
	\end{equation*}
	In other words, the quadratic coupling term is completely negligible relative to the Zeeman term $ \m \cdot \B $. 
	
	\item If we neglect the quadratic coupling term, the Hamiltonian for a charged particle  in a magnetic field reads
	\begin{equation*}
		H = \frac{p^{2}}{2m} - \m \cdot \B + q \phi
	\end{equation*}
	This \Ham is used to analyze the splitting of the electron's energy levels in a hydrogen atom exposed to a weak magnetic field, which is called the normal Zeeman effect. 
	
	We can find the wavefunction reusing the results from the central potential chapter, where the general wavefunction reads
	\begin{equation*}
		\P_{nlm}(\r) = \p_{nl}(r) Y_{l}^{m}(\theta, \varphi)
	\end{equation*}
	and $ \p_{nl} $ is the radial eigenfunction. Although this eigenfunction was originally derived in the context of angular momentum $ \vec{L} $, since $ \m $ and $ \L $ commute (recall the relationship $ \m = \frac{q}{2m}\L $), eigenfunctions of $ \L $ are also eigenfunctions of $ \m $. 
	
	In the absence of a magnetic field, considering only the electrostatic Coulomb potential $ \var{\phi}(r) $, the hydrogen atom's energy eigenfunctions obey
	\begin{equation*}
		H \ket{nlm} = - \frac{\si{Ry}}{n^{2}}\ket{nlm}
	\end{equation*}
	In the presence of a homogeneous magnetic field and the Zeeman coupling term $ - \m \cdot \B $, the energy levels split according to
	\begin{equation*}
		H \ket{nlm} = \left(- \frac{\si{Ry}}{n^{2}} + m \mu_{B}B\right)\ket{nlm}
	\end{equation*}
	Note that each energy level now depends on $ m $ as well as $ n $, which reduces, but does not fully remove, the energy level degeneracy, which is still degenerate with respect to $ l $. 
	
	\end{itemize}

\subsection{Landau Levels}
\begin{itemize}
	\item We consider only a particle in the two-dimensional $ xy $ plane. The Landau gauge potential is used generate a homogeneous magnetic field of the form $ \B = B \uvec{z} $. Two possible gauge potentials generating this magnetic field are
	\begin{equation*}
		\A = x B \uvec{y} \eqtext{and} \A = - y B \uvec{x}
	\end{equation*}
	Both recover $ \B = B \uvec{z} $ via $ \B = \curl \A $; we will work with the latter, i.e. $ \A = - y B \uvec{x} $.
	
%	note that the sum of the two Landau options recovers the symmetric gauge potential in the previous section.
	
	\item In terms of the gauge potential $ \A = - y B \uvec{x} $, the stationary \Schro equation for a particle in an electromagnetic field reads
	\begin{align*}
		H\P &= \frac{(\vec{p} - q\A)^{2}}{2m} + q \phi = \frac{1}{2m} \left(-i\hbar \grad + qyB \right)^{2}\P + q \phi \P \\
		&  =\frac{1}{2m}\left[\left(-i\hbar \pdv{x} + qBy\right)^{2} - \hbar^{2}\pdv[2]{}{y} - \hbar^{2}\pdv[2]{}{z}\right]\P + q\phi \P = E \P
	\end{align*}
	Next, we assume the electric potential depends only on the single coordinate $ y $, ie. $ \phi(\r) = \phi(y) $, and solve the equation with the ansatz
	\begin{equation*}
		\P(\r) = \exp\left[i \left(\frac{p_{x}}{\hbar}x + \frac{p_{z}}{\hbar}z \right)\right]\chi(y)
	\end{equation*}
	The exponential term contains plane waves in the $ x $ and $ z $ directions.
	
	\item Next, as an intermediate step, we note that applying a function of the operator $ \hat{p}_{x} $ to the operator's plane wave eigenfunction is the same as multiplying the plane wave by the momentum operator's eigenvalue $ p_{x} $. In equation form, this reads. 
	\begin{equation*}
		\hat{f}(\hat{p}_{x})e^{i\frac{p_{x}}{\hbar}x} \equiv  \hat{f}\left(-i\hbar \pdv{x}\right)e^{i\frac{p_{x}}{\hbar}x} = f(p_{x}) e^{i\frac{p_{x}}{\hbar}x} 
	\end{equation*}
	In our case, this identity is used to show
	\begin{align*}
		\left(-i\hbar \pdv{x} + qBy\right)^{2}\exp\left[i \left(\frac{p_{x}}{\hbar}x + \frac{p_{z}}{\hbar}z \right)\right]\chi(y) = (p_{x} + qBy)^{2}\exp\left[i \left(\frac{p_{x}}{\hbar}x + \frac{p_{z}}{\hbar}z \right)\right]\chi(y)
	\end{align*}
	where $ p_{x} $ on the right hand side is the eigenvalue of the operator $ \hat{p}_{x} \to -i\hbar \pdv{x}$ on the left hand side, and the corresponding function is $ f(x) = x^{2} $. With this identity in mind, we substitute the ansatz for $ \P $ into the \Schro equation, note that $ \phi = \phi(y) $ acts only on the $ \chi(y) $ factor of $ \P $, evaluate the relevant momentum eigenvalue relations for $ x $ and $ z $, and cancel out the common factor $ \exp\left[i \left(\frac{p_{x}}{\hbar}x + \frac{p_{z}}{\hbar}z \right)\right] $ to get
	\begin{equation*}
		\frac{1}{2m}(p_{x} + qBy)^{2}\chi(y) - \frac{\hbar^{2}}{2m}\dv[2]{}{y}\chi(y) + \frac{p^{2}_{z}}{2m}\chi(y) + q\phi(y)\chi(y) = E\chi(y).
	\end{equation*}
	where we stress that $ p_{x} $ and $ p_{z} $ are scalar eigenvalues and not operators. We consider only a particle restricted to the  $ xy $ plane, so $ p_{z} = 0 $, and the equation simplifies to
	\begin{equation*}
		\frac{1}{2m}\left[(p_{x} + qBy)^{2} - \hbar \dv[2]{}{y}\right]\chi(y) + V(y)\chi(y) = E\chi(y)
	\end{equation*}
	where we have defined $ V(y) = q\phi(y) $ and noted $ p_{z} = 0 $. Note than we have reduced the problem to a one-dimensional problem in the coordinate $ y $. 
	
	\item Next, we defined the cyclotron frequency $  \omega  $ and characteristic magnetic length $ \xi $, defined as
	\begin{equation*}
		\omega = \frac{qB}{m} \eqtext{and} \xi = \sqrt{\frac{\hbar}{qB}}
	\end{equation*}
	We introduce a wave vector $ k $, allowing use to write $ p_{x} = \hbar k $, and finally define a $ k $-displacement $ y_{k} = -\xi^{2}k $. In terms of these new quantities, we can write \Ham in the same form as a displaced harmonic oscillator with an additional potential $ V(y) $, i.e.
	\begin{equation*}
		-\frac{\hbar^{2}}{2m} \dv[2]{\chi(y)}{y} + \frac{m\omega^{2}}{2}(y - y_{k})^{2}\chi(y) + V(y)\chi(y) = E\chi(y)
	\end{equation*}
	
	\item In absence of an electric potential, and thus $ \phi = V = 0 $, the solutions of the above equation are simply the eigenstates of a displaced harmonic oscillator, i.e.
	\begin{equation*}
		\chi_{nk}(y) = \psi_{n}(y - y_{k})
	\end{equation*}
	The complete solution $ \Psi $, again for a particle in the $ xy $ plane with $ p_{z} = 0$, is
	\begin{equation*}
		\P_{nk}(x, y) = e^{i\frac{p_{x}}{\hbar}x}\chi_{nk}(y) = \frac{1}{\sqrt{2\pi}}e^{ikx} \psi_{n}(y - y_{k})
	\end{equation*}
	where the factor $ \sqrt{2\pi} $ is included for normalization. The corresponding energy eigenvalues, i.e.
	\begin{equation*}
		E_{n} = \left(n + \frac{1}{2}\right)\hbar \omega
	\end{equation*}
	are called Landau levels. These energy eigenvalues are highly degenerate, since a continuum of linearly independent eigenstates $ \P_{nk} $, which can exist for any $ k \in \mathbb{R} $, will have the same energy eigenvalue $ E_{n} $. 
	
	\item Since the energy eigenvalue $ E_{n} $ is independent of $ k $, the time evolution of a wavefunction expanded in the $ \{\P_{nk}\} $ basis is trivial, since because of the high degeneracy with respect to $ k $, any linear combination of the stationary states is still a stationary state with the same energy.
	
	As an example, consider an arbitrary wavefunction $ \p $ initially expanded in the plane wave basis, i.e.
	\begin{equation*}
		\p(x, 0) = \frac{1}{\sqrt{2\pi}} \int \F{\p}(k)e^{ikx}\diff k
	\end{equation*}
	The wavefunctions maintains is shape for $ t > 0 $, since the time evolution reads
	\begin{align*}
		\p(x, t) &= \frac{1}{\sqrt{2\pi}} \int \F{\p}(k)e^{ikx}e^{-i\frac{E}{\hbar}t} \diff k = e^{-i\frac{E}{\hbar}t} \frac{1}{\sqrt{2\pi}} \int \F{\p}(k)e^{ikx}\diff k\\
		& = e^{-i\frac{E}{\hbar}t} \p(x, 0)
	\end{align*}
	where we can move the time-dependent factor out of the integral since $ E $ does not depend of $ k $. Thus, the wavefunction's probability density $ \rho = \abs{\p}^{2} $ does not change with time, i.e.
	\begin{equation*}
		\rho(x, t) = \rho(x, 0)
	\end{equation*}
	since the factor $ e^{-i\frac{E}{\hbar}t} $ vanishes when taking $ \p $'s squared absolute value.
	
	\item Because the Landau energy levels are degenerate with respect to $ k $, which is related to $ p_{x} $ via $ p_{x} = \hbar k $, the particle's eigenstates cannot move in the $ x $ direction (if they could move, then $ E $ would depend on $ x $ and thus $ k $). Similarly eigenstates at a given Landau level cannot move in the $ y $ direction, since the states are stationary states. 
	
	The classical analog of this high degeneracy is a particle having fixed energy moving at a given angular speed around a circle with fixed radius, regardless of the position of the circle's center.
	
\end{itemize}

\subsubsection{Some Generalized Solutions}
\begin{itemize}
	\item We can generalize the solution to a \Ham with the additional $ y $-dependent harmonic potential $ V(y) = \frac{1}{2}m\omega_{y}y^{2} $. Since the sum of two separate harmonic potentials is still a harmonic potential, the solution in the presence of the additional $ V(y) $ term is the same as for $ V = 0 $, just with renormalized constants
	\begin{equation*}
		\omega \to \omega' = \sqrt{\omega^{2} + \omega_{y}} \eqtext{and} y_{k} \to y_{k}' = \frac{\omega^{2}}{\omega^{2} + \omega_{y}^{2}}y_{k}
	\end{equation*}
	
	\item In the presence of a homogeneous electric field $ \bm{\mathcal{E}} = \mathcal{E}\uvec{y} $, which corresponds to the linear potential $ V(y) = -q\mathcal{E}y $. The corresponding eigenvalue equations is
	\begin{equation*}
		-\frac{\hbar^{2}}{2m} \dv[2]{\chi(y)}{y} + \frac{m\omega^{2}}{2}(y - y'_{k})^{2}\chi(y) - q \mathcal{E}y\chi(y) = E\chi(y),
	\end{equation*}
	where the new displacement $ y'_{k} $ is
	\begin{equation*}
		y'_{k} = - \xi^{2}k + \frac{m \mathcal{E}}{qB^{2}}
	\end{equation*}
	Next, we define the energy shift
	\begin{equation*}
		\epsilon_{k} = \frac{\mathcal{E}}{B}\hbar k + \frac{m \mathcal{E}^{2}}{2B^{2}},
	\end{equation*}
	In terms of which the new potential in the presence of the electric field $ \bm{\mathcal{E}} = \mathcal{E}\uvec{y} $ reads
	\begin{equation*}
		V'(y) = \frac{m\omega^{2}}{2}(y - y'_{k})^{2} + \epsilon_{k}
	\end{equation*}
	The corresponding energy eigenstates are again shifted harmonic oscillator eigenstates of the form
	\begin{equation*}
		\chi_{nk}(y) = \p_{n}(y - y_{k}') \implies \P_{nk}(x, y) = \frac{1}{\sqrt{2\pi}}e^{ikx} \psi_{n}(y - y'_{k})
	\end{equation*}
	while the corresponding eigenvalues are
	\begin{equation*}
		E_{nk} = \left(n + \frac{1}{2}\right)\hbar \omega + \epsilon_{k}
	\end{equation*}
	Note that the presence of the $ \epsilon_{k} $ term breaks the energy degeneracy with respect to $ k $. Because of the broken degeneracy with respect to $ k $, the corresponding time-dependent wave packets are mobile in the $ x $ direction, and propagate with group velocity
	\begin{equation*}
		v_{\text{group}_{x}} = \pdv{E_{nk}}{[\hbar k]} = \frac{\mathcal{E}}{B}
	\end{equation*}
	In other words, the charged particle moves in the direction $ \bm{\mathcal{E}} \cross \B $, which is the basis for understanding the quantum Hall effect.
	
\end{itemize}

\subsubsection{The Cylindrically-Symmetric Gauge Potential}
\begin{itemize}
	\item We write the \Ham operator in the form
	\begin{equation*}
		H = \frac{\vec{\pi}^{2}}{2m}
	\end{equation*}
	where we have defined the kinetic moment operator 
	\begin{equation*}
		\vec{\pi} = \vec{p} - q \A
	\end{equation*}
	The kinetic moment moment operator is \Herm, i.e. $ \vec{\pi} = \vec{\pi}^{\dagger} $, and depends on the choice of gauge potential $ \A $. 
	
	\item The $ \vec{\pi} $ operator obeys the commutation relation $ [\pi_{x}, \pi_{y}] = i \hbar q B_{z} $, which we show with
	\begin{align*}
		[\pi_{x}, \pi_{y}] & = [p_{x} - qA_{x}, p_{y} - qA_{y}] = -q[p_{x}, A_{y}] - q[A_{x}, p_{y}] \\
		& = i\hbar q \left(\pdv{A_{y}}{x} - \pdv{A_{x}}{y}\right) = i \hbar q B_{z}
	\end{align*}
	where we have used $ [p_{i}, p_{j}] = 0 $ and $ \B = \curl 
	\A $. We can proceed analogously for the other components, which leads to commutation relation 
	\begin{equation*}
		[\pi_{\alpha}, \pi_{\beta}] = i \hbar q \epsilon_{\alpha \beta \gamma} B_{\gamma} \eqtext{or} \vec{\pi} \cross \vec{\pi} = i \hbar q \B
	\end{equation*}
	
	\item We now restrict ourselves to a particle confined in the $ xy $ plane, and define the raising and lowering operators
	\begin{equation*}
		a = \frac{1}{\sqrt{2 \hbar q B}} (\pi_{x} + i \pi_{y}) \eqtext{and} a^{\dagger} = \frac{1}{\sqrt{2 \hbar q B}} (\pi_{x} - i \pi_{y})
	\end{equation*}
	These operators obey the commutation relation $ \big[a, a^{\dagger}\big] = 1 $, which we show with
	\begin{align*}
		\big[a, a^{\dagger}\big] &= \frac{1}{2\hbar q B}[\pi_{x} + i\pi_{y}, \pi_{x} - i \pi_{y}] = \frac{1}{2\hbar q B}\big([\pi_{x}, i\pi_{y}] + [i\pi_{y}, \pi_{x}]\big) \\
		& = \frac{2}{2\hbar q B}[i \pi_{y}, \pi_{x}] = \frac{i}{\hbar q B} (-i\hbar q B) = 1
	\end{align*}
	In terms of $ a $ and $ a^{\dagger} $, the Hamiltonian reads
	\begin{align*}
		H &= \frac{\vec{\pi}^{2}}{2m} = \frac{1}{4m} \big[(\pi_{x} + i \pi_{y})(\pi_{x} - i \pi_{y}) + (\pi_{x} - i \pi_{y})(\pi_{x} + i \pi_{y})\big]\\
		& = \hbar \omega \left(a^{\dagger}a + \frac{1}{2}\right)
	\end{align*}
	
	\item We assume a homogeneous magnetic field $ \B = B \uvec{z} $, which we write in terms of the symmetric gauge potential
	\begin{equation*}
		\A = \frac{B}{2}(-y\uvec{x} + x \uvec{y})
	\end{equation*}
	We continue our analysis in the complex plane (this works well because our particle is restricted to the $ xy $ plane) and define the particle's position vector $ \r = (x, y) $ in terms of the complex number $ z = x + i y $.
	
	\item \textit{Mathematical aside:} We introduce the Wirtinger derivatives, which read
	\begin{equation*}
		\pdv{z} = \frac{1}{2}\left(\pdv{x} - i\pdv{y}\right) \eqtext{and} \pdv{z^{*}} = \frac{1}{2}\left(\pdv{x} + i\pdv{y}\right)
	\end{equation*}
	The Wirtinger derivatives obey the standard properties of complex differentiation, such as the power rule,
	\begin{equation*}
		\pdv{z^{m}}{z} = mz^{m-1} \eqtext{and} \pdv{z}{z^{*}} = 0 = \pdv{z^{*}}{z}
	\end{equation*}
	
	\item In terms of the Wirtinger derivatives, the lowering operator reads 
	\begin{align*}
		a = \frac{1}{\sqrt{2\hbar q B}}\left(i \hbar \pdv{x} + \hbar \pdv{y} - \frac{qB}{2}(-i + ix)\right) = -\frac{i}{\sqrt{2}} \left(2 \xi \pdv{z^{*}} + \frac{z}{2\xi}\right)
	\end{align*}
	where we have defined $ \xi = \sqrt{\frac{\hbar}{qB}} $.
	
	\item Next, we aim to find the particle's energy eigenfunctions. Similar to our treatment of the harmonic oscillator, we apply the lowering operator $ a $ to an as-of-yet unknown ground state:
	\begin{equation*}
		a \p_{0}(z, z^{*}) = 0
	\end{equation*}
	In the coordinate representation of $ a $, using the Wirtinger derivatives, this reads
	\begin{equation*}
		-\frac{i}{\sqrt{2}} \left(2 \xi \pdv{z^{*}} + \frac{z}{2\xi}\right) \p_{0}(z, z^{*}) = 0 \implies 2 \pdv{}{z^{*}}\p_{0}(z, z^{*}) = - \frac{z}{2\xi^{2}}\p_{0}(z, z^{*})
	\end{equation*}
	Note that we have written the wavefunction $ \p_{0} $, which was originally a function of $ x $ and $ y $, as a function of the complex variables $ z $ and $ z^{*} $. Since $ x $ and $ y $ are linearly independent, so are $ z $ and $ z^{*} $. 
	
	\item We then integrate the above equation with respect to $ z^{*} $, which results in
	\begin{equation*}
		\p_{0}(z, z^{*}) = C\exp\left(-\frac{zz^{*}}{4\xi^{2}}\right) = C\exp\left(-\frac{\abs{z}^{2}}{4\xi^{2}}\right) 
	\end{equation*}
	where, because of the integration over $ z^{*} $, the normalization constant $ C $ can also have a dependence on $ z $, i.e. $ C = C(z) $. Note that the arbitrary choice of $ C(z) $ corresponds to the earlier continuous degeneracy of the Landau levels.
	
	\item In general, we can write the analytic function $ C(z) $ as a power series, e.g. $ C(z) = \sum_{m}C_{m}z^{m} $. Since the choice of $ C(z) $ is arbitrary, we will write $ C $ as the single term
	\begin{equation*}
		C = C_{m}z^{m} \implies \p_{0_{m}}(z, z^{*}) = C_{m}z^{m}e^{-\frac{\abs{z}^{2}}{4\xi^{2}}}, \qquad m = 0, 1, 2, \ldots
	\end{equation*}
	To satisfy the normalization condition on $ \p_{0_{m}} $, the constant $ C_{m} $ must obey
	\begin{equation*}
		C_{m}^{2} = \pi m! \big(2\xi^{2}\big)^{m+1}
	\end{equation*}
	Note that $ m $ must be integer-valued to satisfy periodicity of $ \p $ with respect to the azimuthal angle $ \varphi $ on the interval $ [0, 2\pi] $; and non-negative to satisfy the normalization condition.
	
	% Non-negative $ m $ because of some kind of singularity with respect to $ \abs{z}^{-\abs{m}} $.

	\item In terms of both the Cartesian coordinates $ x $ and $ y $ and the polar coordinates $ r $ and $ \varphi $, the eigenfunctions read
	\begin{align*}
		\P_{0_{m}}(x, y) &\equiv \p_{0_{m}}(z, z^{*}) = C_{m}(x + iy)^{m}e^{-\frac{x^{2}+y^{2}}{4 \xi}}\\
		& = C_{m}r^{m}e^{im\varphi}e^{-\frac{r^{2}}{4\xi^{2}}}
	\end{align*}
	Meanwhile, the ground state eigenvalue relation reads
	\begin{equation*}
		H \P_{0_{m}} = \frac{\hbar \omega}{2}\P_{0_{m}} \equiv E_{\text{LLL}}\P_{0_{m}}
	\end{equation*}
	where we have defined the ``lowest Landau level'' $ E_{\text{LLL}} = E_{0} = \frac{\hbar\omega}{2}$, which is just the ground state energy eigenvalue.
	
	We generate excited states with the raising operator $ a^{\dagger} $, e.g.
	\begin{equation*}
		\P_{1_{m}} = a^{\dagger}\P_{0_{m}} \implies \P_{n_{m}} = \big(a^{\dagger}\big)^{n} \P_{0_{m}}
	\end{equation*}
	
	\item Without derivation, the ground state expectation values of $ r $ and $ r^{2} $ are 
	\begin{align*}
		\ev{r} = \sqrt{2}\frac{\big(m + \tfrac{1}{2}\big)!}{m!}\xi \eqtext{and} \ev{r^{2}} = \ev{(x^{2} + y^{2})} = 2(m+1)\xi^{2}
	\end{align*}
	For $ m \gg 1/2 $, the expectation value $ \ev{r} $ approaches $ r \to \sqrt{2m}\xi $, and the uncertainty in radius approaches
	\begin{equation*}
		\Delta r = \sqrt{\ev{r^{2}} - \ev{r}^{2}} \to \sqrt{2(m+1)\xi^{2} - 2m\xi^{2}} = \sqrt{2} \xi
	\end{equation*}
	
\end{itemize}

\subsubsection{Coherent States and the Classical Limit}
\begin{itemize}
	\item In the classical limit, we expect uniform circular motion at the Larmor frequency. We will use the raising and lowering operators, much like when finding the coherent states of a shifted harmonic oscillator. We start with a wave packet shifted to $ z \to (z - z_{0}) $
	\begin{equation*}
		a \p = -i\frac{z_{0}}{\xi}\p \implies \p = C\exp\left[-\frac{1}{4\xi^{2}}\big(\abs{z}^{2} - 2zz_{0} + \abs{z_{0}}^{2}\big)\right]
	\end{equation*}

	\item The corresponding probability density is
	\begin{equation*}
		\rho(x, y) = \abs{\p}^{2} = C^{2}\exp \left\{-\frac{1}{4\xi^{2}}\big[(x - x_{0})^{2} + (y - y_{0})^{2}\big]\right\}
	\end{equation*}
	and is centered at $ \r_{0} = (x_{0}, y_{0}) $. Note that the center $ \r_{0} $ moves clockwise in the plane around a circle of radius $ \abs{r_{0}} = x_{0}^{2} + y_{0}^{2} $ at the cyclotron frequency $ \omega = \frac{qB}{m_{\text{mass}}} $ according to
	\begin{equation*}
		z_{0}(t) = \big[x_{0}(0) + i y_{0}(0)\big]e^{-i\omega t}
	\end{equation*}
	The wave packet's width is constant is becomes negligible in the classical limit, in which the circular orbit occurs on a macroscopic scale, since $ \xi/r \to 0 $.
	
\end{itemize}

\subsection{Local Gauge Transformations}
\begin{itemize}
	\item Recall from the chapter on symmetries that the global gauge transformation $ \P \to e^{i\delta}\P $, in which a wave function is multiplied by a constant phase factor, has no effect on the wavefunction's observable properties, and neither does the general change of basis $ \{\ket{n}\} \to \{e^{i\delta_{n}}\}\ket{n}$. 
	
	We now introduce the local gauge transformation, which is time-dependent and reads
	\begin{equation*}
		\P'(\r, t) = e^{i\delta (\r, t)} \P(\r, t)
	\end{equation*}
	We substitute this transformation into the \Schro equation for a particle in an electromagnetic field and get
	\begin{equation*}
		i\hbar \pdv{t}\P' = \frac{1}{2m}(-i\hbar \grad - q\A')^{2}\P' + q \phi' \P'
	\end{equation*}
	
	\item Next, an intermediate mathematical aside which will help us simplify the expression $ (-i\hbar \grad - q\A')^{2} $. For a transformed wavefunction $ \P' = e^{i\delta(\r, t)}\P $, we have
	\begin{align*}
		\left(i\pdv{x} + f\right)e^{i\delta}\P &= - \pdv{\delta}{x} e^{i\delta }\P + ie^{i\delta}\pdv{\P}{x} + fe^{i\delta}\P\\
		& = e^{i\delta}\left(i \pdv{x} + f - \pdv{\delta}{x}\right)\P \\
	\end{align*}
	Repeating the above procedure recursively leads to
	\begin{equation*}
		\left(i\pdv{x} + f\right)^{2}e^{i\delta}\P = e^{i\delta}\left(i \pdv{x} + f - \pdv{\delta}{x}\right)^{2}\P 
	\end{equation*}
	
	\item Next, we choose the gauge potentials
	\begin{equation*}
		\A' = \A + \grad \Lambda \eqtext{and} \phi' = \phi - \pdv{\Lambda}{t}
	\end{equation*}
	which preserve the magnetic and electric field $ \B' = \B $ and $ \vec{\mathcal{E}}' = \vec{\mathcal{E}} $.
	
	Finally, we choose the phase term $ \delta(\r, t) $ in the local gauge transform to be
	\begin{equation*}
		\delta(\r, t) = \frac{q}{\hbar}\Lambda(\r, t) \implies \P'(\r, t) = e^{i\frac{q}{\hbar}\Lambda(\r, t)} \P(\r, t)
	\end{equation*}
	
	\item In terms of the above gauge transforms and the earlier mathematical aside, the \Schro equation for the original wavefunction $ \P $ reads
	\begin{equation*}
		i\hbar \pdv{t}\P = \frac{1}{2m}\sum_{\alpha}\left(-i\hbar \pdv{x_{\alpha}} - q A'_{\alpha} + \hbar \pdv{\delta}{x_{\alpha}}\right)^{2} \P + \left(q\phi' + \hbar \pdv{\delta}{t}\right)\P
	\end{equation*}
	where we have substituted in the identities
	\begin{equation*}
		-qA_{\alpha} = -qA'_{\alpha} + \hbar \pdv{\delta}{x_{\alpha}} \eqtext{and} q\phi = q\phi' + \hbar \pdv{\delta}{t} 
	\end{equation*}
	These follow from our choice of gauge potential and the earlier mathematical aside, but we state them without full derivation.
	
	\item Matrix elements of observable quantities, which correspond to \Herm operators, are gauge-invariant, i.e.
	\begin{equation*}
		\rho'(\r, t) = \abs{e^{i\delta}\P(\r, t)}^{2} = \rho(\r, t)
	\end{equation*}
	As an example, we consider the kinetic moment operator $ \pi $,
	\begin{align*}
		\mel{\P_{1}'}{\vec{\pi}'}{\P_{2}'} & = \mel{\P_{1}'}{\vec{p} - q\A'}{\P_{2}'} = \mel{e^{i\frac{1}{\hbar}\Lambda}\P_{1}}{\vec{p} - q(\A + \grad \Lambda)}{e^{i\frac{q}{\hbar}\Lambda}\P_{2}'}\\
		& =  \mel{e^{i\frac{1}{\hbar}\Lambda}\P_{1}}{e^{i\frac{q}{\hbar}\Lambda}(\vec{p} - q \A)}{\P_{2}'} = \mel{\P_{1}}{(\vec{p} - q\A)}{\P_{2}}\\
		& = \mel{\P_{1}}{\vec{\pi}}{\P_{2}}
	\end{align*}
	
	\textbf{TODO:} However, the canonical momentum $ \vec{p} $ is not gauge-invariant. (Does this contradict with matrix elements of observables being gauge invariant?)
	
	\iffalse
	As a result of $ \vec{\pi} $ being gauge invariant instead of $ \vec{\p} $, in an electromagnetic field we modify the definitino of probability current to
	\begin{equation*}
		\vec{j}(\r, t) = \frac{1}{m} \Re \big[\P^{*}(\r, t)\vec{\pi}\P(\r, t)\big]
	\end{equation*}
	\fi 
	
\end{itemize}

\subsection{Aharonov Bohm Effect}
\begin{itemize}
	\item We continue our study of a particle in a static (time-independent) magnetic field $ \B = \B(\r) $. Consider a region of space without a magnetic field, which implies
	\begin{equation*}
		\B = \curl \A = 0
	\end{equation*} 
	Since $ \curl \A = 0 $, we can write $ \A $ as the gradient of a scalar field $ \A = \grad \Lambda $. We then have
	\begin{equation*}
		\Lambda(\r) = \Lambda(\r_{0}) + \int_{\r_{0}}^{\r}\A(\t{\r})\cdot \diff \t{\r}
	\end{equation*}
	Note that this relationship is analogous the work done by the electric force, for which $ \curl \vec{\mathcal{E}} = 0 $ allows use to write $ \vec{\mathcal{E}} = - \grad \phi $ and thus
	\begin{equation*}
		q\phi(\r) - q\phi(\r_{0}) = - q\int_{\r_{0}}^{\r} \vec{\mathcal{E}}(\r) \cdot \diff \t{\r} \implies \phi(\r) = \phi(\r_{0})  - \int_{\r_{0}}^{\r} \vec{\mathcal{E}}(\r) \cdot \diff \t{\r} 
	\end{equation*}
	
	\item We choose the origin $ \r_{0} $ to occur in the region without magnetic field, i.e. $ \B(\r_{0}) = 0 $. The constant $ \Lambda(\r_{0}) $ then represents a global phase shift, which we can freely set to zero. The \Schro equation in the region of space with zero magnetic field reads
	\begin{equation*}
		i\hbar \pdv{t}\P = \frac{(\vec{p} - q \A)^{2}}{2m} \P + V\P
	\end{equation*}
	However, we can also work with the transformed \Schro equation
	\begin{equation*}
		i\hbar \pdv{t}\P' = \frac{(\vec{p} - q \A')^{2}}{2m} \P' + V\P' = \frac{p^{2}}{2m}\P' + V\P'
	\end{equation*}
	where the vector potential $ \A' $ (but not necessarily $ \A $) vanishes in the absence of a magnetic field via
	\begin{equation*}
		\A' = \A + \grad(-\Lambda) = \A - \A = 0
	\end{equation*}
	
	\textbf{TODO:} minus signs seem off?
	
	\item Both wavefunctions $ \P $ and $ \P' $ describe the same particle in the absence of a magnetic field, and are related by the local gauge transformation
	\begin{equation*}
		\P'(\r, t) = e^{i\frac{q}{\hbar}} \P(\r, t)
	\end{equation*}
	We then define
	\begin{equation*}
		\P_{\text{A}}(\r) = \exp\left(i\frac{q}{\hbar} \int_{\r_{0}}^{\r}\A(\t{\r})\cdot \diff\t{r}\right) \P_{0}(\r)
	\end{equation*}
	Where $ \P_{\text{A}} $ denotes the wavefunction evaluated in the presence of a non-zero vector potential and $ \P_{0} $ denotes the wavefunction evaluated in the absence of a vector potential.
\end{itemize}
\textbf{Aharonov-Bohm Thought Experiment}
\begin{itemize}
	\item Consider a two-dimensional system in which an electron can move along a straight quantum wire, which briefly splits into two branches, which then join back into a single wire.
	
	\item Assume, in the absence of both a magnetic field and vector potential ($ \B = 0 $ and $ \A = 0 $), that an electron wave packet $ \P $ travels along the wire, splits at the first junction (e.g. Roman numeral I) into two equal parts $ \P_{\text{I}} = \P_{1} + \P_{2} $ where $ \P_{1} $ and $ \P_{2} $ encode the probability for taking either branch 1 and branch 2.
	
	Both $ \P_{1} $ and $ \P_{2} $ propagate along their respective branches and joint back at the second junction (e.g. Roman numeral II) into the wavefunction $ \P_{\text{II}} $.
	
	The total electric current through the wire after the junction is proportional to $ \abs{\P_{\text{ii}}}^{2} $, the probability of finding the particle at the second junction.
	
	\item Next, assume the region between the two branches is permeated with a magnetic field $ \B $, and that $ \B = 0 $ everywhere else, including along the wires carrying the electron wave packet. 
	
	Although $ \B = 0 $ along the wires, the vector potential, defined via $ \B = \curl \A $ ``spills out'' from the region between the branches, and is non-zero along the wires. 
	
	The corresponding wavefunction encoding the electron at the second junction is then
	\begin{equation*}
		\P_{\text{II}} = e^{i\delta_{1}} \P_{0_{1}} + e^{i\delta_{2}}\P_{0_{1}}
	\end{equation*}
	where $ \P_{0_{1}} $ and $ \P_{0_{2}} $ are wavefunctions in the absence of $ \A $, while the phases in each branch are defined via
	\begin{equation*}
		\delta_{1} = \frac{q}{\hbar} \int_{\text{branch 1}} \A(\r) \cdot \diff \r \eqtext{and} \delta_{2} = \frac{q}{\hbar} \int_{\text{branch 2}} \A(\r) \cdot \diff \r
	\end{equation*}
	
	\item Assuming the probabilities of taking either branch are equal, we have $ \P_{0_{1}} = \P_{0_{2}} \equiv \P_{0} $ and the wavefunction at the second junction is
	\begin{equation*}
		\P_{\text{II}} = e^{i\delta_{2}}\left(1 + e^{i(\delta_{1} - \delta_{2})}\right)\P_{0}
	\end{equation*}
	Since the phases $ \delta_{1} $ and $\delta_{2} $ are defined by path integrals over branch 1 and 2, respectively, the phase difference $ \delta_{1} - \delta_{2} $ corresponds to a closed line integral over the loop formed by the two branches:
	\begin{equation*}
		\delta_{1} - \delta_{2} = \frac{q}{\hbar} \oint_{\text{loop}} \A(\r) \cdot \diff \r
	\end{equation*}
	We then rewrite the closed line integral with Stokes law to get
	\begin{equation*}
		\delta_{1} - \delta_{2} = \frac{q}{\hbar} \oiint_{S}\curl \A \cdot \diff \vec{S} = \frac{q}{\hbar}\oiint_{S} \B \cdot \diff \vec{S} = \frac{q}{\hbar}\Phi_{\text{M}}
	\end{equation*}
	where $ S $ is the surface bounded by the loop and $ \Phi_{\text{M}} $ is the magnetic flux through the surface. In other words, the phase difference $ \delta_{1} - \delta_{2} $ is proportional to the magnetic flux through the region between the two branches.
	
	The wavefunction encoding the electron at the second junction is thus
	\begin{equation*}
		\P_{\text{II}} = e^{i\delta_{2}}\left(1 + e^{i\frac{q}{\hbar}\Phi_{\text{M}}}\right)\P_{0}
	\end{equation*}
	
	\item The ratio of electric currents in the presence of a magnetic field in the intra-branch region, with non-zero $ \Phi_{\text{M}} $, and in the absence of a magnetic field, with $ \Phi_{\text{M}} = 0 $, is thus
	\begin{equation*}
		\frac{I_{\text{B}}}{I_{0}} = \frac{\abs{\P_{\text{II}_{\text{B}}}}^{2}}{\abs{\P_{\text{II}}}^{2}} = \frac{\abs{e^{i\delta_{2}}\left(1 + e^{i\frac{q}{\hbar}\Phi_{\text{M}}}\right)\P_{0}}^{2}}{\abs{e^{i\delta_{2}}(1 + 1)\P_{0}}^{2}} = \frac{1}{4}\abs{1 + e^{i\frac{q}{\hbar}\Phi_{\text{M}}}}^{2} = \cos^{2}\left(\frac{q}{2\hbar}\Phi_{\text{M}}\right)
	\end{equation*}
	In other words, the current through the second branch changes with the magnetic field in the region between the two conducting branches, even though the magnetic field is zero along the branches themselves. 
	
	Interpretation: the electron current $ I_{\text{B}} $ oscillates in an interference pattern as a function of a magnetic field that occurs only in a region the electrons don't travel through!
\end{itemize}

\newpage
\section{Spin}
\begin{itemize}
	\item We will denote the spin operator by $ \S = (S_{x}, S_{y}, S_{z}) $. The spin operator, like the angular momentum operator, obeys the fundamental commutation relations
	\begin{equation*}
		[S_{\alpha}, S_{\beta}] = i \hbar \epsilon_{\alpha \beta \gamma}S_{\gamma} \eqtext{or, in vector form,} \S \cross \S = i \hbar \S
	\end{equation*}
	
	\item Like for the angular momentum ladder operators, we introduce the spin ladder operators
	\begin{equation*}
		S_{+} = S_{x} + iS_{y} \eqtext{and} S_{-} = S_{x} - i S_{y}
	\end{equation*}
	The ladder operators are not \Herm, and instead obey $ S_{\pm} = S_{\mp}^{\dagger} $. 
	
	When applied to the angular momentum basis states $ \ket{sm_{s}} $, the spin ladder operators produces
	\begin{equation*}
		S_{\pm_{s}} \ket{sm_{s}} = \hbar \sqrt{s(s+1) - m_{s}(m_{s} \pm 1)}\ket{s, m_{s} \pm 1}
	\end{equation*}
	We recover $ S_{x} $ and $ S_{y} $ from the ladder operators via
	\begin{equation*}
		S_{x} = \frac{1}{2}(S_{+} + S_{-}) \eqtext{and} S_{y} = \frac{1}{2i}(S_{+} - S_{-}) 
	\end{equation*}
	
	\item Spin corresponds to angular momentum with half-integer eigenvalues $ s = \frac{1}{2}, \frac{3}{2}, \ldots $, which do not lead to continuous solutions of the \Schro equation. As a result, the spin eigenvectors do not have a coordinate representation of the form $ \r \ket{sm_{s}} $. 
	
	Instead, we represent the spin eigenstates with \textit{spinors}, which are $ (2s+1) $-tuples in the complex vector space $ \mathbb{C}^{2s + 1} $. 
	
	\item The spin states $ \ket{sm_{s}} $ are eigenstates of both the $ S^{2} $, the squared magnitude of spin, and of $ S_{z} $, the projection of spin onto the $ z $ axis. The relevant eigenvalue relations are
	\begin{equation*}
		S_{z}\ket{sm_{s}} = m_{s}\hbar \ket{sm_{s}} \eqtext{and} S^{2}\ket{sm_{s}} = s(s+1) \hbar^{2} \ket{sm_{s}}
	\end{equation*}
	As for $ l $ and $ m_{l} $, the spin quantum numbers $ s $ and $ m_{s} $ can take on the values
	\begin{equation*}
		s \in \{0, 1, 2, \ldots\} \eqtext{and} m_{s} \in \{-s, -s+1, \ldots, s-1, s\}
	\end{equation*}
	Of course, states with $ s = 0 $ have no spin, and we consider only $ s \geq 1 $ in this chapter.
	
	
\end{itemize}

\subsection{Spin 1/2}
\begin{itemize}
	\item Keep in mind that $ m_{l} $ and $ m_{s} $ are different quantum numbers, and correspond to the operators $ L_{z} $ and $ S_{z} $. When it is clear from context whether we're referring to $ m_{l} $ or $ m_{s} $, we only write $ m $ and drop the subscript, which we will periodically do in the remainder of this section for $ m_{s} \equiv m $. 
	
	\item We give states with spin $ s = 1/2 $ special attention. We write these states
	\begin{equation*}
		\ket{sm} = \ket{\tfrac{1}{2} m} \equiv \ket{\chi_{m_{s}}}
	\end{equation*}
	When $ s = 1/2 $, $ m $ can be either $ -1/2 $ or $ 1/2 $, and we often abbreviate the two possible $ \ket{sm} $  states with arrows:
	\begin{equation*}
		\ket{\tfrac{1}{2}\tfrac{1}{2}} \equiv \ket{\ua} \eqtext{and} \ket{\tfrac{1}{2}-\tfrac{1}{2}} \equiv \ket{\da}
	\end{equation*}
	to indicate ``spin up'' or ``spin down''. These states may also be written as the spinors
	\begin{equation*}
		\ket{\ua}  \equiv \chi_{\ua} = 
		\begin{pmatrix}
			1\\
			0
		\end{pmatrix}
		\eqtext{and}
		\ket{\da}  \equiv \chi_{\da} = 
		\begin{pmatrix}
			0\\
			1
		\end{pmatrix}
	\end{equation*}
	The states $ \ket{\ua} $ and $ \ket{\da} $ form an orthonormal basis spanning the space of eigenstates for particles with spin $ s = 1/2 $.
	
	As a side note, keep in mind that behind the arrow notation is the definition
	\begin{equation*}
		\ket{\tfrac{1}{2}\tfrac{1}{2}} \equiv \ket{\ua} \eqtext{and} \ket{\tfrac{1}{2}-\tfrac{1}{2}} \equiv \ket{\da}
	\end{equation*}
	relating the arrows to the actual values of $ s $ and $ m $.
	
	\item The spin ladder operators act on the states $ \ket{\ua} $ and $ \ket{\da} $ according to
	\begin{equation*}
		S_{+}\ket{\da} = \hbar \ket{\ua} \eqtext{and} S_{-}\ket{\ua} = \hbar \ket{\da}
	\end{equation*}
	In matrix form, in the $ \{\ket{\ua}, \ket{\da}\} $ basis, the ladder operators thus read
	\begin{align*}
		&S_{+} = 
		\begin{pmatrix}
			\mel{\ua}{S_{+}}{\ua} & \mel{\ua}{S_{+}}{\da}\\
			\mel{\da}{S_{+}}{\ua} & \mel{\da}{S_{+}}{\da}
		\end{pmatrix}
		= \hbar
		\begin{pmatrix}
			0 & 1\\
			0 & 0
		\end{pmatrix}\\
		&S_{-} = 
		\begin{pmatrix}
			\mel{\ua}{S_{-}}{\ua} & \mel{\ua}{S_{-}}{\da}\\
			\mel{\da}{S_{-}}{\ua} & \mel{\da}{S_{-}}{\da}
		\end{pmatrix}
		= \hbar
		\begin{pmatrix}
			0 & 0 \\
			1 & 0
		\end{pmatrix}
	\end{align*}
	Note that the matrices maintain the relationship $ S_{\pm} = S_{\mp}^{\dagger} $.
	
	\item In matrix form, the spin components $ S_{x} $ and $ S_{y} $, which we can construct directly from the $ S_{+} $ and $ S_{-} $ matrices, are
	\begin{align*}
		& S_{x} = \frac{1}{2}(S_{+} + S_{-}) = \frac{\hbar}{2}
		\begin{pmatrix}
			0 & 1\\
			1 & 0
		\end{pmatrix}
		\\
		& S_{y} = \frac{1}{2i}(S_{+} - S_{-}) = \frac{\hbar}{2}
		\begin{pmatrix}
			0 & - i\\
			i & 0
		\end{pmatrix}
	\end{align*}
	We find the $ z $ component $ S_{z} $ with direct calculation:
	\begin{equation*}
		S_{+} = 
		\begin{pmatrix}
			\mel{\ua}{S_{z}}{\ua} & \mel{\ua}{S_{z}}{\da}\\
			\mel{\da}{S_{z}}{\ua} & \mel{\da}{S_{z}}{\da}
		\end{pmatrix}
		= \frac{\hbar}{2}
		\begin{pmatrix}
			\braket{\ua}{\ua} & - \braket{\ua}{\da}\\
			\braket{\da}{\ua} & - \braket{\da}{\da}
		\end{pmatrix}
		= \frac{\hbar}{2}
		\begin{pmatrix}
			1 & 0\\
			0 & -1
		\end{pmatrix}
	\end{equation*}
	The corresponding eigenvectors (spinors) for $ S_{x} $, $ S_{y} $ and $ S_{z} $ are
	\begin{equation*}
		\chi_{x} = \frac{1}{\sqrt{2}}
		\begin{pmatrix}
			1\\
			\pm 1
		\end{pmatrix} \qquad 
		\chi_{y} = \frac{1}{\sqrt{2}}
		\begin{pmatrix}
			1\\
			\pm i
		\end{pmatrix} \qquad
		\chi_{z} =
		\begin{pmatrix}
			1\\
			0
		\end{pmatrix}
		\text{ and }
		\begin{pmatrix}
			1\\
			0
		\end{pmatrix}
	\end{equation*}
	
	
	\item For $ \alpha \in \{x, y, z\} $, the squared components $ S_{\alpha}^{2} $ read
	\begin{equation*}
		S_{\alpha}^{2} = \frac{\hbar^{2}}{4}
		\begin{pmatrix}
			1 & 0\\
			0 & 1
		\end{pmatrix}
	\end{equation*}
	The squared components obey the commutation relations
	\begin{equation*}
		[S_{\alpha}, S_{\beta}^{2}] = 0 \eqtext{an} [S_{\alpha}^{2}, S_{\beta}^{2}] = 0
	\end{equation*}
	Finally, the squared spin operator $ S^{2} $ acts on the states $ \ket{\chi_{m}} \equiv \ket{\tfrac{1}{2}m} $ according to
	\begin{equation*}
		S^{2}\ket{\chi_{m}} = \sum_{\alpha}S_{\alpha}^{2} \ket{\chi_{m}} = \frac{3}{4}\hbar^{2} \ket{\chi_{m}}
	\end{equation*}
	
	\item As a final note, for larger spins $ s = \frac{3}{2}, \frac{5}{2}, \ldots $, we write the spin wavefunction $ \ket{\p} $ by expanding $ \ket{\p} $ in the $ \{\ket{sm}\} $ basis:
	\begin{equation*}
		\ket{\p} = \sum_{s = 1/2}^{\infty}\sum_{m = -s}^{s}c_{sm}\ket{sm}, \qquad c_{sm} = \braket{sm}{p}
	\end{equation*}
	
\end{itemize}

\subsubsection{The Pauli Spin Matrices}
\begin{itemize}
	\item We often analyze states with spin $ s = 1/2 $ using the Pauli spin matrices $ \sigma_{x} $, $ \sigma_{y} $ and $ \sigma_{z} $, which read
	\begin{equation*}
		\sigma_{x} = 
		\begin{pmatrix}
			0 & 1\\
			1 & 0
		\end{pmatrix} \qquad 
		\sigma_{y} = 
		\begin{pmatrix}
			0 & -i\\
			i & 0
		\end{pmatrix} \qquad 
		\sigma_{x} = 
		\begin{pmatrix}
			1 & 0\\
			0 & -1
		\end{pmatrix}
	\end{equation*}
	Together with the $ 2 \cross 2 $ identity matrix $ \II $, the Pauli matrices provide a convenient basis in which to expand an arbitrary $ 2 \cross 2 $ matrix. 
	
	\item In terms of the Pauli spin matrices, the spin operator reads
	\begin{equation*}
		\S = \frac{\hbar}{2}\vec{\sigma} \eqtext{where} \vec{\sigma} = (\sigma_{x}, \sigma_{y}, \sigma_{z})
	\end{equation*}

	\item The Pauli spin matrices are Hermitian, and obey the following properties:
	\begin{equation*}
		\sigma_{\alpha} = \sigma_{\alpha}^{\dagger} \qquad \sigma_{\alpha}^{2} = \II \qquad \det \sigma_{\alpha} = -1 \qquad \tr \sigma_{\alpha} = 0
	\end{equation*}
	Each of the spin matrices has eigenvalues $ \lambda_{\pm} = \pm 1 $---note that the eigenvalues must be equal and opposite to satisfy $ \tr \sigma_{\alpha} = 0 $.
	
	\item The product of two spin matrices obeys the general formula
	\begin{equation*}
		\sigma_{\alpha}\sigma_{\beta} = \delta_{\alpha \beta} \II + i \epsilon_{\alpha \beta \gamma}\sigma_{\gamma}
	\end{equation*}
	
	\item The commutation relations between the spin matrices are analogous to the spin operator commutation relations, and read
	\begin{equation*}
		[\sigma_{\alpha}, \sigma_{\beta}] = 2i \epsilon_{\alpha \beta \gamma} \sigma_{\gamma}
	\end{equation*}
	More so, we use matrix multiplication to derive the anti-commutator relation
	\begin{equation*}
		\{\sigma_{\alpha}, \sigma_{\beta}\} = \sigma_{\alpha}\sigma_{\beta} + \sigma_{\beta}\sigma_{\alpha} = 2 \delta_{\alpha \beta} \II
	\end{equation*}
	
	\item Finally, in terms of the spin matrices, an arbitrary vector $ \vec{a} \in \mathbb{R}^{3} $ can be written 
	\begin{equation*}
		 \vec{a} \cdot \vec{\sigma} = \sum_{\alpha}a_{\alpha}\sigma_{\alpha} = (\vec{a} \cdot \vec{\sigma})^{\dagger} 
	\end{equation*}
	Note that $ \sigma_{\alpha} $ is a $ 2 \cross 2 $ matrix, while $ a_{\alpha} $ is a scalar. Meanwhile, for two vectors $ \vec{a}, \vec{b} \in \mathbb{R}^{2} $ we have
	\begin{align*}
		(\vec{a} \cdot \vec{\sigma})(\vec{b} \cdot \vec{\sigma}) &= \sum_{\alpha, \beta} a_{\alpha}\sigma_{\alpha}b_{\beta}\sigma_{\beta} = \vec{a}\cdot \vec{b}\II + i \sum_{\alpha, \beta} \epsilon_{\alpha \beta \gamma}a_{\alpha}b_{\beta} \sigma_{\gamma}\\
		& = (\vec{a}\cdot \vec{b})\II + i(\vec{a}\cross \vec{b})\cdot \vec{\sigma}
	\end{align*}
	Again, we stress that $ \sigma_{\alpha} $  and $ \II$ are $ 2 \cross 2 $ matrices, while $ a_{\alpha} $ and $ b_{\alpha} $ are scalars.
	
%	TODO: quaternions?
	
\end{itemize}

\subsection{Rotation of Spinors}
\begin{itemize}
	\item For particles with spin $ s = 1/2 $, spinors, which describe general spin states, are determined by $ 2s + 1 = 2 $ coordinates, e.g. $ a, b \in \mathbb{C} $, that satisfy the normalization condition $ \abs{a}^{2} + \abs{b}^{2} = 1 $. Some notation:
	\begin{equation*}
		\chi = 
		\begin{pmatrix}
			a\\
			b
		\end{pmatrix}
		= a \chi_{\ua} + b\chi_{\da} \qquad \ket{\chi} = a \ket{\ua} + b \ket{\da}
	\end{equation*}
	The product of two spinor states reads
	\begin{equation*}
		\braket{\chi_{1}}{\chi_{2}} = \big(a_{1}^{*} \, \ b_{1}^{*}\big)\cdot 
		\begin{pmatrix}
			a_{2}\\
			b_{2}
		\end{pmatrix}
		= a_{1}^{*}a_{2} + b_{1}^{*}b_{2}
	\end{equation*}
	
	\item The expectation values of the Pauli spin matrices in a given spin state $ \ket{\chi} $ are
	\begin{align*}
		& \ev{\sigma_{x}} = \mel{\chi}{\sigma_{x}}{\chi} = \big(a^{*} \, \ b^{*}\big) 
		\begin{pmatrix}
			0 & 1\\
			1 & 0
		\end{pmatrix}
		\begin{pmatrix}
			a\\
			b
		\end{pmatrix}
			= \big(a^{*} \, \ b^{*}\big)\cdot 
		\begin{pmatrix}
			b\\
			a
		\end{pmatrix}
		= 2 \Re \big[a^{*}b\big]\\
		& \ev{\sigma_{y}} = \mel{\chi}{\sigma_{y}}{\chi} = \big(a^{*} \, \ b^{*}\big) 
		\begin{pmatrix}
			0 & -i\\
			i & 0
		\end{pmatrix}
		\begin{pmatrix}
			a\\
			b
		\end{pmatrix}
			= \big(a^{*} \, \ b^{*}\big)\cdot 
		\begin{pmatrix}
			-ib\\
			ia
		\end{pmatrix}
		= 2 \Im \big[a^{*}b\big]\\
		& \ev{\sigma_{z}} = \mel{\chi}{\sigma_{z}}{\chi} = \big(a^{*} \, \ b^{*}\big) 
		\begin{pmatrix}
			1 & 0\\
			0 & -1
		\end{pmatrix}
		\begin{pmatrix}
			a\\
			b
		\end{pmatrix}
			= \big(a^{*} \, \ b^{*}\big)\cdot 
		\begin{pmatrix}
			a\\
			-b
		\end{pmatrix}
		= \abs{a}^{2} - \abs{b}^{2}
	\end{align*}
	
	\item We rotate a spinor by the angle $ \phi $ about the axis $ \uvec{n} $ with the unitary rotation operator
	\begin{equation*}
		U(\phi \uvec{n})\ket{\chi} = \exp \left(-i\phi \frac{\uvec{n}\cdot \S}{\hbar}\right) \ket{\chi}
	\end{equation*}
	We begin by writing the rotation operator as a Taylor series in terms of the spin matrix vector $ \vec{\sigma} $:
	\begin{equation*}
		U(\phi \uvec{n}) = \exp \left(-i\phi \frac{\uvec{n}\cdot \S}{\hbar}\right) = \exp \left(-\frac{i\phi}{2}\uvec{n}\cdot \vec{\sigma}\right) = \sum_{k = 0}^{\infty}\frac{1}{k!}\left(-\frac{i\phi}{2}\uvec{n}\cdot \vec{\sigma}\right)^{k}
	\end{equation*}
	We then write $ \uvec{n} \cdot \vec{\sigma} $ in the form
	\begin{equation*}
		(\uvec{n} \cdot \vec{\sigma} )^{k} = 
		\begin{cases}
		\II & k \text{ even}\\
		\uvec{n} \cdot \vec{\sigma} & k \text{ odd}
		\end{cases}
	\end{equation*} 
	The rotation operator then reduces to the linear function
	\begin{align*}
		U(\phi \uvec{n}) = \sum_{k = 0}^{\infty}\frac{1}{k!}\left(-\frac{i\phi}{2}\uvec{n}\cdot \vec{\sigma}\right)^{k} = &\left[1 - \frac{1}{2!}\left(\frac{\phi}{2}\right)^{2} + \frac{1}{4!}\left(\frac{\phi}{2}\right)^{4} \mp \cdots \right]\II \\
		& - i\left[\frac{\phi}{2} - \frac{1}{3!}\left(\frac{\phi}{2}\right)^{3} + \frac{1}{5!}\left(\frac{\phi}{2}\right)^{5} \mp \cdots \right]\uvec{n} \cdot \vec{\sigma}
	\end{align*}
	Using the power series definitions of the sine and cosine function, this becomes
	\begin{equation*}
		U(\phi \uvec{n}) = \cos \left(\frac{\phi}{2}\right)\II - i \sin \left(\frac{\phi}{2}\right)\uvec{n} \cdot \vec{\sigma}
	\end{equation*}
	Note that rotating a spinor by an angle $ \phi = 2\pi $ corresponds to multiplication by $ -\II $, and not simply the identity $ \II $. We must rotate a spinor around ``twice'', by an angle of $ 4\pi $, to reach its original orientation.
	
	\item We can parameterize an arbitrary spinor $ \ket{\chi} $ with two angles. We begin by rotating a spin-up state $ \ket{\ua} $ by an angle $ \theta $ about the $ y $ axis, followed by a rotation by the angle $ \phi $ about the $ z $ axis. The result is
	\begin{align*}
		\ket{\chi} &= U(\phi \uvec{e}_{z})U(\theta \uvec{e}_{y})\ket{\ua} = e^{-i\frac{\phi}{2}\sigma_{z}} e^{-i\frac{\theta}{2}\sigma_{y}}\ket{\ua}\\
		& = e^{-i\frac{\phi}{2}}\left[\cos \left(\frac{\theta}{2}\right)\ket{\ua} + e^{i \phi} \sin \left(\frac{\theta}{2}\right)\ket{\da}\right]
	\end{align*}
	Since a wavefunction is determined only up to a constant phase factor, we can neglect the coefficient $ e^{-i\frac{\phi}{2}} $ go get
	\begin{equation*}
		\ket{\chi(\theta, \phi)} = \cos \left(\frac{\theta}{2}\right)\ket{\ua} + e^{i \phi} \sin \left(\frac{\theta}{2}\right)\ket{\da}
	\end{equation*}
	
	\item As an example, if we rotate the spinor only about the $ y $ axis, we have
	\begin{equation*}
		U(\theta \uvec{e}_{y}) = e^{-i\frac{\theta}{2}\sigma_{y}} = 
		\begin{pmatrix}
			\cos \frac{\theta}{2} & - \sin \frac{\theta}{2} \\
			\sin \frac{\theta}{2} & \cos \frac{\theta}{2} 
		\end{pmatrix}
	\end{equation*}
	Note that if we rotate $ \ket{\ua} \equiv \ket{\chi_{z}} $, which is an eigenstate of $ S_{z} $, by an angle $ \pi/2 $ in the $ x $ direction, we end up with an eigenstate of the operator $ S_{x} $:
	\begin{equation*}
		U\left(\frac{\pi}{2}\uvec{e}_{y} \right) 
		\begin{pmatrix}
			1\\
			0
		\end{pmatrix}
		= \frac{1}{\sqrt{t}} 
		\begin{pmatrix}
			1\\
			1
		\end{pmatrix}
		= \chi_{x}
	\end{equation*}
	since $ S_{x}\ket{\chi_{x}} = \frac{\hbar}{2}\ket{\chi_{x}} $. 
	
	\item We then repeat the procedure, rotating the $ S_{z} $ eigenstate $ \ket{\ua} \equiv \ket{\chi_{z}} $ by an arbitrary angle $ \theta $ to get
	\begin{equation*}
		U(\theta \uvec{e}_{y})
		\begin{pmatrix}
			1\\
			0
		\end{pmatrix}
		=
		\begin{pmatrix}
			\cos \frac{\theta}{2}\\
			\sin \frac{\theta}{2}
		\end{pmatrix}
		= \chi_{x}
	\end{equation*}
	The resulting state is an eigenstate of the projection of spin in the direction of the vector $ \uvec{n}_{0} = \cos \theta \uvec{e}_{x} + \sin \theta \uvec{e}_{z} $, since
	\begin{align*}
		(\uvec{n}_{0}\cdot \vec{S})\ket{\chi_{z}(\theta)} &= (\sin \theta \sigma_{x} + \cos \theta \sigma_{z})\ket{\chi_{z}(\theta)} = \frac{\hbar}{2}
		\begin{pmatrix}
			\cos \theta & \sin \theta\\
			\sin \theta & - \cos \theta
		\end{pmatrix}
		\ket{\chi_{z}(\theta)}\\
		& = \frac{\hbar}{2}\ket{\chi_{z}(\theta)}
	\end{align*}
	The above unit vector was constructed with $ \phi = 0 $. For a more general direction $ \uvec{n} = (\cos \phi \sin \theta, \sin \phi \sin \theta, \cos \theta) $, we have
	\begin{equation*}
		\uvec{n}(\theta, \phi) \cdot \S \ket{\chi(\theta, \phi)} = \frac{\hbar}{2} \ket{\chi(\theta, \phi)}
	\end{equation*}
	
	\item Finally, we can recover the direction $ \uvec{n} $ via
	\begin{equation*}
		\mel{\chi(\theta, \phi)}{\vec{\sigma}}{\chi(\theta, \phi)} = \uvec{n}(\theta, \phi)
	\end{equation*}
	We prove this by components. Starting with $ \sigma_{x} $, we have
	\begin{align*}
		\mel{\chi(\theta, \phi)}{\vec{\sigma}}{\chi(\theta, \phi)} &= \left(\cos\frac{\theta}{2}, \, e^{-i\phi}\sin \frac{\theta}{2}\right)
		\begin{pmatrix}
			0 & 1\\
			1 & 0
		\end{pmatrix}
		\begin{pmatrix}
			\cos \frac{\theta}{2}\\
			e^{i\phi}\sin \frac{\theta}{2}
		\end{pmatrix}\\
		& = \left(\cos\frac{\theta}{2}, \, e^{-i\phi}\sin \frac{\theta}{2}\right)
		\begin{pmatrix}
			e^{i\phi}\sin \frac{\theta}{2}\\
			\cos \frac{\theta}{2}
		\end{pmatrix}\\
		& = \sin \theta \cos \phi = \uvec{n}\cdot \uvec{e}_{x}
	\end{align*}
	The calculation for $ \sigma_{y} $ and $ \sigma_{z} $ is analogous.
	
\end{itemize}

\subsection{Time Reversal of Spinors}
\begin{itemize}
	\item Recall from the symmetry chapter that the time reversal operator for a particle with spin $ s = 1/2 $ is 
	\begin{equation*}
		\T = i \sigma_{y} KT \equiv \tau KT, \qquad \tau = i \sigma_{y} =
		\begin{pmatrix}
			0 & 1\\
			-1 & 1
		\end{pmatrix}
	\end{equation*}
	where $ K $ is the complex conjugation operator and $ T $ changes the sign of time. We begin by confirming the commutator and anti-commutator relations
	\begin{align*}
		&\{\sigma_{x}, \tau\} = \{\sigma_{z}, \tau\} = 0 \eqtext{and} [\sigma_{y}, \tau] = 0\\
		&[\sigma_{x}, K] = [\sigma_{z}, K] = 0 \eqtext{and} \{\sigma_{y}, \tau\} = 0
	\end{align*}
	From these relations, it follows that $ \T $ reverses spin, which we show with the components of $ \vec{\sigma} = (\sigma_{x}, \sigma_{y}, \sigma_{z}) $:
	\begin{equation*}
		\begin{pmatrix}
			0 & 1\\
			-1 & 0
		\end{pmatrix}
		K T \sigma_{\alpha} = - \sigma_{\alpha}
		\begin{pmatrix}
			0 & 1\\
			-1 & 0
		\end{pmatrix}
		KT
		\implies \T \S = - \S \T
	\end{equation*}
	
	\item The matrix elements of $ \vec{\sigma} $ ovey
	\begin{align*}
		\mel{\T \chi_{1}}{\vec{\sigma}}{\T \chi_{2}} &= \mel{i\sigma_{y}KT\chi_{1}}{\vec{\sigma}i \sigma_{y}K}{T\chi_{2}} = - \mel{i\sigma_{y}KT\chi_{1}}{i \sigma_{y}K \vec{\sigma}}{T\chi_{2}} \\
		& - \mel{KT\chi_{1}}{K \vec{\sigma}}{T\chi_{2}} = -\mel{T\chi_{1}}{\vec{\sigma}}{T\chi_{2}}^{*}\\
		& -\mel{\chi_{1}(-t)}{\vec{\sigma}}{\chi_{2}(-t)}^{*}
	\end{align*}
	The above result implies
	\begin{equation*}
		\mel{\T \chi}{\vec{\sigma}}{\T \chi} = -\mel{\chi(-t)}{\vec{\sigma}}{\chi(-t)}
	\end{equation*}
	
	\item Time-dependent spinors $ \chi(t) $ transform under time reversal according to
	\begin{equation*}
		\T\chi(t) = \T
		\begin{pmatrix}
		a(t)\\
		b (t)
		\end{pmatrix}
		=
		\begin{pmatrix}
		b^{*}(-t) \\
		-a^{*}(-t)
		\end{pmatrix}
	\end{equation*}
	
	\item Finally, we write the time-reversed state as
	\begin{equation*}
		\T \ket{\chi(\theta, \phi)} = \ket{\chi(\theta + \pi, \phi)} = \ket{\t{\chi}(\theta, \phi)}
	\end{equation*}
	This state is a second eigenstate of the earlier equation $ \uvec{n}(\theta, \phi) \cdot \S \ket{\chi(\theta, \phi)} = \frac{\hbar}{2} \ket{\chi(\theta, \phi)} $
	\begin{equation*}
		\uvec{n}(\theta, \phi) \cdot \S \ket{\T\chi(\theta, \phi)} = -\frac{\hbar}{2} \ket{\T \chi(\theta, \phi)}
	\end{equation*}
	
\end{itemize}

\subsection{Changing the Axis of Quantization}
\begin{itemize}
	\item The basis states $ \ket{\ua} $ and $ \ket{\da} $ correspond to a choice of the $ z $ axis as the axis of quantization. 
	
	More generally, recall that the states $ \ket{\chi(\theta, \phi)}  $ and $ \ket{\T \chi(\theta, \phi)}  $ are eigenstates of the projection of spin $ \vec{S} $ onto the direction
	\begin{equation*}
		\uvec{n} = (\cos \phi \sin \theta, \sin \phi \sin \theta, \cos \theta)
	\end{equation*}
	We can use these states as basis vectors with respect to the new quantization axis $ \uvec{n} $. This this cases the basis vectors are
	\begin{equation*}
		\ket{\chi_{m}(\uvec{n})} = 
		\begin{cases}
			\cos \theta \ket{\ua} + e^{i\phi} \sin \theta \ket{\da} & m = \frac{1}{2}\\
			\sin \theta \ket{\ua} - e^{i\phi} \cos \theta \ket{\da} & m = -\frac{1}{2}
		\end{cases}
	\end{equation*}
	
	\item In terms of the new basis vectors $ \ket{\chi_{m}(\uvec{n})} $, we can write an arbitrary spin state $ \ket{\p} $ as
	\begin{equation*}
		\ket{\p} = \sum_{m} c_{m}\ket{\chi_{m}(\uvec{n})}, \qquad c_{m} =  \braket{\chi_{m}(\uvec{n})}{\p}
	\end{equation*}
	
\end{itemize}

\subsection{Coupling of Spin and and Electromagnetic Field}
\begin{itemize}
	\item We define the spin magnetic moment
	\begin{equation*}
		\m_{S} = \frac{ge}{2m} \S
	\end{equation*}
	where, for an electron, $ g = 2 $. The above magnetic moment corresponds to the anomalous Zeeman coupling of a spin $ s = 1/2 $ particle with a magnetic field:
	\begin{equation*}
		H_{\text{anomalous}} = - \m_{S} \cdot \B
	\end{equation*}
	
	\item Next, we consider a semi-classical picture of an electron orbiting a hydrogen nucleus. The electron feels both an electric interaction due to the Coulomb force, and, in its own coordinate system, a magnetic field 
	\begin{equation*}
		\B = -\frac{1}{c^{2}}(\vec{v} \cross \vec{\mathcal{E}})
	\end{equation*}
	which arises because in the electron's coordinate system, the positively-charged proton orbits the electron at speed $ \uvec{v} $, which leads to a magnetic force via the Lorentz interaction.
	
	\item In a spherically-symmetric potential $ V = V(r) $ we have
	\begin{equation*}
		\vec{\mathcal{E}} = - \frac{1}{q}\grad V(r) = \frac{1}{q}\pdv{V}{r} \frac{\r}{r}
	\end{equation*}
	The corresponding magnetic field reads
	\begin{equation*}
		\B = -\frac{1}{qc^{2}}\left(\vec{v} \cross \pdv{V}{r}\frac{\r}{r}\right) = \frac{1}{qmc^{2}}\frac{1}{r}\pdv{V}{r}\L
	\end{equation*}
	
	\item We then apply the quantum-mechanical properties of the angular momentum operator $ \L $ and use the result in the Zeeman formula with $ \m = \m_{S} $. The resulting effective Hamiltonian is
	\begin{equation*}
		H_{\text{LS}} = \frac{1}{m^{2}c^{2}}\frac{1}{r}\pdv{V}{r} \L \cdot \S
	\end{equation*}
	This heuristically-derived result is too large by a factor of two with respect to the correct result derived from the Dirac equation for relativistic particles, which is beyond the scope of this course. 
	
	\item Finally, we mention the coupling of spin to angular momentum for a particle moving in a plane that is orthogonal to a homogeneous electric field. As before, we use the Lorentz expression for magnetic field $ \B = -\frac{1}{c^{2}}(\vec{v} \cross \vec{\mathcal{E}}) $ and the Zeeman coupling. The result is
	\begin{equation*}
		\m \cdot (\vec{v} \cross \vec{\mathcal{E}}) \implies \vec{\sigma} \cdot (\vec{p} \cross \vec{\mathcal{E}}) = (\vec{\sigma} \cross \vec{p}) \cdot \vec{\mathcal{E}}
	\end{equation*}
	This expression results in the Rashba coupling term
	\begin{equation*}
		H_{\text{Rashba}} = \alpha (\vec{\sigma} \cross \vec{p}) \cdot \uvec{e}_{z}
	\end{equation*}
\end{itemize}


\subsection{More on the Pauli Formalism}
\begin{itemize}
	\item  We now consider a spin wavefunction $ \P $ whose components are both time-dependent spinors:
	\begin{equation*}
		\P(\r, t) = 
		\begin{pmatrix}
			\p_{\ua}(\r, t)\\
			\p_{\da}(\r, t)
		\end{pmatrix}
	\end{equation*}
	The two components encode the probability amplitudes of detecting the particle with either spin up or spin down near the location $ \vec{r} $ at the time $ t $---the respective amplitudes are
	\begin{equation*}
		\rho_{\ua}(\r, t) = \abs{\p_{\ua}(\r, t)}^{2} \eqtext{and} \rho_{\da}(\r, t) = \abs{\p_{\da}(\r, t)}^{2}
	\end{equation*}
	
	\item The total probability density is $ \rho = \rho_{\ua} + \rho_{\da} $ and must satisfy the normalization condition
	\begin{equation*}
		\int \rho(\r, t)\dr \equiv 1
	\end{equation*}
	
	\item We can parameterize the state $ \P $ at a position $ (\r, t) $ using four dependent real functions $ \rho, \delta, \theta $ and $ \phi $, in the form
	\begin{equation*}
		\P = \p
		\begin{pmatrix}
			\cos \frac{\theta}{2}\\
			e^{i\phi}\sin \frac{\theta}{2}
		\end{pmatrix},
		\qquad
		\p = e^{i\delta}\sqrt{\rho}
	\end{equation*}
	We then write the state as the product of the position-dependent state $ \ket{\p} $ and the normalized spin state $ \ket{\chi} $:
	\begin{equation*}
		\ket{\P} = \ket{\p(\r, t)} \otimes \ket{\chi(\theta(\r, t), \phi(\r, t))}
	\end{equation*}
	
	\item In the corresponding \Schro equation
	\begin{equation*}
		i \hbar \pdv{t} \P = H \P
	\end{equation*}
	the \Ham is written as a $ 2 \cross 2 $ matrix. For example, in the presence of a magnetic field and homogeneous electric field, and taking into account Zeeman and Rashba coupling:
	\begin{align*}
		H = &\left(\frac{p^{2}}{2m} + V(r) - \frac{q}{2m}\L \cdot \B\right)\II\\
		& - \frac{qe}{2m}\S \cdot \B + \frac{1}{m^{2}c^{2}}\frac{1}{r}\pdv{V}{r} \L \cdot \S + \alpha (\vec{\sigma}\cross \vec{p})\cdot \uvec{e}_{z}
	\end{align*}
	This corresponds to an expansion of the \Ham operator in the basis of Pauli spin matrices $ \{\sigma_{x}, \sigma_{y}, \sigma_{z}, \II\} $:
	\begin{equation*}
		H = E_{0}\II + \sum_{\alpha}E_{\alpha}\sigma_{\alpha}
	\end{equation*}
	The first three terms, involving $ p^{2} $, $ V(r) $ and $ \L \cdot \B $, correspond to the energy $ E_{0} $, while the remaining coupling terms are higher-order corrections.
	
\end{itemize}


\newpage
\section{Addition of Angular Momentum}

\subsection{Particles With Spin $ s = 1/2 $}
\begin{itemize}
	\item We now consider the possibility of a quantum system consisting of multiple particles, each with their own angular momentum. For the majority of this section we will focus on simple system: a hydrogen atom with an electron in the ground state with angular momentum quantum number $ l = 0 $. The second particle is of course the proton in the nucleus. Both particles have spin $ s_{i} = 1/2 $. The spin operators and basis vectors for each particle are
	\begin{equation*}
		S_{i}^{2} = s_{i}(s_{i} + 1)\hbar^{2}\ket{s_{i}m_{i}} \eqtext{and} S_{z_{i}}\ket{s_{i}m_{i}} = m_{i}\hbar \ket{s_{i}}\ket{m_{i}}
	\end{equation*}
	where $ i = 1 $ corresponds to the electron and $ i = 2 $ to the proton. 
	
	\item The electron's and proton's angular momentum operators commute with each other, which is summarized in the commutation relation
	\begin{equation*}
		\big[S_{\alpha_{i}}, S_{\beta_{j}}\big] = i \hbar \delta_{ij}\epsilon_{\alpha \beta \gamma}S_{\gamma_{j}}, \qquad \alpha \in \{x, y, z\}
	\end{equation*}
	Both sets of spin operators can be written in terms of the Pauli spin matrices as
	\begin{equation*}
		\vec{S}_{i} = \frac{\hbar}{2}\vec{\sigma}, \qquad \vec{\sigma} = \left(\sigma_{x}, \sigma_{y}, \sigma_{z}\right)
	\end{equation*}
	
	\item We expect the hydrogen atom's total spin $ \vec{S} $ is the sum of the spins of the constituent electron and proton. More generally, for a system of $ N $ particles each with spin $ \S_{i} $, we expect the total spin $ \S $ to be
	\begin{equation*}
		\S = \sum_{i}^{N}\S_{i}
	\end{equation*}
	Similarly, we expect the total system's spin expectation value to obey
	\begin{equation*}
		\mel{\P}{\S}{\P} = \sum_{i}\mel{\chi_{i}}{\S_{i}}{\chi_{i}}
	\end{equation*}
	where $ \P $ is the system's complete wavefunction and $ \ket{\chi_{i}} $ are the spin states of the constituent particles. 
	
	\item For these definitions to hold, the operators $ \S_{i} $ of each constituent particle act only the spin basis states $ \ket{s_{i}m_{i}} $ corresponding to that particular particle. With this requirement in mind, we define the total spin of a two particle system as
	\begin{equation*}
		\S = \S_{1}\otimes \II_{2} + \II_{1} \otimes \, \S_{2}
	\end{equation*}
	where $ \II_{i} $ is the identity operator for the spin subspace spanned by the $ i $th particle's spin basis states $ \ket{\{s_{i}m_{i}\}} $. Similarly, the total spin state $ \ket{s_{1}s_{2}m_{1}m_{2}} $ is written
	\begin{equation*}
		\ket{s_{1}s_{2}m_{1}m_{2}}  = \ket{s_{1}m_{1}} \otimes \ket{s_{2}m_{2}} 
	\end{equation*}
	\textit{Shorthand Notation}: Since the spins $ s_{1} $ and $ s_{2} $ are both fixed at the constant values $ s_{1,2} = 1/2 $, we can leave the $ s_{i} $ terms implicit and write the spin state simply as
    \begin{equation*}
        \ket{s_{1}s_{2}m_{1}m_{2}} \equiv \ket{m_{1}} \otimes \ket{m_{2}} 
    \end{equation*}
        
    \item In the $ \ket{m_{1}m_{2}} $ notation, the total spin operator $ S_{z} $ acts on the basis state as
    \begin{align*}
        S_{z}\ket{m_{1}m_{2}} &= (S_{z_{1}}\otimes \II_{2} + \II_{1}\otimes S_{z_{2}})\ket{m_{1}} \otimes \ket{m_{2}}\\
        &= m_{1}\hbar \ket{m_{1}} \otimes \ket{m_{2}} + \ket{m_{1}}\otimes m_{2}\hbar \ket{m_{2}}\\
        &=(m_{1} + m_{2}) \hbar \ket{m_{1}m_{2}}
    \end{align*}
    In other words---the total spin operator $ S_{z} $ has the expected property of an angular momentum operator in that it has the eigenvalue $ m_{1} + m_{2}F $ when acting on the state $ \ket{m_{1}m_{2}} $. 

    \item As any angular momentum quantity, the total spin operator $ \vec{S} $ obeys the usual angular momentum commutation relations, which we can derive as follows:
    \begin{align*}
        [S_{\alpha}, S_{\beta}] &= \big[ S_{\alpha_{1}} \otimes \II_{2} + I_{1} \otimes S_{\alpha_{2}}, S_{\beta_{1}}\otimes \II_{2} + I_{1}\otimes S_{\beta_{2}} \big]\\
        & = [S_{\alpha_{1}} \otimes \II_{2}, S_{\beta_{1}}\otimes\II_{2}] + [\II_{2} \otimes S_{\alpha_{2}}, \II_{1} \otimes S_{\beta_{2}}]\\
        & = [S_{\alpha_{1}}, S_{\beta_{1}}] \otimes \II_{2} + \II_{1} \otimes [S_{\alpha_{2}}, S_{\beta_{2}}]\\
        & = i \hbar \epsilon_{\alpha\beta\gamma}S_{\gamma},
    \end{align*}
    where the last equality uses the identities $ [S_{\alpha_{i}}, S_{\beta_{i}}] = i \hbar \epsilon_{\alpha\beta\gamma}S_{\gamma_{i}} $. Note also the use of $ [S_{\alpha_{1}} \otimes \II_{2}, \II_{1} \otimes S_{\beta_{2}}] = 0 $, which follows from $ \big[S_{\alpha_{i}}, S_{\beta_{j}}\big] = i \hbar \delta_{ij}\epsilon_{\alpha \beta \gamma}S_{\gamma_{j}} $ (a result of $ S_{i} $ and $ S_{j} $ acting on independent Hilbert spaces).
    
    \item The total spin operator $ \S $ is often analyzed in terms of ladder operators, just like the spin operator for a single particle. The total ladder operators are defined as
    \begin{equation*}
        S_{\pm} = S_{x} \pm iS_{y} = S_{\pm_{1}} \otimes \II_{2} + \II_{1} \otimes S_{\pm_{2}}.
    \end{equation*}

    \item The basis states $ \{\ket{m_{1}m_{2}}\} $ span the system's total spin vector space and are eigenstates of the total angular momentum operator $ S_{z} $, individual operators $ S_{z_{i}} $ and individual operators $ S_{i}^{2} $. 

    Note, however, that the states $ \ket{m_{1}m_{2}} $ are not eigenstates of the total angular momentum operator $ S^{2} $---only the states $ \ket{sm} $ are eigenstates of $ S^{2} $. The states $ \ket{sm} $ obey
    \begin{equation*}
        S^{2}\ket{sm} = s(s+1)\hbar^{2}\ket{sm} \qquad \text{and} \qquad S_{z}\ket{sm} = m\hbar \ket{sm}
    \end{equation*}
    
    \item Although $ \ket{m_{1}m_{2}} $ are not directly eigenstates of $ S^{2} $, we can expand the eigenstates $ \ket{sm} $ in the $ \ket{m_{1}m_{2}} $ basis in the form
    \begin{equation*}
        \ket{sm} = \sum_{m_{1}m_{2}} s_{m_{1}m_{2}}\ket{m_{1}m_{2}}
    \end{equation*}
    where we sum over all $ m_{i} = \pm 1/2 $ obeying the condition $ m = m_{1} + m_{2} $. 
    
    \item Next, we introduce the earlier arrow notation, which generalizes to a two-particle system with states $ \ket{m_{1}m_{2}}  $ as follows:
    \begin{align*}
        \ket{\tfrac{1}{2}\tfrac{1}{2}} &= \ket{\ua} \otimes \ket{\ua} \equiv \ket{\ua\ua}\\
        \ket{-\tfrac{1}{2}\tfrac{1}{2}} &= \ket{\da} \otimes \ket{\ua} \equiv \ket{\da\ua}\\
        \ket{\tfrac{1}{2}-\tfrac{1}{2}} &= \ket{\ua} \otimes \ket{\da} \equiv \ket{\ua\da}\\
        \ket{-\tfrac{1}{2}-\tfrac{1}{2}} &= \ket{\da} \otimes \ket{\da} \equiv \ket{\da\da}
    \end{align*}
    Don't forget that the state $ \ket{m_{1}m_{2}} $ is short for the state $ \ket{s_{1}m_{2}s_{2}m_{2}} $---the $ s_{i} $ are left out for conciseness but are still implicitly there. Note also that we have left tensor product $ \otimes $ implicit in the last equality of each line. Leaving the tensor product implicit in multi-particle spin systems in conventional, just like leaving the hat on operators implicit. The need for the tensor product is usually clear from context.

    \item The largest possible value of $ m $ for the state $ \ket{\ua \ua} $ is $ m = \tfrac{1}{2} + \tfrac{1}{2} = 1 $. The corresponding $ \ket{sm} $ state for $ m = 1 $, i.e. $ \ket{sm} = \ket{11} $ is also an eigenstate of $ S^{2} $, as shown below:
    \begin{equation*}
        S^{2}\ket{sm} = S^{2}\ket{11} = 1(1 + 1)\hbar^{2}\ket{11} = 2\hbar^{2}\ket{\ua\ua}.
    \end{equation*}
    
    \item Next, we applyg the ladder operator $ S_{-} $ to the state $ \ket{11} $, which gives
    \begin{equation*}
        S_{-}\ket{11} = \hbar \sqrt{1(1+1) - 1(1-1)} \ket{10} = \sqrt{2}\hbar \ket{10},
    \end{equation*}
    which we can also write in the form
    \begin{align*}
        S_{-}\ket{11} & = \big( S_{-_{1}}\II_{2} + \II_{1} S_{-_{2}} \big)\ket{\ua\ua} = \hbar \ket{\da\ua} + \hbar \ket{\ua\da}\\
        &= \hbar \big( \ket{\da\ua} + \ket{\ua\da} \big)
    \end{align*}
    In other words, we have derived the identity
    \begin{equation*}
        \sqrt{2}\hbar \ket{10} = \hbar \big( \ket{\da\ua} + \ket{\ua\da} \big) \implies \ket{10} = \frac{\hbar}{\sqrt{2}} \big( \ket{\da\ua} + \ket{\ua\da} \big)
    \end{equation*}
    Following a similar procedure with $ s = 1 $ and $ m = -1 $, we could derive the identity
    \begin{equation*}
        \ket{1, -1} = \ket{\da\da}.
    \end{equation*}
    The above three states $ \ket{sm} = \ket{1m} $ with $ m = -1, 0, 1 $ are called the \textit{triplet} states for a two-particle spin system with individual spins $ s_{1,2} = 1/2 $.

    \item The fourth possible state for the two-particle spin $ 1/2 $ system is the \textit{singlet} state $ \ket{sm} = \ket{00} $ with $ s = 0 $ and $ m = 0 $. We cannot construct the single state using the ladder operators, since the ladder operators preserve the quantum number $ s $, and we have no way of generating $ s = 0 $ from a $ s = 1 $ state. 

    Instead, we generate the singlet state by expanding the state $ \ket{sm} = \ket{00} $ in the $ \ket{m_{1}m_{2}} $ basis:
    \begin{equation*}
        \ket{00} = c_{\tfrac{1}{2},-\tfrac{1}{2}} \ket{\ua\da} + c_{-\tfrac{1}{2}, \tfrac{1}{2}}\ket{\da \ua}
    \end{equation*}
    Because the states $ \ket{10} $ and $ \ket{00} $ have different values of $ s $, they are orthogonal and obey $ \braket{10}{00} = 0 $, which implies
    \begin{equation*}
        c_{-\tfrac{1}{2}, \tfrac{1}{2}} = - c_{\tfrac{1}{2}, -\tfrac{1}{2}}
    \end{equation*}
    The complete of triplet and singlet states is thus
    \begin{align*}
        \ket{1m} &= 
        \begin{cases}
            \ket{\ua\ua} & m = 1\\
            \tfrac{1}{\sqrt{2}} \big( \ket{\ua\da} + \ket{\da\ua} \big) & m = 0\\
            \ket{\da\da} & m = -1
        \end{cases}\\
        \ket{00} &= \tfrac{1}{\sqrt{2}} \big( \ket{\ua\da} - \ket{\da \ua} \big)
    \end{align*}
   

    \item \textit{Note on notation}: From here forward, in addition to leaving out the tensor product symbol $ \otimes $, we will also leave the identity operator implicit when writing the product of independent operators. Here is an example:
    \begin{equation*}
        S^{2} = S_{1}^{2}\otimes\II_{2} + \II_{1} \otimes S_{2}^{2} \to S^{2} = S_{1}^{2} + S_{2}^{2}
    \end{equation*}
    The implicit presence of the tensor product and relevant identity operators are understood from context.
	
\end{itemize}

\textbf{Heisenberg Coupling}
\begin{itemize}
    \item The Heisenberg coupling involves two particles with spin $ s_{i} = 1/2 $ and reads
    \begin{equation*}
        H = J_{0} \S_{1} \cdot \S_{2}
    \end{equation*}
    where $ J_{0} $ is called the exchange coupling constant. When $ J_{0} < 0 $, system's spins $ \S_{1} $ and $ \S_{2} $ are aligned in the ground state (ferromagnetic coupling). When $ J_{0} > 0 $, the spins $ \S_{1} $ and $ \S_{2} $ are oppositely oriented in the ground state (anti-ferromagnetic coupling). 

    \item Next, we use the identity
    \begin{equation*}
        S^{2} = \S \cdot \S = (\S_{1} + \S_{2}) \cdot (\S_{1} + \S_{2}) = S_{1}^{2} + S_{2}^{2} + 2 \S_{1} \cdot \S_{2}
    \end{equation*}
    to write the Heisenberg coupling term in the form
    \begin{equation*}
        H = \frac{J_{0}}{2}\left( S^{2} - \frac{3}{2} \hbar^{2} \right).
    \end{equation*}
    
    \item Note that the coupling term is rotationally invariant, which implies the commutation relation $ [\S, H] = 0 $. Because of the relation $ [\S, H] = 0 $, the quantities $ S^{2} $ and $ S_{z} $ are conserved, and the singlet and triplet states are eigenstates of the equation
    \begin{equation*}
        H \ket{sm} = \frac{J_{0}\hbar^{2}}{2} \left( s(s + 1) - \frac{3}{2} \right)\ket{sm}
    \end{equation*}
    with corresponding energy eigenvalues
    \begin{equation*}
        E_{s} = J_{0}\hbar^{2} 
        \begin{cases}
            \frac{1}{4} & s = 1 \quad \text{(triplet)}\\
            -\frac{3}{4} & s = 0 \quad \text{(singlet)}
        \end{cases}
    \end{equation*}
    The ground state (with lowest energy) is thus the singlet state, while the triplet states are triply degenerate with energy $ \Delta E = J_{0}\hbar^{2} $ above the ground state.

\end{itemize}

\subsection{Addition of Angular Momenta with the Clebsch Gordan Coefficients}
\begin{itemize}
    \item In the previous section, we consider addition of angular momentum operators $ \S_{1} $ and $ S_{2} $ corresponding to particles with spins $ s_{1} = s_{2} = 1/2 $. In this section, we will generalize the addition of angular momenta to arbitrary operators $ \J_{1} $ and $ \J_{2} $ that obey the usual angular momentum commutation relations
    \begin{align*}
        \big[ J_{\alpha_{i}}, J_{\beta_{j}} \big] = i \hbar \delta_{ij}\epsilon_{\alpha\beta\gamma}J_{\gamma_{j}} 
    \end{align*}
    
    \item Leaving the tensor product and identity operators implicit, the total angular momentum operator $ \J $ and corresponding basis states are
    \begin{equation*}
        \ket{j_{1}m_{1}j_{2}m_{2}} = \ket{j_{1}m_{1}}\ket{j_{2}m_{2}} \qquad \text{and} \qquad \J = \J_{1} + \J_{2}.
    \end{equation*}
    
    \item By making the substitution $ J_{\alpha_{i}} \to S_{\alpha_{i}} $ and $ (j_{i}, j) \to (s_{i}, s) $, we can reuse most of the identities related to addition of angular momentum that were derived in the previous section.

    Be begin by expanding the eigenstatates of the operators $ \J^{2} $ and $ J_{z} $ in the basis $ \big\{ \ket{j_{1}m_{1}j_{2}m_{2}} \big\} $ in the form
    \begin{equation*}
        \ket{j_{1}j_{2}jm} = \sum_{m_{1} = - j_{1}}^{j_{1}}\sum_{m_{2} = - j_{2}}^{j_{2}} \braket{j_{1}m_{1}j_{2}m_{2}}{jm} \ket{j_{1}m_{1}j_{2}m_{2}}
    \end{equation*}
    The coefficients $ \braket{j_{1}m_{1}j_{2}m_{2}}{jm} $ in the expansion are called the Clebsch-Gordon coefficients and encode the transformation between the total angular momentum basis $ \ket{jm} $ and the product basis $ \ket{m_{1}m_{2}} $. 

    \item Finally, we note that just like in the addition of spins, when adding angular momenta the total quantum number $ m $ must obey $ m = m_{1} + m_{2} $ and that the values of $ j $ must satisfy the corresponding inequality
    \begin{equation*}
        \abs{j_{1} + j_{2}} \leq j \leq j_{1} + j_{2}
    \end{equation*}

\end{itemize}

% A state with ``good'' angular momentum is an eigenstate of the total angular momentum operator J^{2}
\subsubsection{Example: Addition of Angular Momenta in the Case $ l \cross 1/2 $}
\begin{itemize}
    \item As an exercise in the addition of angular momentum, we will find the total angular momentum basis representation of an electron in the first excited state of a hydrogen atom. The excited electron has angular momentum $ \J_{1} = \L $ and basis states $ \{\ket{lm_{l}}\} $ while the proton in the hydrogen nucleus has angular momentum $ \J_{2} = \S $ and basis states $ \{\ket{\ua}, \ket{\da}\} $. 

    The total angular momentum is $ \J = \L + \S $, and the corresponding basis is
    \begin{equation*}
        \{ \ket{j_{1}j_{2}jm} \} \equiv \{\ket{lsjm}\} = \left\{ \ket{l \tfrac{1}{2}jm} \right\}
    \end{equation*}
    where the spin $ 1/2 $ proton's angular momentum $ j_{2} $ is fixed at $ j_{2} \equiv s = 1/2 $. For $ l = 0 $ and $ l= 1 $, the electron's basis states are the spherical harmonics
    \begin{equation*}
        Y_{lm_{l}}(\theta, \phi) = \braket{\r}{lm_{l}}
    \end{equation*}
    Finally, recall that the angular momentum quantum numbers must satisfy
    \begin{equation*}
        \abs{s + l} \leq j \leq s + l
    \end{equation*}
    

    \item We begin with the ground state $ l = 0 $ and $ s = 1/2 $, in which case the total angular momentum quantum number can be only $ j = 1/2 $. The corresponding states in the total angular momentum $ \ket{lsmj} $ and product $ \ket{lm_{l}sm_{s}} $ bases are
    \begin{equation*}
        \ket{0 \tfrac{1}{2} \tfrac{1}{2} \tfrac{1}{2}} = \ket{00}\ket{\ua} \qquad \text{and} \qquad \ket{0 \tfrac{1}{2} \tfrac{1}{2} -\tfrac{1}{2}}  = \ket{00}\ket{\da}
    \end{equation*}
    
    \item We now consider the first excited electron state with $ l = 1 $, for which we can have both $ j = 3/2 $ and $ j = 1/2 $. Here is the plan: We will begin with the state with the largest value of $ m $, which occurs for $ j = 3/2 $, and generate the $ j = 3/2 $ states with lower $ m $ using the ladder operator $ J_{-} $. Finally, we find the $ j = 1/2 $ state using the orthonormality of the basis states. 

    \item The state with the largest possible value of $ m $ is $ \ket{1 \tfrac{1}{2} \tfrac{3}{2} \tfrac{3}{2}} = \ket{11}\ket{\ua} $. We then apply the ladder operator $ J_{-} = S_{-} + L_{-} $ to determine the state with $ \ket{jm} = \ket{\tfrac{3}{2}\tfrac{1}{2}} $. This reads:
    \begin{align*}
        J_{-} \ket{1 \tfrac{1}{2} \tfrac{3}{2}\tfrac{3}{2}} &= \hbar \sqrt{3} \ket{1 \tfrac{1}{2} \tfrac{3}{2} \tfrac{1}{2}}\\
        (L_{-} + S_{-}) \ket{11}\ket{\ua} &= \hbar \sqrt{2} \ket{10}\ket{\ua} + \hbar \ket{11} \ket{\da}\\
        \ket{1 \tfrac{1}{2}\tfrac{3}{2}\tfrac{1}{2}} &= \sqrt{\tfrac{2}{3}}\ket{10}\ket{\ua} + \sqrt{\tfrac{1}{3}} \ket{11}\ket{\da}
    \end{align*}
    
    \item We then repeat the procedure to generate the state with $ \ket{jm} = \ket{\tfrac{3}{2}-\tfrac{1}{2}} $. This reads
    \begin{align*}
        J_{-}\ket{1 \tfrac{1}{2}\tfrac{3}{2}\tfrac{1}{2}} &= 2 \hbar \ket{1 \tfrac{1}{2}\tfrac{3}{2}-\tfrac{1}{2}}\\
        (L_{-} + S_{-})\ket{10}\ket{\ua} &= \sqrt{2}\hbar \ket{1-1}\ket{\ua} + \hbar \ket{10}\ket{\da}\\
        (L_{-} + S_{-}) \ket{11}\ket{\da} &= \sqrt{2}\hbar \ket{10}\ket{\da} + 0\\
        \ket{1 \tfrac{1}{2}\tfrac{3}{2}-\tfrac{1}{2}} & = \sqrt{\tfrac{1}{3}} \ket{1-1}\ket{\ua} + \sqrt{\tfrac{2}{3}} \ket{10}\ket{\da}
    \end{align*}

    \item On more application of the ladder operator leads to the state with $ \ket{jm} = \ket{\tfrac{3}{2}-\tfrac{3}{2}} $, which is
    \begin{equation*}
        \ket{1 \tfrac{1}{2}\tfrac{3}{2} -\tfrac{3}{2}} = \ket{1-1}\ket{\da}
    \end{equation*}
    
    \item Finally, we determine the states with $ j = 1/2 $ and $ \ket{jm} = \ket{\tfrac{1}{2} \pm \tfrac{1}{2}} $ with the orthogonality condition
    \begin{equation*}
        \braket{1 \tfrac{1}{2}\tfrac{3}{2}\pm \tfrac{1}{2}}{1 \tfrac{1}{2}\tfrac{1}{2}\pm \tfrac{1}{2}} = 0,
    \end{equation*}
    since these two states have different values of $ j $. The desired states are then
    \begin{align*}
        \ket{1 \tfrac{1}{2}\tfrac{1}{2}\tfrac{1}{2}} &= \sqrt{\tfrac{2}{3}}\ket{10} \ket{\ua} - \sqrt{\tfrac{1}{3}} \ket{11}\ket{\da}\\
        \ket{1 \tfrac{1}{2}\tfrac{1}{2}-\tfrac{1}{2}} &= -\sqrt{\tfrac{2}{3}}\ket{1-1} \ket{\ua} - \sqrt{\tfrac{1}{3}} \ket{10}\ket{\da}\\
    \end{align*}
\end{itemize}

\subsubsection{TODO: Table of Clebsch-Gordan Coefficients}

\newpage
\section{Perturbation Theory}

\subsection{The Rayleigh-\Schro Method for a Non-Degenerate Spectrum}
\begin{itemize}
    \item We consider quantum system with a \Ham of the form 
    \begin{equation*}
        H = H_{0} + H',
    \end{equation*}
    where the term $ H_{0} $ is a good approximation for the system, while the term $ H' $ is a \textit{perturbation} term with a secondary effecto on the system. We assume we are able to analytically solve the stationary \Schro equation for $ H_{0} $, which reads
    \begin{equation*}
        H_{0} = \ket{n_{0}} = E_{n}^{(0)} \ket{n_{0}}.
    \end{equation*}
    More so, we assume $ H_{0} $'s eigenstates are orthonormal and obey $ \braket{m_{0}}{n_{0}} = \delta_{mn} $, and that the energy eigenvalues $ E_{n}^{(0)} $ are non-degenerate. 

    We write the perturbation term in the form $ H' = \lambda V $ where $ V $ has units of energy and $ \lambda $ is a dimensional parameter encoding the strength of the perturbation. 

    \item Our goal is to solve the stationary \Schro equation for the \textit{total} \Ham $ H = H_{0} + H' $, which reads
    \begin{equation*}
        H \ket{n} = E_{n}\ket{n}.
    \end{equation*}
    Our first step is to expand the desired eigenstates $ \ket{n} $ and energy eigenvalues $ E_{n} $ in powers of the perturbation parameter $ \lambda $ in the form
    \begin{align*}
        &E_{n} = E_{n}^{(0)} + \lambda E_{n}^{(1)} + \lambda^{2} E_{n}^{(2)} + \cdots\\
        &\ket{n} = \ket{n_{0}} + \lambda \ket{n_{1}} + \lambda^{2} \ket{n^{2}} + \cdots,
    \end{align*}
    where $ E_{n}^{(j)} $ and $ \ket{n_{j}} $ are progressively higher-order corrections to the total \Ham $ H $'s $ n $-th eigenvalue $ E_{n} $ and eigenstate $ \ket{n} $, respectively.

    \item We proceed by multipliying the equation for $ \ket{n} $ by $ \bra{n_{0}} $, which gives
    \begin{equation*}
        \braket{n_{0}}{n} = 1 + \lambda \braket{n_{0}}{n_{1}} + \lambda^{2}\braket{n_{0}}{n_{2}} + \cdots
    \end{equation*}
    We then make an important assumption: we assume the complete solution for $ \ket{n} $ is well approximated by the lowest-order approximation $ \ket{n_{0}} $, allowing us to temporarily assume $ \braket{n_{0}}{n} = 1 $. Under this assumption, the above equation simplifies to
    \begin{equation*}
        \lambda \braket{n_{0}}{n_{1}} + \lambda^{2}\braket{n_{0}}{n_{2}} + \cdots = 0,
    \end{equation*}
    which is satisfied by all $ \ket{n_{j}} $ only if $ \braket{n_{0}}{n_{j}} = \delta_{0j} $ for all $ j \in \mathbb{N} $.

    \item We then subsitute the expressions for $ H $, $ \ket{n} $ and $ E_{n} $ into the stationary \Schro equation $ H \ket{n} = E_{n} \ket{n} $ to get
    \begin{equation*}
        (H_{0} + \lambda V)\big( \ket{n_{0}} + \lambda \ket{n_{1}}  + \cdots \big) = \big( E_{n}^{(0)} + \lambda E_{n}^{(1)} + \cdots \big)\big( \ket{n_{0}} + \lambda \ket{n_{1}} + \cdots \big).
    \end{equation*}
    We then equate then equate the coefficients of each power of $ \lambda $. The result up to $ \lambda^{j} = \lambda^{3} $ is
    \begin{equation*}
        \begin{array}{l|r c l}
            \lambda^{0} & H_{0} \ket{n_{0}} & = & E_{n}^{(0)} \ket{n_{0}}\\
            \lambda^{1} & H_{0}\ket{n_{1}} + V \ket{n_{0}} & = & E_{n}^{(0)}\ket{n_{1}} + E_{n}^{(1)}\ket{n_{0}}\\
            \lambda^{2} & H_{0}\ket{n_{2}} + V \ket{n_{1}} & = & E_{n}^{(0)}\ket{n_{2}} + E_{n}^{(1)}\ket{n_{1}} + E_{n}^{(2)}\ket{n_{0}}\\
            \lambda^{3} & H_{0}\ket{n_{3}} + V \ket{n_{2}} & = & E_{n}^{(0)}\ket{n_{3}} + E_{n}^{(1)}\ket{n_{2}} + E_{n}^{(2)}\ket{n_{1}} + E_{n}^{(3)}\ket{n_{1}}
        \end{array}
    \end{equation*}
    The first equation is trivial and represents the known eigenvalue relation for the \Ham $ H_{0} $. 

    The second equation with $ \lambda^{1} $ is more useful. We first multiply the equation by $ \bra{n_{0}} $ to get
    \begin{equation*}
        E_{n}^{(0)} \braket{n_{0}}{n_{1}} + \mel{n_{0}}{V}{n_{0}} = E_{n}^{(0)}\braket{n_{0}}{n_{1}} + E_{n}^{(1)}\braket{n_{1}}{n_{1}}.
    \end{equation*}
     We then apply the orthonormality condition $ \braket{n_{0}}{n_{j}} = \delta_{0j} $ to get 
     \begin{equation*}
         E_{n}^{(1)} = \mel{n_{0}}{V}{n_{0}} = V_{nn}.
     \end{equation*}

     \item We can perform an analogous procedure with the equations for higher powers $ \lambda^{j} $ to get expressions for the higher-order corrections $ E_{n}^{(1)} $. For example, using the equation for $ \lambda^{2} $ and multiplying by $ \bra{n_{0}} $ as before gives
    \begin{equation*}
        E_{n}^{(0)} \braket{n_{0}}{n_{2}} + \mel{n_{0}}{V}{n_{1}} = E_{n}^{(0)}\braket{n_{0}}{n_{2}} + E_{n}^{(1)}\braket{n_{0}}{n_{1}} + E_{n}^{(2)}\braket{n_{0}}{n_{0}}, 
    \end{equation*}
    which produces $ E_{n}^{(2)} = \mel{n_{0}}{V}{n_{1}}$ after applying $ \braket{n_{0}}{n_{j}} = \delta_{0j} $. 

    In general, the expression for the $ j $-th correction to the energy $ E_{n} $ is
    \begin{equation*}
        E_{n}^{(j)} = \mel{n_{0}}{V}{n_{j-1}}.
    \end{equation*}
    
    \item Recall that we have assumed we can solve the stationary \Schro equation $ H \ket{n} = E_{n} \ket{n} $, which means we can diagonalize the \Ham $ H_{0} $. Working in the $ H_{0} $ basis, we can write the identity operator in the form
    \begin{equation*}
        \II = \sum_{m}\ket{m_{0}}\bra{m_{0}}.
    \end{equation*}
    We can then expand each higher-order correction $ \ket{n_{j}} $ in the $ H_{0} $ basis to get
    \begin{equation*}
        \ket{n_{j}} = \sum_{m \neq n}\ket{m_{0}}\braket{m_{0}}{n_{j}}
    \end{equation*}
    Note that the term $ \ket{n_{0}} $ is left out of the sum, which corresponds to the identity $ \braket{n_{0}}{n_{j}} = 0 $. 

    With this expression for $ \ket{n_{j}} $ in the $ H_{0} $ in mind, we multiply the earlier equation of $ \lambda^{1} $ by $ \ket{m_{0}} $ and then apply $ \ket{n_{1}} = \sum_{m \neq n}\ket{m_{0}}\braket{m_{0}}{n_{1}} $ to get
    \begin{equation*}
        \left( E_{n}^{(0)} - E_{m}^{(0)} \right)\braket{m_{0}}{n_{1}} = \mel{m_{0}}{V}{n_{0}} = V_{mn}
    \end{equation*}
    The coefficients in the expansion of $ \ket{n_{1}} $ are thus
    \begin{equation*}
        \braket{m_{0}}{n_{1}} = \frac{V_{mn}}{E_{n}^{(0)} - E_{m}^{(0)}},
    \end{equation*}
    which implies the first-order correction $ \ket{n_{1}} $ to the state $ \ket{n} $ is
    \begin{equation*}
        \ket{n_{1}} = \sum_{m\neq n} = \frac{V_{mn}}{E_{n}^{(0)} - E_{m}^{(0)}} \ket{m_{0}}.
    \end{equation*}

    \item Using the just-derived result for $ \ket{n_{1}} $, the second-order energy correction $ E_{n}^{(2)} $ is then
    \begin{equation*}
        E_{n}^{(2)} = \mel{n_{0}}{V}{n_{1}} = \sum_{m \neq n} \frac{\abs{V_{mn}}^{2}}{E_{n}^{(0)} - E_{m}^{(0)}}.
    \end{equation*}
    Note that in practice, the second-order energy correction $ E_{n}^{(2)} $ and the first-order wavefunction correction $ \ket{n_{1}} $ usually lead to satisfactory approximations of the exact quantities $ E_{n} $ and $ \ket{n} $.

    \item Next, as an exercise in the Rayleigh-\Schro method, we will determine the second-order wavefunction correction $ \ket{n_{2}} $, which will also reveal the third-order energy correction $ E_{n}^{(3)} $. 

    We begin by multiplying the earlier equation for $ \lambda^{2} $ by $ \bra{m_{0}} $, which gives
    \begin{align*}
        \mel{m_{0}}{H_{0}}{n_{2}} + \mel{m_{0}}{V}{n_{1}} = E_{n}^{(0)} \braket{m_{0}}{n_{2}} + E_{n}^{(1)} \braket{m_{0}}{n_{1}} + E_{n}^{(2)}\braket{m_{0}}{n_{0}}
    \end{align*}
    This equation, after substituting in the results derived earlier in this section, comes out to, in order,
    \begin{equation*}
        E_{m}^{(0)}\braket{m_{0}}{n_{2}} + \sum_{l \neq n}\frac{V_{ml}V_{ln}}{E_{n}^{(0)} - E_{l}^{(0)}} = E_{n}^{(0)}\braket{m_{0}}{n_{2}} + E_{n}^{(1)} \frac{V_{mn}}{E_{n}^{(0)} - E_{m}^{(0)}} + E_{n}^{(2)}\cdot 0.
    \end{equation*}
    We then substitute in $ E_{n}^{(1)} = V_{nn} $ and rearrange to get
    \begin{equation*}
        \sum_{l \neq n} \frac{V_{ml}V_{ln}}{E_{n}^{(0)} - E_{l}^{(0)}} = \left( E_{n}^{(0)} - E_{m}^{(0)} \right)\braket{m_{0}}{n_{2}} + \frac{V_{nn}V_{mn}}{E_{n}^{(0)} - E_{m}^{(0)}}
    \end{equation*}
    The coefficients $ \braket{m_{0}}{n_{2}} $ in the expansion of $ E_{n}^{(3)} $ are thus
    \begin{equation*}
        \braket{m_{0}}{n_{2}} = \sum_{n \neq l} \frac{V_{ml}V_{ln}}{\big( E_{n}^{(0)} - E_{m}^{(0)} \big)\big( E_{n}^{(0)} - E_{l}^{(0)} \big)} - \frac{V_{mn}v_{nn}}{\big( E_{n}^{(0)} - E_{m}^{(0)} \big)^{2}},
    \end{equation*}
    and the second-order wavefunction correction is
    \begin{equation*}
        \ket{n_{2}} = \sum_{m \neq n} \left[ \sum_{n \neq l} \frac{V_{ml}V_{ln}}{\big( E_{n}^{(0)} - E_{m}^{(0)} \big)\big( E_{n}^{(0)} - E_{l}^{(0)} \big)} - \frac{V_{mn}V_{nn}}{\big( E_{n}^{(0)} - E_{m}^{(0)} \big)^{2}}  \right] \ket{m_{0}}
    \end{equation*}

    \item In terms of the just-derived $ \ket{n_{2}} $, the third-order energy correction $ E_{n}^{(3)} $ is
    \begin{equation*}
        E_{n}^{(3)} = \sum_{m \neq n} \braket{m_{0}}{n_{2}} V_{nm},
    \end{equation*}
    where the expression for for $ \ket{n_{2}} $ is left out for conciseness.
    
    
    \item Finally, we consider the normalization of the $ \ket{n} $ states. Up to the first-order correction in $ \lambda $, the $ \ket{n} $ are already normalized, since
    \begin{align*}
        \braket{n}{n} &= \big( \ket{n_{0}} + \lambda \ket{n_{1}} \big)\big( \ket{n_{0}} + \lambda \ket{n_{1}} \big)\\
        &= \braket{n_{0}}{n_{0}} + \lambda \braket{n_{0}}{n_{1}} + \lambda \braket{n_{1}}{n_{0}} + \mathcal{O}(\lambda^{2})\\
        & = 1 + \mathcal{O}(\lambda^{2})
    \end{align*}
    However, the $ \ket{n} $ are not normalized in the second-order correction, in which case we have
    \begin{align*}
        \braket{n}{n} &= \braket{n_{0}}{n_{0}} + \lambda^{2} \braket{n_{1}}{n_{1}} + \mathcal{O}(\lambda^{3})\\
        & = 1 + \lambda^{2} \sum_{m \neq n} \frac{\abs{V_{nm}}^{2}}{\big( E_{n}^{(0)} - E_{m}^{(0)} \big)^{2}} + \mathcal{O}(\lambda^{3})
    \end{align*}
    If we want the $ \ket{n} $ states to be normalized we have two options:
    \begin{enumerate}
        \item Renormalize the entire $ \ket{n} $ state into
        \begin{equation*}
           \ket{n} \to \bigg[ \sum_{j} \braket{n_{j}}{n_{j}} \bigg]^{-1/2} \ket{n}
        \end{equation*}
        
        \item Renormalize only the second-order correction via
        \begin{equation*}
            \ket{n_{2}} \to \ket{n_{2}} - \frac{\lambda}{2} \sum_{m \neq n} \frac{\abs{V_{nm}}^{2}}{\big( E_{n}^{(0)} - E_{m}^{(0)} \big)^{2}}\ket{n_{0}}.
        \end{equation*}
        In terms of this renormalization, the product $ \braket{n}{n} $ comes out to
        \begin{equation*}
            \braket{n}{n} = 1 + \mathcal{O}(\lambda^{3}).
        \end{equation*}
        
    \end{enumerate}
    
\end{itemize}

\textbf{Example: The Anharmonic Oscillator: TODO}

\subsection{Degenerate Perturbation Theory}

\begin{itemize}
    \item As in the previous subsection, we again consider a \Ham of the form
    \begin{equation*}
        H = H_{0} + \lambda V.
    \end{equation*}
    This time, however, we assume the energy eigenvalues of the unperturbed \Ham $ H_{0} $ are degenerate. As before, our goal is to solve the stationary \Schro equation
    \begin{equation*}
        H \ket{n} = E \ket{n}
    \end{equation*}
    for the \Ham $ H $'s eigenstates $ \ket{n} $ and eigenvalues $ E_{n} $. 
\end{itemize}

\subsubsection{Example: Doubly Degenerate Energies}
\begin{itemize}
    \item We begin with a concrete example: when $ H_{0} $'s eigenstates are doubly degenerate. The beginning steps of the analysis are the same as for a non-degenerate spectrum. We first expand $ H $'s eigenstates and eigenvalues in the form
    \begin{align*}
        &E_{n} = E_{n}^{(0)} + \lambda E_{n}^{(1)} + \lambda^{2} E_{n}^{(2)} + \cdots\\
        &\ket{n} = c_{1}\bket{n_{0}^{(1)}} + c_{1}\bket{n_{0}^{(2)}} + \lambda \ket{n_{1}} +  \cdots.
    \end{align*}
    Note that because the unperturbed \Ham $ H_{0} $'s energy levels are doubly degenerate, the zeroth-order energy $ E_{n}^{(0)} $ (which is associated to $ H_{0} $) corresponds to two linear independent wavefunction corrections, which we have written as the linear combination $ c_{1}\bket{n_{0}^{(1)}} + c_{1}\bket{n_{0}^{(2)}} $. 

    \item Next, working only up to first order in $ \lambda $, we substitute the expansions for $ \ket{n} $ and $ E_{n} $ into the stationary \Schro equation and equate the coefficients for $ \lambda^{0} $ and $ \lambda^{1} $ to get the three equations
    \begin{align*}
        H_{0} \bket{n_{0}^{(1)}} &= E_{n}^{(0)} \bket{n_{0}^{(1)}}\\
        H_{0} \bket{n_{0}^{(2)}} &= E_{n}^{(0)} \bket{n_{0}^{(2)}}\\
        H_{0}\ket{n_{1}} + c_{1}V \bket{n_{0}^{(1)}} + c_{2}V \bket{n_{0}^{(2)}} &= E_{n}^{(0)} \ket{n_{1}} + E_{n}^{(1)} \big( c_{1}\bket{n_{0}^{(1)}} + c_{2}\bket{n_{0}^{(2)}} \big).
    \end{align*}
    
    \item Next, we multiply the last equation through by $ \bket{n_{0}^{(1)}} $ to get an equation for the coefficients $ c_{1} $ and $ c_{2} $:
    \begin{align*}
        \bmel{n_{0}^{(1)}}{H_{0}}{n_{1}} + c_{1}\bmel{n_{0}^{(1)}}{V}{n_{0}^{(1)}} &+ c_{2}\bmel{n_{0}^{(1)}}{V}{n_{0}^{(2)}} \\
        & = E_{n}^{(0)} \bbraket{n_{0}^{(1)}}{n_{1}} + E_{n}^{(1)}\big( c_{1}\bbraket{n_{0}^{(1)}}{n_{0}^{(1)}} + c_{2} \bbraket{n_{0}^{(1)}}{n_{0}^{(2)}} \big).
    \end{align*}
    Applying the orthonormality of the $ \ket{n_{j}} $ states simplifies this to
    \begin{equation*}
        0 + c_{1}V_{11} + c_{2}V_{12} = 0 + E_{n}^{(0)}(c_{1} + 0).
    \end{equation*}
    We then perform a similar procedure in which we multiply the last equation through by $ \bket{n_{0}^{(2)}} $ to get a second equation for the coefficients $ c_{1} $ and $ c_{2} $. In one place, the first and second equations are
    \begin{align*}
        & c_{1}V_{11} + c_{2}V_{12} = E_{n}^{(0)} c_{1}\\
        & c_{1}V_{21} + c_{2}V_{22} = E_{n}^{(1)} c_{2}.
    \end{align*}
    This system of equations is a $ 2 \cross 2 $ eigenvalues problem of the form
    \begin{equation*}
        \begin{bmatrix}
            V_{11} & V_{12}\\
            V_{21} & V_{22}
        \end{bmatrix}
        \begin{bmatrix}
            c_{1}\\
            c_{2}
        \end{bmatrix}
        = E_{n}^{(1)}
        \begin{bmatrix}
            c_{1}\\
            c_{2}
        \end{bmatrix}.
    \end{equation*}
    Solving this eigenvalue problem determines the desired first-order perturbation theory correction---the eigenvalues are the two first-order energy corrections $ E_{n_{1}}^{(1)} $ and $ E_{n_{2}}^{(1)} $ and the eigenvectors determine the coefficients $ c_{1} $ and $ c_{2} $ of the eigenstate corrections $ \bket{n_{0}^{(1)}} $ and $ \bket{n_{0}^{(2)}} $. 
\end{itemize}

\subsubsection{The General $ N $-Times Degenerate Case}
\begin{itemize}
    \item The procedure for an $ N $-times degenerate energy level is a generalization of the $ N = 2 $ case and leads to a corresponding $ N \cross N $ eigenvalue problem with matrix elements $ V_{ij} $ given by
    \begin{equation*}
        V_{ij} = \bmel{n_{0}^{(i)}}{V}{n_{0}^{(j)}}
    \end{equation*}
    Solving the $ N \cross N $ eigenvalue problem's corresponding characteristic polynomial
    \begin{equation*}
        \det \big[ V_{ij} - E_{n}^{(1)}\delta_{ij} \big] = 0
    \end{equation*}
    leads to the energy eigenvalue corrections $ E_{n_{l}}^{(1)} $ where $ l = 1, 2, \ldots N $. 

    Solving the associated eigenvector problem leads to the coefficients $ c_{i_{l}} $ for the eigenstate corrections $ \bket{n_{0}^{(i)}} $.
    
    \item To final notes: In the above case, the energy corrections $ E_{n_{l}}^{(1)} $ and the diagonal matrix elements $ V_{ii} $ obey the relationship
    \begin{equation*}
        \sum_{l = 1}^{N} E_{n_{l}}^{(1)} = \sum_{i = 1}^{N}V_{ii}.
    \end{equation*}
    
    Second, when the matrix elements $ \bmel{n_{0}^{(i)}}{V}{n_{0}^{(j)}} $ are small (e.g. much less than one), it is also important to consider transitions between excited states with different values of $ n $ and $ m $. In this case, we redefine the matrix elements to be
    \begin{equation*}
        V_{ij} \to V_{ij} + \sum_{m\neq n} \frac{V_{im}V_{mj}}{E_{n}^{(0)} - E_{m}^{(0)}},
    \end{equation*}
    where we leave out the case $ m = n $ to avoid a zero in the denominator.
    
\end{itemize}

\subsection{Time-Dependent Perturbation Theory}

\begin{itemize}
    \item We now consider a quantum system described by a time-dependent \Ham of the form
    \begin{equation*}
        H(t) = H_{0} + \lambda V(t),
    \end{equation*}
    where $ H_{0} $, as before, is a time-independent, diagonalizable \Ham that reasonably approximates the system, while $ V(t) $ is a time-dependent perturbation term of secondary influence and $ \lambda $ parameterizes the perturbation's strength. In time dependent perturbation theory, the goal is to solve for the system's time-dependent wavefunction $ \ket{\psi(t)} $ given an initial state $ \ket{\psi(0)} $.

    \item We write the known eigenvalue relation for $ H_{0} $ in the form
    \begin{equation*}
        H_{0} \ket{n} = E_{n} \ket{n}
    \end{equation*}
    and assume the orthonormality relation $ \braket{m}{n} = \delta_{mn} $. We then begin our analysis by writing the initial state in the $ H_{0} $ basis:
    \begin{equation*}
        \ket{\p(0)} = \sum_{n}c_{n}(0)\ket{n}.
    \end{equation*}
    Similarly, we can write the time-dependent wavefunction in $ H_{0} $ basis in the general form
    \begin{equation*}
        \ket{\psi(t)} = \sum_{n}c_{n}(t)e^{-i \frac{E_{n}}{\hbar}t}\ket{n},
    \end{equation*}
    where the coefficients $ c_{n}(t) $ are time-dependent, since $ H(t) $ is time-dependent. 

    \item We then substitute the general expression for $ \ket{\p(t)} $ into the \Schro equation,
    \begin{equation*}
        i \hbar \pdv{t}\ket{\psi(t)} = H \ket{\psi(t)},
    \end{equation*}
    which leads to
    \begin{equation*}
        i \hbar \sum_{n}\left( \pdv{c_{n}(t)}{t}e^{-i \frac{E_{n}}{\hbar}t} - i \frac{E_{n}}{\hbar}c_{n}(t)e^{-i \frac{E_{n}}{\hbar}t} \right)\ket{n} = \sum_{n}\big( E_{n} + \lambda V(t) \big)c_{n}(t)e^{-i \frac{E_{n}}{\hbar}t} \ket{n}.
    \end{equation*}
    We then multiply the equation by the basis vector $ \bra{m} $, apply $ \braket{m}{n} = \delta_{mn} $, and cancel like terms to get
    \begin{equation*}
        i \hbar \pdv{c_{m}(t)}{t}e^{-i \frac{E_{m}}{\hbar}t} = \lambda \sum_{n} \mel{m}{V(t)}{n}e^{-i \frac{E_{n}}{\hbar}t} c_{n}(t).
    \end{equation*}
    This is a system of coupled differential equations $ c_{n}(t) $.

    \item Alternatively, we can write the above system of equations in the matrix form
    \begin{equation*}
        i \hbar \pdv{t} 
        \begin{pmatrix}
            c_{1}(t)\\
            c_{2}(t)\\
            \vdots\\
            c_{m}(t)\\
            \vdots
        \end{pmatrix}
        = \lambda
        \begin{pmatrix}
            V_{11}(t) & V_{12}(t) & \cdots & V_{1n}(t) & \cdots \\
            V_{21}(t) & V_{22}(t) & \cdots & V_{2n}(t) & \cdots \\
            \vdots & \vdots & \ddots & \vdots & \cdots \\
            V_{m1}(t) & V_{m2}(t) & \cdots & V_{mn}(t) & \cdots \\
            \vdots & \vdots &  & \vdots & \ddots 
        \end{pmatrix}
        \begin{pmatrix}
            c_{1}(t)\\
            c_{2}(t)\\
            \vdots\\
            c_{m}(t)\\
            \vdots
        \end{pmatrix},
    \end{equation*}
    or the more compact equation
    \begin{equation*}
        i \hbar \pdv{t} \vec{c}(t) = \lambda \mat{V}(t) \vec{c}(t).
    \end{equation*}
    The time-dependent matrix elements $ V_{mn}(t) $ are given by
    \begin{equation*}
        V_{mn}(t) = \mel{m}{V(t)}{n} e^{-i \frac{E_{n} - E_{m}}{\hbar}t},
    \end{equation*}
    where $ E_{n} $ and $ E_{m} $ are $ H_{0} $'s energy eigenvalues. This formalism is called the interaction or Dirac picture, and the system's quantum state is determined by the vector $ \vec{c}(t) $.

    Finally, note that the entire time-dependent theory discussed so far is exact---we will only make perturbation approximations in later sections.
    
\end{itemize}

\subsubsection{Two-State System and Rabi Oscillations}
\begin{itemize}
    \item As a simple example of the time-dependent formalism, we consider a two-state system corresponding to the two quantum states $ \ket{1} $ and $ \ket{2} $ and  $ 2\cross 2 $ Hamiltonian $ H_{0} $. We take our time-dependent perturbation term to be a harmonic oscillation of the form $ V(t) \sim e^{-i\omega t} $. A physical example of such a state would be an atom with two energy levels upon which we shine monochromatic light.

    \item We begin by writing the two state system's Hamiltonian in the form
    \begin{equation*}
        H = H_{0} + V(t),
    \end{equation*}
    where $ H_{0} $ and $ V(t) $ are given by
    \begin{equation*}
        H_{0} = 
        \begin{pmatrix}
            E_{1} & 0\\
            0 & E_{2}
        \end{pmatrix}
        \qquad \text{and} \qquad 
        V(t) = \hbar
        \begin{pmatrix}
            0 & \Delta e^{i\omega t}\\
            \Delta e^{-i\omega t} & 0
        \end{pmatrix}.
    \end{equation*}

    \item The system's time-dependent \Schro equation reads
    \begin{equation*}
        i \hbar 
        \begin{pmatrix}
            \dot{c}_{1}(t)\\
            \dot{c}_{2}(t)
        \end{pmatrix}
        = \hbar
        \begin{pmatrix}
            0 & \Delta e^{i(\omega - \tilde{\omega})t}\\
            \Delta e^{i(\omega - \tilde{\omega})t} & 0
        \end{pmatrix}
        \begin{pmatrix}
            c_{1}(t)\\
            c_{2}(t)
        \end{pmatrix}
    \end{equation*}
    where we have defined $ \hbar \tilde{\omega} = E_{2} - E_{1} $.
    
    \item We assume the system's initial state is the ground state $ \ket{\psi(0)} = \ket{1} $, which implies
    \begin{equation*}
        c_{1}(0) = 1 \qquad \text{and} \qquad c_{2}(t) = 0.
    \end{equation*}
    In this case, the probabilities $ P_{2} $ and $ P_{1} $ of the system occupying the states $ \ket{2} $ and $ \ket{1} $, respectively, oscillate according to
    \begin{equation*}
        P_{2}(t) = \frac{\Delta^{2}}{\Omega}\sin^{2}\Omega t \qquad \text{and} \qquad P_{1} = 1 - P_{2},
    \end{equation*}
    where we have defined the Rabi frequency $ \Omega $ according to
    \begin{equation*}
        \Omega = \sqrt{\Delta^{2} + \tfrac{1}{4}(\omega - \tilde{\omega})^{2}}.
    \end{equation*}
    The sinusoidal oscillations of the occupation probabilities $ P_{1} $ and $ P_{2} $ are called Rabi oscillations, and have maximum amplitude, exactly equal to one, when $ \omega = \tilde{\omega} $.
    
\end{itemize}

\subsubsection{Time-Dependent Perturbation Approach}
\begin{itemize}
    \item We now consider a perturbation that begins at the time $ t = 0 $, of the form
    \begin{equation*}
        V(t) = 
        \begin{cases}
            0 & t < 0\\
            V(t) & t \geq = 0
        \end{cases}
    \end{equation*}
    As before, we work with the \Ham $  H = H_{0} + \lambda V(t) $, where $ H_{0} $ is time-independent and diagonalizable. 

    \item We begin by assuming the system occurs in one of the $ H_{0} $ eigenstates, i.e. 
    \begin{equation*}
        \ket{\psi(0)} = \ket{m}, \qquad \text{and thus} \qquad  c_{k}^{(0)}(0) = \delta_{km}.
    \end{equation*}
    We then substitute this expression into the time-dependent \Schro equation to get
    \begin{equation*}
        i \hbar \pdv{t}c_{k}^{(1)}(t) = \sum_{n} V_{kn}(t)c_{n}^{(0)}(t) = V_{km}(t).
    \end{equation*}
    The coefficients in the first-order perturbation approximation are then 
    \begin{equation*}
        c_{k}^{(1)}(t) = \frac{1}{i \hbar} \int_{0}^{t}V_{km}(t')\diff t'.
    \end{equation*}
    
    \item Using the just-derived $ c_{k}^{(1)} $, the state $ \ket{\psi(t)} $ is given to first-order in $ \lambda $ as
    \begin{equation*}
        \ket{\psi(t)} = \sum_{k}\left[ c_{k}^{(0)}(t) + \lambda c_{k}^{(1)}(t) + \mathcal{O}(\lambda^{2})\right]e^{-i \frac{E_{k}}{\hbar}t}\ket{k}.
    \end{equation*}
    The first-order approximation is valid in the regime $ \big| c_{k}^{(1)}(t) \big| \ll 1 $. In practice, the first-order approximation gives an acceptable result for many physical problems.
    
    
\end{itemize}

\subsubsection{Fermi's Golden Rule}
\begin{itemize}
    \item We now consider a step-like perturbation of the form
    \begin{equation*}
        V(t) = 
        \begin{cases}
            0 & t < 0\\
            V & t \geq 0
        \end{cases}.
    \end{equation*}
    As before, we work with the general \Ham $ H = H_{0} + \lambda V(t) $ where $ H_{0} $ is time-independent and diagonalizable with eigenstates $ \ket{n} $.

    \item We assume our system is in the initital state $ \ket{\psi(0)} = \ket{m} $ (i.e. in an $ H_{0} $ eigenstate). We then aim to determine the probability of finding the system in the final state $ \ket{k} $ as a function of time.

    Since $ V $ is constant, the matrix elements $ V_{km} = \mel{k}{V}{m} $ are also constant, and the coefficients in the first-order perturbation approximation are
    \begin{equation*}
        c_{k}^{(1)}(t) = \frac{1}{i \hbar} \int_{0}^{t} V_{km}e^{-i \frac{E_{m} - E_{k}}{\hbar}t'} \diff t' = \frac{V_{km}}{i \hbar} \frac{e^{i\omega_{mk}t} - 1}{- i \omega_{mk}},
    \end{equation*}
    where we have defined $ \hbar \omega = E_{k} - E_{m} $. 

    \item In terms of $ c_{k}^{(1)}(t) $ and $ \omega_{mk} $, the probability of a transition from $ \ket{m} $ to $ \ket{k} $ is
    \begin{align*}
        P_{km} &= \abs{c_{k}^{(1)}}^{2} = \frac{\abs{V_{km}}^{2}}{\hbar^{2}} \frac{\abs{e^{-i\omega_{mk}t} - 1}^{2}}{\omega^{2}_{mk}} = \frac{\abs{V_{km}}^{2}}{\hbar^{2}} \frac{\sin^{2}\left( \tfrac{1}{2} \omega_{mk}t \right)}{\left( \tfrac{1}{2}\omega_{mk} \right)^{2}}\\
        & \equiv \frac{2\pi}{\hbar}\abs{V_{km}}^{2} \delta_{t}(E_{k} - E_{m})t,
    \end{align*}
    where we have defined the function
    \begin{equation*}
        \delta_{t}(x) \equiv \frac{1}{\pi} \frac{\sin^{2}(xt)}{x^{2}t}.
    \end{equation*}
    
    \item Interpretation: note that $ \delta_{t} $ approaches the Dirac delta function in the limit $ t \gg \frac{\hbar}{\abs{E_{k} - E_{m}}} $. In this regime, there is appreciable likelihood only for transitions two states with energies in a neighborhood of the initial energy $ E_{m} $.

    Meanwhile, for $  t \ll \frac{\hbar}{\abs{E_{k} - E_{m}}}  $, the function $ \delta_{t} $ is very wide, which corresponds to appreciable probability for transitions to states with energies in the range $ E_{m} \pm \frac{\hbar}{t} $.

    In general, the width of the function $ \delta_{t}(E_{k} - E_{m}) $ decreases with time according to
    \begin{equation*}
        \Delta E \sim \frac{\hbar}{t}.
    \end{equation*}
    
    \item Finally, we stress that this analysis of the transition probability $ P_{km} $, which uses the first-order time-dependent perturbation theory, rests on the assumption $ P_{km}(t) \ll 1 $, which is necessary for the first-order approximation to hold.

    \item Next, we consider the special case in which our system transitions from the intial $ H_{0} $ eigenstate $ \ket{m} $ to a final state $ \ket{k} $ for there are many states with energies very close to $ E_{k} $. We describe the densely-spaced states using the concept of a density of states (as in e.g. thermodynamics), which we define as
    \begin{equation*}
        \rho(E_{k}) = \dv{N}{E_{k}},
    \end{equation*}
    where $ \diff N $ represents the states available in a small energy band $ \diff E_{k} $ centered around $ E_{k} $. 

    \item The total probability for a transition to any state in the neighborhood of $ E_{k} $ is an integral over $ E_{k} $ of the form
    \begin{equation*}
        P(t) = \int P_{km}(t)\rho(E_{k})\diff E_{k}.
    \end{equation*}
    For large $ t \gg \frac{\hbar}{\abs{E_{k} - E_{m}}} $, the transition probability increases with time as
    \begin{equation*}
        P(t) = \frac{2\pi}{\hbar} \abs{V_{km}}^{2}\rho(E_{m})t.
    \end{equation*}
    The rate of change of probability with respect to time, which we will denote $ w_{km} $, is thus constant and obeys
    \begin{equation*}
        w_{m \to k} \equiv \dv{P(t)}{t} = \frac{2\pi}{\hbar} \abs{V_{km}}^{2}\rho(E_{m})
    \end{equation*}
    This important result is called Fermi's golden rule.
    
\end{itemize}

\textbf{Example: The Fermi Golden Rule and Radioactive Decay}
\begin{itemize}
    \item We consider the decay of $ N $ radioactive nuclei, where we expect that at a time $ t $ after their creation, $ \diff N $ nuclei decay in the time interval $ \diff t $, which we write in terms of the transition rate $ w(t) $ as
    \begin{equation*}
        - \diff N = N \diff P = N w(t) \diff t.
    \end{equation*}
    In general, a nucleus can decay according to different modes (e.g. $ \alpha $ decay, $ \beta $ decay, etc...), which case the total decay rate $ w $ is the sum of the decay rate for each individual mode, i.e. $ w = \sum_{i} w_{i} $, which leads to the familiar radioactive decay law
    \begin{equation*}
        N = N_{0} e^{-t/\tau}, \qquad \tau = \frac{1}{w}.
    \end{equation*}
    
    \item For a quantum system with discretely spaced energy levels, the coefficients $ c_{k}(t) $, for small times in the regime $ t \ll \frac{\hbar}{\abs{E_{k} - E_{m}}} $, are proportional to $ t $, which corresponds to a decay rate 
    \begin{equation*}
        w(t) = \frac{4}{\hbar} \abs{V_{km}}^{2}t.
    \end{equation*}
    The corresponding dependence of the number of nuclei $ N $ is 
    \begin{equation*}
        - \dv{N}{t} = \frac{4}{\hbar}\abs{V_{km}}^{2}t N \implies N(t) = N_{0} e^{-\frac{2}{\hbar}\abs{V_{km}}^{2}t^{2}}.
    \end{equation*}
    
\end{itemize}

\subsection{Time Evolution Operator for a Time-Dependent \Ham}
\begin{itemize}
    \item Recall from the section time evolution that for a time-independent \Ham $ H_{0} $, the time evolution operator reads
    \begin{equation*}
        U(t) = e^{-i \frac{H_{0}}{\hbar}t}.
    \end{equation*}
    In this section, we will generalize this result and define a time-evolution operator applicable to the time-dependent \Ham $ H = H_{0} + V(t) $. 

    \item We begin with the \Schro equation 
    \begin{equation*}
        i \hbar \pdv{t} \vec{c}(t) = \lambda \mat{V}(t) \vec{c}(t).
    \end{equation*}
    We then solve the equation iteratively for the coefficients $ c_{k}(t) $ by substituting in the first-order approximation
    \begin{equation*}
        c_{k}^{(1)}(t) = \frac{1}{i \hbar} \int_{0}^{t}V_{km}(t')\diff t'.
    \end{equation*}
    We write $ c_{k}(t) $ as the power series $ c_{k}(t) = \sum_{j}\lambda^{j}c_{k}^{(j)}(t) $, and use the first order approximation to find the second-order term via
    \begin{equation*}
        i \hbar \pdv{t} c_{k}^{(2)}(t) = \sum_{n} V_{kn} (t) c_{n}^{(1)}(t) = \frac{1}{i \hbar}\sum_{n}V_{kn}(t)\int_{0}^{t}V_{nm}(t')\diff t'.
    \end{equation*}
    We repeat the process iteratively, which leads to the series expression
    \begin{align*}
        c_{k}(t) = \delta_{km} &+ \lambda \frac{1}{ i \hbar} \int_{0}^{t} V_{km}(t')\diff t'\\
        & + \lambda^{2} \left( \frac{1}{i \hbar} \right)^{2} \sum_{n}\int_{0}^{t} \left[ \sum_{n} V_{kn}(t') \int_{0}^{t'} V_{nm}(t'')\diff t'' \right]\diff t' + \cdots.
    \end{align*}
    If we set $ \lambda = 1 $, this gives an exact, if not particularly practical, solution to the time-dependent wavefunction
    \begin{equation*}
        \ket{\psi(t)} = \sum_{n}c_{n}(t)e^{-i \frac{E_{n}}{\hbar}t}\ket{n}.
    \end{equation*}
    
    \item We can write the above results more compactly in terms of a generalized time operator in the form
    \begin{equation*}
        \vec{c}(t) = U(t)\vec{c}(0),
    \end{equation*}
    where the time-operator is defined as
    \begin{equation*}
        U(t) = \II + \sum_{n = 1}^{\infty} \left( \frac{1}{i \hbar} \right)^{n} \int_{0}^{t}V(t_{1}) \int_{0}^{t_{1}}V(t_{2}) \cdots \int_{0}^{t_{n-1}} V(t_{n} - 1)\diff t_{n-1} \cdots \diff t_{2} \diff t_{1}.
    \end{equation*}
    The potential energy operators $ V(t) $ are applied in the given order for the times
    $ 0 \leq t_{n-1} \leq \cdots \leq t_{2} \leq t_{1} \leq t $, which we can write yet more concisely by defining a time operator $ \operatorname{T} $ via
    \begin{equation*}
        U(t) = \operatorname{T} \left\{ \exp \left[ \frac{1}{i \hbar} \int_{0}^{t}V(t')\diff t'\right] \right\}.
    \end{equation*}
    The thus-defined time-evolution operator $ U(t) $ is unitary and an appropriate generalization of the expression
    \begin{equation*}
        U(t) = e^{-i \frac{H_{0}}{\hbar}t}
    \end{equation*}
    for a time-dependent \Ham $ H_{0} $.
    
\end{itemize}

\subsection{Adiabatic Transistions and Quantum Phases}
\begin{itemize}
    \item We now consider a quantum system described by a time-varying parameter $ \lambda(t) $, and thus, implicitly, a time-varying \Ham $ H(\lambda(t)) $. As a concrete example, consider a particle in the ground state of an infinite potential well with time-varying width $ l(t) $, initially equal to $ l_{0} $ at $ t = 0 $. 

    Adiabatic transitions are those transitions in which the system parameter changes slowly enough that the system remains in the ground state throughout the transition. 

    \item Note that a quantitative analysis of the condition ``changes slowly enough'' is beyond the scope of this course; we consider only the qualitative condition, that the reate of change of the paratemeter with respect to time is small relative to the energy level between the system's states, i.e.
    \begin{equation*}
        \dv{\lambda}{t} \ll \frac{\abs{E_{m} - E_{n}}}{\hbar} \lambda(t).
    \end{equation*}
    This condition ensures that the probability for transitions to excited states is negligible, meaning the particle is likely to stay in its ground state, as required for an adiabatic transition.

    \item In the more general case, we consider a system described by $ n $ time-parameters $ \Q(t) = (q_{1}(t), q_{2}(t), \cdots, q_{n}(t)) $, with a corresponding implicitly time-dependent \Ham $ H = H(\Q(t)) $. 

    Let $ \ket{\psi_{n}(\Q)} $ denote the system's eigenstates for a given value of the parameters $ \Q $, with corresponding energy eigenvalue relation
    \begin{equation*}
        H(\Q) \ket{\psi_{n}(\Q)} = E_{n}(\Q) \ket{\psi_{n}(\Q)}.
    \end{equation*}
    The parameters, and thus the eigenstate $ \ket{\psi_{n}(\Q)} $, change with time, so that the exact value of $ \ket{\psi_{n}(\Q)} $ is different at each value of $ t $. As a result, the wavefunction $ \Psi_{n}^{(0)}(\r, t) = \braket{\r}{\psi_{n}(\Q)} $ is not in general of the time-dependent \Schro equation, i.e.
    \begin{equation*}
        i \hbar \pdv{t} \ket{\Psi_{n}^{(0)}} \neq H(\Q) \ket{\Psi_{n}^{(0)}(\r, t)},
    \end{equation*}
    since the stationary states themselves change with time.
    
    \item We now assume the parameters change adiabatically, which means the system remains in the $ n $-th eigenstate $ \ket{\psi_{n}(\Q)} $ of the \Ham $ H(\Q) $ throughout the change. Although the stationary state $ \ket{\psi_{n}(\Q)} $ is constant, we allow for a potentially-changing global phase $ \phi_{n}(t) $, and thus solve the time-dependent \Schro equation with the ansatz
    \begin{equation*}
        \ket{\P_{n}} = e^{i\phi_{0}(t)} \ket{\Psi_{n}^{(0)}},
    \end{equation*}
    which we substitute into the \Schro equation $ i \hbar \pdv{t} \ket{\P_{n}} = H \ket{\Psi_{n}} $ to get
    \begin{equation*}
        i \hbar \left( i \dv{\phi_{n}}{t} e^{i\phi_{n}}\ket{\Psi_{n}^{(0)}} + e^{i\phi_{n}}\pdv{t}\ket{\Psi_{n}^{(0)}}\right) = H e^{i\phi_{n}}\ket{\Psi_{n}^{(0)}} = E_{n}e^{i\phi_{n}} \ket{\Psi_{n}^{(0)}}.
    \end{equation*}
    Note that all quantities are dependent on $ \Q(t) $ and thus implicitly on time, which we have left implicit for conciseness.

    \item Next, we multiply the equation through by $ \bbra{\Psi_{n}^{(0)}} $ to get
    \begin{equation*}
        i \hbar \left( i \dv{\phi_{n}}{t}\braket{\Psi_{n}^{(0)}}{\Psi_{n}^{(0)}} + \mel{\Psi_{n}^{(0)}}{\pdv{t}}{\Psi_{n}^{(0)}} \right) = E_{n} \braket{\Psi_{n}^{(0)}}{\Psi_{n}^{(0)}},
    \end{equation*}
    where we assume the state $ \Psi_{n}^{(0)} $ is normalized at every time $ t $.

    \item Next, we separate the phase $ \phi_{n} $ into two components $ \phi_{n} = \gamma_{n} + \theta_{n} $, chosen such that
    \begin{equation*}
        i \hbar \left( i \dv{\gamma_{n}}{t} + \mel{\Psi_{n}^{(0)}}{\pdv{t}}{\Psi_{n}^{(0)}} \right) = \underbrace{\left( E_{N} + \hbar \dv{\theta_{n}}{t} \right)}_{0},
    \end{equation*}
    which implies
    \begin{equation*}
        \theta_{n} = -\frac{1}{\hbar} \int_{0}^{t}E_{n}(t')\diff t' \qquad \text{and} \qquad \dv{\gamma_{n}}{t} = i \mel{\Psi_{n}^{(0)}}{\pdv{t}}{\Psi_{n}^{(0)}}.
    \end{equation*}

    \item Next, keeping the $ q_{i}(t) $ parameters' time dependence in mind, we rewrite the wavefunction's time derivative as
    \begin{equation*}
        \pdv{\P_{n}^{(0)}}{t} = \sum_{i} \pdv{\Psi_{n}^{(0)}}{q_{i}}\dot{q}_{i} \equiv (\gq\psi_{n}) \cdot \dot{\Q}.
    \end{equation*}
    In terms of this notation for $ \Psi_{n}^{(0)} $'s time derivative, the phase term $ \gamma_{n}(t) $ is given by
    \begin{align*}
        \gamma_{n}(t) &= i \int \mel{\Psi_{n}^{(0)}}{\pdv{t}}{\Psi_{n}^{(0)}} \diff t = i\int_{0}^{t} \braket{\psi_{n}}{\gq \psi_{n}} \cdot \dot{\Q} \diff t\\
        & = i\int_{\Q(0)}^{\Q(t)} \braket{\psi_{n}}{\gq\psi_{n}} \cdot \diff \Q.
    \end{align*}
    We have thus separated the phase $ \phi_{n} $ into two parts:
    \begin{equation*}
        \phi_{n}(t) = \theta_{n} + \gamma_{n} = - \frac{1}{\hbar}\int_{0}^{t}E_{n}(t')\diff t' + \int_{\Q(0)}^{\Q(t)} i \braket{\psi_{n}}{\gq \psi_{n}} \cdot \diff \Q.
    \end{equation*}
    The first component $ \theta_{n} $ is called the dynamic phase and the component $ \gamma_{n} $ is called the geometric or Berry phase. In the adiabatic limit, the Berry phase does not depend explicitly on time, but instead on the path $ \Q $ traced out by the system in the parameter space.

    The matrix element $ \mathcal{A}_{n}(\Q) = i \braket{\psi_{n}}{\gq \psi_{n}}$  is called the Berry connection or Berry potenial. Note that the Berry potential is real-valued, which follows from
    \begin{equation*}
        \gq \big[ \braket{\psi_{n}}{\psi_{n}} \big] = \gq [1] = 0 = \braket{\gq \psi_{n}}{\psi_{n}} + \braket{\psi_{n}}{\gq \psi_{n}} = 2 \Re \braket{\psi_{n}}{\gq \psi_{n}}.
    \end{equation*}
    Evidently, $ \mathcal{A}_{n} = 0 $ for real functions $ \psi_{n} $. 

    \item Next, assume the trajectory $ \mathcal{C} $ traced out by the parameters $ \Q $ in parameter space is closed at the time $ t_{0} $. In this case, the \Ham operator returns to its initial form: $ H(\Q(t_{0})) = H(\Q(0)) $, and the corresponding state at time $ t_{0} $ is multiplied by both the dynamic phase $ \theta $, and the Berry phase
    \begin{equation*}
        \gamma_{\text{Berry}}(t_{0}) = \oint_{\mathcal{C}} \mathcal{A}_{n}(\Q) \cdot \diff \Q.
    \end{equation*}
    In a three-dimensional parameter space, in which $ \Q = (q_{1}, q_{2}, q_{3}) $, the can also define a Berry-curvature
    \begin{equation*}
        \Omega_{n}(\Q) = \gq \cross \mathcal{A}_{n}(\Q).
    \end{equation*}
    Then, using Stokes' theorem, we can write the Berry phase in terms of an integral over the surface $ \mathcal{S} $ defined by the closed curve $ \mathcal{C} $, in the form
    \begin{equation*}
        \gamma_{n} = \iint_{\mathcal{S}}\Omega_{n}(\Q)\cdot \diff \vec{S}.
    \end{equation*}
    Without derivation, we note that if the surface $ \mathcal{S} $ is closed, the Berry phase is a multiple of $ 2\pi $, and the multiple is called a Chern number.
    
\end{itemize}

\subsection{The WKB Approximation}
    In the context of quantum mechanics WKB method is an expansion in terms of $ \hbar $, which represents an expansion with respect to the classical limit $ \hbar \to 0 $. 
\begin{itemize}

    \item We begin by writing the system's wavefunction $ \Psi $ with the ansatz 
    \begin{equation*}
        \Psi(\r, t) = e^{\frac{i}{\hbar}S(\r, t)}, \qquad S(\r, t) \in \mathbb{C}.
    \end{equation*}
    We then substitute this ansatz into the \Schro equation,
    \begin{equation*}
        i \hbar \pdv{\Psi}{t} = i \frac{\hbar^{2}}{2m}\laplacian \Psi + V \Psi,
    \end{equation*}
    which leads to the partial differential equation
    \begin{equation*}
        - \pdv{S}{t}\Psi = \left( \frac{1}{2m}(\grad S)^{2} - i \frac{\hbar^{2}}{2m} \laplacian S + V \right)\Psi.
    \end{equation*}
    Assuming $ \Psi \neq 0 $, we can then divide through by $ \Psi $ to get
    \begin{equation*}
        -\pdv{S}{t} = \frac{1}{2m}(\grad S)^{2} + V - i \frac{\hbar^{2}}{2m}\laplacian S.
    \end{equation*}
    
    \item Next, we note that in the limit $ \hbar \to 0 $, the above equation reduces to
    \begin{equation*}
        \pdv{S}{t} = \frac{1}{2m}(\grad S)^{2} + V,
    \end{equation*}
    which is a classical Hamilton-Jacobi equation for the principle Hamiltonian function $ S $ for a classical particle with velocity $ \vec{v} = \frac{1}{m}\grad S $. From classical mechanics, we know that Hamilton-Jacobi theory is equivalent to the formalism of Newton's second law, so the limit $ \hbar \to 0 $ essentially recovers classical mechanics.
    
    \item We proceed with the WKB method by expanding $ S $ in powers of $ \hbar $, in the form
    \begin{equation*}
        S = S_{0} + \hbar S_{1} + \hbar^{2}S_{2} + \cdots,
    \end{equation*}
    where the zero-th order expansion in $ \hbar $ produces the aforementioned classical equation
    \begin{equation*}
        \pdv{S}{t} = \frac{1}{2m}(\grad S)^{2} + V,
    \end{equation*}
    while the first order in $ \hbar $ corresponds to a first-order quantum correction
    \begin{equation*}
        - \pdv{S_{1}}{t} = \frac{1}{2m} \big( 2 \grad S_{0} \cdot \grad S_{1} - i \laplacian S_{0} \big).
    \end{equation*}
    Analysis in the regime of this first-order approximation is called the WKB or semi-classical approximation.

    \item As an example, we consider the one-dimensional stationary states
    \begin{equation*}
        \Psi(x, t) = e^{-i \frac{E}{\hbar}t} \psi(x).
    \end{equation*}
    We will write $ S $ and $ \psi $ as the time-independent functions
    \begin{equation*}
        S(x) S_{0}(x) + \hbar S_{1}(x) + \mathcal{O}(\hbar^{2}) \qquad \text{and} \qquad e^{\frac{i}{\hbar}S(x)}.
    \end{equation*}
    
    \item We begin by solving for $ S_{0}(x) $ using the zero-order classical equation, which gives
    \begin{equation*}
        E = \frac{1}{2m}\left( \dv{S_{0}(x)}{x} \right)^{2} + V(x)
    \end{equation*}
    and
    \begin{equation*}
        S_{0}(x) = \pm \int_{x_{0}}^{x} \sqrt{2m \big[ E - V(x') \big]}\diff x' \equiv \pm \int_{x_{0}}^{x}p(x')\diff x',
    \end{equation*}
    where we have defined $ p(x) = \sqrt{2m [E - V(x)]} $. We then use this expression for $ S_{0}(x) $ to solve for $ S_{1} $ via
    \begin{equation*}
        2 \dv{S_{0}(x)}{x} \dv{S_{1}(x)}{x} = i \dv[2]{S_{0}(x)}{x},
    \end{equation*}
    which leads to
    \begin{equation*}
        S_{1}(x) = \frac{i}{2} \int_{x_{0}}^{x}\frac{S''_{0}(\tilde{x})}{S'_{0}(\tilde{x})} \diff \tilde{x} = \frac{i}{2} \ln \frac{p(x)}{p(x_{0})}.
    \end{equation*}
    The approximate solution for the wavefunction in the WKB approximation is thus
    \begin{equation*}
        \psi_{\text{wkb}}(x, t) = \frac{C}{\sqrt{p(x)}}\exp \left[ -i \frac{E}{\hbar}t \pm \frac{i}{\hbar}\int_{x}^{x_{0}} p(x')\diff x' \right].
    \end{equation*}
    This solution takes the form of a modulated wave, with a position-dependent wave vector
    \begin{equation*}
        k(x) = \frac{p(x)}{\hbar}.
    \end{equation*}
    The plus/minus sign in the exponent encodes the wave's direction, just like in a plane wave, and depends on the problem's concrete boundary conditions. The constants $ C $ and $ x_{0} $ are determed by normalization and boundary conditions, respectively.

    Finally, we note that the WKB approximation does not hold in the neighborhood of classical turning points where $ p(x) = 0 $, since this produces a singularity in the denominator. We analyze such turning points using complex-valued connection formulas, but this is beyond the scope of this course.
    
    
\end{itemize}

\subsection{The Variational Method}
\begin{itemize}
    \item We consider a \Ham with orthornormalized but otherwise unknown eigenstates and eigenvalues satisfying
    \begin{equation*}
        H \ket{n} = E_{n}\ket{n}, \qquad \braket{m}{n} = \delta_{mn}.
    \end{equation*}
    We then expand an arbitrary wavefunction $ \ket{\psi} $ in the $ \big\{ \ket{n} \big\} $ basis in the form
    \begin{equation*}
        \ket{\psi} = \sum_{n} c_{n}\ket{n} = \sum_{n} \braket{n}{\p}\ket{n}.
    \end{equation*}
    
    \item The energy expectation value $ \ev{H} $ for the state $ \ket{\psi} $ is
    \begin{align*}
        \mel{\psi}{H}{\psi} &= \mel{\sum_{m} \braket{m}{\p}\bra{m}}{H}{\sum_{n} \braket{n}{\p}\ket{n}}\\
        & = \braket{\sum_{m} \braket{m}{\p}\bra{m}}{\sum_{n} \braket{n}{\p}E_{n}\ket{n}}\\
        &= \sum_{n} E_{n}\abs{c_{n}}^{2}.
    \end{align*}
    We can bound this result from above according to
    \begin{equation*}
        E_{0}\sum_{n} \abs{c_{n}}^{2} \leq \sum_{n} E_{n}\abs{c_{n}}^{2},
    \end{equation*}
    where $ E_{0} $ is the ground state energy, which obeys $ E_{0} \leq E_{1} \leq \cdots \leq E_{n} \leq \cdots $. This upper bound implies
    \begin{equation*}
        \mel{\psi}{H}{\psi} \geq E_{0} \sum_{n}\abs{c_{n}}^{2} = E_{0} \braket{\psi}{\psi},
    \end{equation*}
    which we finally rearrange to get the important result
    \begin{equation*}
        E_{0} \leq \frac{\mel{\psi}{H}{\psi}}{\braket{\psi}{\psi}}.
    \end{equation*}
    In other words, the $ H $ ground state energy $ E_{0} $ is smaller than the energy expectation value for any state $ \ket{\psi} $ expanded in the $ H $ basis $ \big\{ \ket{n} \big\} $.
    
    \item The just-derived inequality forms the basis for a variational method of ground state energy calculation of some \Ham $ H $. The variational method proceeds as follows:
    \begin{enumerate}
        \item First, we choose $ \ket{\psi} $ to be as good an approximation as possible to the \Ham's ground state, and require that $ \ket{\psi} $ has the same symmetry properties as $ H $. 

        \item We then parametrize $ \psi $ with $ n $ variational parameters $ \alpha_{i} $ in the form $ \psi = \psi(\alpha_{1}, \alpha_{2}, \ldots, \alpha_{n}; x) $. 

        Using this trial function, we then calculate the expectation value 
        \begin{equation*}
            \mel{\psi}{H}{\psi} = \mathcal{E}(\alpha_{1}, \ldots, \alpha_{n}).
        \end{equation*}

        \item Next, by varying the parameteris $ \alpha_{i} $, we determine the minimum energy $ \mathcal{E}_{\text{min}} $, which is the best approximation to the ground state within the space of trial functions generated by the parameters $ \alpha_{1}, \ldots, \alpha_{n} $. 
        
        \item Because of the earlier inequality, $ \mathcal{E}_{\text{min}} $ must obey
        \begin{equation*}
            E_{0} \leq \mathcal{E}_{\text{min}},
        \end{equation*}
        which theoretically allows us to arbitrarily increase the number of parameters and trial functions, and repeat the process of finding the minimum energy expectation value $ \mathcal{E}_{\text{min}} $ until $ \mathcal{E}_{\text{min}} $ is an arbitrarily close approximation of $ E_{0} $. Keep in mind that $ E_{0} \leq \mathcal{E}_{\text{min}} $ gaurantees we will never drop below $ E_{0} $ when decreasing $ \mathcal{E}_{\text{min}} $, which is why we can be sure we will get an arbitrarily close approximation of $ E_{0} $ instead.
    \end{enumerate}
    
    \item As an example of the variational method, we return to Rayleigh-\Schro method for solving a \Ham of the form
    \begin{equation*}
        H = H_{0} + \lambda V,
    \end{equation*}
    where $ \lambda V $ is a secondary perturbation term and $ H_{0} $ has the known eigenvalue relation
    \begin{equation*}
        H_{0} \ket{\varphi_{n}^{(0)}} = E_{n}^{(0)}\ket{\varphi_{n}^{(0)}}.
    \end{equation*}
    We then calculate the expectation value $ \ev{H} $ to first order in $ \lambda $, which reads
    \begin{equation*}
        \ev{H} = \mel{\psi}{H}{\psi} = E_{0}^{(0)} + \lambda \bmel{\varphi_{0}^{(0)}}{V}{\varphi_{0}^{(0)}} + \mathcal{O}(\lambda^{2}) = E_{0}^{(0)} + E_{0}^{(1)} + \mathcal{O}(\lambda^{2})
    \end{equation*}
    The inequality underlying the variational principle gaurantees $ E_{0} \leq \ev{H} $, which implies
    \begin{equation*}
        E_{0} \leq E_{0}^{(0)} + E_{0}^{(1)}.
    \end{equation*}
    In other words, the \Ham $ H $'s true ground state energy $ E_{0} $ is less than the sum of the first two perturbative energy corrections $ E_{0}^{(0)} $ and $ E_{0}^{(1)} $.
    
\end{itemize}

\newpage
\section{Scattering}

\subsection{Scattering in One Dimension}
\begin{itemize}
    \item We first consider one-dimensional scattering problems involving a \Ham of the form
    \begin{equation*}
        H = - \frac{\hbar^{2}}{2m}\dv[2]{}{x} + V(x).
    \end{equation*}
    In the absence of a potential, i.e. for $ V(x) = 0 $, the \Ham $ H $'s stationary states are plane waves with wavefunctions and energies of the form
    \begin{equation*}
        \braket{x}{\psi} \sim e^{\pm i kx} \qquad \text{with} \qquad E_{p} = \frac{p^{2}}{2m} > 0.
    \end{equation*}
    Each energy eigenvalue $ E_{p} $ is doubly degerate, since the linearly independent states with momenta $ \pm p = \pm \hbar k $ have the same energy.

    \item In scattering problems we have $ V(x) \neq 0 $; however, we assume the potential is nonzero only in a finite region $ x \in [x_{a}, x_{b}] $. More so, we assume the potential does not allow for bound states. In such situations, we write stationary scattering states in the form
    \begin{equation*}
        \psi(x) = 
        \begin{cases}
            A_{1} e^{ikx} + B_{1} e^{-ikx} & x < x_{a}\\
            \psi_{ab}(x) & x \in [x_{a}, x_{b}]\\
            A_{2} e^{-ikx} + B_{2} e^{ikx} & x > x_{a}.
        \end{cases}
    \end{equation*}
    In other words, we assume the wavefunction is a linear superposition of plane waves in the region of zero potential and some wavefunction $ \psi_{ab} $ in the region with non-zero potential $ V(x), x \in [x_{a}, x_{b}] $. The $ A $ coefficients denote waves moving toward the potential barrier and the $ B $ coefficients denote waves moving away from the barrier.

    \item Next, we introduce two two-dimensional column vectors, $ \Psi_{\text{in}} $ and $ \Psi_{\text{out}} $, which encode the waves moving toward and moving away from the potential, respectively. The vectors are:
    \begin{align*}
        & \Psi_{\text{in}} = 
        \begin{pmatrix}
            A_{1}\\
            A_{2}
        \end{pmatrix}
        && \Psi_{\text{out}} = 
        \begin{pmatrix}
            B_{1} \\
            B_{2}
        \end{pmatrix}\\
        & \Psi_{\text{in}}^{\dagger} = \big( A_{1}^{*}, A_{2}^{*} \big)
        && \Psi_{\text{out}}^{\dagger} = \big( B_{1}^{*}, B_{2}^{*} \big).
    \end{align*}

    \item The system's associated probability current density is constant and equal to
    \begin{equation*}
        j_{a} = \frac{\hbar k}{m} \big( \abs{A_{1}}^{2} - \abs{B_{1}}^{2} \big) = j_{b} = \frac{\hbar k}{m} \big( \abs{B_{2}}^{2} - \abs{A_{2}}^{2} \big),
    \end{equation*}
    from which follows the fundamental relationship between the coefficients $ A $ and $ B $, namely:
    \begin{equation*}
        \abs{A_{1}}^{2} + \abs{A_{2}}^{2} = \abs{B_{1}}^{2} + \abs{B_{2}}^{2}.
    \end{equation*}
    In other words, the sum of the currents entering the potential equals the sum of the currents leaving the potential, which may also be written
    \begin{equation*}
        \Psi_{\text{out}}^{\dagger} \Psi_{\text{out}}^{\dagger} = \Psi_{\text{in}}^{\dagger} \Psi_{\text{in}}^{\dagger}.
    \end{equation*}
    
\end{itemize}

\subsubsection{The Scattering Matrix}
\begin{itemize}
    \item The coefficient vectors $ \Psi_{\text{in}} $ and $ \Psi_{\text{out}} $ are related by a scattering matrix $ \SS $ via
    \begin{equation*}
        \Psi_{\text{out}} = \SS \Psi_{\text{in}}.
    \end{equation*}
    The scattering matrix is unitary, which we can prove via
    \begin{equation*}
        \Psi_{\text{out}}^{\dagger} \Psi_{\text{out}} = \Psi_{\text{out}}^{\dagger} \SS \Psi_{\text{in}} = \Psi_{\text{in}}^{\dagger} \SS^{\dagger} \SS \Psi_{\text{in}}
    \end{equation*}
    Since $ \Psi_{\text{out}}^{\dagger} \Psi_{\text{out}}^{\dagger} = \Psi_{\text{in}}^{\dagger} \Psi_{\text{in}}^{\dagger} $, it follows that $ \SS^{\dagger} \SS = \mathbf{I} $ and thus $ \SS^{\dagger} = \SS^{-1} $, which means $ \SS $ is indeed unitary.

    \item We will parameterize the scattering matrix in the form
    \begin{equation*}
        \SS = 
        \begin{pmatrix}
            r & t'\\
            t & r'
        \end{pmatrix} \in \mathbb{C}^{2 \cross 2},
    \end{equation*}
    from which follow the relationships
    \begin{equation*}
        \begin{pmatrix}
            r\\
            t
        \end{pmatrix} 
        = \SS
        \begin{pmatrix}
            1\\
            0
        \end{pmatrix}
        \qquad \text{and} \qquad 
        \begin{pmatrix}
            t'\\
            r'
        \end{pmatrix}
        = \SS
        \begin{pmatrix}
            0\\
            1
        \end{pmatrix}.
    \end{equation*}
    The parameters $ t $ and $ t' $, which are in general complex numbers, encode the probability for a wave packet to pass through the potential (to transmit, hence the letter $ t $), while the parameters $ r $ and $ r' $ encode the probability for a wave packet to reflect off the potential barrier.
    
    \item The scattering matrix parameters are related according to
    \begin{equation*}
        \SS^{\dagger} \SS = \mathbf{I} \iff 
        \begin{cases}
            1 = \abs{t}^{2} + \abs{r}^{2} = \abs{t'}^{2} + \abs{r'}^{2}\\
            0 = r^{*} t' + t^{*} r' = (t')^{*} r + (r')^{*}t.
        \end{cases} 
    \end{equation*}
    Additionally, the fact that the scattering matrix is unitary produces the following, perhaps less obvious relationships
    \begin{equation*}
        \SS \SS^{\dagger} = \mathbf{I} \iff 
        \begin{cases}
            1 = \abs{t'}^{2} + \abs{r}^{2} = \abs{t}^{2} + \abs{r'}^{2}\\
            0 = r^{*}t + (t')^{*}r' = t*r + (r')^{*}t'.
        \end{cases}  
    \end{equation*}
    
    \item The scattering matrix is parameterized by four complex numbers, $ t $, $ t' $, $ r $ and $ r' $, which corresponds to eight independent real-valued parameters. However, because of the condition that $ \S $ is unitary creates certain restrictions on the parameters, namely:
    \begin{equation*}
        \abs{t} = \abs{t'} \qquad \abs{r} = \abs{r'} \qquad r' = -r^{*}\frac{t'}{t^{*}}.
    \end{equation*}
    These three relationships between the parameters imply that $ \SS $ is fact described by five real-valued parameters instead of eight. With these relationships in mind, the scattering matrix may be written
    \begin{equation*}
        \SS = 
        \begin{pmatrix}
            r & t'\\
            t & r'
        \end{pmatrix}
        = 
        \begin{pmatrix}
            r & t'\\
            t & - r^{*}\frac{t'}{t^{*}}
        \end{pmatrix}
        = 
        \begin{pmatrix}
            r & e^{i\phi}t^{*}\\
            t & - e^{i\phi}r^{*}
        \end{pmatrix},
    \end{equation*}
    where $ \det \SS = -e^{i\phi} $.
    
    
\end{itemize}

\subsubsection{Time-Reversal Invariance}
\begin{itemize}
    \item We now consider a system invariant under time inversion, using the simple example of a free-particle. In this case, for any $ \psi $ solving the stationary \Schro equation in the regions $ x < x_{a} $ and $ x > x_{b} $, the time-transformed wavefunction
    \begin{equation*}
        \T \psi(x) = \psi^{*}(x) = 
        \begin{cases}
            A^{*}_{1} e^{-ikx} + B^{*}_{1} e^{ikx} & x < x_{a}\\
            A^{*}_{2} e^{ikx} + B^{*}_{2} e^{-ikx} & x > x_{a}
        \end{cases}
    \end{equation*}
    also solves the stationary \Schro equation. Time invariance thus switches and conjugates the ingoing and outgoing vectors via
    \begin{equation*}
        \T \Psi_{\text{in}} = 
        \begin{pmatrix}
            B_{1}^{*}\\
            B_{2}^{*}\\
        \end{pmatrix}
        = \Psi^{*}_{\text{out}}
        \qquad \text{and} \qquad 
        \T \Psi_{\text{out}} = 
        \begin{pmatrix}
            A_{1}^{*}\\
            A_{2}^{*}\\
        \end{pmatrix}
        = \Psi^{*}_{\text{in}}.
    \end{equation*}

    \item Since the time-reversed states also solve the \Schro equation, the scattering matrix must be symmetric. To prove this, we begin with the relationship $ \Psi_{\text{out}} = \SS \Psi_{\text{in}} $. We then replace $ \Psi $ with the equivalent solution $ \T \Psi $ and apply the above transformation rules of $ \P_{\text{in}} $ and $ \Psi_{\text{out}} $ under $ \T $ to get
    \begin{equation*}
        \T \Psi_{\text{out}} = \SS \T \Psi_{\text{in}} \implies \Psi_{\text{in}}^{*} = \SS \Psi_{\text{out}}^{*}.
    \end{equation*}
    We then multiply the equation from the left by $ \SS^{\dagger} $, apply the unitary identity $ \SS^{\dagger}\SS = \mat{I} $, and take the equation's complex conjugate to get
    \begin{equation*}
        \SS^{\dagger}\Psi_{\text{in}}^{*} = \SS^{\dagger}\SS \Psi_{\text{out}}^{*} = \Psi_{\text{out}} \implies (\SS^{\dagger})^{*} \Psi_{\text{in}} \equiv \SS^{T}\Psi_{\text{in}} = \Psi_{\text{out}}.
    \end{equation*}
    Comparing this to the original relationship $ \SS \Psi_{\text{in}} = \Psi_{\text{out}} $ implies $ \SS = \SS^{T} $, i.e. the scattering matrix is symmetric.

    \item The symmetry condition $ \SS = \SS^{T} $ implies $ t = t' $, which allows us to write the scattering matrix in the form
    \begin{equation*}
        \SS = 
        \begin{pmatrix}
            r & t\\
            t & r'
        \end{pmatrix}
        = 
        \begin{pmatrix}
            r & t\\
            t & - r^{*}\frac{t'}{t^{*}}
        \end{pmatrix}
        = 
        \begin{pmatrix}
            r & \abs{t}e^{i\frac{\phi}{2}}\\
            \abs{t}e^{i\frac{\phi}{2}} & - e^{i\phi}r^{*}
        \end{pmatrix}, \qquad \det \SS = -e^{i\phi}.
    \end{equation*}
    In passing, we note that in the presence of an external magnetic field, the \Ham for a charged particle is no longer invariant under time reversal, which case the scattering matrix obeys $ \SS_{-\B} = \SS_{\B}^{T} $.

\end{itemize}

\subsubsection{Parity Invariance}
\begin{itemize}
    \item We now consider systems that are invariant under parity transformation. Such systems obey $ V(-x) = V(x) $ and thus $ x_{a} = - x_{b} $. In this case, for any $ \psi $ solving the stationary \Schro equation in the regions $ x < x_{a} $ and $ x > x_{b} $, the parity-transformed wavefunction
    \begin{equation*}
        \Par \psi(x) = \psi(-x) = 
        \begin{cases}
            B_{2} e^{-ikx} + A_{2} e^{ikx} & x < x_{a}\\
            B_{1} e^{ikx} + A_{1} e^{-ikx} & x > \abs{x_{a}}
        \end{cases}
    \end{equation*}
    also solves the stationary \Schro equation. The parity operator thus transforms the roles of the coefficients in the form $ A_{1,2} \longleftrightarrow B_{1,2} $. As a result, the parity operator transforms the ingoing and outgoing vectors in the form
    \begin{equation*}
        \Par \Psi_{\text{in}} = \sigma_{x} \Psi_{\text{in}} \qquad \text{and} \qquad \Par \Psi_{\text{out}} = \sigma_{x}\Psi_{\text{out}},
    \end{equation*}
    where $ \sigma_{x} = \begin{pmatrix}
        0 & 1\\
        1 & 0
    \end{pmatrix} $ is the first Pauli spin matrix. As an example, we consider the transformation of $ \Psi_{\text{in}} $:
    \begin{equation*}
        \Par \Psi_{\text{in}} = \Par
        \begin{pmatrix}
            A_{1}\\
            A_{2}
        \end{pmatrix}
        = 
        \begin{pmatrix}
            A_{2}\\
            A_{1}
        \end{pmatrix}
        = 
        \begin{pmatrix}
            0 & 1\\
            1 & 0
        \end{pmatrix}
        \begin{pmatrix}
            A_{1}\\
            A_{2}
        \end{pmatrix}
        = \sigma_{x} \Psi_{\text{in}}.
    \end{equation*}
    
    \item The parity-transformed states are related by the scattering matrix according to
    \begin{equation*}
        \sigma_{x} \Psi_{\text{out}} = \SS \sigma_{x} \Psi_{\text{in}} \implies \sigma_{x}^{2} \Psi_{\text{out}} \equiv \mat{I} \Psi_{\text{out}} = \sigma_{x}\SS \sigma_{x} \Psi_{\text{in}}.
    \end{equation*}
    Comparing the last equality to the general relationship $ \Psi_{\text{out}} = \SS \Psi_{\text{in}} $ implies $ \sigma_{x} \SS \sigma_{x} $, which in turn implies
    \begin{equation*}
        \sigma_{x} \SS \sigma_{x} = \sigma_{x}
        \begin{pmatrix}
            r & t'\\
            t & r'
        \end{pmatrix}
        \sigma_{x} = 
        \begin{pmatrix}
            r' & t\\
            t' & r
        \end{pmatrix}
        \SS = 
        \begin{pmatrix}
            r & t'\\
            t & r'
        \end{pmatrix},
    \end{equation*}
    or, in other words, $ r = r' $ and $ t = t' $. This condition on $ r $ and $ t $ implies the scattering matrix for a parity-invariant system is symmetric with respect to both diagonals and can be written in the form
    \begin{equation*}
        \SS = 
        \begin{pmatrix}
            r & t\\
            t & r
        \end{pmatrix} 
        = - e^{i\tau}
        \begin{pmatrix}
            i \abs{r} & \abs{t}\\
            \abs{t} & i \abs{r}
        \end{pmatrix}, 
        \qquad t = e^{i\tau}\abs{t}.
    \end{equation*}
    
    
\end{itemize}

\subsubsection{Multiple Scattering Channels}
\begin{itemize}
    \item We now consider a system with additional degrees of freedom that can change after scattering, for example a particle with spin $ s = 1/2 $ whose projection can reverse after scattering. In this case, the ingoing and outgoing vectors $ \Psi_{\text{in}} $ and $ \Psi_{\text{out}} $ are formed of $ 2N $ components, while the scattering matrix becomes a $ 2N \cross 2N $ block-diagonal matrix whose block diagonal elements are $ 2 \cross 2 $ matrices corresponding to the scattering matrix for each individual scattering channel.

    \item In passing, we mention the possibility of a particle changing effective mass from e.g. $ m_{1} $ to $ m_{2} $ when passing through different regions. In this case, the scattering coefficients change according to
    \begin{equation*}
        \begin{Bmatrix}
            A_{1}\\
            B_{1}
        \end{Bmatrix}
        \to \sqrt{\frac{m_{1}}{p_{1}}}
        \begin{Bmatrix}
            A_{1}\\
            B_{1}
        \end{Bmatrix} 
        \qquad \text{and} \qquad 
        \begin{Bmatrix}
            A_{2}\\
            B_{2}
        \end{Bmatrix}
        \to \sqrt{\frac{m_{2}}{p_{2}}}
        \begin{Bmatrix}
            A_{2}\\
            B_{2}
        \end{Bmatrix}
    \end{equation*}
    where we have defined $ p_{1,2} = \sqrt{2m_{1,2}(E - V_{1,2})} $.

\end{itemize}

\subsubsection{The Transfer Matrix}
\begin{itemize}
    \item In addition to the scattering matrix approach, we often analyze one-dimensional systems in terms of transfer matrices, which we discuss in this section. Scattering matrices relate the waves on one side of a potential barrier to the waves on the other side in the form
    \begin{equation*}
        \begin{pmatrix}
            A_{1}\\
            B_{1}
        \end{pmatrix}
        = \M
        \begin{pmatrix}
            A_{2}\\
            B_{2}
        \end{pmatrix}
        \qquad \text{where} \qquad \M = 
        \begin{pmatrix}
            \MM_{11} & \MM_{12}\\
            \MM_{21} & \MM_{22}
        \end{pmatrix}.
    \end{equation*}
    
    \item To relate the scattering matrix $ \SS $ to the transfer matrix $ \M $, we begin with
    \begin{equation*}
        \begin{pmatrix}
            r\\
            t
        \end{pmatrix} 
        = \SS
        \begin{pmatrix}
            1\\
            0
        \end{pmatrix}
        \qquad \text{and} \qquad 
        \begin{pmatrix}
            t'\\
            r'
        \end{pmatrix}
        = \SS
        \begin{pmatrix}
            0\\
            1
        \end{pmatrix}.
    \end{equation*}
    Meanwhile, the scattering matrix obeys 
    \begin{equation*}
        \begin{pmatrix}
            1 \\
            r
        \end{pmatrix}
        = \M
        \begin{pmatrix}
            t\\
            0
        \end{pmatrix}
        \qquad \text{and} \qquad 
        \begin{pmatrix}
            0\\
            t'
        \end{pmatrix}
        = \M
        \begin{pmatrix}
            r'\\
            1
        \end{pmatrix},
    \end{equation*}
    from which follows the relationship
    \begin{equation*}
        \M = 
        \begin{pmatrix}
            \frac{1}{t} & - \frac{r'}{t}\\
            \frac{r}{t} & t' - \frac{rr'}{t}
        \end{pmatrix}
        \qquad \text{where} \qquad \det \M = \frac{t'}{t}.
    \end{equation*}
    Under the condition $ t = t' $, the relationship between $ \SS $ and $ \M $ simplifies to
    \begin{equation*}
        \M = 
        \begin{pmatrix}
            \frac{1}{t} & \frac{r^{*}}{t^{*}}\\[0.5mm]
            \frac{r}{t} & \frac{1}{t^{*}}
        \end{pmatrix}
        \qquad \text{and} \qquad \SS = \frac{1}{\MM_{11}}
        \begin{pmatrix}
            \MM_{21} & 1\\
            1 & - \MM_{21}^{*}
        \end{pmatrix}.
    \end{equation*}
    
    \item The transfer matrix approach is well-suited to systems involving a series of  multiple (e.g. $ N $) potential barriers, with endpoints at the positions $ x_{1}, x_{2}, \ldots, x_{2N}$. In this case, we define the wave amplitudes $ A $ and $ B $ with respect to the shifted plane waves $ e^{\pm i k(x - x_{i})} $, split the $ x $ axis into $ 2N + 1 $ regions, and calculate the $ 2N - 1 $ transfer matrices $ \M_{ij} = \M_{12}, \M_{23}, \ldots, \M_{(2N-1) 2N} $ connecting the regions. In this case, the transfer matrix for the entire $ x $ axis is simply the product
    \begin{equation*}
        \M_{1(2N)} = \M_{12}\M_{23} \cdots \M_{(2N-1)2N}.
    \end{equation*}
    We can then use the total transfer matrix together with the equations in the previous bullet point to calculate the corresponding total transfer matrix $ \SS_{1(2N)} $.

\end{itemize}

\subsection{Scattering States}
\begin{itemize}
    \item We now consider solutions of the stationary \Schro equation of the form
    \begin{equation*}
        H_{0} \ket{\psi_{\vec{p}}^{(0)}} = E_{p}\ket{\psi_{\vec{p}}^{(0)}} \qquad \text{and} \qquad \hat{p} \ket{\psi_{\vec{p}}^{(0)}} = \vec{p}\ket{\psi_{\vec{p}}^{(0)}}.
    \end{equation*}
    The vectors $ \ket{\psi_{\vec{p}}^{(0)}} $ don't directly represent physical states, but instead form a basis for expanding wave packets encoding the system's actual quantum state. The energies $ E_{p} $ are degenerate with respect to the choice of direction of momentum. 

%TODO come back and review this (begining of chapter 11.2 in book)
    \item We often analyze eigenstates in terms of a Fourier transform to momentum space, using the fact that the momentum operator is self-adjoint. As an example, we consider a one-dimensional system that obeys
    \begin{equation*}
        \braket{x}{p} = \psi_{p}^{0}(x) = Ce^{i \frac{p}{\hbar}x}
    \end{equation*}
    with the periodic boundary conditions
    \begin{equation*}
        \psi_{p}(x + L) = \psi_{p}(x) \implies p_{n} = \frac{2\pi \hbar}{L}, \quad n = 0, 1, 2, \ldots
    \end{equation*}
    and orthogonal eigenstates
    \begin{equation*}
        \braket{m}{n} = \frac{e^{2 \pi i(n - m)} - 1}{2\pi i (n-m)}L \abs{C}^{2} \quad \text{for } n \neq m.
    \end{equation*}
    For a choice of normalization constant $ C = L^{-1/2} $, the eigenstates are normalized and obey $ \braket{m}{n} = \delta_{mn} $. Although the energy spectrum is discrete, in the limit of large $ L $ the spacing between energy levels grows arbitrarily small. 

    \item Alternatively, we can normalize states in terms of the Dirac delta function:
    \begin{equation*}
        \braket{x}{p} = \psi_{p}^{0}(x) = \frac{1}{\sqrt{2\pi \hbar}} e^{i \frac{p}{\hbar}x},
    \end{equation*}
    where the eigenstates obey $ \braket{p_{1}}{p_{2}} = \delta(p_{1} - p_{2}) $ and the identity operator is written
    \begin{equation*}
        \II = \int_{-\infty}^{\infty} \ket{p} \bra{p} \diff p.
    \end{equation*}
    This notation allows for orthogonal eigenstates in the case of a continuous spectrum, with momentum eigenvalues $ p \in \mathbb{R} $. 
    
    \item Regardless of the choice of normalization, in the presence of the potential barrier $ V $ we define scattering states $ \ket{\psi_{\vec{p}}^{+}} $ according to
    \begin{equation*}
        (H_{0} + V) \ket{\psi_{\vec{p}}^{+}} = E_{\vec{p}}\ket{\psi_{\vec{p}}^{+}},
    \end{equation*}
    which we choose so that in the limit of a vanishing potential $ V(\vec{r}) \to 0 $ the scattering states approach plane waves $ \ket{\psi_{\vec{p}}^{0}} $. 

    \item For a one-dimensional system, a scattering state is written in the form
    \begin{equation*}
        \psi_{p}^{+}(x) = C
        \begin{cases}
            e^{i \frac{p}{\hbar} x} + r e^{-i \frac{p}{\hbar} x} & x < x_{a}\\
            \psi_{ab}(x) & x \in [x_{a}, x_{b}]\\
            t e^{i \frac{p}{\hbar}x} & x > x_{b},
        \end{cases}
    \end{equation*}
    where $ r $ and $ t $ are parameters from the scattering matrix $ \SS $ and encode reflection and transmission, respectively. In the presence of a potential, momentum is in general not conserved, i.e. $ [\vec{p}, H] \neq 0 $. However, the energy eigenvalues $ E_{\vec{p}} $ are still degenerate, since each plane wave $ \psi_{\vec{p}}^{0}(x) $ correponds to a scattering state $ \psi_{\vec{p}}^{+}(x) $ with equal energy.

    \item A one dimensional system has two degenerate states, corresponding to the momentum eigenvalues $ \pm \abs{p} $. This double denegeracy motives the definition of a quantum number $ \eta = \pm 1$ encoding the direction of wave propagation. The corresponding states, i.e.
    \begin{equation*}
        \bket{\abs{p}, \eta} = \ket{p}, \quad \text{with eigenvalues} \quad p = \eta \abs{p},
    \end{equation*}
    are eigenstates of the system's Hamiltonian and can thus be described in terms of the eigenvalue $ E $ instead of $ p $. As a result, we can write the completeness relation in two equivalent forms:
    \begin{equation*}
        \II = \sum_{\eta} \int_{0}^{\infty} \ket{p, \eta} \bra{p, \eta} \diff p = \sum_{\eta} \int_{0}^{\infty} \ket{E, \eta} \bra{E, \eta} \diff E,
    \end{equation*}
    where the bases $ \{\ket{p, \eta}\} $ and $ \{E, \eta\} $ differ only in the normalization of the basis vectors. The state $ \ket{E, \eta} $ is normalized according to
    \begin{equation*}
        \braket{E_{1}, \eta_{1}}{E_{2}, \eta_{2}} = \delta_{\eta_{1}\eta_{2}}\delta(E_{1} -E_{2}).
    \end{equation*}

    \item For momentum eigenvalues $ p_{i} \sqrt{2mE_{i}} > 0 $ we rewrite the Dirac delta function as follows
    \begin{align*}
        \delta(E_{1} - E_{2}) &= \delta \left( \frac{p_{1}^{2}}{2m} - \frac{p_{2}^{2}}{2m} \right) = 2m \delta \big[ (p_{1} - p_{2})(p_{1} + p_{2}) \big] = \frac{2m}{p_{1} + p_{2}}\delta(p_{1} - p_{2})\\
        & = \frac{m}{p_{1}} \delta(p_{1} - p_{2}),
    \end{align*}
    where we have used the delta function properties $ \delta(ax) = \frac{1}{\abs{a}}\delta(x) $ and $ f(x)\delta(x-a) = f(a)\delta(x-a)^{3} $. In terms of the above expression for the delta function, we have
    \begin{equation*}
        \braket{x}{E, \eta} = \sqrt{\frac{m}{p}} \braket{x}{p, \eta} = \frac{1}{\sqrt{2\pi \hbar}} \left( \frac{m}{2E} \right)^{1/4} \exp \left( i\eta \sqrt{\frac{2m E}{\hbar^{2}}} \right),
    \end{equation*}
    which is a third way to normalize a plane wave. An analogous normalization relationship holds between the corresponding scattering scates $ \psi_{E\eta}^{+} = \sqrt{\frac{m}{p}} \psi_{p\eta}^{+} $.

\end{itemize}

\subsubsection{Formal Scattering Theory and the \Mol Operator}
\begin{itemize}
    \item A particle is in the initial asymptotic incident state $ \ket{\psi_{\vec{p}_{0}}^{0}} $, which approaches a plane wave with momentum $ \vec{p}_{0} $ at $ t \to -\infty $. 

    At the time $ t \sim 0 $, the particle (wave packet) occurs in the region of the potential barrier and occurs in the scattering state $ \ket{\psi_{\vec{p}_{0}}^{0}} $, where the scattering state is a superposition of all possible \textit{final} asymptotic states $ \ket{\psi_{\vec{p}}^{0}} $. 

    \item We can describe the quantum scattering process in terms of the \Mol operator $ \Omega_{+} $. First, the \Mol operator applies the time evolution operator $ e^{-i H_{0} t} $ to the asymptotic incidence state $ \ket{\psi_{\vec{p}_{0}}^{0}} $, which maps the state back in time to $ t \to -\infty $. At time $ t \to \infty $, the asymptotic state $ \ket{\psi_{\vec{p}_{0}}^{0}} $ equals the the particle's initial state $ \ket{\psi_{\vec{p}_{0}}^{+}} $. 

    Then, we apply the time evolution operator $ e^{iHt} $, using the \Ham $ H = H_{0} + V $, to return the particle to the present:
    \begin{equation*}
        \ket{\psi_{\vec{p}_{0}}^{+}} = \lim_{t \to - \infty} e^{iHt}e^{-iH_{0}t} \ket{\psi_{\vec{p}_{0}}^{0}} = \Omega_{+} \ket{\psi_{\vec{p}_{0}}^{0}}.
    \end{equation*}

    Meanwhile, the operator $ \Omega_{-} $ transfers the asymptotic state $ \ket{\psi_{\vec{p}}^{0}} $ to the scattering state $ \ket{\psi_{\vec{p}_{0}}^{-}} $, via
    \begin{equation*}
        \ket{\psi_{\vec{p}}^{-}} = \lim_{t \to \infty} e^{iHt}e^{-iH_{0}t}\ket{\psi_{\vec{p}}^{0}} = \Omega_{-} \ket{\psi_{\vec{p}}^{0}}.
    \end{equation*}
    The state $ \ket{\psi_{\vec{p}}^{0}} $ is a second possible scattering state, which approaches the asymptotic state $ \ket{\psi_{\vec{p}}^{0}} $ in the limit $ t \to \infty $. 

    \item The asymptotic states are related by the scattering operator $ S $, which is represented by the scattering matrix. The operator's matrix elements
    \begin{equation*}
        S_{\vec{p}\vec{p}_{0}} = \mel{\psi_{\vec{p}}^{0}}{S}{\psi_{\vec{p}_{0}}^{0}}
    \end{equation*}
    are the probability amplitudes $ P_{\vec{p}\vec{p}_{0}} = \abs{S_{\vec{p}\vec{p}_{0}}}^{2} $ of detecting a particle in one of the asymptotic states $ \ket{\psi_{\vec{p}}^{0}} $. 

    The scattering states, like the asymptotic states, thus form a complete system of basis vectors so that we can expand the state $ \ket{\psi_{\vec{p}_{0}}^{0}} $ in the basis $ \{\ket{\psi_{\vec{p}}^{0}}\} $ to get
    \begin{equation*}
        \ket{\psi_{\vec{p}_{0}}^{+}} = \int \ket{\psi_{\vec{p}}^{-}} \braket{\psi_{\vec{p}}^{-}}{\psi_{\vec{p}_{0}}^{+}}\ddp  = \Omega_{+} \ket{\psi_{\vec{p}_{0}}^{0}}.
    \end{equation*}
    As time approaches $ t \to \infty $, the state $ \ket{\psi_{\vec{p}}^{-}} $ uniquely evolves into the asymptotic state $ \ket{\psi_{\vec{p}}^{0}} $. As a result, the probability amplitude $ \braket{\psi_{\vec{p}}^{-}}{\psi_{\vec{p}_{0}}^{+}} $ equals $ S_{\vec{p}\vec{p}_{0}} $, i.e.
    \begin{equation*}
        S_{\vec{p}\vec{p}_{0}} = \mel{\psi_{\vec{p}}^{0}}{S}{\psi_{\vec{p}_{0}}^{0}} = \braket{\psi_{\vec{p}}^{-}}{\psi_{\vec{p}_{0}}^{+}} = \mel{\psi_{\vec{p}}^{0}}{\Omega_{-}^{\dagger}\Omega_{+}}{\psi_{\vec{p}_{0}}^{0}}.
    \end{equation*}
    As a result, the scattering operator can be written in terms of the \Mol operator as
    \begin{equation*}
        S = \Omega_{-}^{\dagger}\Omega_{+}.
    \end{equation*}
    
    \item Finally, we note that the matrix elements $ S_{\vec{p}\vec{p}_{0}} $ can be decomposed into a component that preserves the momentum of the asymptotic states and a component that preserves only energy; the decomposition reads
    \begin{equation*}
        S_{\vec{p}\vec{p}_{0}} = \braket{\psi_{\vec{p}}^{0}}{\psi_{\vec{p}_{0}}^{0}} - 2\pi i \delta(E_{p} - E_{p_{0}})\mel{\psi_{\vec{p}}^{0}}{V}{\psi_{\vec{p}_{0}}^{+}},
    \end{equation*}
    where we can write the matrix element $ \mel{\psi_{\vec{p}}^{0}}{V}{\psi_{\vec{p}_{0}}^{+}} = \mel{\psi_{\vec{p}}^{0}}{T}{\psi_{\vec{p}_{0}}^{0}} $ in terms of the operator $ T = V\Omega_{+} $.
    
    
\end{itemize}

\subsubsection{Example: A Gaussian Wave Packet}
\begin{itemize}
    \item We consider scattering on a narrow, repulsive potential, which we model with a delta function with a positive coefficient:
    \begin{equation*}
        V(x) = \lambda \delta(x), \quad \lambda > 0.
    \end{equation*}
    The transmission and reflection amplitudes are given by
    \begin{equation*}
        t_{\lambda}(p) = \frac{p}{p + ip_{\lambda}} \qquad \text{and} \qquad r_{\lambda}(p) = \frac{-ip_{\lambda}}{p + ip_{\lambda}},
    \end{equation*}
    where $ p_{\lambda} = \lambda m $ and $ m $ is the scattered particle's mass. Note that because the potential is even, the problem is invariant under parity transformations and thus obeys $ t'_{\lambda} = t_{\lambda} $ and $ r'_{\lambda} = r_{\lambda} $.

    \item We represent the scattered particle with a Gaussian wavepacket moving along the $ x $ axis from left to right with group velocity $ v_{\text{g}} = \frac{\ev{p}}{m} $ and thus central position value $ \ev{x} = x_{0} + v_{\text{g}}t $. The corresponding wavefunction is
    \begin{equation*}
        \ket{\psi^{+}(t)} = \int \F{\psi}(p) e^{-i \frac{p^{2}}{2m \hbar}t } \ket{\psi_{p}^{+}} \diff p,
    \end{equation*}
    where the scattering basic vector $ \ket{p_{p}^{+}} $ is given by 
    \begin{equation*}
        \psi_{p}^{+}(x) = C
        \begin{cases}
            e^{i \frac{p}{\hbar} x} + r e^{-i \frac{p}{\hbar} x} & x < x_{a}\\
            \psi_{ab}(x) & x \in [x_{a}, x_{b}]\\
            t e^{i \frac{p}{\hbar}x} & x > x_{b}.
        \end{cases}
    \end{equation*}
    
    \item Without a detailed derivation, we qualitatively consider four regimes:
    \begin{enumerate}
        \item At early times, the particle is modeled by a standard Gaussian wave packet with a sigmoid shape. 

        \item As the particle approaches the potential barrier at $ x = 0 $ the wave packet begines to deform. 

        \item Once the wave packet's center reaches the potential barrier, the reflected and incident waves temporarily interfere to create a typical interference pattern on the left of the barrier. 

        \item After the wave packet's center has pass the barrier, both the reflected and transmitted waves are on their way towards the left and right, respectively, and the interference pattern loses its resolution. 

        \item At time well after the wave packet's center reached the barrier, the transmitted and reflected waves travel in their respective directions as essentially free wave packets, and begin to approach the asymptotic plane wave states $ \psi_{\pm p_{0}}^{0} \propto e^{\pm i \frac{p_{0}}{\hbar} x} $ as $ t \to \infty $.
    \end{enumerate}
    
\end{itemize}

\subsubsection{Example: Alpha Decay}
\begin{itemize}
    \item We now consider the example of radioactive alpha decay. The alpha particle initially occurs in a ground state between two potential barriers, encoded by the potential
    \begin{equation*}
        V(x) = \lambda(t)\delta(x) + \lambda_{\infty}\delta(x - L),
    \end{equation*}
    where the two constants $ \lambda $ and $ \lambda $ are very large, i.e. $ \lambda, \lambda_{\infty} \to \infty $, corresponding to a large potential barrier.

    \item At $ t = 0 $, the barrier $ \lambda(t) $ at $ x = 0 $ rapidly decreases to its final value $ \lambda(t) \to \lambda_{0} $. The scattering basis vector is zero for $ x \geq L $, since any waves reflect off the steep barrier $ \lambda_{\infty} $ at $ x = L $, and thus obeys
    \begin{equation*}
        \psi_{p}^{+}(x) = C
        \begin{cases}
            e^{i \frac{p}{\hbar} x} + B_{1} e^{-i \frac{p}{\hbar} x} & x < 0 \\
            \psi_{0_{L}}(x) & x \in [0, L]\\
            0 & x > L.
        \end{cases}
    \end{equation*}
    where we write the wavefunction in the region $ x \in [0, L] $ with the ansatz
    \begin{equation*}
        \Psi_{0_{L}} = A_{2} e^{-i \frac{p}{\hbar}x} + B_{2} e^{i \frac{p}{\hbar}x}.
    \end{equation*}
    The coefficients at $ x = 0 $ are related by the same symmetric scattering matrix from the Gaussian example via
    \begin{equation*}
        \begin{pmatrix}
            B_{1}\\
            B_{2}
        \end{pmatrix}
        = \SS
        \begin{pmatrix}
            1\\
            A_{2}
        \end{pmatrix},
    \end{equation*}
    and can be determined from the equation
    \begin{equation*}
        B_{2} e^{i \frac{p}{\hbar} L} = - A_{2} e^{-i \frac{p}{\hbar}L},
    \end{equation*}
    which follows from the continuity condition $ \psi_{0_{L}} = 0 $ at the boundary $ x = L $.

    \item We write the initial state at $ t = 0 $ in the basis of scattering states, where the time evolution proceeds according to the earlier equation
    \begin{equation*}
        \ket{\psi^{+}(t)} = \int \F{\psi}(p) e^{-i \frac{p^{2}}{2m \hbar}t } \ket{\psi_{p}^{+}} \diff p.
    \end{equation*}
    When the left barrier $ \lambda(t) $ at $ x = 0 $ decreases to its final value at $ t = 0 $, a probability current $ j_{x} < 0 $ begins to flow to the left, and the probability for finding the particle between the potential barriers begins to decrease. 

    \item We now return to the alpha particle's scattering state $ \psi_{p}^{+} $. Because all incident waves are reflected from the right barrier $ \lambda_{\infty} $ at $ x = 0 $, the components of the particle's wavepacket reflecting from $ \lambda_{\infty} $ also scatter off $ \lambda_{0}x $. The scattering matrix is thus relevant only for the region $ x < 0 $, to the left of the first barrier, where a scattering state is described by the reflection coefficient $ B_{1} $ via
    \begin{equation*}
        \psi_{p}^{+}(x) \big|_{x < 0} = C \left( e^{i \frac{p}{\hbar}x} + B_{1} e^{- i \frac{p}{\hbar}x} \right).
    \end{equation*}
    As a result, the scattering matrix $ \SS $ reduces to a scalar value $ \mathrm{S}_{11} $ relating the incident and reflected wave. This relationship reads 
    \begin{equation*}
        \mathrm{S}_{11} = e^{2 i \delta} = 1 + \left( e^{2i \delta} - 1 \right) = 1 + 2ie^{i\delta}sin \delta,
    \end{equation*}
    which is a scalar analog of the earlier matrix equation
    \begin{equation*}
        S_{\vec{p}\vec{p}_{0}} = \braket{\psi_{\vec{p}}^{0}}{\psi_{\vec{p}_{0}}^{0}} - 2\pi i \delta(E_{p} - E_{p_{0}})\mel{\psi_{\vec{p}}^{0}}{V}{\psi_{\vec{p}_{0}}^{+}}.
    \end{equation*}
    
\end{itemize}

\subsection{Scattering in Three Dimensions}
\begin{itemize}
    \item First, we give an overview of the scattering process:
    \begin{itemize}
        \item An incident quantum particle, described by a wave packet, approaches a scatterer (a target or potential barrier) and scatters from the target in a given spatial direction, where we then detect the particle.

        \item The incident wave packet must initially be wide enough that its width does not increase dramatically (with respect to the initial width) over the course of the scattering process. The wavepacket must be large with respect to the target and small with respect to the dimensinos of the laboratory---the second condition ensures the wave packet does not simultaneously cover the scatterer and the detector. 

        The wavepacket's width is determed by the cross-sectional width (cross-sectional area) of the beam of incident particles. 

        \item After the scattering interaction between incident particle and the target, we observe two wave packets: one continues past the target in the original direction of incidence, and one continues at an angle with respect to the direction of incidents and corresponds to the scattered particles. 

        \item The scattering cross section is the fundamental measurable quantity used to analyze scattering processes, and represents the number of particles scattered in a given element of solid angle per unit time per unit incident current.

    \end{itemize}

    \item We will work with particles expanded in a plane wave basis
    \begin{equation*}
        \psi_{k}(\r) = C e^{i \vec{k} \cdot \r},
    \end{equation*}
    where the constant $ C $ is independent of the choice of orthonormalization of the basis vectors. The probability current density for these plane waves is given by
    \begin{equation*}
        \vec{j}_{0} = j_{} \frac{\vec{p}}{p}, \qquad j_{0} = \frac{p}{m} \abs{C}^{2},
    \end{equation*}
    where $ j_{0} $ encodes the probability of a particle being incident on the surface $ S_{0} $ in the time interval $ \Delta t $. In this case we can choose the normaliation constant 
    \begin{equation*}
        \abs{C}^{2} = \frac{m}{pS_{0}\Delta t},
    \end{equation*}
    which represents a fourth possible means of naormalizing plane waves. 

    \item A typical scattering experiment involves a beam of incident particles with cross section $ S_{0} $ moving along the $ z $ axis and described by the probability current density $ j_{N} = \frac{N}{S_{0} \Delta t} = N j_{0} $.
    
\end{itemize}

\subsubsection{Expansion In Terms of Spherical Waves}

\begin{itemize}
    \item We consider the scattering state $ \psi_{k}^{+}(\r) $, which, in the limit of a vanishing scattering potential, approaches the plane wave $ \psi_{k}(\r) = Ce^{i k z} $. We begin our analysis by writing the plane wave in a spherical wave basis, i.e. 
    \begin{equation*}
        e^{ikz} = e^{ikr \cos \theta} = \sum_{l = }^{\infty} (2l + 1)i^{l}j_{l}(kr)P_{l}(\cos \theta),
    \end{equation*}
    where $ j_{l}(x) $ are the spherical Bessel functions and $ P_{l}(x) $ are the Legendre polynomials. 

    Next, we expand the spherical wave representation in the limit of large distance of the particle from the target, in which case the the Bessel function approaches the asymptotic limit $ j_{l}(x) \sim \frac{1}{x} \sin \big( x - l \frac{\pi}{2} \big) $. In this limit, the plane wave becomes
    \begin{equation*}
        \psi_{k}^{0} \equiv Ce^{ikz} \to C \sum_{l = 0}^{\infty} (2l + 1)\frac{i^{l}}{2ikr}\left[ e^{i(kr - l \frac{\pi}{2})} - e^{-i(kr - l \frac{\pi}{2})} \right]P_{l}(\cos \theta).
    \end{equation*}
    
    \item We assume the potential is spherically symmetric, in which case the problem's angular momentum is conserved, meaning we can expand the scattering state $ \lambda_{k}^{+} $ in terms of the angular momentum eigenstates analogously to the expansion of the plane wave $ \psi_{k}^{0} $. However, when expanding the scattering states in the angular momentum basis, we must also account for the scattering amplitude $ S_{l}(k) $. With the consideration of $ S_{l}(k) $ in mind, the expansion of the scattering states reads
    \begin{equation*}
        \psi_{k}^{+}(\r) \to C \sum_{l = 0}^{\infty} (2l + 1)\frac{i^{l}}{2ikr}\left[ S_{l}(k)e^{i(kr - l \frac{\pi}{2})} - e^{-i(kr - l \frac{\pi}{2})} \right]P_{l}(\cos \theta).
    \end{equation*}
    
    \item Expansion in terms of spherical waves is needed in three dimensions because, in three dimensions, a particle can scatter in any spatial direction. Every partial wave has an associated reflection amplitude in the radial direction. 

    For a given $ l $, a scattering state has the same structure as a wave reflected from an infinite potential barrier in the previous section, i.e. 
    \begin{equation*}
        \psi_{k}^{+}(\r) = C \left( e^{i \vec{k} \cdot \r} + B_{1} e^{- i \vec{k} \cdot \r} \right).
    \end{equation*}
    where the origin $ r = 0 $ corresponds to the infinite potential barrier. 
    
    \item In elastic scattering, probability is conserved in each scattering channel and the problem's scattering matrix is thus unitary, which allows us to introduce the phase term $ \delta_{l}(k) $ and write the scattering matrix components as
    \begin{equation*}
        \mathrm{S}_{l}(k) = e^{2i \delta_{l}(k)} = 1 + 2i e^{i \delta_{l}(k)} \sin \delta_{l}(k), \qquad \abs{S_{l}(k)} = 1.
    \end{equation*}
    We then decompose the scattering states into a non-scattered plane wave, which continues in the initial direction of incidence, and a scattered spherical wave. This decomposition reads
    \begin{align*}
        \psi_{\vec{k}}^{+}(\r) &\to C e^{ikz} + C \left[ \sum_{l = 0}^{\infty} (2l + 1) \frac{S_{l}(k) - 1}{2ik} P_{l}(\cos \theta) \right]\frac{e^{ikr}}{r}\\
        & = C \left( e^{ikz} + \frac{f(\theta, \phi)}{r} e^{ikr} \right),
    \end{align*}
    where we have defined the scattering amplitude
    \begin{equation*}
        f(\theta, \phi) = \frac{1}{k} \sum_{l = 0}^{\infty} (2l + 1)e^{i\delta_{l}(k)} \sin \delta_{l}(k) P_{l}(\cos \theta).
    \end{equation*}
    For a spherically symmetric potential, the scattering amplitude depends only on the polar angle because of rotational invariance about the $ z $ axis. 

    \item Finally, we note: the symbol $ + $ in the scattering state $ \psi_{\vec{k}}^{+} $ denotes the outgoing waves $ \frac{f}{r}e^{ikr} $, while the scattering state $ \psi_{\vec{k}}^{-} $ corresponds to the incident wave $ \frac{f}{r}e^{-ikr} $. The scattering amplitude $ f $ is porportional to the problem's $ T $ operator, which we defined in the section on the \Mol operator as $ T = V \Omega_{+} $, where $ \Omega_{+} $ is the \Mol operator.

    
\end{itemize}

\subsubsection{Scattering Cross Section}
\begin{itemize}
    \item Consider a beam of incident particles in a scattering experiment, where $ N $ particles in the beam pass through the beam's cross section $ S_{0} $ in the time $ \Delta t $, of which 
    \begin{equation*}
        N_{\text{s}} = P N
    \end{equation*}
    are scattered, where $ P $ is the probability of a given particle in the incident beam scattering off the target in any spatial direction. 

    The scattering cross section is defined as the ratio between the number of scattered particles per unit time and the number current density of incoming particles, defined as $ j_{N} = \frac{N}{S_{0}\Delta t}$. In equation form, the scattering cross section is denoted by $ \sigma $ and is defined as
    \begin{equation*}
        \sigma = \frac{1}{j_{N}}  \frac{N_{\text{s}}}{\Delta t} = \frac{1}{j_{0}} \frac{N_{\text{s}}}{N} \frac{1}{\Delta t} = \frac{1}{j_{0}} \frac{P}{\Delta t},
    \end{equation*}
    where $ j_{N} = N j_{0} $. Alternatively, we can write the scattering cross section in the form
    \begin{equation*}
        \sigma = \frac{N_{\text{s}}}{N}S_{0} = P S_{0},
    \end{equation*}
    which we interpret as a surface, with surface area $ \sigma $, through which a particle that has passed through the beam cross section $ S_{0} $ passes through with probability $ P = \frac{N_{\text{s}}}{N} = \frac{\sigma}{S_{0}}$.

    \item Next, we define the probability $ \diff P(\r) $ of a particle scattering into the element of solid angle $ \diff \Omega $ in the time interval $ \Delta t $. This probability is defined as
    \begin{equation*}
        \diff P(\r) = j_{\r} \diff S \Delta t = j_{0} \abs{f(\theta, \phi)}^{2} \diff \Omega \Delta t,
    \end{equation*}
    where $ j_{\r} $ is the probability current density in the direction $ \uvec{r} = \r /r $ and $ \diff S = r^{2} \diff \Omega $. We have related the current $ j_{\r} $ to the scattering amplitude $ f(\theta, \phi) $via
    \begin{equation*}
        j_{\r}(\theta, \phi) = \vec{j}(\r) \cdot \uvec{r} \stackrel{r \to \infty}{\longrightarrow} j_{0} \frac{\abs{f(\theta, \phi)}^{2}}{r^{2}}.
    \end{equation*}

    \item The differential cross section, denoted by $ \diff \sigma $, is defined as
    \begin{equation*}
        \diff \sigma = \frac{1}{j_{N}} \frac{\diff N_{\text{s}}}{\Delta t} = \frac{1}{j_{0}} \frac{\diff P}{\Delta t} = \abs{f(\theta, \phi)}^{2} \diff \Omega.
    \end{equation*}
    The differential cross section is independent of the scattering state normalization constant $ C $, and is related to the scattering amplitude via
    \begin{equation*}
        \frac{\diff \sigma}{\diff \Omega} = \abs{f(\theta, \phi)}^{2}.
    \end{equation*}
    Note that the expression is not a derivative of $ \sigma $ with respect to $ \Omega $---all of $ \frac{\diff \sigma}{\diff \Omega}  $ together is the differential cross section. The total and differential cross section are related by
    \begin{equation*}
        \sigma = \iint \frac{\diff \sigma}{\diff \Omega} \diff \Omega,
    \end{equation*}
    which should be read as an integral of the differential cross section over the entire solid angle. 
    
    \item The probability $ P $ of a particle scattering in any spatial direction in the time interval $ \Delta t $ is determined by the total cross section
    \begin{align*}
        \frac{1}{j_{0}}\frac{P}{\Delta t} = \sigma &= \iint \frac{\diff \sigma}{\diff \Omega} \diff \Omega = \int_{0}^{2\pi}\int_{0}^{\pi} \abs{f(\theta, \phi)}^{2} \sin \theta \diff \theta \diff \phi \\
        & = \int_{0}^{2\pi}\int_{0}^{\pi} \abs{\frac{1}{k} \sum_{l = 0}^{\infty} (2l + 1)e^{i\delta_{l}(k)} \sin \delta_{l}(k) P_{l}(\cos \theta)}^{2} \sin \theta \diff \theta \diff \phi \\
        & = \frac{4\pi}{k^{2}} \sum_{l = 0}^{\infty} (2l + 1) \sin^{2}\delta_{l}(k),
    \end{align*}
    where we have evaluated the integral using the orthogonality of the Legendre polynomials, i.e.
    \begin{equation*}
        \iint P_{l}(\cos \theta) P_{l'}(\cos \theta) \diff \Omega = \frac{4 \pi}{2l + 1}\delta_{ll'},
    \end{equation*}
    and the Legendre polynomial identity $ P_{l}(1) = 1 $.

    \item Because the total number of particles is conserved in scattering processes, the number of particles continuing passed the scatterer in the original direction of incident is naturally less than the number in the incident beam, which is encoded in the unitary nature of the scattering matrix in the relationship $ \abs{S_{l}(k)} = 1 $. 

    The scattering amplitude for scattering in the forward direction is described by
    \begin{align*}
        \Im \big[ f(\theta, \phi)\big |_{\theta = 0} \big] &= \frac{1}{k} \sum_{l = 0}^{\infty} (2l + 1) \Im \big[ e^{i\delta_{l}(k)} \big] \sin \delta_{l}(k) \\
        & = \frac{1}{k} \sum_{l = 0}^{\infty} (2l + 1) \sin^{2} \delta_{l}(k)\\
        &= \frac{k}{4\pi} \sigma.
    \end{align*}
    This result, which is derived using the Legendre polynomial identity $ P_{l}(1) = 1 $, is known as the optical theorem. 
    
    
\end{itemize}

\newpage
\section{Quantum Measurements}
\subsection{Stern-Gerlach Experiment}

\begin{itemize}
    \item Recall the axiom: that after a quantum measurement, the system occurs in an eigenstate of the operator being measured, and we measure an eigenvalue.

    \item Note that the Stern-Gerlach experiment occured before the evolution of the new quantum mechanics in 1925. 

    \item Two states. See picture

    Oven with silver at 1050 celsius. 

    Stress that the silver is chosen so the atoms are neutral and don't move in the magnetic field like a classical charged particle. 

    Beam of silver atoms in $ y $ direction through two filters.

    Through a large magnet---nonhomogeneous. 

    All inside a box in a vacuum. 

    \item In the absence of a magnetic field expect a single dot on the screen. 

    Instead get a funny-looking pattern of order 0.1 millimeter. Without a field is a straight line. 

    \item This device allows measurement of both spin and momentum. 

    \item Inside the magnet the spins split apart and get different momenta. 
\end{itemize}

\subsection{Stern-Gerlach SemiClassical}
\begin{itemize}
    \item Nonhomog magnetic momentum, assuming $ \curl \B = 0 $ and $ \m = \text{constant} $.
    \begin{equation*}
        \vec{F} = (\vec{\mu} \cdot \grad)\B(\r) = \grad (\m \cdot \B(\r)) = - \grad H_{\text{Zeeman}}
    \end{equation*}
    Recall
    \begin{equation*}
        H_{\text{Zeeman}} = - \m \cdot \B(\r)
    \end{equation*}
    although correct $ H_{\text{Z}}  $ is only for a homogeneous field.

    So that's the force on our particle.

    \item Assume nonhomog field
    \begin{equation*}
        \B(\r) = \B_{0} + \sum_{a} \pdv{\B}{x_{\alpha}}\bigg |_{0} x_{\alpha} + \cdots
    \end{equation*}
    Assume $ \B = B \uvec{e}_{z} $. 


    \item The field points occurs in the $ (x, z) $ plane. 
    
    And then $ \div \cdot \B  = 0 $ and so
    \begin{equation*}
        \pdv{B_{x}}{x} + \pdv{B_{z}}{z} = 0
    \end{equation*}
    
    \item Recall the beam is very small relative to the device dimensions. 

    We assume the force on the particle is
    \begin{equation*}
        F = + \mu \pdv{B_{z}}{z} \bigg |_{z = 0}
    \end{equation*}
    The force on the particle in the magnetic depends on the orientation of its magnetic moment---the particle will either bend up or down.

    Particle is in the magnet for the time
    \begin{equation*}
        \tau \sim \frac{L}{v_{y}}, \qquad m v_{z} = F \tau.
    \end{equation*}
    Second expression is impulse. For $ \mu B_{z} > 0 $, the force is $ F > 0 $ ie. upward. 

    We then have $ z_{\ua \da} = \pm v_{z} t $
    
    \item Assume particle enters magnet in quantum state $ \ket{\chi} $:
    \begin{equation*}
        \chi = 
        \begin{pmatrix}
            \cos \frac{\theta}{2}\\
            \sin \frac{\theta}{2}
        \end{pmatrix}
    \end{equation*}
    
    Probability for up is 
    \begin{equation*}
        P_{\ua} = \abs{\braket{\ua}{\chi}}^{2} = \cos^{2}\frac{\theta}{2} \qquad \text{and} \qquad P_{\da} = \abs{\braket{\da}{\chi}}^{2} = \sin^{2} \frac{\theta}{2} = 1 - P_{\da}
    \end{equation*}
    Expected value of $ \sigma_{z} $ is
    \begin{equation*}
        \ev{\sigma_{z}} = \mel{\chi}{\sigma_{z}}{\chi} = 
        \left( \cos \frac{\theta}{2}, \sin \frac{\theta}{2} \right) 
        \begin{pmatrix}
            1 & 0\\
            0 & -1
        \end{pmatrix}
        \begin{pmatrix}
            \cos \frac{\theta}{2}\\
            \sin \frac{\theta}{2}
        \end{pmatrix}
        = P_{\ua} - P_{\da} = \cos \theta
    \end{equation*}
    
    
\end{itemize}

\subsection{Quantum Stern-Gerlach Experiment}
\begin{itemize}
    \item Goal is to find time evolution of the Stern-Gerlach wavepacket. 

    Using anomolous Zeeman
    \begin{equation*}
        H = - \frac{\hbar^{2}}{2m}\laplacian - \frac{q}{m}\S \cdot \B
    \end{equation*}
    Energy and force, two constants
    \begin{equation*}
        E_{0} = \frac{q \hbar}{2m} B_{0} \qquad F = \frac{q \hbar}{2m} \pdv{B_{z}}{z} \bigg |_{z = 0}
    \end{equation*}

    and so
    \begin{equation*}
        H = H_{0} - E_{0} \sigma_{z} - F z \sigma_{z}
    \end{equation*}
    Note $ -\grad F z \sigma_{z} = F\sigma_{z} $.

    Write this as $ H = H_{0} + H_{\text{interaction}} $.
    
    The coordinate $ z $ is macroscopically measurable, we call it a pointer. 

    The $ H_{\text{int}} $ connects the measurable pointer and the quantum operator quantities.

    \item Magnet at $ y = 0 $. Wavepacket travels along $ y $ axis with speed $ v_{y} $:
    \begin{equation*}
        \frac{1}{m} \ev{p} = v_{y}. 
    \end{equation*}
    Speed is large, so the wp doesn't widen significantly in the magnet. 


    At time $ t = 0 $ the wp is just before the magnet at $ y \sim \frac{L}{2} $. 

    \begin{equation*}
        \Psi(\r, 0) = \psi(\r, 0)\chi = \psi(\r, 0) 
        \begin{pmatrix}
            \cos \frac{\theta}{2}\\
            \sin \frac{\theta}{2}
        \end{pmatrix}
    \end{equation*}
    Work in the spinor formalism. 

    Factor according to 
    \begin{equation*}
        \Psi(\r, t) = \psi_{0}(x, y, t) 
        \begin{pmatrix}
            \psi_{\ua}(z, t)\\
            \psi_{\da}(z, t)
        \end{pmatrix}
    \end{equation*}
    The $ x, y $ components don't affect the experiment. The spinor contains the probabilities for the incident particle occurring in either spin up or spin down.
    
    \item After wp passes through magnet, it splits into two wps. 

    Now consider $ \psi(\r, t) $
    \begin{equation*}
        \p(\r, t) = \psi_{0}(x, y, t) \int_{-\infty}^{\infty} \tilde{\psi}e^{i \frac{p}{\hbar}z - i \frac{E_{p}}{\hbar}t}\diff p 
    \end{equation*}
    where $ E_{p} = \frac{p^{2}}{qm} $. The integrand is $ \psi_{z}(z, t) $. Note that $ \ev{p} = 0 $ and $ p = p_{z} $.


    Assume system is exposed to interaction only briefly and that $ B $ is very strong. $ \tau = \frac{L}{v_{y}} $. 
    
    We thus consider only $ H_{\text{int}} $, since $ H_{0} $ is negligible. 

    \Schro equation for $ z $ dynamics
    \begin{equation*}
        i \hbar \pdv{t} 
        \begin{pmatrix}
            \psi_{\ua}(z, t)\\
            \psi_{\da}(z, t)
        \end{pmatrix}
        = - \left( \frac{\hbar^{2}}{2m}\pdv[2]{}{z} + E_{0}\sigma_{z} \pm F z \sigma_{z} \right)
        \begin{pmatrix}
            \psi_{\ua}(z, t)\\
            \psi_{\da}(z, t)
        \end{pmatrix}
    \end{equation*}
    At $ t = 0 $ we have
    \begin{equation*}
        \psi_{\ua}(z, t) = \p_{z}(z, t) \cos^{2}\frac{\theta}{2}
    \end{equation*}


    At later times
    \begin{equation*}
        \Psi(\r, t) = U \Psi(\r, 0) = e^{-i \frac{H_{\text{int}}}{\hbar}\tau}\Psi(\r, 0)
    \end{equation*}
    We leave out $ H_{0} $. 
    \begin{equation*}
        U = U_{\text{SG}} = e^{i \delta\phi \sigma_{z}} e^{i \frac{\delta\psi}{\hbar} \sigma_{z}z} 
    \end{equation*}
    where
    \begin{equation*}
        \delta\phi = \frac{E_{0}}{\hbar} \tau \qquad \text{and} \qquad \delta p = F\tau
    \end{equation*}


    Stress short time so leave out $ H_{0} $
    \item Now consider time evolutoin
    \begin{equation*}
        U_{\text{SG}}e^{\pm i \frac{p}{\hbar}z} \to e^{i \delta\phi} e^{i \frac{p + \delta p }{\hbar}z}
    \end{equation*}
    and so
    \begin{equation*}
        \begin{pmatrix}
            \psi_{\ua}(z, \tau)\\
            \psi_{\da}(z, \tau)
        \end{pmatrix}
        = \int_{-\infty}^{\infty} \tilde{\psi(p)}
        \begin{pmatrix}
            e^{i \delta \phi} e^{i \frac{p + \delta p}{\hbar}z}e^{-i \frac{E_{p+ \delta p}}{\hbar}\tau} \cos \frac{\theta}{2}\\
            e^{-i \delta \phi} e^{i \frac{p - \delta p}{\hbar}z}e^{-i \frac{E_{p - \delta p}}{\hbar}\tau} \sin \frac{\theta}{2}
        \end{pmatrix}
        \diff p
    \end{equation*}
    And then define new variable $ \tilde{p} = p + \delta p $ and $ p = \tilde{p} - \delta p $ to get
    \begin{equation*}
        \begin{pmatrix}
            \psi_{\ua}(z, \tau)\\
            \psi_{\da}(z, \tau)
        \end{pmatrix}
        = \int_{-\infty}^{\infty}
        \begin{pmatrix}
            \tilde{\psi}(p - \delta p) e^{i \delta\phi} \cos \frac{\theta}{2}\\
            \tilde{\psi}(p + \delta p) e^{-i \delta\phi} \sin \frac{\theta}{2}
        \end{pmatrix}
        e^{i \frac{p}{\hbar}z - i \frac{E_{p}}{\hbar}\tau} \diff p
    \end{equation*}
    And then send $ \tau \to 0 $. This is the wavefunction just after the interaction. 

    And one more notation of two lines up with energy written more clearly---note that this energy from $ H_{0} $ vanishes anyway as $ \tau \to 0 $. 
    \begin{equation*}
        \begin{pmatrix}
            \psi_{\ua}(z, \tau)\\
            \psi_{\da}(z, \tau)
        \end{pmatrix}
        = \int_{-\infty}^{\infty}
        \begin{pmatrix}
            \tilde{\psi}(p - \delta p) e^{i \delta\phi} \cos \frac{\theta}{2} e^{- i \frac{(p + \delta p)^{2}}{2m \hbar}\tau}\\
            \tilde{\psi}(p + \delta p) e^{-i \delta\phi} \sin \frac{\theta}{2}e^{- i \frac{(p + \delta p)^{2}}{2m \hbar}\tau}
        \end{pmatrix}
        e^{i \frac{p}{\hbar}z} \diff p
    \end{equation*}
    
    \item And then consider probability density
    \begin{equation*}
        \rho(z, t) = \rho_{\ua}(z, t) + \rho_{\da}(z, t)
    \end{equation*}
    The result is a curve $ N(z) $ corresponding to particle detection along the $ z $ axis of the detector. It has one peak at positive $ z $, on at negative $ z $, and a gap in between. The relative sizes of the peaks correspond to the initial probabilities in the inputted spin wavefunction $ \chi(0) $.
    
    Stress the macroscopic measurement is measuring the coordinate $ z $ of where particles are detected. 

    Okay then
    \begin{equation*}
        \Psi = \psi_{0}(x, y, t)
        \begin{pmatrix}
            \psi_{\ua}\\
            0
        \end{pmatrix}
    \end{equation*}
    So basically wavefunction collapse in that we either detect in spin up or down only. 
    \begin{equation*}
        \begin{pmatrix}
            \psi_{\ua}\\
            \psi_{\da}
        \end{pmatrix}
        \to 
        \frac{1}{\sqrt{\int_{-\infty}^{\infty} \abs{\psi_{\ua} (z, t)}^{2}\diff z}}
        \begin{pmatrix}
            \psi_{\ua}\\
            0
        \end{pmatrix}
        = \frac{1}{\sqrt{\cos^{2}\frac{\theta}{2}}}
        \begin{pmatrix}
            \psi_{\ua}\\
            0
        \end{pmatrix}
    \end{equation*}
    This is non-unitary time evolution because of wavefunction collapse. 
    
\end{itemize}

\subsection{General Quantum Measurement}
\begin{itemize}
    \item Classical measurement: determine property of a system that the system intrinsically has, whether we see the property or not.

    \item Bell: quantum mechanics can only predict the result of an experiment. It does not predict the system's property in the classical sense. 

    Over the process of the experiment, the system assumes a value that we then detect.

    \item VN or strong measurement: particle initially described by a linear superposition of all possible states, collapses into a single eigenstate. 

    But this is not really correct, we cannot say even that the particle is in a linear superposition of states, we in fact don't know anything about the particle's intrinsic properties. 

    All QM allows us to measure is the statistical probability with which we will detect outcomes from a large number of experiments. 

    However, after the measurement outcome, we can be sure the particle has the property we measured. We just know nothing about the particle's property before the measurement.

\end{itemize}

\subsubsection{VM Measurement}

\begin{itemize}
    \item Assume system with $ H = H_{0} + H_{\text{int}}(q, \hat{p}, t) $ where $ q $ and $ \hat{p} $ are position and momentum operators. 

    Short time $ \tau $ and some operator. Example of coupling of operator with momentum
    \begin{equation*}
        H_{\text{int}} = \frac{g}{\tau} \hat{A} \hat{p}
    \end{equation*}

    $ A $ has a basis
    \begin{equation*}
        \hat{A}\ket{n} = a_{n}\ket{n}
    \end{equation*}
    
    Initial state $ \ket{\Psi(0)} = \ket{\psi} \otimes \ket{\phi} $. The second state is supporting state
    \begin{equation*}
        \ket{\Psi(0)} = \sum_{n}\ket{n}\braket{n}{\psi} \otimes \ket{\phi(q)}
    \end{equation*}
    where $ c_{n} \braket{n}{\psi} $.
    
    Assume $ H_{\text{int}} $ turns on only for a short time, like a small pulse.


    \begin{equation*}
        \ket{\Psi(\tau)} = e^{-i \frac{H}{\hbar}\tau} \ket{\Psi(0)} = e^{-ig \hat{A}\hat{p}} \ket{\Psi(0)} \equiv U \ket{\Psi(0)}
    \end{equation*}
    and so 
    \begin{equation*}
        \to \sum_{N} e^{-ig \hat{A}\hat{p}} c_{n}\ket{n} \otimes \ket{\phi(q)} = \sum_{n}c_{n}\ket{n}\otimes e^{i g a_{n} \hat{p}}\ket{\phi(q)} = \sum_{n}c_{n}\ket{n} \otimes \ket{\phi(q - ga_{n})} 
    \end{equation*}
    e.g. for spin
    \begin{equation*}
        = c_{\ua}\ket{\ua} \phi(q - ga_{\ua}) + c_{\da}\ket{\da}\phi(q - ga_{\ua})
    \end{equation*}
    and for $ t > \tau $ after the interaction
    \begin{equation*}
        \ket{\psi(t)} = e^{-i \frac{H_{0}}{\hbar}(t- \tau)} \ket{\psi(\tau)}
    \end{equation*}
    
    \item We now consider coupling with a coordinate
    \begin{equation*}
        H_{\text{int}} = - \frac{g}{\tau}\hat{A} q
    \end{equation*}
    and then
    \begin{align*}
        \ket{\psi(0)} &= \sum_{n}c_{n}\ket{n} \otimes \int \tilde{\phi}(p)\ket{p}\diff p
    \end{align*}
    and $ U = e^{+ig \hat{A}q} $ and then
    \begin{align*}
        \ket{\Psi(\tau)} & = \sum_{n}c_{n}\ket{n} \otimes \int \tilde{\phi}(p)e^{ig \frac{a_{n}}{\hbar}q} \ket{p} \diff p \\
        & = \sum_{_{n}c_{n}}\ket{n} \otimes \int \tilde{\phi}(p - ga_{n})\ket{p} \diff p
    \end{align*}
    supposedly we called $ ga_{n} \delta p_{n} $ in spin.

    and then expect
    \begin{equation*}
        \ev{q}_{n} = \frac{g a_{n}}{m}t, \qquad  \ev{p} = ga_{n}
    \end{equation*}
    This is unitary so far.
    
    \item And to measure $ \hat{A} $ we have
    \begin{equation*}
        \ket{\Psi(t_{0})} = e^{-i \frac{H_{0}}{\hbar}(t_{0} - \tau)}\ket{\Psi(\tau)}
    \end{equation*}
    This cgoes into a quantum system regime, then a few interactions $ H_{\text{int}} $, $ H_{\text{int}_{1}} $, $ H_{\text{int}_{2}} $ and so on...

    And in the end we measure position, the pointer. The series of interactions progressively brings the quantities in the experiment closer to the macroscopically measurable quantity. The interactions are all unitary in the quantum regime. 

    And then the wavefunction ends up in our macroscopic detector, and we detect it, and it is normalized. 

    We have 
    \begin{equation*}
        \ket{\psi} \to \ket{n}
    \end{equation*}
    i.e. some superposed $ \ket{\psi} $ to a single eigenstate $ \ket{n} $, where the probability for detecting that $ \ket{n} $ is $ P_{n} = \abs{c_{n}}^{2} $, e.g. in the SG experiment these were the $ \cos^{2}\frac{\theta}{2} $ and $ \sin^{2}\frac{\theta}{2} $ terms. 
    
    % \item In general, we note that there exists a measurement operator $ M_{n} $ e.g. a projector $ \ket{n}\bra{n} = P_{n} $ which we normalize so something like
    % \begin{equation*}
    %     \ket{\psi} \stackrel{M_{n}}{\longrightarrow} \frac{M_{n}\ket{\psi}}{\sqrt{\mel{\psi}{M_{n}^{\dagger}M_{n}}{\psi}}}
    % \end{equation*}
    % if $ M_{n} = P \to \ket{n} $ but idk wtf this means lol.
    
\end{itemize}
\textbf{What Comes Out of the SG Oven?}
\begin{itemize}
    \item Consider concept of a pure state, for which we have
    \begin{equation*}
        \ket{\chi} = \sum_{n} c_{n}\ket{n}
    \end{equation*}


    \item We have assumed so far for a beam particle
    \begin{equation*}
        \chi = 
        \begin{pmatrix}
            c_{\ua}\\
            c_{\da}
        \end{pmatrix}
    \end{equation*}
    But really this a pure state---actually more complicated. Because of entanglement with everything else in the oven. 


    \item Expected value
    \begin{align*}
        \ev{A} &= \mel{\chi}{A}{\chi} = \sum_{mn}c_{n}^{*}c_{m}\mel{n}{A}{m}  \\
        \equiv & = \sum_{mn} \rho_{mn}A_{nm} = \tr \rho A
    \end{align*}
    where we have defined the operator
    \begin{equation*}
        \rho = \sum_{mn}\rho_{mn}\ket{m}\bra{n}
    \end{equation*}
    We can also write $ \rho = \ket{\chi}\bra{\chi} = P_{\chi} $. 

    We call this a density matrix for a pure state.
    
    \item And now a mixed state. A hypothetical box with two particles, together described by
    \begin{equation*}
        \ket{\Psi} = \sum_{m m_{1}} c_{mm_{1}} \ket{m}\otimes \ket{m_{1}}
    \end{equation*}
    where $ \ket{m_{1}} $ is the basis of the first particle. Note that this $ \ket{\Psi} $ is a complicated state---it is a big linear combination. It is not a pure state.
    
    \item Idea: One particle exits the box.

    We write density matrix for what is outside of the box, i.e. a mixed-state density matrix.
    \begin{equation*}
        \ev{A} = \tr(\rho A) 
    \end{equation*}
    And then we speak of quantum entanglement I think of the particle inside and outside of the box. 


    \item In SG we have not one but $ \sim 10^{23} $ particles in the box and a few particles flying out, and these particles are coupled to all of the $ \sim 10^{23} $ left in the box.

    
    \item And then
    \begin{equation*}
        \rho = \sum_{mn} \rho_{mn} \ket{m}\bra{n}
    \end{equation*}
    As earlier. 
    
    
\end{itemize}

\textbf{More on Mixed States}
\begin{itemize}
    \item Particle in box with $ \ket{m_{1}} $ and particle out with $ \ket{m} $. Total state
    \begin{equation*}
        \ket{\Psi} = \sum_{mm_{1}}c_{mm_{1}} \ket{m}\otimes \ket{m_{1}}
    \end{equation*}
    consider some Operator $ A = A\otimes \II $. Find expval for total system
    \begin{equation*}
        \ev{A} = \mel{\Psi}{A\otimes \II}{\Psi} = \sum_{} c_{nn_{1}}^{*}c_{mm_{1}}\mel{n}{A}{m} \braket{n_{1}}{m_{1}}
    \end{equation*}
    Everything with index 1 is in the box and orthonormal and $ \braket{n_{1}}{m_{1}} = \delta_{m_{1}n_{1}} $.
    \begin{equation*}
        \ev{A} = \sum_{nm n_{1}} c_{nm_{1}}^{*} c_{mm_{1}} A_{nm}
    \end{equation*}
    We then write the sum in the form
    \begin{equation*}
        \ev{A} = \sum_{nm}\sum_{m_{1}} c_{nm_{1}}^{*} c_{mm_{1}}A_{nm}
    \end{equation*}
    We call $ \rho_{mn} = c_{nm_{1}}^{*} c_{mm_{1}} $, index order is important, and get
    \begin{equation*}
        \ev{A} = \sum_{mn}\rho_{mn}A_{nm} = \tr \big[\rho A \big]
    \end{equation*}
    and note density matrix for the mixed state is
    \begin{equation*}
        \rho_{mn} = \sum_{m_{1}}c_{nm_{1}}^{*} c_{mm_{1}}
    \end{equation*}
    

    
    
\end{itemize}

\textbf{Properties of the Density Matrix}
\begin{itemize}
    \item Used for systems of many particles. 

    $ N $ particles in box, one flies out. Total psi
    \begin{equation*}
        \ket{\Psi} = \sum_{m} \sum_{m_{1}m_{2}\cdots} c_{m m_{1}\cdots m_{N}} \ket{m} \otimes \big( \ket{m_{1}}\otimes \cdots \otimes \ket{m_{N}} \big)
    \end{equation*}
    Find expavale of operator for the outer particle, assuming the $ c $ are known. 
    \begin{equation*}
        \ev{A} = \mel{\Psi}{A}{\Psi}
    \end{equation*}
    Density matrix is
    \begin{equation*}
        \rho_{mn} = \sum_{m_{1} \cdots m_{N}} c_{n n_{1}n_{N}}^{*} c_{m m_{1}m_{N}}^{*}= \sum \abs{c_{m m_{1}m_{N}}^{*} }^{2}
    \end{equation*}
    And then in eigenbasis:
    \begin{equation*}
        \rho = 
        \begin{pmatrix}
            p_{1} & 0\\
            0 & p_{2}
        \end{pmatrix}
        \qquad p_{1}, p_{2} \geq 0
    \end{equation*}
    And then
    \begin{equation*}
        \rho \ket{k} = p_{k}\ket{k} \qquad \rho = \sum_{k} p_{k} \ket{k}\bra{k}
    \end{equation*}
    From before $ \tr \rho = p_{1} + p_{2} \equiv 1 $ and so 
    \begin{equation*}
        \sum_{k} p_{k} = 1
    \end{equation*}
    and $ 0 \leq p_{k} \leq 1 $. 

    \item in the eigenbasis, the expval for the measured operator is
    \begin{equation*}
        \ev{A} = \tr \rho A = \sum_{k} p_{k} \mel{k}{A}{k} = \sum_{k}p_{k}A_{kk}
    \end{equation*}
    I think we've assumed $ A $ is spin-like and hence the two by two matrix. 

    \item The trace obeys
    \begin{equation*}
        \tr \rho^{2} = \sum_{k} p_{k}^{2} \leq \sum_{k} p_{k} = \tr \rho
    \end{equation*}
    So the sruaed trace is always less than or equal the normal trace. They're equal if $ p_{k} = \delta_{kk_{0}} $ but what is that?
    

    \item Assume two particle box system. $ m_{1} $ in box with mutual $ \ket{\Psi} $ for both. As usual expval of operator is
    \begin{equation*}
        \ev{A} = \tr \rho A 
    \end{equation*}
    and $ \rho^{2} = \rho $ for a pure state and $ \rho = \ket{\chi}\bra{\chi} $. 

    And then
    \begin{equation*}
        \ket{\Psi} = \ket{\chi} \otimes \ket{\chi_{1}}
    \end{equation*}
    And now we're assuming what we measure outside the box is uncoupled from the particle in the box, so we're dealing with a pure state now wut lol. Oh I thinks its once we measure $ \ev{A} $ that we get a pure state.
    
\end{itemize}

\textbf{Back to SG System}
\begin{itemize}
    \item So $ N \sim 10^{23} $ particles in oven. We assume they are independent wherever its convenient. What's the density matrix for a particle that flies out of the box. 

    \item In analogy with statistical physics. Assume thermodynamic equilibrium in the box. Use ansatz
    \begin{equation*}
        \rho = \frac{1}{Z} e^{-\beta H} \qquad \beta = \frac{1}{k_{B}T}
    \end{equation*}
    where $ H $ is \Ham operator. Assume magnetic field coupling
    \begin{equation*}
        H = - \frac{\hbar^{2}}{2m}\laplacian - \m \cdot \B \equiv H_{0} - \m \cdot \B
    \end{equation*}

    \item Next step is to find $ \rho $ and $ \ev{A} $.

    eigenbasis of $ \rho $ is of $ H $
    \begin{equation*}
        H \ket{k} = E_{k}\ket{k}
    \end{equation*}
    and then
    \begin{equation*}
        \rho = \frac{1}{Z} \sum_{k} e^{-\beta E_{k}} \ket{k}\bra{k}, 
    \end{equation*}
    Trace is one. 
    \begin{equation*}
        Z = \sum_{k} e^{-\beta E_{k}}
    \end{equation*}


    \item Example with spin in $ \ket{\ua} $ and $ \ket{\da} $ basis:
    \begin{equation*}
        \rho = \frac{1}{e^{+\beta E_{B}} + e^{-\beta E_{B}}} \left( e^{\beta E_{B}} \ket{\ua} \bra{\ua} + e^{-\beta E_{B}} \ket{\da}\bra{\da} \right)
    \end{equation*}
    where $ E_{B} = - \mu B $.

    And then suppose we want to find $ \sigma_{Z} = \ket{\ua} \bra{\ua} - \ket{\da} \bra{\da}  $.

    As usual formula
    \begin{equation*}
        \ev{\sigma_{z}} = \tr \rho \sigma_{z} = \mel{\ua}{\rho\sigma_{z}}{\ua} + \mel{\da}{\rho \sigma_{z}}{\da} = \frac{e^{\beta E_{B}} - e^{-\beta E_{B}}}{e^{\beta E_{B}} + e^{-\beta E_{B}}} = - \tanh \frac{\mu B_{0}}{k_{B} T}
    \end{equation*}
    In other words, the average spin orientation $ \ev{\sigma_{z}} $ depends on the value of $ \frac{\mu B_{0}}{k_{B} T} $, which can be controlled with temperature or $ B_{0} $.
    
\end{itemize}

\textbf{Two Particles}
\begin{itemize}
    \item Wavefunction 
    \begin{equation*}
        \ket{\Psi} = c \ket{\ua} \otimes \left( a \ket{\da} + b \ket{\ua} \right)
    \end{equation*}
    Which is a product of pure states. Density matrix is then $ \rho = \ket{\ua}\bra{\ua} $. 

    \item Example: mixed
    \begin{equation*}
        \ket{\Psi} = a \ket{\ua} \otimes \ket{\ua} + b \ket{\ua} \otimes \ket{\da}
    \end{equation*}
    This is a mixed state. The particle outside the hypothetical box is in a mixed state and has 
    \begin{equation*}
        \rho = \abs{a}^{2} \ket{\ua} \bra{\ua} + \abs{b}^{2} \ket{\ua}\ket{\da} + \cdots 
    \end{equation*}
    Just in passing...
    
    
\end{itemize}

\subsubsection{Decoherence: More on SG Experiment}
\begin{itemize}
    \item We consider the second quantum approach with two splitting wave packets. 

    Decoherence involves the transition from the quantum to the macroscopic regime, e.g. when we observe the outcome of a measurement.

    \item Recall
    \begin{equation*}
        \Psi(\r, t) = \psi_{0}(x, y, t) \left( c_{1}
        \begin{pmatrix}
            1\\
            0
        \end{pmatrix}\phi_{\ua}(z, t)
        + c_{2}
        \begin{pmatrix}
            0\\
            1
        \end{pmatrix}
        \phi_{\da}(z, t) \right)
    \end{equation*}
    where
    \begin{equation*}
        \phi_{\ua, \da} (z, t) \sim e^{-i \frac{\delta p}{\hbar} (z \mp vt)} \phi_{0}(z \mp vt)
    \end{equation*}
    These two functions encode probability amplitudes for spin up or down. 
    
    The real and imaginary components of $ \phi{\ua \da} $ oscillate in $ z $ space. They have wavelength much less than the characteristic dimensions of the experiment, e.g. the spacing $ \Delta z $ on the measurement screen. 

    Example: for $ \Delta z \sim \SI{0.1}{\milli \meter} $ there are $ \sim 10^{6} $ wavelengths in one $ \Delta z $ in the SG experiment.
    
    \item Measure $ \ev{\sigma_{z}} $ and get
    \begin{equation*}
        \ev{\sigma_{z}}  = \int \abs{c_{1}^{2} \phi_{0} (z - vt)}a - \abs{c_{2} \phi_{0} (z + vt)}^{2} \diff z = \abs{c_{1}}^{2} - \abs{c_{2}}^{2} = P_{\ua} - P_{\da}
    \end{equation*}

    \item The idea for decoherence seems to be that for e.g. $ \ev{\sigma_{x}} $ which varies largely with $ z $, then idk something wierd happens haha 
   
    
\end{itemize}

\subsection{Quantum Entanglement}
\begin{itemize}
    \item Two particles. Describe with wavefunction such that distance $ l $ between particles is known e.g.
    \begin{equation*}
        \psi(x_{1}, x_{2}) = \exp \left[ - \frac{(x_{1} - x_{2} - l)^{2}}{4 \Delta x^{2}} \right]
    \end{equation*}
    assume $ \frac{\Delta x}{l} \ll 1 $. Problem is say, we detect particle 1 somewhere. Then the location of particle 2 is already known from the wavefunction, which raises questions of sort of instantaneous communication between the particles.
    
    \item EPR paradox. We encode the two particles with the mixed state
    \begin{equation*}
        \ket{\psi} = a \ket{\ua} \otimes \ket{\da} + b \ket{\da} \otimes \ket{\ua} = \frac{1}{\sqrt{2}} \big( \ket{\ua}\ket{\da} - \ket{\da}\ket{\ua} \big)\psi(\r_{1}, \r_{2})
    \end{equation*}
    Confine the two particles in a small enclosure at $ t = 0 $. Open the box at $ t = 0^{+} $. 

    No magnetic field. Total spin of system is zero---particles then must have opposite spin. Idea is once we detect spin of one particle, we immediately know the spin projection of the second particle. 

    Wavefunction after measurement and detecting particle 1 in e.g. spin up simplifies to
    \begin{equation*}
        \ket{\Psi} = \ket{\ua}\ket{\da}
    \end{equation*}
    We immediately know the second particle is spin $ \ket{\da} $, even if it is an arbitrary distance from the first particle. So information seems to travel instantaneously. 
    
    
    
    
    
\end{itemize}


\end{document}

