\documentclass[11pt, a4paper]{article}
\usepackage[T1]{fontenc}
\usepackage{mwe}
\usepackage[margin=3.5cm]{geometry}
\usepackage{amsmath}
\usepackage{amssymb}
\usepackage{bm} % for bold vectors in math mode
\usepackage{physics} % many useful physics commands
\usepackage[separate-uncertainty=true]{siunitx} % for scientific notation and units

\usepackage{graphicx}
\graphicspath{{"../figures/"}}
\usepackage[section]{placeins} % to keep figures in their sections
\usepackage{subcaption} % for subfigure captions
\usepackage[export]{adjustbox} % centers figures wider than text width

\usepackage{xcolor}  % to color hyperref links
\usepackage[colorlinks = true, allcolors=blue]{hyperref}

\setlength{\parindent}{0pt} % to stop indenting new paragraphs
\newcommand{\eqtext}[1]{\qquad \text{#1} \qquad}
\newcommand{\diff}{\mathop{}\!\mathrm{d}} % differential


\begin{document}
\title{Fourier-Transform Spectroscopy}
\author{Elijan Mastnak}
\date{Winter Semester 2020-2021}
\maketitle
%\tableofcontents
	
\section{Tasks}
\begin{enumerate}
	\item Investigate the relationship between the interferogram and spectrum's widths.
	
	\item Investigate which quantities determine the spectrum's resolution.
\end{enumerate}

\section{Equipment,  Procedure and Data}

\subsection{Equipment}
Although the experiment was performed in an online simulation, a physical experiment would involve:
\begin{enumerate}
	\item Light sources: helium-neon laser, a mercury bulb, and a white-light flashlight
	
	\item Michelson interferometer with a movable mirror
	
	\item Photodiode and ammeter to measure photocurrent
\end{enumerate}

\subsection{Procedure}
Use the online \texttt{iPython} simulator to measure the interferogram of a helium-neon laser, a mercury bulb, and a white flashlight.

\subsection{Data}
\textbf{Independent Variable:} Displacement $ x $ of the interferometer's movable mirror relative to the equilibrium position.

\vspace{2mm}
\textbf{Dependent:} Photodiode current $ I_{\text{diode}} $, which is proportional to the intensity of the detected light.

\vspace{2mm}
\textbf{Parameters:} Mirror displacement step size: \SI{0.05}{\micro \meter}. 

\section{Discussion}

\subsection{Interferogram and Spectrum}
\begin{itemize}
	\item A purely sinusoidal interferogram, corresponding to monochromatic light, (i.e. an ideal laser) has a spectrum with a single discrete component.
	
	\item An interferogram with a finite number of frequency components (e.g. a mercury lamp with emission peaks at a few specific frequencies) has a Fourier spectrum with finitely many discrete spikes at the corresponding frequencies. 
	
	\item An interferogram with an appreciable non-zero response only in small displacement range (e.g. approximately white light) has a continuous Fourier spectrum.
\end{itemize}


\subsection{Spectral Resolution}
\begin{itemize}
	\item The energy resolution in Fourier spectroscopy is proportional to the maximum displacement $ x_{\text{max}} $ of the movable interferometer mirror from the equilibrium position. 
	
%	The relationship between wavelength resolution and maximum displacement is
%	\begin{equation*}
%		\frac{\lambda}{\Delta \lambda} = 4 kx_{\text{max}}
%	\end{equation*}
	
	\item This is the manifestation of a more general result, namely that the resolution of a discrete Fourier transform in the conjugate domain (e.g. frequency $ f $, wave vector $ k $, etc...) improves with the length of the input signal in the original domain (e.g. time $ t $, position $ x $, etc...). 
	
	\item Meanwhile, the spectral bandwidth of the Fourier transform improves with the sample rate of the original signal.
	
\end{itemize}

\section{Results}

The simulated experiment produced the following emission line spectrum for the mercury lamp:
\begin{table}[htb!]
\centering
\begin{tabular}{c|c|c}
	Wave vector $ 2\pi k $ $ [\si{\micro \meter^{-1}}] $  & Wavelength $ \lambda $ $ [\si{\nano \meter}] $ & Frequency $ f $ $ [\si{\tera \hertz}] $\\
	\hline
	1.78 & 562 & 534 \\
	1.83 & 546 & 549 \\
	2.30 & 435 & 690 \\
	2.48 & 403 & 744
\end{tabular}
\caption{Measured emission line date for the mercury vapor lamp. The reference values for the same lines are 578.2, 546.1, 435.8 and \SI{404.7}{\nano \meter}. }
\end{table}

Meanwhile, the helium-neon laser showed the following spectral characteristics:
\begin{equation*}
	2\pi k = \SI{1.58}{\micro \meter^{-1}} \qquad \lambda = \SI{633}{\nano \meter} \qquad f = \SI{474}{\tera \hertz}
\end{equation*}
The reference wavelength for a helium-neon laser is $ \lambda_{\text{HeNe}} = \SI{632.8}{\nano \meter} $. Of course, the agreement between the measured and reference spectral data for both the mercury lamp and helium-neon laser is artificial, since the simulation was reverse-engineered to produce the reference values.

\newpage


\section{Interferogram and Spectrum Plots}

\begin{figure}[htb!]
\centering
\includegraphics[width=\linewidth]{gauss}
\caption{Michelson interferogram and Fourier spectra of a (computer simulated) white flashlight, mercury lamp and helium-neon laser. The interferogram is apodized with a Gauss filter.}
\end{figure}


\begin{figure}
\centering
\includegraphics[width=\linewidth]{cosine}
\caption{Michelson interferogram and Fourier spectra of a (computer simulated) white flashlight, mercury lamp and helium-neon laser. The interferogram is apodized with a cosine filter.}
\end{figure}


\begin{figure}
\centering
\includegraphics[width=\linewidth]{bartlett}
\caption{Michelson interferogram and Fourier spectra of a (computer simulated) white flashlight, mercury lamp and helium-neon laser. The interferogram is apodized with a Bartlett (absolute value) filter.}
\end{figure}


\begin{figure}
\centering
\includegraphics[width=\linewidth]{box}
\caption{Michelson interferogram and Fourier spectra of a (computer simulated) white flashlight, mercury lamp and helium-neon laser. The interferogram is apodized with a box filter.}
\end{figure}


\appendix

\section{Theory}
%Advantages:
%\begin{itemize}
%	\item We measure the entire the light current's entire spectral bandwidth, which improves signal to noise ratio.
%\end{itemize}

The intensity (irradiance) of light incident on the detector is proportional to the time average of the energy:
\begin{equation*}
	I_{\text{det}} \propto \ev{\abs{E_{\text{det}}}^{2}}
\end{equation*}
When both split beams incident on the detector are monochromatic and equally intense, we can write the intensity on the detector as
\begin{equation*}
	I_{\text{det}}(x) = I_{0}\big[1 + \cos(2k x)\big]
\end{equation*}
where $ I_{0} $ is the intensity of the incident light on the beam splitter, $ k $ is the light's wave vector and $ x $ is the displacement of the movable interferometer mirror. If the incident contain a spectral distribution $ S(k) $ of many frequencies, the corresponding intensity on the detector is
\begin{equation*}
	I_{\text{det}} = \int_{0}^{\infty}S(k)(1 + \cos(2kx)) \diff k
\end{equation*}

The convolution of the spectrum with a unit step function spreads out the spectral peaks and degrades their resolution.


\end{document}