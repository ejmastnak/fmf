\documentclass[11pt, a4paper]{article}
\usepackage[T1]{fontenc}
\usepackage{mwe}
\usepackage[margin=3.5cm]{geometry}
\usepackage{amsmath}
\usepackage{amssymb}
\usepackage{bm} % for bold vectors in math mode
\usepackage{physics} % many useful physics commands
\usepackage[separate-uncertainty=true]{siunitx} % for scientific notation and units

\usepackage{graphicx}
\graphicspath{{"../figures/"}}
\usepackage[section]{placeins} % to keep figures in their sections
\usepackage{subcaption} % for subfigure captions
\usepackage[export]{adjustbox} % centers figures wider than text width

\usepackage{xcolor}  % to color hyperref links
\usepackage[colorlinks = true, allcolors=blue]{hyperref}

\setlength{\parindent}{0pt} % to stop indenting new paragraphs
\newcommand{\eqtext}[1]{\qquad \text{#1} \qquad}
\newcommand{\diff}{\mathop{}\!\mathrm{d}} % differential
\newcommand{\isoptope}[2]{${}^{#2}${#1}}


\begin{document}
\title{Scintillation Detector Spectroscopy}
\author{Elijan Mastnak}
\date{Winter Semester 2020-2021}
\maketitle

\tableofcontents
		
\newpage

\section{Tasks}
\begin{enumerate}
	
	\item With both the single and multi-channel analyzers, calibrate the scintillation spectrometer using the \isoptope{Na}{22} source, then measure the spectra and gamma ray energies of the \isoptope{Cs}{137} and \isoptope{Co}{60} sources.

	
	\item Identify the photopeak, Compton peak, and back-scattering peak in the \isoptope{Ce}{137} spectrum. Estimate the energy of the back-scattering peak.
	
	\item Measure the energy resolution of the photopeak for all three sources. Qualitatively discuss how the resolution changes with energy.
	
%	\item \textbf{TODO} Measure the \isoptope{Cs}{137} spectrum again with a collimator between the source and detector. Compare the resolution of the collimated and un-collimated spectral peaks.
	
	\item Estimate the NaI(Ti) crystal's photon absorption efficiency in the \isoptope{Ce}{137} photopeak.
	
\end{enumerate}

\section{Equipment and Procedure}
\subsection{Equipment}
\begin{enumerate}
	\item A scintillation detector, consisting of a NaI(Ti) crystal, photomultiplier, and pre-amplification units. Used to convert gamma decays into measurable current pulses. CAEN N471 high-voltage power supply for the photomultiplier.
	
	\item Single-channel and multi-channel analyzers to measure the gamma ray spectrum. 
	
	\item \isoptope{Na}{22}, \isoptope{Cs}{137} and \isoptope{Co}{60} gamma ray sources.
\end{enumerate}

\subsection{Sketch of Procedure}
\begin{enumerate}
	\item Connect all elements to their power supplies. Place the \isoptope{Na}{22} source on scintillator and observe the gamma ray current pulses on the oscilloscope. 
	
	\item For each source, use the single-channel analyzer to measure the dependence of pulse counts on the photomultiplier's lower level.
	
	\item Use the multi-channel analyzer to measure the background spectrum. Then use the MCA to measure each source's spectrum. In raw form, the spectrum is counts as a function of channel number. 
	
\end{enumerate}


\subsection{Data}
\textbf{Part 1 (SCA):} Independent variables are the SCA's lower level and window size (in volts) and measurement time. Dependent variable is the number of gamma ray counts in a given SCA window over a given measurement time.

\vspace{2mm}
\textbf{Part 2 (MCA):} Independent variables are channel number and measurement time. Dependent variable is gamma ray counts per channel over a given measurement time.


\section{Analysis}
\subsection{Single-Channel Analyzer}
Using the SCA \isoptope{Na}{22} data, I identified the SCA lower level numbers $ n_{1} $ and $ n_{2} $ corresponding to the  $ E_{1} = \SI{0.51}{\mega \electronvolt} $ and $ E_{2} = \SI{1.277}{\mega \electronvolt} $ peak, and then interpolated a linear $ E(n) $ characteristic with
\begin{equation*}
	E(n) = E_{1} + \left(\frac{E_{2}-E_{1}}{n_{2}-n_{1}}\right)(n - n_{1})
\end{equation*}
I then estimated the energies of the \isoptope{Cs}{137} and \isoptope{Co}{60} peaks using the $ E(n) $ equation. Table \ref{spegam:table:sca-energies} shows the results.

\begin{table}[h]
\begin{center}
    \begin{tabular}{c|c|c|c}
         Isotope &  Lower Level [V] &  Energy Estimate [\si{\mega \electronvolt}]& True Energy [\si{\mega \electronvolt}]\\
        \hline {\rule{0pt}{2.6ex}} \hspace{-7pt}  % for space below hline
        \isoptope{Na}{22} &  $ 1.6 \pm 0.2 $ & - & 0.511\\
        \isoptope{Na}{22} &  $ 3.8 \pm 0.2 $ & - & 1.277\\
        \isoptope{Cs}{137} &  $ 2.4 \pm 0.2 $ & $ 0.80 \pm 0.38 $ & $ 0.667 $\\
        \isoptope{Co}{90} &  $ 4.4 \pm 0.2 $ & $ 1.48 \pm 0.17 $ & 1.173\\
        \isoptope{Co}{90} &  $ 5.0 \pm 0.2 $ & $ 1.68 \pm 0.16 $ & 1.332
	\end{tabular}
	\caption{Rough estimates of spectral line energies using the single-channel analyzer.}
	\label{spegam:table:sca-energies}
\end{center}
\end{table}
\textit{Note}: If I remember correctly, I slightly adjusted the photomultiplier's amplification setting when switching to the cobalt source, which explains the systematically too-large energy estimate for the cobalt peaks.

\begin{figure}
	\centering
	\includegraphics[width=\linewidth]{sca}
	\caption{The spectrum of each source measured with the single-channel analyzer. Note the very coarse measurements---see Figure \ref{spegam:fig:mca} for a higher-quality measurement using a multi-channel analyzer.}
	\label{spegam:fig:sca}
\end{figure}

\subsection{Multi-Channel Analyzer} \label{spegam:ss:mca}
To account for different samples (in particularly the background) being measured for different times, I first converted count data to activity by dividing counts by measurement time.
\begin{equation*}
	A = \frac{N}{t_{\text{live}}}
\end{equation*}
I then subtracted the background spectrum from the sodium, cobalt, and cesium data. I then repeated the calibration procedure from channel number to energy using the known values of the two sodium spectral lines.
\begin{equation*}
	E(n) = E_{1} + \left(\frac{E_{2}-E_{1}}{n_{2}-n_{1}}\right)(n - n_{1})
\end{equation*}
With this relationship, I found the energy $ E $ of the main spectral lines using the known channel number $ n $. Table \ref{spegam:table:mca-energies} shows the results.


\begin{table}[h]
\begin{center}
    \begin{tabular}{c|c|c|c}
         Isotope &  Channel &  Energy Estimate [\si{\mega \electronvolt}]& True Value [\si{\mega \electronvolt}]\\
        \hline {\rule{0pt}{2.6ex}} \hspace{-7pt}  % for space below hline
        \isoptope{Na}{22} &  $ 264 \pm 1 $ & - & 0.511\\
        \isoptope{Na}{22} &  $ 636 \pm 1 $ & - & 1.277\\
        \isoptope{Cs}{137} &  $ 339 \pm 1 $ & $ 0.665 \pm 0.021 $ & $ 0.667 $\\
        \isoptope{Co}{60} &  $ 589 \pm 1 $ & $ 1.180 \pm 0.006 $ & 1.173\\
        \isoptope{Co}{60} &  $ 667 \pm 1 $ & $ 1.341 \pm 0.006 $ & 1.332
	\end{tabular}
	\caption{Estimates of spectral line energies using the multi-channel analyzer.}
	\label{spegam:table:mca-energies}
\end{center}
\end{table}

\begin{figure}
	\centering
	\includegraphics[width=\linewidth]{mca}
	\caption{The spectrum of each source measured with the multi-channel analyzer.}
	\label{spegam:fig:mca}
\end{figure}

\subsection{Cesium Spectrum}
Figure \ref{spegam:fig:cesium} identifies the \isoptope{Ce}{137} backscatter peak, Compton edge, and photopeak.. I used a 5-point moving average to help estimate the corresponding energies as:
\[
\begin{array}{ll}
	\text{Backscatter peak: } & \SI{0.18\pm 0.01}{\mega \electronvolt}\\
	\text{Compton edge: } & \SI{0.47\pm 0.01}{\mega \electronvolt}\\
	\text{Photopeak: } & \SI{0.667\pm 0.003}{\mega \electronvolt}
\end{array}
\]
Note that the backscatter peak's energy is roughly the photopeak energy minus the energy of the Compton edge. The uncertainties on the backscatter and Compton energies are rough visual estimates---the photopeak uncertainty is separation between neighboring points.


\begin{figure}
	\centering
	\includegraphics[width=\linewidth]{cesium-mca}
	\caption{Prominent components of the \isoptope{Ce}{137} spectrum with energy calibrated as described in Subsection \ref{spegam:ss:mca}.}
	\label{spegam:fig:cesium}
\end{figure}


\subsection{Resolutions}
I fit a Gauss curve to each photopeak peak to find the peaks standard deviation $ \sigma $, found the FWHM with $ \text{FWHM} \approx 2.355 \sigma $,  then found the peak's resolution $ R $ with
\begin{equation*}
	R = \frac{\Delta E}{E}
\end{equation*}
where $ E $ is the peak's energy and $ \Delta E $ is the FWHM. Note that resolution decreases (improves) with increasing energy. This is expected---from the Poisson statistics of light production in the scintillator we have $ \Delta E \propto E $ and thus $ R \propto \frac{1}{\sqrt{E}}$. In other words, the observed decrease in $ R $ with respect to $ E $ agrees with theory.

\begin{table}[h]
\begin{center}
    \begin{tabular}{c|c|c|c}
         Isotope & Line Energy [\si{\mega \electronvolt}]& FWHM [\si{\mega \electronvolt}] & Resolution\\
        \hline {\rule{0pt}{2.6ex}} \hspace{-7pt}  % for space below hline
        \isoptope{Na}{22} & $ 0.511 \pm 0.001 $ & $ 0.047 \pm 0.000 $ & $ 0.092 \pm 0.001 $ \\
        \isoptope{Cs}{137} & $ 0.665 \pm 0.021 $ &  $ 0.052 \pm 0.000 $ & $ 0.078 \pm 0.003 $ \\
        \isoptope{Co}{60} & $ 1.180 \pm 0.006 $ & $ 0.083 \pm 0.001 $ & $ 0.071 \pm 0.001 $ \\
        \isoptope{Co}{60} &  $ 1.341 \pm 0.006 $ & $ 0.084 \pm 0.001 $ & $ 0.062 \pm 0.001 $
	\end{tabular}
	\caption{Resolution of spectral lines. The uncertainty in sodium is relatively low because of the low uncertainty in line energy---we assumed the known value \SI{0.511}{\mega \electronvolt} \textit{a priori}. To three decimal places, the uncertainty in FWHM, estimated from the covariance matrix returned by SciPy's \texttt{curve\_fit}, is essentially zero.}
	\label{spegam:table:resolution}
\end{center}
\end{table}

\begin{figure}
	\centering
	\includegraphics[width=\linewidth]{resolutions}
	\caption{Full-width at half maximum of each photopeak. The analysis is performed by fitting a Gauss curve to each photopeak using SciPy's \texttt{curve\_fit}.}
	\label{spegam:fig:resolutions}
\end{figure}

\subsection{Photon Absorption Efficiency}
I estimated the NaI crystal's photon absorption efficiency with the \isoptope{Ce}{137} source as
\begin{equation*}
	\eta = \frac{N_{\text{peak}}}{N_{\text{all}}} = \frac{A_{\text{peak}}}{A_{\text{all}}}
\end{equation*}
where $ N_{\text{peak}} $ is the number of pulses registered in the photopeak and $ N_{\text{all}} $ is total number of gamma rays in the solid angle $ 2\pi $, and the $ A $ values are the corresponding activities. I was given the source's half-life and January 2013 activity: 30.7 years and \SI{9250}{\becquerel}, respectively. I estimated the October 2020 activity using the radioactive decay law
\begin{equation*}
	A(t) = A_{0}e^{-t/\tau}  \implies  A_{\text{2020}} = A_{\text{2013}}\exp(-\ln 2 \cdot \frac{\SI{7.83}{years}}{\SI{30.7}{years}} ) = \SI{7750}{\becquerel}
\end{equation*}
$ A_{\text{2020}} $ represents isotropic activity over the entire solid angle $ 4\pi $ (not $ 2\pi $), so I divided $ A_{\text{2020}} $ by a factor of 2, and then multiplied $ A_{\text{2020}} $ by 0.946, since the rate of \isoptope{Cs}{137} \SI{0.667}{\mega \electronvolt} emission is 0.946 times the source activity (see Appendix \ref{spegam:ss:decay-theory}). From Figure \ref{spegam:fig:cesium}, we see activity at the cesium photopeak is about $ A_{Cs} = \SI{19.5}{\becquerel}$. The crystal's efficiency is thus
\begin{equation*}
	\eta = \frac{N_{\text{peak}}}{N_{\text{all}}} = 2\cdot \frac{A_{\text{Cs}}}{0.946 \cdot A_{\text{2020}}} = 2\cdot \frac{\SI{19.5}{\becquerel}}{0.946 \cdot \SI{7750}{\becquerel}} \approx 0.0053
\end{equation*}

\section{Error Analysis}
\subsection{Line Energies with Single-Channel Analyzer}
Input data is lower level $ n $ and the known \isoptope{Na}{22} energies $ E_{1} $ and $ E_{2} $. I estimated the lower-level position error as $ u_{n} = \SI{0.2}{\volt} $, the separation between points and took $ E_{1,2} $ to have zero uncertainty. I modified the interpolation formula as follows
\begin{equation*}
	E(n) = E_{1} + \left(\frac{E_{2}-E_{1}}{n_{2}-n_{1}}\right)(n - n_{1}) \equiv E(n) = E_{1} + \frac{E_{2}-E_{1}}{\Delta n_{1}} \Delta n_{2}
\end{equation*}
where I introduce the two lower-level differences $ \Delta n_{1} $ and $ \Delta n_{2} $ as input quantities, each with error $ u_{1,2} = 2 \cdot \SI{0.2}{\volt} = \SI{0.4}{\volt} $, since error adds under subtraction. The sensitivity coefficients are
\begin{equation*}
	c_{1} = \pdv{E}{\Delta n_{1}} = -\Delta n_{2} \frac{E_{2}-E_{1}}{\Delta n_{1}^{2}} \eqtext{and} c_{2} = \pdv{E}{\Delta n_{2}} = \frac{E_{2}-E_{1}}{\Delta n_{1}}
\end{equation*}
while the error on the energy estimate $ E $ is 
\begin{equation*}
	u_{E} = \sqrt{(u_{1}c_{1})^{2} + (u_{2}c_{2})^{2}}
\end{equation*}
Table \ref{spegam:table:sca-energies} shows the results. This analysis assumes, perhaps too liberally, a simple linear relationship between $ E $ and $ n $.

\subsection{Line Energies with Multi-Channel Analyzer}
Since I used the same linear interpolation procedure to find the line energies with the MCA as with the SCA, the corresponding error analysis for the MCA is completely analogous the above approach for the PCA. I simply replaced the role of the SCA lower level with MCA channel number, which I took to have an uncertainty of $ u_{c} \pm 1 $ at the peak positions. Table \ref{spegam:table:mca-energies} shows the results.

\subsection{Energy Resolution}
Input quantities are line energy $ E $ and peak FWHM $ \Delta E $. Table \ref{spegam:table:mca-energies} shows the error in line energies $ u_{E} $. I estimated error in FWHM $ u_{\Delta E} $ as the square root of the corresponding element in the covariance matrix returned by \texttt{curve\_fit} when fitting a bell curve to the peak. 
\begin{equation*}
	R = \frac{\Delta E}{E}
\end{equation*}
Sensitivity coefficients are
\begin{equation*}
	c_{\Delta E} = \pdv{R}{\Delta E} = \frac{1}{E} \eqtext{and} c_{E} = \pdv{R}{E} = -\frac{\Delta E}{E^{2}}
\end{equation*}
The associated error is
\begin{equation*}
	u_{R} = \sqrt{(u_{\Delta E}c_{\Delta E})^{2} + (u_{E}c_{E})^{2}}
\end{equation*}

\subsection{Photon Absorption Efficiency}
I did not calculate error on the estimated efficiency, since the efficiency value is intended more as an order of magnitude estimate than an exact result. Additionally, the two input quantities, half-life and initial activity, are given without any uncertainty. I would have to ``invent'' uncertainties in these two quantities if I were to calculate error in efficiency, which would then be somewhat arbitrary anyway.

\section{Results}
The estimates of the photopeak energies using the single channel analyzer, shown in Table \ref{spegam:table:sca-energies} and reprinted below, were poor. As discussed in the analysis, I believe the values for cobalt are additionally skewed by my having tinkered with the photomultiplier's amplification settings without re-calibrating the scintillation detector.
\begin{table}[h]
\begin{center}
    \begin{tabular}{c|c|c}
         Isotope &  Energy Estimate [\si{\mega \electronvolt}]& True Energy [\si{\mega \electronvolt}]\\
        \hline {\rule{0pt}{2.6ex}} \hspace{-7pt}  % for space below hline
        \isoptope{Cs}{137} & $ 0.80 \pm 0.38 $ & $ 0.667 $\\
        \isoptope{Co}{90} & $ 1.48 \pm 0.17 $ & 1.173\\
        \isoptope{Co}{90} & $ 1.68 \pm 0.16 $ & 1.332
	\end{tabular}
	\caption{Rough estimates of spectral line energies using the single-channel analyzer.}
\end{center}
\end{table}

The estimates of the photopeak energy using the multi-channel analyzer are much closer to the true values. The results, from Table \ref{spegam:table:sca-energies}, are re-printed below:
\begin{table}[h]
\begin{center}
    \begin{tabular}{c|c|c}
         Isotope &  Energy Estimate [\si{\mega \electronvolt}]& True Energy [\si{\mega \electronvolt}]\\
        \hline {\rule{0pt}{2.6ex}} \hspace{-7pt}  % for space below hline
        \isoptope{Cs}{137} &  $ 0.665 \pm 0.021 $ & $ 0.667 $\\
        \isoptope{Co}{60} & $ 1.180 \pm 0.006 $ & 1.173\\
        \isoptope{Co}{60} & $ 1.341 \pm 0.006 $ & 1.332
	\end{tabular}
	\caption{Estimates of spectral line energies using the multi-channel analyzer.}
\end{center}
\end{table}

The sources' photopeak resolutions ranged from about 5 to 10 percent. As theoretically expected, resolution improved with increasing photopeak energy. The results are below:
\begin{table}[h]
\begin{center}
    \begin{tabular}{c|c|c}
         Isotope & Line Energy [\si{\mega \electronvolt}]& Resolution\\
        \hline {\rule{0pt}{2.6ex}} \hspace{-7pt}  % for space below hline
        \isoptope{Na}{22} & $ 0.511 \pm 0.001 $ & $ 0.092 \pm 0.001 $ \\
        \isoptope{Cs}{137} & $ 0.665 \pm 0.021 $ &  $ 0.078 \pm 0.003 $ \\
        \isoptope{Co}{60} & $ 1.180 \pm 0.006 $ & $ 0.071 \pm 0.001 $ \\
        \isoptope{Co}{60} &  $ 1.341 \pm 0.006 $ & $ 0.062 \pm 0.001 $
	\end{tabular}
	\caption{Resolution of spectral lines. Uncertainty in \isoptope{Na}{22} is relatively low because the line energy of $ \SI{0.511}{\mega \electronvolt} $ was assumed with low uncertainty \textit{a priori}.}
\end{center}
\end{table}

The crystal's efficiency for photon absorption, estimated using the photon activity registered in the \isoptope{Cs}{137} photopeak, was roughly $ \eta \approx 0.005 $. 

\iffalse
\section{Theoretical Questions}
\begin{enumerate}
	\item\textit{Explain the position of the photon escape peak given that the bound electron energies in the iodine $ K, L_{\text{III}}, $ and $ L_{\text{II}} $ shells are \SI{33.2}{\kilo \electronvolt}, \SI{4.54}{\kilo \electronvolt} and \SI{4.85}{\kilo \electronvolt} respectively.}
	
	\item \textit{Qualitatively explain how the spectrum would change for a source with \SI{2}{\mega \electronvolt} gamma rays in the center of a very large NaI crystal.} 
	
	
	From very bottom of instructions section 5.2.4: The ratio between the height of the Compton plateau and the photopeak depends somewhat on gamma ray energy and particularly on the crystal size. The larger the crystal, the more events are registered in the photopeak. So a large crystal with high energy would mean a relatively large photopeak.
	
	\item \textit{To get a positive signal from the photomultiplier, you would take the signal from the last dynode instead of from the anode resistor. Explain why, and discuss how the signal's amplitude would change.}
	
	\item \textit{Would the experiment work if you grounded the photomultiplier anode instead of the cathode? What would be the advantages and disadvantages of grounding the cathode. Keep in mind that photomultiplier voltages reach \SI{2500}{V}.}
\end{enumerate}
\fi


\appendix

\section{Theory}
The experiment's goal is to measure the energy of gamma rays. This done indirectly by measuring the energy of photons created by the gamma rays e.g. in our case with a scintillation detector.

\subsection{Sodium, Cesium and Cobalt Decay} \label{spegam:ss:decay-theory}
\begin{itemize}
	\item Sodium scheme: \isoptope{Na}{22} has a 90.2 percent probability for $ \beta^{+} $ decay and a 9.7 percent probability for electron capture into an excited state of \isoptope{Ne}{22}. Nearly 100 percent of the time, the neon nucleus decays to the ground state by emitting a \SI{1.27}{\mega \electronvolt} photon. Meanwhile, the positron from beta decay annihilates with a surrounding electron and produces two \SI{0.511}{\mega \electronvolt} gamma rays.
	
	The rates of \SI{1.27}{\mega \electronvolt} and \SI{0.511}{\mega \electronvolt} gamma ray emission are equal to the source activity and ($ 2 \cdot 0.9 = 1.8 $) times the source activity, respectively.
	
	
%	\href{https://www.ld-didactic.de/software/524221en/Content/Appendix/Na22.htm}{Link to Sodium spectrum}. \href{http://ns.ph.liv.ac.uk/~ajb/radiometrics/glossary/sodium22.html}{Link to sodium decay scheme}.
	
	\item Cesium: \isoptope{Cs}{137} has a 5.4 percent chance to decay directly into the \isoptope{Ba}{137} ground state via $ \beta^{-} $ decay and a 94.6 percent probability for $ \beta^{-} $ decay into an excited \isoptope{Ba}{137} nucleus, which then decays to the \isoptope{Ba}{137} ground state by emission of a \SI{0.667}{\mega \electronvolt} photon. The rate of \SI{0.667}{\mega \electronvolt} gamma emission is thus 0.946 times the source activity.
	
%	\href{https://www.ld-didactic.de/software/524221en/Content/Appendix/Cs137.htm}{Link to cesium decay scheme}. 
	
	\item Cobalt: \isoptope{Co}{60} decays by $ \beta^{-} $ emission to an excited \isoptope{Ni}{60} state \SI{2.507}{\mega \electronvolt} above the ground state. This excited state decays by emission of a \SI{1.175}{\mega \electronvolt} gamma ray, followed within a picosecond or so by a \SI{1.332}{\mega \electronvolt} gamma ray. The rate of emission for both gamma rays approximately equals the  source activity.
\end{itemize}


\subsection{Gamma Ray Interaction with Crystal}
\begin{itemize}
	\item Photoelectric effect: incident gamma ray gives up its energy to eject a bound inner shell electron (binding energy of order \SI{10}{\kilo \electronvolt}) from one of the scintillator crystal's atoms. All gamma ray energy is deposited into scintillator. Corresponds to the \textit{photopeak}.
	
	\item Compton effect: when gamma rays first strike the lead shield and then Compton scatter back into the scintillator at reduced energy. Corresponds to the low-energy \textit{back-scatter peak}.
	
	\textit{Compton plateau}: corresponds to gamma rays Compton scattering off atoms in the scintillator crystal. The recoiling electron’s energy is deposited in the crystal, while the scattered gamma photon exits the crystal undetected.
	
	The electron's recoil energy depends on the angle of the scattered photon. Recoil energy varies from a maximum when the photon back-scatters (corresponding to the \textit{Compton edge}), to zero when the photon is scattered in the forward direction.
	
	\item Pair production: In the strong electric fields near scintillator crystal nuclei, a gamma ray can create an electron-positron pair as long as the gamma ray energy exceed the \SI{1.022}{\mega \electronvolt} rest mass energy of an electron-positron pair. Gamma energy in excess of \SI{1.022}{\mega \electronvolt} contributes to the kinetic energy of the electron and positron; the extra kinetic energy is quickly absorbed in the crystal.
	
	When the positron gets to low enough energy, it annihilates with an electron in the crystal, which produces two \SI{0.511}{\mega \electronvolt} annihilation gamma rays.
	
	If both annihilation gamma rays are absorbed in the crystal, the total absorbed energy is the original gamma energy, and the event contributes to the \textit{photopeak}.
	
	If one or both of the annihilation gamma rays escape from the crystal, the event registers in one of two small peaks (either the \textit{single} or \textit{double escape peak}) that are either \SI{0.511}{\mega \electronvolt} or \SI{1.022}{\mega \electronvolt} below the photopeak.
\end{itemize}

\textbf{Quick Review of Photoelectric Effect and Company}
\begin{itemize}
	\item Photoelectric effect: Photons incident on material eject electrons.
	\item Compton scattering: scattering of photons by a charged particle with an accompanying wavelength shift for the scattered photon.
	\item Pair production: production of a particle and antiparticle from a neutral boson, in our case electron-positron production from a photon.
\end{itemize}


\begin{figure}
\centering
\includegraphics[width=\linewidth]{photomultiplier}
\caption{Schematic of a scintillation detector (from Wikipedia)}
\end{figure}


\subsection{How the Detector Works}
\begin{itemize}
	\item Gamma ray from the radioactive source pass through the NaI crystal. The gamma rays interact with the crystal and create scintillation photons (in the visible and ultraviolet spectrum). 
	
	\item The scintillation photons strike the photomultiplier's cathode and eject electrons via the photoelectric effect. The electrons emitted from the cathode are accelerated by a high voltage through a dynode chain before reaching a collector anode. The dynodes serve as electron multipliers, and the entire dynode chain typically amplifies the electron current by a factor of order $ 10^{6} $. 
	
	\item The anode connects to an series of amplifiers, which convert the collected charge to a proportional voltage pulse. Because the number of scintillation photons produced in the NaI crystal is proportional to the absorbed gamma ray energy, so to are the number of photoelectrons from the cathode, the final anode charge, and the amplitude of the preamp and amplifier voltage pulses. The overall effect is that the final pulse height is proportional to the gamma ray energy absorbed in crystal.
\end{itemize}

\end{document}