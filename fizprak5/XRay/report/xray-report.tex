\documentclass[11pt, a4paper]{article}
\usepackage[T1]{fontenc}
\usepackage{mwe}
\usepackage[margin=3.5cm]{geometry}
\usepackage{amsmath}
\usepackage{amssymb}
\usepackage{bm} % for bold vectors in math mode
\usepackage{physics} % many useful physics commands
\usepackage[separate-uncertainty=true]{siunitx} % for scientific notation and units

\usepackage{graphicx}
\graphicspath{{"../figures/"}}
\usepackage[section]{placeins} % to keep figures in their sections
\usepackage{subcaption} % for subfigure captions
\usepackage[export]{adjustbox} % centers figures wider than text width

\usepackage{xcolor}  % to color hyperref links
\usepackage[colorlinks = true, allcolors=blue]{hyperref}

\setlength{\parindent}{0pt} % to stop indenting new paragraphs
\newcommand{\eqtext}[1]{\qquad \text{#1} \qquad}

\newcommand{\diff}{\mathop{}\!\mathrm{d}} % differential


\begin{document}
\title{X-Rays}
\author{Elijan Mastnak}
\date{Winter Semester 2020-2021}
\maketitle

\tableofcontents
\newpage


\section{Tasks}
\begin{enumerate}
	\item Use the ionization cell to measure the average dose strength in an x-ray beam.
	
	\item Measure the polarization of primary x-ray beams
	
	\item Measure the polarization of scattered x-ray beams
\end{enumerate}


\section{Equipment and Procedure}
\subsection{Equipment}
\begin{itemize}
	\item Lehr und Didaktiksysteme 554811 x-ray apparatus, associated software, and computer.

	\item Ionization cell, with a power supply and voltage-regulation dials.
	
	\item Multimeters to measure voltage applied to ionization cell and ionization current.
	
	\item Polarization measurement module with scatterers Geiger-Muller counter

\end{itemize}

\subsection{Sketch of Procedure}
\begin{enumerate}
	
	\item Set up the ionization cell in the main compartment of the X-ray apparatus; set x-ray anode voltage and current. Measure the the voltage drop across the output resistor as a function of voltage applied to the ionization cell for three difference anode voltages
	
	\item Remove the ionization cell and set up the polarization module, initially with one scatterer; turn on the computer and open the x-ray apparatus software. Use the software to measure the number of radiation counts for two perpendicular orientations of the scatterer-Geiger apparatus.
	
	\item Modify the polarization module to use two scatterers. Repeat the x-ray count measurement for two perpendicular orientations of the scatterer-Geiger apparatus.
	
\end{enumerate}

\subsection{Data}
\textbf{Ionization Measurements:} Independent variable is the voltage $ U_{\text{cell}} $ applied to the ionization cell and the dependent variable is the voltage drop $ U_{R} $ across the output resistor, which is converted to ionization cell current $ I_{\text{cell}} $ with the known output resistor value $ R_{\text{out}} $. The $ U_{R}(U_{\text{cell}}) $ characteristic is measured for three values of x-ray anode voltage.

\vspace{2mm}
\textbf{Polarization Measurement:} The independent variable is the spatial orientation of the scatterer relative to the incident x-ray beam (and the measurement time). The dependent variable is the number of GM radiation counts in the measurement period. 


\section{Analysis}

\subsection{Ionization and Dose Rate}
I first converted voltage drop $ U_{R} $ across the output resistor to current through the ionization cell with Ohm's law
\begin{equation*}
	I_{\text{cell}} = \frac{U_{R}}{R_{\text{out}}}
\end{equation*}
where $ R_{\text{out}} = \SI{1}{\giga \ohm} $ is the output resistance. I then plotted the $ I_{\text{cell}}(U_{\text{cell}}) $ ionization curve for each value of the anode voltage, shown in Figure \ref{xray:fig:ionization}. 
%\textit{Note:} the ionization curves in Figure \ref{xray:fig:ionization} give the impression that ionization current begins to plateau with large $ U_{\text{cell}} $. This is only because I did not measure large enough values of $ U_{\text{cell}} $. In fact, the ionization current should begin decrease slightly for large $ U_{\text{cell}} $.
\begin{figure}
	\includegraphics[width=\linewidth]{ionization}
	\caption{Ionization curve for three values of x-ray anode voltage.}
	\label{xray:fig:ionization}
\end{figure}
Using the ionization current, I then estimated the exposition dose rate in the ionization cell as a function of applied cell voltage $ U_{\text{cell}} $ using
\begin{equation*}
	\dv{X}{t} = \frac{I_{\text{cell}}}{\rho V_{\text{cell}}}
\end{equation*}
where $\rho$ and $ V_{\text{cell}} $ are the density and volume of the air in the ionization cell, respectively. The cell has a uniform trapezoidal base, so the cell volume is
\begin{equation*}
	V_{\text{cell}} = A_{\text{base}} \cdot h \approx \SI{144}{\centi \meter^{2}} \cdot \SI{3}{\centi \meter} = \SI{432}{\centi \meter^{3}}
\end{equation*}
where $ A_{\text{base}} $ is the area of the trapezoidal base and $ h $ is the cell height. I took the density of air to be $ \SI{1.23}{\kilogram \, \meter^{-3}} $. Figure \ref{xray:fig:dose-rate} shows the results. The dose rate in the cell is of order $ \SI{3}{\micro \ampere \cdot \kilogram^{-1}} $ and increases with increasing cell voltage before plateauing at a saturated value. Logically, the dose rate increases with increasing anode voltage.

\begin{figure}
	\includegraphics[width=\linewidth]{dose-rate}
	\caption{Dose rate in the ionization cell for three values of x-ray anode voltage.}
	\label{xray:fig:dose-rate}
\end{figure}

\subsection{Polarization}
This section uses a Cartesian coordinate system whose axes align with the axes of the cuboid x-ray apparatus. The x-rays propagate along the $ y $ axis, which points toward the protective circular cover on the wall nearest the camera. The $ z $ axis points parallel to the apparatus's ceiling (i.e. ``up'') and the $ x $ axis points toward the sliding glass door. Table \ref{xray:table:polarization-data} shows the polarization measurement data.

%If the electrons decelerated only along the $ y $ axis, the emitted x-rays would be polarized in the $ y $ direction and propagate only the in the $ xz $ plane.

\vspace{2mm}
\textbf{One Scatterer:} The x-rays are incident on the scatterer in the $ y $ direction; I measured the scattered radiation over a ten-second period using a cylindrical Geiger-Muller counter whose cross section was pointed in the $ x $ and $ z $ directions. First, I divided counts by measurement time to get two rates $ R_{x} $ and $ R_{z} $. I then estimated the incident x-ray's polarization in the incident $ y $ direction using
\begin{equation*}
	\eta_{y} = \frac{R_{z} - R_{x}}{R_{x} + R_{z}} = \frac{\SI{296.6}{s^{-1}} -  \SI{267.3 }{s^{-1}}}{\SI{296.6}{s^{-1}} + \SI{267.3 }{s^{-1}}} \approx 0.055
\end{equation*}
There is almost no polarization. This is, I believe, expected---the x-rays would be polarized in the $ z $ direction, the direction of electron incidence on the anode.
%Using the given, online data:
%\begin{equation*}
%	\eta = \frac{\SI{227.9}{s^{-1}} - \SI{194.5}{s^{-1}}}{\SI{227.9}{s^{-1}} + \SI{194.5}{s^{-1}}} \approx 0.079
%\end{equation*}


\begin{table}[h]
\begin{center}
    \begin{tabular}{c|ccccccc}
        & $ N_{1} $ & $ N_{2} $ & $ N_{3} $ & $ N_{4} $ & $ N_{5} $ & $ \bar{N} $ & $ \sigma_{N} $ \\
        \hline
        Single scatter-$ x $ & 2702 & 2683 & 2675 & 2659 & 2646 & 2673 & 22 \\
        Single scatter-$ z $ & 2955 & 2979 & 2965 & 2951 & 2982 & 2966 & 14 \\
        Two scatters-$ x $ & 1.8 & 1.1 & 2.1 & 1.4 & 1.5 & 1.58 & 0.38\\
        Two scatters-$ y $ & 7.2 & 7.0 & 6.3 & 6.2 & 7.3 & 6.80 & 0.51
	\end{tabular}
	\caption{5 sample runs, average, and standard deviation of X-ray counts in a ten-second period using a Geiger-Muller counter. }
	\label{xray:table:polarization-data}
\end{center}
\end{table}


\textbf{Two Scatterers:}
In this case, the secondary x-rays scattered from the first scatterer propagate in the $ z $ direction, and we use the Geiger-Muller counter with its cross section pointed in the $ x $ and $ y $ directions. 
\begin{equation*}
	\eta_{z} = \frac{R_{y} - R_{x}}{R_{x} + R_{y}} = \frac{\SI{0.680}{s^{-1}} - \SI{0.158}{s^{-1}} }{\SI{0.680}{s^{-1}} + \SI{0.158}{s^{-1}}} \approx 0.623
\end{equation*}
In this case, the scattered radiation is partially polarized in the plane $ xy $ plane, perpendicular to its direction of incidence. Theory predicts a perfect $ \eta_{z} = 1 $, but because of non-ideal measurement conditions, we get roughly $ \eta_{z} = 0.5 $.



\section{Error Analysis}
\subsection{Ionization Dose Rate}
\textbf{Current:} Input data is voltage $ U_{R} $ across output resistor, converted to cell current $ I_{\text{cell}} $ using
\begin{equation*}
	I_{\text{cell}} = \frac{U_{R}}{R_{\text{out}}}
\end{equation*}
I estimated the error $ u_{\text{R}} $ as one half of difference between the maximum and minimum value plus  \SI{0.1}{\volt}, which is the smallest significant digit reported by the multimeter. With no better options available, I estimated the error on $ R_{\text{out}} $ as 15 percent error, which is roughly typical for an average resistor. Sensitivity coefficients are
\begin{equation*}
	c_{U} = \pdv{I_{\text{cell}}}{U_{R}} = \frac{1}{R_{\text{out}}} \eqtext{and} c_{R} = \pdv{I_{\text{cell}}}{R_{\text{out}}} = -\frac{U_{R}}{R_{\text{out}}^{2}}
\end{equation*}
Error is 
\begin{equation*}
	u_{I} = \sqrt{(u_{U}c_{U})^{2} + (u_{R}c_{R})^{2}}
\end{equation*}


\textbf{Dose Rate:} Next, when calculating dose rate via
\begin{equation*}
	\dv{X}{t} = \frac{I_{\text{cell}}}{\rho V}
\end{equation*}
Input quantities with error are $ I_{\text{cell}} $, found above, $ V $, which I'll assume to have about $ 25 $ percent error since I estimated it visually, and $ \rho $, with e.g. 5 percent error. So $ \rho = \SI{1.23 \pm 0.06}{\kilogram \, \meter^{-3}} $, and volume $ V = \SI{430 \pm 100}{\centi \meter^{3}} $. Sensitivity coefficients are
\begin{equation*}
	c_{I} = \frac{1}{\rho V} \qquad c_{\rho} = -\frac{I_{\text{cell}}}{\rho^{2} V} \qquad c_{V} = -\frac{I_{\text{cell}}}{\rho V^{2}}
\end{equation*}
Error is 
\begin{equation*}
	u_{X} = \sqrt{(u_{I}c_{I})^{2} + (u_{\rho}c_{\rho})^{2} + (u_{V}c_{V})^{2}}
\end{equation*}

\subsection{Polarization}
Working in counts per ten seconds instead of rate, I introduced the quantities $ N_{+} $ and $ N_{-} $ and rearranged the polarization formula to read
\begin{equation*}
	\eta = \frac{N_{2} - N_{1}}{N_{2} + N_{1}} \equiv \frac{N_{-}}{N_{+}}
\end{equation*}
The input data are $ N_{+} $ and $ N_{-} $, and the associated errors $ u_{-} $ and $ u_{+} $ are the sum of the error on $ N_{1} $ and $ N_{2} $, which I estimated as $ u_{N} = \frac{\sigma_{N}}{\sqrt{5}} $ (see Table \ref{xray:table:polarization-data}). 



Sensitivity coefficients are
\begin{equation*}
	c_{+} = \pdv{\eta}{N_{+}} = - \frac{N_{-}}{N_{+}^{2}} \eqtext{and} c_{-} = \pdv{\eta}{N_{-}} = \frac{1}{N_{+}}
\end{equation*}
and error is
\begin{equation*}
	u_{\eta} = \sqrt{(u_{+}c_{+})^{2} + (u_{-}c_{-})^{2}}
\end{equation*}

\textbf{Single Scatterer}
\begin{align*}
	& N_{+} = \bar{N}_{x} + \bar{N}_{z} = 2966 + 2673 = 5639  \\ 
	& N_{-} = \bar{N}_{z} - \bar{N}_{x} = 2966 - 2673 = 293  \\ 
	& u_{+} = u_{-} = \frac{\sigma_{x} + \sigma_{z}}{\sqrt{5}} = \frac{22 + 14}{\sqrt{5}} \approx 16
\end{align*}
The result---implemented in Python---is $ u_{\eta} = 0.003 $. Error is low because of small standard deviation.

\vspace{2mm}
\textbf{Two Scatterers}
\begin{align*}
	& N_{+} = \bar{N}_{x} + \bar{N}_{z} = 6.80 + 1.58 = 8.38  \\ 
	& N_{-} = \bar{N}_{z} - \bar{N}_{x} = 6.80 - 1.58 = 5.22  \\ 
	&u_{+} = u_{-} = \frac{\sigma_{x} + \sigma_{z}}{\sqrt{5}} = \frac{0.38 + 0.51}{\sqrt{5}} \approx 0.40
\end{align*}
The result---implemented in Python---is $ u_{\eta} = 0.056 $.






\section{Results}

\appendix

\section{Theory}

\textbf{X-Ray Emission}
\begin{itemize}
	\item Electrons from a cathode are accelerated by a large potential difference into a target anode, where x-rays are emitted in the \textit{brehmstrallung} interaction between the electrons and metal nuclei. This radiation corresponds to the continuous part of the x-ray spectrum. 
	
	\item An accelerated electron passing a nucleus in the target anode loses speed as it emits electromagnetic radiation. The radiation's frequency $ \nu $ is related to the kinetic energy $ \Delta T $ lost by the electron via
	\begin{equation*}
		\Delta T = h \nu
	\end{equation*}
	where $ h $ is Planck's constant. The max frequency occurs when the electron loses its entire kinetic energy $ T $:
	\begin{equation*}
		h \nu_{\text{max}} = T = e_{0}V \implies \lambda_{\text{min}} = \frac{e_{0}V}{hc}
	\end{equation*}
	
	\item If the accelerated electrons have large enough energies, they can also free atomic electrons from the nuclei's inner electron shells. Atomic electrons from the outer shells filling the holes in the inner shells then emit \textit{characteristic x-ray} of discrete energy characteristic of the element and initial/final electron shells. Characteristic x-rays correspond to the discrete spikes in the x-ray spectrum. 
\end{itemize}

\textbf{Ionization Cell}
\begin{itemize}
	\item A basic ionization cell is a parallel-plate capacitor wired to a large potential difference. An x-ray incident on the space between the capacitor plates frees photoelectrons via the photoelectric effect, which ionizes the matter between the plates. The resulting ionized matter---pairs of molecular anions and free electrons---accelerates across the capacitor's potential difference, giving rise to an electric current pulse in the presence of an x-ray pulse. 
	
	For x-ray photon currents of order greater than $ \sim \SI{e9}{\second^{-1}} $, the pulses average to a macroscopically measurable electric current.
	
	\item  Of course, all ionized pairs don't reach the capacitor electrodes; some will recombine in the space between the plates. Recombination dominates at low electric field strengths and nearly disappears for large electric fields. This trend appears when measuring current as a function of capacitor voltage at a constant x-ray intensity: current initially grows with increasing capacitor voltage before reaching a saturated value. For vary large voltages current begins to increases again, but this effect is beyond the scope of this experiment.
	
	\item X-ray beams are usually characterized by the exposition dose $ X $, defined as the electric charge $ \Delta Q $ of a given sign freed by ionizing radiation in a mass $ \Delta m $. 
	\begin{equation*}
		X = \frac{\Delta Q}{\Delta m}
	\end{equation*}
	The SI units of exposition dose are $ \si{\coulomb \, \kilogram^{-1}} $.
	
	\item The exposition dose rate is defined as
	\begin{equation*}
		\dv{X}{t} = \frac{\Delta Q}{\Delta t \Delta m} = \frac{I}{\Delta m} = \frac{I}{\rho \Delta V}
	\end{equation*}
	where $ I $ is the current of charged particles freed by the ionizing radiation and $ \rho = \frac{m}{V} $ is the density of the matter exposed to the radiation.
	 
\end{itemize}

\textbf{X-Ray Polarization}
\begin{itemize}
	\item We take a simplified classical approach and assume the accelerated electrons behave as harmonically oscillating classical charge. If the charge oscillates in e.g. the $ y $ direction, its position and acceleration are
	\begin{equation*}
		y = A \sin \omega t \eqtext{and} a_{y} = -A\omega^{2} \sin \omega t
	\end{equation*}
	The accelerating charge emits electromagnetic radiation, which we describe in terms of the electric field strength $ \bm{E} $ and magnetic flux density $ \bm{B} $; $ \bm{E} $ is parallel to the charge's velocity and is perpendicular to the direction of wave propagation, while $ \bm{B} $ is perpendicular to both $ \bm{E} $ and the direction of wave propagation.
	
	\item If the charge oscillates along the $ y $ axis, $ \bm{E} $ will always point along the $ y $ axis, and we say the radiation is \textit{linearly polarized} in the $ y $ direction. The energy current of linearly polarized radiation is anisotropic; it is largest in the equatorial plane of wave propagation and is zero along the axis of charge oscillation. 
	
	\item Next, suppose we have multiple charges whose oscillation directions are uniformly distributed in the $ y-z $  plane. In this case the electromagnetic radiation is linearly polarized in the $ y $ and $ z $ directions and is unpolarized in the $ x $ direction. 
	
	In a similar scenario in which oscillation is \textit{not} uniformly distributed in the $ y-z $ plane, the radiation is still linearly polarized in the $ y $ and $ z $ directions and partially polarized in the $ x $ direction. 
	
	\item If the electrons in the target anode decelerated only in a direction parallel to their initial velocity (e.g. the $ y $ direction), we would observe linearly polarized x-rays which would propagate in the $ x-z $ plane. Because the incident electrons tend to deviate from their initial direction as they pass through the anode before the brehmstrallung process begins, we see only partial polarization in the $ x-z $ plane.
	
\end{itemize}

\begin{figure}
\centering
\includegraphics[width=\linewidth]{polarization}
\caption{Understanding x-ray polarization with an oscillating electric charge. The left figure is linearly polarized in the $ y $ direction, while the right figure is only partially polarized.}
\end{figure}


\textbf{Coherent X-Ray Scattering}
\begin{itemize}
	\item When x-rays interact with \textit{bound} electrons, the photon energy remains unchanged\footnote{When x-rays interact with free electrons, the photon energy decreases}. We call this process elastic, or coherent scattering. Elastic scattering is used to measure x-ray polarization. 
	
	\item In the classical model of elastic scattering, an electromagnetic wave of frequency $ \nu $ incident on an electron excites the electron, causing it to oscillate with the same frequency $ \nu $. The oscillating electron then returns the absorbed energy by subsequently emitting electromagnetic radiation of frequency $ \nu $. This secondary radiation propagates through space as is characteristic for an oscillating classical charge. 
	
	\item The excited electron oscillates in the same direction as the electric field vector $ \bm{E} $ of the incident radiation, i.e. in a plane perpendicular to the direction of the incident wave's propagation. 
	
	If the incident wave is polarized, the charge oscillates equally in all directions in the plane perpendicular to the propagation direction; if the the incident radiation is partially polarized, some oscillation directions are more pronounced than others; if the incident radiation is linearly polarized, the charge oscillates in a line.
	
	\item Elastically scattered electromagnetic radiation emitted from an oscillating charge that propagates in a plane perpendicular to the incident radiation is linearly polarized. 
	
	The energy current density radiated by an oscillating charge varies as $ \sim \sin^{2} \theta $, where $ \theta $ is the angle between the direction of charge oscillation and the direction of the radiated wave. We thus see maximum radiation in the plane perpendicular to the charge's oscillation (where $ \theta = \frac{\pi}{2} $) and no radiation along the direction of oscillation (where $ \theta = 0 $). 
	
	\item The angular distribution of elastically scattered radiation depends on the polarization of the incident radiation. 
	
	Suppose the incident radiation propagates in the e.g. $ y $ direction. If the incident radiation is polarized in the $ z $ direction, there is no scattered radiation in the $ z $ direction; if the incident radiation is partially polarized in the $ z $ direction, the strength of the scattered radiation is smaller in the $ z $ direction than in the $ x $ direction; only if the incident radiation is unpolarized is the scattered radiation equally strong in the $ x $ and $ z $ directions. 
	
	\item By measuring the strength of elastically scattered radiation we can thus determine the polarization of the incident radiation. For an incident beam propagating in the $ y $ direction, we place a scatterer in the beam and then measure the angular distribution of scattered radiation in the $ x-y $ plane with a counter for ionizing radiation (e.g. a Geiger-Muller counter). The distribution is circular for unpolarized radiation and elliptical for partially polarized radiation. In practice, we rarely measure the entire angular distribution but only the intensity values $ I_{x} $ and $ I_{z} $. We then define polarization $ \eta $ as
	\begin{equation*}
		\eta = \frac{I_{z} - I_{x}}{I_{x} + I_{z}}
	\end{equation*}
\end{itemize}




\end{document}