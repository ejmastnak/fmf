\documentclass[11pt, a4paper]{article}
\usepackage{mwe}
\usepackage{amsmath}
\usepackage{mathtools}
\usepackage{graphicx}
\usepackage{bm} % for bold vectors in math mode
\usepackage{physics} % for differential notation, etc...
\usepackage[separate-uncertainty=true]{siunitx}

\usepackage{graphicx}
\graphicspath{{"../figures/"}}
\usepackage[section]{placeins} % to keep figures in their sections
\usepackage{subcaption} % for subfigures
\usepackage[export]{adjustbox} % for centered figures larger than text width

\usepackage[margin=3.5cm]{geometry}
\usepackage{xcolor}  % to color hyperref links
\usepackage[colorlinks = true, allcolors=blue]{hyperref}

\setlength{\parindent}{0pt} % to stop indenting new paragraphs
\newcommand{\diff}{\mathop{}\!\mathrm{d}} % differential
\newcommand{\eqtext}[1]{\qquad \text{#1} \qquad}


\begin{document}
\title{Microwaves}
\author{Elijan Mastnak}
\date{Winter Semester 2020-2021}
\maketitle
\tableofcontents
		
\section{Tasks}
\begin{enumerate}
	\item Measure the microwave frequency using the built-in resonator
	
	\item Measure the standing microwave signal in a waveguide with both short-circuit and loaded termination, and use the measurement to find the load's impedance.
	
	\item Measure the dependence of the klystron modes on the reflection voltage and measure the power dissipated by a thermistor at each klystron mode.

\end{enumerate}


\begin{figure}
\centering
\includegraphics[width=0.8\linewidth]{klystron}
\caption{Cross-sectional view of a reflex klystron.}
\end{figure}

\section{Equipment, Procedure and Data}

\subsection{Equipment}
\begin{itemize}
	\item Reflex klystron microwave system including a microwave source, power supply with variable reflection voltage, adjustable attenuation, a resonator cavity with adjustable chamber height, waveguide and microwave probe, and short-circuit termination.
	
	\item Voltmeter to measure klystron reflection voltage.
	
	\item Oscilloscope to observe the microwave probe signal. 
	
	\item Thermistor and power meter, with appropriate waveguide coupling, used to measure microwave power.
\end{itemize}

\subsection{Procedure}
\begin{itemize}
	\item Set the klystron reflection voltage: Observe the dependence of the microwave probe signal on the oscilloscope as a function of reflection voltage. Set reflection voltage so that probe signal is most negative (roughly $ V_{\text{ref}} = \SI{144}{\volt} $). This minimum probe signal corresponds to a klystron mode.
	
	\item With the reflection voltage set, adjust the sliding waveguide tuner until the probe signal again reaches a minimum value.
	
	\item Find microwave frequency by adjusting resonator cavity depth so that the oscilloscope probe signal falls in absolute value. Note the resonator screw position, and convert the screw position to frequency with a calibration table.
	
	\item Measure the microwave apparatus' tuning curve with short-circuit coupling: move the probe mount along the sliding guide at uniform speed and observe the corresponding probe signal on the oscilloscope. Repeat the measurement with a bolometer termination.
	
	\item Measure power of various klystron modes using the bolometer.
	
\end{itemize}

\subsection{Data}
\textbf{Independent Variable}: Klystron reflection voltage, position of microwave probe along the waveguide, and position of the resonant chamber screw (when measuring microwave frequency with the resonance chamber).

\vspace{2mm}
\textbf{Dependent Variable:} Voltage signal from the microwave probe, observed on an oscilloscope.

\vspace{2mm}
\textbf{Parameters:} Waveguide width $ a = \SI{2.3}{\centi \meter} $ and sliding probe position from \SIrange{0.0}{5.7}{\centi \meter}.

\section{Analysis}

\begin{figure}[htb!]
	\includegraphics[width=\linewidth]{frequency}
	\caption{Estimating the microwave frequency using the resonator's calibration curve and the position $ h $ of the resonator cavity screw at resonance.}
	\label{mikval:fig:frequency}
\end{figure}


\subsection{Finding Frequency with Resonator Cavity}
When the resonator cavity depth is set to allow resonance, some 40 percent of the microwaves' power is used in the resonator, and the probe signal correspondingly drops. The manufacturer provides calibration data do convert from screw position to microwave frequency. The relationship is linear and shown in Figure \ref{mikval:fig:frequency}; the equation of the line is
\begin{equation*}
	f(h) = \big(- 0.005 \cdot h + 10.5 \big) \si{\giga \hertz}
\end{equation*}
In my case, the cavity reached resonance at a screw position of $ h = 408 $ (without units). The corresponding frequency is
\begin{equation*}
	f(408) = \big(- 0.005 \cdot 408 + 10.5 \big) \si{\giga \hertz} = \SI{8.46}{\giga \hertz}
\end{equation*}



\subsection{Short-Circuit and Load-Terminated Microwave Signals}
Figure \ref{mikval:fig:curves} shows the microwave signal along the waveguide for both short-circuit and load termination, where the load is a thermistor connected to a power meter.

\begin{figure}
	\includegraphics[width=\linewidth]{curves}
	\caption{Microwave signals in a waveguide terminated by an external load and short-circuit.}
	\label{mikval:fig:curves}
\end{figure}

\subsubsection{Microwave Frequency from Standing Wave Signals}
The distance $ \Delta x_{\text{sc}} $ between two minima on the short-circuit curve is half the wavelength $ \lambda' $ of the microwaves in the waveguide, from which I found $ \lambda' $:
\begin{equation*}
	\lambda' = 2 \Delta x_{\text{sc}} = 2 \cdot \SI{2.443}{\centi \meter} \approx \SI{4.89}{\centi \meter}
\end{equation*}
The microwave wavelength $ \lambda $ in vacuum and the wavelength $ \lambda' $ in the waveguide are connected by the following relationship, which I used to find $ \lambda $:
\begin{equation*}
	\lambda' = \frac{\lambda}{\sqrt{1 - \frac{\lambda^{2}}{4a^{2}}}} \implies 	\lambda = \frac{\lambda'}{\sqrt{1 + \frac{\lambda'^{2}}{4a^{2}}}} = \frac{\SI{4.89}{\centi \meter}}{\sqrt{1 + \frac{(\SI{4.89}{\centi \meter})^{2}}{4(\SI{2.30}{\centi \meter})^{2}}}} \approx \SI{3.35}{\centi \meter}
\end{equation*}
The microwave frequency $ f $ is thus
\begin{equation*}
	f = \frac{c}{\lambda} = \frac{\SI{3.00e8}{\meter \, \second^{-1}}}{\SI{3.35e-2}{\meter}} \approx \SI{8.96}{\giga \hertz}
\end{equation*}


\subsubsection{Standing Wave Ratio}
The standing wave ratio $ s $ for the load-terminated curve is
\begin{equation*}
	s = \sqrt{\frac{h_{\text{min}}}{h_{\text{max}}}} = \sqrt{\frac{\SI{1.42}{\volt}}{\SI{3.86}{\volt}}} = 0.61
\end{equation*}
where $ h_{\text{min}} $ and $ h_{\text{max}} $ are the minimum and maximum of the load-terminated curve shown in Figure \ref{mikval:fig:curves}. The square root corrects for the quadratic characteristic of the diode in the microwave probe.



\subsubsection{Impedance of Terminating Load}
To find the terminating load's resistance $ \xi $ and reactance $ \eta $, we first find the product $ \beta x_{\text{min}} $. We find the waveguide quantity $ x'_{\text{min}} $ from the difference between minima of the two curves in Figure \ref{mikval:fig:curves}, from which calculate $  \beta x_{\text{min}}  $ using
\begin{equation*}
	\frac{x_{\text{min}}'}{\lambda'} = \frac{x_{\text{min}}}{\lambda} = \frac{\beta x_{\text{min}}}{2\pi} \implies \beta x_{\text{min}} = 2\pi \frac{x_{\text{min}}'}{\lambda'} = 2\pi \frac{\SI{1.43}{\centi \meter}}{\SI{4.89}{\centi \meter}} = 1.83
\end{equation*}
We then use $ \beta x_{\text{min}} $ and the SWR $ s $ find $ \xi $ and $ \eta $ normalized by the waveguide's characteristic impedance $ Z_{0} $. The results are
\begin{equation*}
	\frac{\eta_{R}}{Z_{0}} = \frac{(s^{2} - 1) \tan \beta x_{\text{min}}}{1 + s^{2}\tan^{2}\beta x_{\text{min}}} = \cdots = 0.39
\end{equation*}
and 
\begin{equation*}
	\frac{\xi_{R}}{Z_{0}} = s \cdot \left(1 - \frac{\eta_{R}}{Z_{0}}\tan \beta x_{\text{min}}\right) = \cdots = 1.48
\end{equation*}
The load impedance, normalized by $ Z_{0} $, is thus
\begin{equation*}
	\frac{Z_{R}}{Z_{0}} = \frac{\xi_{R}}{Z_{0}} + i\frac{\eta_{R}}{Z_{0}} = 1.48 + 0.39 i
\end{equation*}


\subsection{Power Dissipated at Klystron Modes}
the true power $ P_{\text{0}} $ corresponding to microwaves at each klystron mode are
\begin{equation*}
	P_{0} = \frac{P_{\text{bol}}}{1 - \abs{r_{R}}^{2}} \eqtext{where}  \abs{r_{R}}^{2} = \left(\frac{1 - s}{1 + s}\right )^{2}
\end{equation*}
where $ P_{\text{bol}} $ is the power measured on the bolometer and $ s $ is the standing wave ratio. The reflection coefficient $ \abs{r_{R}}^{2} $ takes the value
\begin{equation*}
	\abs{r_{R}}^{2} = \left(\frac{1 - s}{1 + s}\right )^{2} = \left(\frac{1 - 0.61}{1 + 0.61}\right)^{2} = 0.0600
\end{equation*}
Because the reflection coefficient is small, the measured bolometer power $ P_{\text{bol}} $ is only slightly less than the microwave power. Figure \ref{mikval:fig:power} shows the power $ P_{0} $  at each klystron mode found with the above relationships.

\begin{figure}
	\includegraphics[width=\linewidth]{mode-powers}
	\caption{Microwave power at reflection voltages corresponding to various klystron modes. The power of a given mode increases with increasing reflection voltage.}
	\label{mikval:fig:power}
\end{figure}



\section{Error Analysis}

\subsection{Microwave Frequency with Resonator Cavity}
\begin{equation*}
	f(h) = \big(- 0.005 \cdot h + 10.5 \big) \si{\giga \hertz}
\end{equation*}
Input data is screw position $ h = 408$. It is given in the simulated measurements without an explicit uncertainty, so I assumed an uncertainty $ \delta h = 0.05 \cdot h $, i.e. 5 percent of the measured value. Using the first-order Taylor approximation
\begin{equation*}
	f(h + \delta h) \approx f(h) + \delta h \cdot f'(h) \implies f(h + \delta h) - f(h) \approx \delta h \cdot f'(h),
\end{equation*}
I assumed an uncertainty
\begin{equation*}
	\delta f = \abs{\delta h \cdot f'(h)} = (0.05 \cdot h) \cdot \SI{0.005}{\giga \hertz} \approx \SI{0.10}{\giga \hertz} 
\end{equation*}

\subsection{Short-Circuit and Load-Terminated Signal Analysis}

\subsubsection{Microwave Frequency with Standing Wave Signals}
Using the microwave wavelength $ \lambda ' = 2\Delta x_{\text{sc}} $  in the waveguide, the microwave wavelength in vacuum is
\begin{equation*}
	\lambda = \frac{\lambda'}{\sqrt{1 + \frac{\lambda'^{2}}{4a^{2}}}} = \frac{2\Delta x_{\text{sc}}}{\sqrt{1 + \left(\frac{\Delta x_{\text{sc}}}{a}\right)^{2}}}
\end{equation*}
The microwave frequency $ f $ in vacuum is thus
\begin{equation*}
	f = \frac{c}{\lambda} = \frac{c}{2\Delta x_{\text{sc}}}\sqrt{1 + \left(\frac{\Delta x_{\text{sc}}}{a}\right)^{2}} = \frac{c}{2}\sqrt{\frac{1}{(\Delta x_{\text{sc}})^{2}} + \frac{1}{a^{2}}}
\end{equation*}
Input data are distance $ \Delta x_{\text{sc}} $ between successive minima in the short-circuited signal and the waveguide width $ a $, with associated uncertainties $ u_{x} $ and $ u_{a} $. The numerical values of the input data are below---I assumed the uncertainties.
\begin{equation*}
	\Delta x_{\text{sc}} = \SI{2.44 \pm 0.09}{\centi \meter} \eqtext{and} a = \SI{2.30 \pm 0.05}{\centi \meter}
\end{equation*}
Assuming the uncertainty in $ c $ is negligible, the sensitivity coefficients are
\begin{align*}
	&c_{x} = \pdv{f}{\Delta x_{\text{sc}}} = -\frac{c}{2(\Delta x_{\text{sc}})^{3}}\left(\frac{1}{(\Delta x_{\text{sc}})^{2}} + \frac{1}{a^{2}}\right)^{-1/2}\\
	&c_{a} = \pdv{f}{a} \hspace{6mm}= -\frac{c}{2a^{3}}\left(\frac{1}{(\Delta x_{\text{sc}})^{2}} + \frac{1}{a^{2}}\right)^{-1/2}
\end{align*}
The associated uncertainty in microwave frequency is
\begin{equation*}
	u_{\text{f}} = \sqrt{(c_{x}u_{x})^2 + (c_{a}u_{a})^2} = \cdots = \SI{0.19}{\giga \hertz}
\end{equation*}

\subsubsection{Standing Wave Ratio}
The standing wave ratio for microwaves in the load-terminated waveguide is
\begin{equation*}
	s = \sqrt{\frac{h_{\text{min}}}{h_{\text{max}}}} 
\end{equation*}
Input data are $ h_{\text{min}} $ and $ h_{\text{max}} $, the microwave signal values at the extrema of the load-terminated signal, with associated measurement uncertainties $ u_{\text{min}} $ and $ u_{\text{max}} $. Assuming a 2 percent uncertainty on $ h_{\text{min}} $ and $ h_{\text{max}} $, the numerical values are
\begin{equation*}
	h_{\text{min}} = \SI{1.42 \pm 0.03}{V} \eqtext{and} h_{\text{max}} = \SI{3.86 \pm 0.08}{V}
\end{equation*}
Sensitivity coefficients are
\begin{equation*}
	c_{\text{min}} = \pdv{s}{h_{\text{min}}} = \frac{1}{2h_{\text{max}}} \sqrt{\frac{h_{\text{max}}}{h_{\text{min}}}} \eqtext{and} 	c_{\text{max}} = \pdv{s}{h_{\text{max}}} = \frac{-1}{2h^{2}_{\text{max}}} \sqrt{\frac{h_{\text{max}}}{h_{\text{min}}}}
\end{equation*}
The associated uncertainty is
\begin{equation*}
	u_{\text{s}} = \sqrt{(c_{\text{min}}u_{\text{min}})^2 + (c_{\text{max}}u_{\text{max}})^2} = \cdots = 0.01
\end{equation*}

\subsubsection{The Product $ \beta x_{\text{min}} $}
The product $ \beta x_{\text{min}}  $ is found with
\begin{equation*}
	\beta x_{\text{min}} = 2\pi \frac{x_{\text{min}}'}{\lambda'} = \pi \frac{x_{\text{min}}'}{\Delta x_{\text{sc}}}
\end{equation*}
Input quantities are distance $ \Delta x_{\text{sc}} $ between successive minima in the short-circuited signal and distance $ x_{\text{min}}' $ between minima of the short-circuit and load-terminated curves with corresponding uncertainties $ u_{\text{sc}} $ and $ u'_{\text{min}} $. Numerical values are
\begin{equation*}
	\Delta x_{\text{sc}} = \SI{2.44 \pm 0.09}{\centi \meter} \eqtext{and} x'_{\text{min}} = \SI{1.43 \pm 0.06}{\centi \meter}
\end{equation*}
The associated sensitivity coefficients are
\begin{align*}
	&c_{\text{sc}} = \pdv{}{\Delta x_{\text{sc}}}\big[\beta x_{\text{min}} \big] = - \pi \frac{x_{\text{min}}'}{(\Delta x_{\text{sc}})^{2}}\\
	&c'_{\text{min}} = \pdv{}{x'_{\text{min}}}\big[\beta x_{\text{min}} \big] = \frac{\pi}{\Delta x_{\text{sc}}}
\end{align*}
The corresponding uncertainty in the product $ \beta x_{\text{min}} $ is then
\begin{equation*}
	u_{\beta} = \sqrt{(c_{\text{sc}}u_{\text{sc}})^2 + (c'_{\text{min}}u'_{\text{min}})^2} = \cdots = 0.10
\end{equation*}

\subsubsection{Load Reactance}
Load reactance, normalized with the waveguide's natural impedance, is found with
\begin{equation*}
	\frac{\eta_{R}}{Z_{0}} = \frac{(s^{2} - 1) \tan \beta x_{\text{min}}}{1 + s^{2}\tan^{2}\beta x_{\text{min}}}
\end{equation*}
The input quantities are $ s $ and the product $ \beta x_{\text{min}} $. Their sensitivity coefficients are
\begin{align*}
	&c_{s} = \pdv{s}\left[\frac{\eta_{R}}{Z_{0}}\right] \hspace{10mm} = \frac{2s\tan \beta x_{\text{min}}\left(1 + \tan^{2} \beta x_{\text{min}}\right)}{\left(1+s^{2}\tan^{2} \beta x_{\text{min}}\right)^2}\\
	&c_{\beta} = \pdv{[\beta x_{\text{min}}]}\left[\frac{\eta_{R}}{Z_{0}}\right] = \frac{\left(\sec ^2\beta x_{\text{min}}-s^2\sec ^2\beta x_{\text{min}}\tan ^2\beta x_{\text{min}}\right)\left(s^2-1\right)}{\left(1+s^2\tan ^2\beta x_{\text{min}}\right)^2}
\end{align*}
The associated error is
\begin{equation*}
	u_{\xi} = \sqrt{(c_{s}u_{s})^{2} + (c_{\beta}u_{\beta})^{2}} = \cdots = 0.11
\end{equation*}
where $ u_{s} $ and $ u_{\beta} $ are found above.

\subsubsection{Load Resistance}
Resistance is found with
\begin{equation*}
	\frac{\xi_{R}}{Z_{0}} = s \cdot \left(1 - \frac{\eta_{R}}{Z_{0}}\tan \beta x_{\text{min}}\right)
\end{equation*}
Sensitivity coefficients are
\begin{align*}
	&c_{s} = \pdv{s}\left[\frac{\xi_{R}}{Z_{0}}\right] \hspace{11mm}= 1 - \frac{\eta_{R}}{Z_{0}}\tan \beta x_{\text{min}} \\
	&c_{\beta} = \pdv{[\beta x_{\text{min}}]}\left[\frac{\xi_{R}}{Z_{0}}\right]  \hspace{1mm} = -s\frac{\eta_{R}}{Z_{0}}\sec^{2} \beta x_{\text{min}}\\
	&c_{\eta} = \pdv{\big[\eta_{R}/Z_{0}\big]}\left[\frac{\xi_{R}}{Z_{0}}\right] = - s \tan \beta x_{\text{min}}
\end{align*}
The associated error is
\begin{equation*}
	u_{\xi} = \sqrt{(c_{s}u_{s})^{2} + (c_{\beta}u_{\beta})^{2} + (c_{\eta}u_{\eta})^{2}} = \cdots = 0.44
\end{equation*}
where $ u_{s} $, $ u_{\beta} $ and $ u_{\eta} $ are found above.

\subsection{Power of Klystron Modes}
The power $ P_{0} $ of each klystron mode is found with
\begin{equation*}
	P_{0} = \frac{P_{\text{bol}}}{1 - \abs{r_{R}}^{2}} = \frac{P_{\text{bol}}}{1 - \left(\frac{1 - s}{1 + s}\right )^{2}}
\end{equation*}
Input data are the power $ P_{\text{bol}} $ measured on the bolometer, with corresponding uncertainty $ u_{\text{bol}} = \SI{0.1}{\milli \watt} $, and the standing wave ratio $ s $ for the load-terminated waveguide, whose uncertainty is found above. Their sensitivity coefficients are
\begin{align*}
	&c_{\text{bol}} = \pdv{P_{0}}{P_{\text{bol}}} = \frac{1}{1 - \left(\frac{1 - s}{1 + s}\right )^{2}} = \frac{s^{2}+2s +1}{4s}\\
	&c_{s} = \pdv{P_{0}}{s} = \frac{P_{\text{bol}}\left(s+1\right)\left(s-1\right)}{4s^2}
\end{align*}
The associated error in the microwave power $ P_{0} $ is 
\begin{equation*}
	u_{P} = \sqrt{(c_{\text{bol}}u_{\text{bol}})^{2} + (c_{s}u_{s})^{2}}
\end{equation*}
where $ u_{s} $ is found above. The graph in Figure \ref{mikval:fig:power} includes error bars showing the corresponding error in microwave power $ P_{0} $.

\section{Results}
The microwave frequency, found with the resonator cavity is
\begin{equation*}
	\boxed{f_{\text{resonator}} = \SI{8.46 \pm 0.10}{\giga \hertz}}
\end{equation*}
Microwave frequency, found from the standing wave wavelength, is
\begin{equation*}
	\boxed{f_{\text{sw}} = \SI{8.96 \pm 0.19}{\giga \hertz}}
\end{equation*}
The standing wave ratio for microwaves in the load-terminated waveguide was
\begin{equation*}
	\boxed{s = 0.61 \pm 0.01}
\end{equation*}
The load resistance and reactance, normalized by the waveguide's characteristic impedance, were
\begin{equation*}
\boxed{	\frac{\xi_{R}}{Z_{0}} = 1.48 \pm 0.44} \eqtext{and} \boxed{\frac{Z_{R}}{Z_{0}} = 0.39 \pm 0.11}
\end{equation*}



\end{document}