\documentclass[11pt, a4paper]{article}
\usepackage{mwe}
\usepackage{amsmath}
\usepackage{mathtools}
\usepackage{graphicx}
\usepackage{bm} % for bold vectors in math mode
\usepackage{physics} % for differential notation, etc...
\usepackage[separate-uncertainty=true]{siunitx}

\usepackage{graphicx}
\graphicspath{{"../figures/"}}
\usepackage[section]{placeins} % to keep figures in their sections
\usepackage{subcaption} % for subfigures
\usepackage[export]{adjustbox} % for centered figures larger than text width
\usepackage[margin=3cm]{geometry}
\usepackage[colorlinks = true, allcolors = blue]{hyperref}

\setlength{\parindent}{0pt} % to stop indenting new paragraphs
\newcommand{\diff}{\mathop{}\!\mathrm{d}} % differential
\newcommand{\eqtext}[1]{\qquad \text{#1} \qquad}


\begin{document}
\title{Electron Spin Resonance}
\author{Elijan Mastnak}
\date{Winter Semester 2020-2021}
\maketitle
\tableofcontents

\newpage


\section{Tasks}
\begin{enumerate}
	\item Determine the electron $ g $ factor and ratio $ \frac{B}{\nu} $ using the DPPH sample.
	\item Estimate the width of the ESR absorption line.
\end{enumerate}
For background theory, see \hyperref[elespinres:s:theory]{Appendix \ref{elespinres:s:theory}}. To jump to the analysis, see \hyperref[elespinres:s:analysis]{Section \ref{elespinres:s:analysis}}.

\section{Measurement Instruments and Samples}

\subsection{Equipment}
The main pieces of equipment used in the experiment are:
\begin{itemize}
	\item A DPPH (2,2-diphenyl-1-picrylhydrazyl) sample and regenerative oscillator used to study the ESR signal.
	
	\item A power supply powering the main inductor used to generate the magnetic field $ B $ applied to the DPPH sample.
	
	\item A function generator controlling the current supplied to the main inductor.
	
	\item Secondary inductors used to slightly modulate the magnetic field about the resonance value $ B_{0} $, which makes it easier to measure the ESR signal.
	
	\item A lock-in amplifier to detect the ESR signal. The amplifier is tuned to the modulation frequency of the magnetic field and only detects signals in a small band around the modulation frequency, improving signal-to-noise ratio.
	
	\item An oscilloscope to view the regenerative oscillator and lock-in amplifier's signals.
\end{itemize}

\subsection{Overview of Procedure}
\begin{enumerate}
	\item Determine the regenerative oscillator's frequency using an oscilloscope---use the \texttt{FREQUENCY} function in \texttt{Measure} mode. Expect about \SI{80}{\mega \hertz}.
	
	\item Measure the large inductor's dimensions---you'll need enough information (e.g. number of coils, diagonal distance...) to later calculate the magnetic field inside.
	
	\item Measure the dependence of lock-in amplifier signal $ U $ on current $ I $ through the large inductor at three different oscillation frequencies.
	
	Change the current through the large inductor using the function generator using the left-most \texttt{Coarse} dial on the \texttt{TRIPLE POWER SUPPLY} panel.
	
	Change oscillator frequency using the oscillator's \texttt{Frequency} dial.
	
	Measure the lock-in amplifier's signal using the oscilloscope's \texttt{AVERAGE} function in \texttt{MEASURE} mode.
	
\end{enumerate}


\section{Data} 
An overview of the data in the experiment:
\begin{itemize}\renewcommand\labelitemi{--}
	\item Independent variable: current $ I $ supplied to the main inductor, measured in \si{\milli \ampere} using the function generator.
	
	\item Dependent variable: voltage $ U $, measured in \si{\milli \volt}, of the signal outputted by the lock-in amplifier, which is in turn proportional to the strength of the ESR signal from the DPPH sample. $ U $ is measured using the oscilloscope's \texttt{MEASURE} mode. I used a four-sample average, and took $ U $ to be the difference between the maximum and minimum observed value over an observation period of about twenty seconds.
	
	\item Regenerative oscillator frequency $ \omega_{0} $, measured in \si{\mega \hertz} using the oscilloscope's \texttt{MEASURE} mode. The $ U(I) $ relationship is measured for three values of $ \omega_{0} $. 
	
	\item Additional geometric data: inductor length, diameter, and distance between inner and outer radius, measured in \si{\centi \meter} using a ruler and calipers.
\end{itemize}


\section{Analysis} \label{elespinres:s:analysis}
\subsection{Converting from Current to Magnetic Field}
I first calculated the inductor's diagonal distance $ d $ using
\begin{equation*}
	d = \sqrt{l^{2} + (\Delta y)^{2}} = \sqrt{(\SI{12.2}{\centi \meter})^{2} + (\SI{12.8}{\centi \meter})^{2}} = \SI{17.7}{\centi \meter}
\end{equation*}
where $ l $ is the inductor's length (along the longitudinal axis) and $ \Delta y  = D - \Delta a $ is vertical distance between coil's upper and lower midpoints of the coils (measured in a cross-sectional plane perpendicular to the longitudinal axis.)


I then converted current $ I $ through the inductor to magnetic field $ B $ along the longitudinal axis with
\begin{equation*}
	B = N\frac{\mu_{0}I}{d}
\end{equation*}
where $ N = 1557 $ is the number of coils and $ d $ is the inductor's diagonal length.

\subsection{Absorption Line Derivative}
Figure \ref{elespinres:fig:absorption-line-derivative} shows the $ U(B) $ signal, which corresponds to the absorption line's derivative. I estimated the width of the absorption line from the horizontal distance on the $ B $ axis between the derivative's extrema, shown in Table \ref{elespinres:table:B-extrema}.

\begin{table}[h]
\begin{center}
 \begin{tabular}{c|c|c|c|c}
     $ \omega_{0} $ [\si{\mega \hertz}] &  $ B_{\text{min}} $ [\si{\milli \tesla}] &  $ B_{\text{max}} $ [\si{\milli \tesla}] &  $ \Delta B $ [\si{\milli \tesla}] & $ B_{0} $ [\si{\milli \tesla}] \\
     \hline
     73.0 & 2.585 & 2.720 & 0.135 & 2.65\\
     80.5 & 2.845 & 2.985 & 0.140 & 2.91\\
     86.0 & 3.070 & 3.205 & 0.135 & 3.13
	\end{tabular}
	\caption{The $ U(B) $ signal's extrema for three values of resonance frequency $ \omega_{0} $. $ \Delta B $ and $ B_{0} $ denote the width of the absorption line and resonance magnetic field, respectively. The $ B_{\text{min}} $ and $ B_{\text{max}} $ points carry an uncertainty of $ \SI{0.01}{\milli \tesla} $.}
	\label{elespinres:table:B-extrema}
\end{center}
\end{table}

\begin{figure}
	\centering
	\includegraphics[width=\linewidth]{lockin-vs-B}
	\caption{Lock-in amplifier voltage $ U $ versus inductor magnetic field $ B $ for three different resonance frequencies. The $ U(B) $ signal corresponds to the absorption line's derivative, and $ U(B) $'s intersection with the $ B $ axis is the resonance field $ B_{0} $.}
	\label{elespinres:fig:absorption-line-derivative}
\end{figure}


\subsection{Electron g Factor}
I first found the resonance magnetic field value $ B_{0} $ by interpolating a line between the $ B(U) $ signal's extrema and finding the line's intersection with the $ B $ axis according to
\begin{equation*}
	B_{0} = B_{\text{max}} - \frac{U_{\text{max}}}{k} \eqtext{where} k = \frac{B_{\text{max}} - B_{\text{min}}}{U_{\text{max}} - U_{\text{min}}}
\end{equation*}
The interpolation approach seemed more thorough than just visually estimating the $ U(B) $ signal's intersection with the $ B $ axis. I then found the electron $ g $ factor using
\begin{equation*}
	g = \frac{h \omega_{0}}{\mu_{B}B_{0}}
\end{equation*}
where $ B_{0} $ is the resonance magnetic field and $ \omega_{0} $ is the resonance frequency.


\begin{table}[h]
\begin{center}
 \begin{tabular}{c|c|c}
     Frequency $ \omega_{0} $ [\si{\mega \hertz}] & Ratio $ \omega_{0}/B_{0} $ [$ \si{\giga \hertz\, \tesla^{-1}} $] & Electron $ g $ factor  \\
     \hline
     73.0 & 27.51 $ \pm $ 0.28 & 1.97 $ \pm $ 0.02 \\
     80.5 & 27.64 $ \pm $ 0.27 & 1.98 $ \pm $ 0.02 \\
     86.0 & 27.44 $ \pm $ 0.24 & 1.96 $ \pm $ 0.02
	\end{tabular}
	\caption{Resonance ratio $ \omega_{0}/ B_{0} $ and electron $ g $ factor for three resonance frequencies. }
	\label{elespinres:table:ratio-g-factor}
\end{center}
\end{table}
	
	

\section{Error Analysis}

\subsection{Error in Inductor Diagonal Distance}
The input data is $ l = \SI{13.2 \pm 0.2}{\centi \meter} $ and $ \Delta y = \SI{12.8\pm 0.4}{\centi \meter} $, from which I found $ d $ with
\begin{equation*}
	d = \sqrt{l^{2} + (\Delta y)^{2}}
\end{equation*}
The associated sensitivity coefficients are
\begin{equation*}
	c_{l} = \pdv{d}{l} = \frac{l}{\sqrt{l^{2} + (\Delta y)^{2}}} = 0.72 \eqtext{and} c_{y} = \pdv{d}{[\Delta y]} = \frac{\Delta y}{\sqrt{l^{2} + (\Delta y)^{2}}} = 0.70
\end{equation*}
The uncertainty $ u_{d} $ is
\begin{equation*}
	u_{d} = \sqrt{(c_{l}u_{l})^{2} + (c_{y}u_{y})^{2}} \approx \SI{0.3}{\centi \meter} 
\end{equation*}

\subsection{Error in Magnetic Field Data Points}
Input values carrying uncertainty are inductor diagonal distance $ d = \SI{17.7 \pm 0.3}{\centi \meter} $ and inductor current $ I $, with uncertainty $ u_{I} = \SI{0.5}{\milli \ampere} $. I estimated $ u_{I} $ as half of function generator's smallest displayed decimal value, which was $ \SI{1}{\milli \ampere} $. Ideally, I would estimate current error as a percent of the full-scale reading, but I did not know the function generator's accuracy. From $ I $ and $ d $, I found $ B $ using
\begin{equation*}
	B = N\frac{\mu_{0}I}{d}
\end{equation*}
The associated sensitivity coefficients are
\begin{equation*}
	c_{I} = \pdv{B}{I} = \frac{N\mu_{0}}{d} \eqtext{and} c_{d} = \pdv{B}{d} = - N\frac{\mu_{0}I}{d^{2}}
\end{equation*}
and the propagated error in magnetic field $ u_{B} $ is
\begin{equation*}
	u_{B} = \sqrt{(c_{I}u_{I})^{2} + (c_{d}u_{d})^{2}}
\end{equation*}
I implemented this expression programmatically and calculated the uncertainty of each $ B $ point individually when plotting the $ U(B) $ signal in Figure \ref{elespinres:fig:absorption-line-derivative}---the values are of order $ \SI{0.005}{\milli \tesla} $.

\subsection{Error In Extrema Positions and Resonance Field}
\begin{itemize}
	\item First, I estimated the error in the $ B $ positions of $ U(B) $ signal's extrema $ B_{\text{max}} $ and $ B_{\text{min}} $ as $ \SI{0.01}{\milli \tesla} $---this is the $ B $ spacing between two $ (B, U) $ points near the extrema (see Figure \ref{elespinres:fig:absorption-line-derivative}) and is roughly twice the uncertainty on a standard $ B $ data point.
	
	\item I then estimated the error of the absorption line width $ \Delta B $ as $ \SI{0.02}{\milli \tesla} $. This is twice the error of a single extrema point; $ \Delta B $ is the difference of $ B_{\text{max}} $ and $ B_{\text{min}} $ and error adds during subtraction.
	
	\item Along similar lines, I estimated the error of the resonance field $ B_{0} $ as $ \SI{0.02}{\milli \tesla} $. My reasoning is that the information of $ B_{0} $'s position is encoded in the positions of the two extrema points  $ B_{\text{max}} $ and $ B_{\text{min}} $. As an estimate, I thus combined the uncertainty of  $ B_{\text{max}} $ and $ B_{\text{min}} $ to get $ \SI{0.02}{\milli \tesla} = 2 \cdot \SI{0.01}{\milli \tesla} $.
	
\end{itemize}


\subsection{Error in Resonance Ratio and Electron $ g $ Factor}
\textbf{Resonance Ratio:} The input quantities are $ \omega_{0} $, with uncertainty $ \SI{0.5}{\mega \hertz} $, and $ B_{0} $, with uncertainty $ \SI{0.02}{\milli \tesla} $. I found the resonance ratio with
\begin{equation*}
	\mathcal{R} = \frac{\omega_{0}}{B_{0}} 
\end{equation*}
The sensitivity coefficients are
\begin{equation*}
	c_{\omega} = \pdv{\mathcal{R}}{\omega_{0}} = \frac{1}{B_{0}} \eqtext{and} c_{B} = \pdv{\mathcal{R}}{B_{0}} = -\frac{\omega_{0}}{B^{2}_{0}}
\end{equation*}
and the error is
\begin{equation*}
	u_{\mathcal{R}} = \sqrt{(c_{\omega}u_{\omega})^{2} + (c_{B}u_{B})^{2}}
\end{equation*}
The value of $ u_{\mathcal{R}} $ for each value of $ \omega_{0} $ is shown in the second column of Table \ref{elespinres:table:ratio-g-factor}.



\vspace{3mm}
\textbf{$ \bm{g} $-Factor}
The input quantities are the same as for $ \mathcal{R} $. The $ g $ factor is found with
\begin{equation*}
	g = \frac{h \omega_{0}}{\mu_{B}B_{0}} = \frac{\hbar}{\mu_{B}} \mathcal{R}
\end{equation*}
where $ B_{0} $ is the resonance magnetic field and $ \omega_{0} $ is the resonance frequency. Using the above results for $ \mathcal{R} $, the error is
\begin{equation*}
	u_{g} = \frac{h}{\mu_{B}} u_{\mathcal{R}} = \frac{h}{\mu_{B}} \sqrt{(c_{\omega}u_{\omega})^{2} + (c_{B}u_{B})^{2}}
\end{equation*}
The value of $ u_{g} $ for each value of $ \omega_{0} $ is shown in the third column of Table \ref{elespinres:table:ratio-g-factor}.

\section{Results}
The width of the absorption line for resonance frequencies of $ \SI{73.0}{\mega \hertz} $, $ \SI{80.5}{\mega \hertz} $ and $ \SI{86.0}{\mega \hertz} $ where $ \SI{0.135}{\milli \tesla} $, $ \SI{0.140}{\milli \tesla} $ and $ \SI{0.135}{\milli \tesla} $, respectively. Since the uncertainty in $ \Delta B $ is $ \SI{0.02}{\milli \tesla} $, it is unreasonable to give $ \Delta B $ to more than two decimal places, and we can round the above values to $ \SI{0.14}{\milli \tesla} $. The result is then
\begin{equation*}
	\boxed{	\Delta B = \SI{0.14 \pm 0.02}{\milli \tesla}}
\end{equation*}
The values of the ratio $ \omega_{0} / B_{0}$ and electron $ g $-factor are shown in Table \ref{elespinres:table:ratio-g-factor}, which, I'm reprinting below for convenience.

\begin{table}[htb!]
\begin{center}
 \begin{tabular}{c|c|c}
     $ \omega_{0} $ [\si{\mega \hertz}] & $ \omega_{0}/B_{0} $ [$ \si{\giga \hertz\, \tesla^{-1}} $] & $ g $ factor  \\
     \hline
     73.0 & 27.51 $ \pm $ 0.28 & 1.97 $ \pm $ 0.02 \\
     80.5 & 27.64 $ \pm $ 0.27 & 1.98 $ \pm $ 0.02 \\
     86.0 & 27.44 $ \pm $ 0.24 & 1.96 $ \pm $ 0.02
	\end{tabular}
\end{center}
\end{table}
The average values---rounded to a reasonable three significant digits---are
\begin{equation*}
	\boxed{\frac{\omega_{0}}{B_{0}} = \SI{26.5 \pm 0.3}{\giga \hertz\, \tesla^{-1}}} \eqtext{and} \boxed{g = 1.97 \pm 0.02}
\end{equation*}
For reference, the correct values are $ \omega_{0}/ B_{0} \approx \SI{28.0}{\giga \hertz \, \tesla^{-1}} $ and $ g \approx 2.00 $. 

The fact the three values of both $ \omega_{0}/ B_{0}  $ and $ g $ are quite precise, but not perfectly accurate, suggests the presence of a systematic error. I may be wrong, but I believe the value of the inductor's diagonal distance $ d $ is at fault here. If found $ d $ only indirectly with a rough calculation, and wouldn't be surprised if my result is systematically off. A fault value of $ d $ would skew the value of magnetic field $ B $, which would then propagate to $ \omega_{0}/ B_{0}  $ and $ g $ view the resonance field $ B_{0} $. 


\appendix

\section{Theory} \label{elespinres:s:theory}
\begin{itemize}
	
	\item An electron has spin $ S = \frac{1}{2} $ and a magnetic moment with magnitude of one Bohr magneton
	\begin{equation*}
		\mu_{B} = \frac{e\hbar}{2m_{e}c} \approx \SI{9.27e-24}{\joule \, \tesla^{-1}}.
	\end{equation*}
	In an external magnetic field $ B_{0} $ there are two possible spin states: $ m_{s} = \frac{1}{2} $, corresponding to spin up (parallel to the external field) and $ m_{s} = -\frac{1}{2} $, corresponding to spin down (anti-parallel to the external field). The energy between the spin-up and spin-down states is
	\begin{equation*}
		\Delta E = E_{\uparrow} - E_{\downarrow} = g \mu_{B} B_{0}
	\end{equation*}
	where $ g \approx 2 $ is the Land\'{e} $ g $ factor for an electron.
	
	\item We can excite a transition between the up and down states with electromagnetic radiation with frequency $ \nu $ satisfying the condition
	\begin{equation*}
		\Delta E = g \mu_{B} B_{0} = h \nu
	\end{equation*}
	This relationships connects the frequency of the radiation with the resonance value of the magnetic field $ B_{0} $. The resonance frequency $ \nu $ is thus a function of the magnetic field $ B_{0} $. For a free electron, the ratio $ \frac{\nu}{B_{0}} $ is roughly
	\begin{equation*}
		\frac{\nu}{B_{0}} = \frac{g \mu_{B}}{h} \approx \SI{28.026}{\giga \hertz \, \tesla^{-1}}
	\end{equation*}
	Typical energies differences $ \Delta E $ are quite low compared to photon energies in the infrared and visible range, so ESR signals are typically quite weak. 
	
	\item The relative population of the spin-up and spin-down energy levels in an ESR sample is distributed according to the Boltzmann distribution
	\begin{equation*}
		\frac{n_{\uparrow}}{n_{\downarrow}} = \exp(-\frac{\Delta E}{k_{B}T}) = \exp(-\frac{h \nu}{k_{B}T})
	\end{equation*}
	As an example, at room temperature and frequency $ \nu = \SI{100}{\mega \hertz} $ the relative population difference is
	\begin{equation*}
		\frac{n_{\uparrow} - n_{\downarrow}}{n_{\downarrow}} = \frac{n_{\uparrow}}{n_{\downarrow}} - 1 \approx \SI{2e-5}{} \qquad \text{at } T =  \SI{300}{\kelvin} \text{ and } \nu = \SI{100}{\mega \hertz} 
	\end{equation*}
	The net absorption of radiation, and thus ESR's sensitivity, depends on the population difference $ n_{\uparrow} - n_{\downarrow} $, which depends on radiation frequency $ \nu $, which in turn depends on the resonance field $ B_{0} $. The higher the frequency, the smaller the ratio $ \frac{n_{\uparrow}}{n_{\downarrow}} $, the larger the difference $  n_{\uparrow} - n_{\downarrow} $, and the larger the ESR sensitivity.  Because of electron interactions (e.g. with the sample's crystal lattice, with other electrons, with nuclei, etc...) the ESR resonance lines are not perfectly sharp, but are somewhat spread out.
\end{itemize}

\iffalse
\textbf{The Experiment}
\begin{itemize}
	\item The magnetic field is not constant. It is amplitude-modulated with a frequency of around $ \SI{215}{Hz} $. Note that the field's oscillation amplitude is considerably smaller than the field's constant component or average value.
	
	\item Twice per oscillation period, the magnetic field crosses the resonance value $ B_{0} $. Absorption increases, resulting in an absorption line on the oscilloscope. 
	
	\item The lock-in amplifier's reference signal is the voltage supplied to the modulation inductor. I suppose that is the... smaller inductor???
	
	The absorption line appears only if the amplitude of the modulation is larger than the width of the absorption line.
	
	Because the amplitude of modulation is typically smaller, in which case we get a signal at the modulation frequency whose amplitude is proportional to the derivative of the absorption line with respect to the constant component of the magnetic field. 
	
	The signal on the oscilloscope is weak an comparable to background noise; we use a phase detector to improve the signal-to-noise ratio. 
	
	\item We can use a phase detector in measurements with a reference signal of the form
	\begin{equation*}
		U_{\text{ref}} = U_{0}\cos (\omega t + \phi). 
	\end{equation*}
	In our case the reference signal 
	
	The measured signal 
	\begin{equation*}
		U_{\text{sig}} = A(t) \cos (\omega t) 
	\end{equation*}
	is the directed output of the regenerative oscillator, partially covered with noise. Its frequency is the same as the reference frequency $ \omega $. The two signals differ in phase by $ \phi $.  
	
	\item The phase detector works as an analog signal multiplier. Setting $ U_{0} = 1$, we have
	\begin{align*}
		U_{\text{out}} &= U_{\text{ref}} U_{\text{sig}} = A(t) \cos(\omega t + \phi)\cos(\omega t)\\
		&\frac{A(t)}{2}[\cos \phi + \cos(2\omega + \phi)]
	\end{align*}
	The carrier frequency of the signal $ A(t) $ is not $ \omega $ but zero. If we filter the output signal with an RC low-pass filter, the component with frequency $ 2\omega $ is filtered out, as are all components of $ A(t) $ with frequency larger than $ \omega_{0} = \frac{1}{RC} $. For a typical time constant $ RC = \SI{1}{\second} $, the phase detector passes frequencies in a $ \SI{1}{\hertz} $ band of the modulation frequency $ \omega $. 
	
	\item The output of the regenerative oscillator has a bandwidth of a few kilohertz (which we can observe on the oscilloscope). The theoretically improved signal-to-noise ratio is
	\begin{equation*}
		\frac{SNR_{\text{detector}}}{SNR_{\text{oscillator}}} = \sqrt{\frac{\Delta f_{\text{detector}}}{\Delta f_{\text{oscillator}}}} \approx 50
	\end{equation*}
\end{itemize}
\fi


\iffalse
\textbf{What Signals to Expect}
\begin{itemize}
	\item The $ U(B) $ from the lock-in amplifier describes the rate of energy absorption in the sample with respect to magnetic field $ B $. The energy absorption itself is a classic bell curve centered at the resonance field $ B_{0} $. 
	
	
	\item The magnetic field is $ B $ as a function of time $ t $. The field has a constant component determined by the current through the inductor and the equation $ B = N\frac{\mu_{0}I}{d} $. 
	
	The signal is amplitude-modulated, so it oscillates slightly about its constant average value. The oscillation amplitude is much smaller than the average value.
	
	Supposedly the modulation frequency is about \SI{215}{\hertz}. 
	
	\item The oscilloscope shows two signals on its two channels. These are:
	\begin{enumerate}
		\item The output from the lock-in amplifier, which is a constant value of voltage vs. time (on channel 1).
		
		\item The signal from the regenerative oscillator is a sinusoidal signal of voltage $ U $ vs. time $ t $. Frequency is about $ \SI{80}{\mega \hertz} $.
	\end{enumerate}
	
\end{itemize}
\fi



\end{document}



