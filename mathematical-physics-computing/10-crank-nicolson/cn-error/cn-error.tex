\documentclass[11pt, a4paper]{article}
\usepackage[T1]{fontenc}
\usepackage{mwe}
%\usepackage[margin=3.5cm]{geometry}
\usepackage[margin=3cm,bottom=3cm,top=2.5cm]{geometry}
\usepackage{amsmath}
\usepackage{amssymb}
\usepackage{bm} % for bold vectors in math mode
\usepackage{physics} % many useful physics commands
\usepackage[separate-uncertainty=true]{siunitx} % for scientific notation and units

\usepackage{graphicx}
\graphicspath{{"figures/"}}
\usepackage[section]{placeins} % to keep figures in their sections
\usepackage[export]{adjustbox} % for subcaptionbox figures
\usepackage{subcaption}  % for figures with subcaptions

\usepackage[most, minted]{tcolorbox} % for displaying code

\usepackage{xcolor}  % to color hyperref links
\usepackage[colorlinks = true, allcolors=blue]{hyperref}

\setlength{\parindent}{0pt} % to stop indenting new paragraphs
\newcommand{\diff}{\mathop{}\!\mathrm{d}} % differential
\newcommand{\eqtext}[1]{\qquad \text{#1} \qquad}
\newcommand{\mat}[1]{\mathbf{#1}}

\newcommand{\schro}{Schr\"{o}dinger\xspace}

\newtcblisting{python}{%
	listing engine=minted,
	minted language=python,
	listing only,
	breakable,
	enhanced,
	minted options = {
		linenos, 
		breaklines=true, 
		tabsize=2,
		fontsize=\footnotesize, 
		numbersep=2mm
	},
	overlay={%
		\begin{tcbclipinterior}
			\fill[gray!25] (frame.south west) rectangle ([xshift=4mm]frame.north west);
		\end{tcbclipinterior}
	}   
}


\begin{document}
%\title{}
%\author{Elijan Jakob Mastnak\\[1mm]\small{Student ID: 28181157}}
%\date{December 2020}
%\maketitle

\begin{center}
{\scshape \Large \textbf{Error with the Generalized Crank-Nicolson Method}\par}
\vspace{2mm}
{Elijan Jakob Mastnak\\[1mm]\small{Student ID: 28181157}}
\end{center}
\vspace{10mm}

A quick case study in how the values of $ r $ and $ M $ (which determine the order of the time and position derivative approximation, respectively) used in the Crank-Nicolson method used to solve the 10th MFP report affect the accuracy of the numerical solution (see also Ref. \cite{vandijk}). The error plotted in Figures \ref{diff:fig:qho-error} and \ref{diff:fig:free-error} (below) is found according to
\begin{equation}
	\mathcal{E} = \int_{x_{0}}^{x_{j}} \abs{\psi(x, t_{N}) - \psi_{\text{analytic}}(x, t_{N})}^{2} \diff x \label{diff:eq:error}
\end{equation}
where $ \psi $ and $ \psi_{\text{analytic}} $ are the numerical and analytic wavefunction solutions, respectively, at the simulation end time $ t_{N} $. 


\begin{figure}[htb!]
\centering
\includegraphics[width=\linewidth]{qho-error}
\caption{Error in the coherent state solution after a ten-period simulation as a function of $ r $ and $ M $---note the logarithmic scale. High $ M $ improves the solution more than high $ r $ (compare the curves at $ r \equiv 1$ and $ M \equiv 1 $. Found with $ \Delta t = 0.2\pi $; error is calculated according to Equation \ref{diff:eq:error}. Note also the onset of a plateau at $ r \gtrsim 7 $ and $ M \gtrsim 8 $ beyond which the error does not improve---this evidently corresponds to the regime of non-negligible floating point error, as investigated more thoroughly in the Airy function report.}
\label{diff:fig:qho-error}
\end{figure}

\begin{figure}[htb!]
\centering
\includegraphics[width=\linewidth]{free-error}
\caption{Error in the numerical solution for the wave packet at $ a \approx 1.0 $ as a function of $ r $ and $ M $---note the logarithmic scale. Higher $ r $ improves the solution more than high $ M $ (compare the curves at $ r \equiv 1$ and $ M \equiv 1 $. Found with $ N = 500 $. Note that the solution does not improve for $ r \gtrsim 10 $ and $ M \gtrsim 4 $, corresponding to the regime of non-negligible floating point error.}
\label{diff:fig:free-error}
\end{figure}

\begin{thebibliography}{}
\setlength{\itemsep}{.2\itemsep} \setlength{\parsep}{.5\parsep}

\bibitem{vandijk} W. van Dijk and F. M. Toyama. ``Accurate numerical solutions of the time-dependent \schro equation.'' Phys. Rev. E \textbf{75}, 036707 (2007). \url{https://arxiv.org/pdf/physics/0701150.pdf}
\end{thebibliography}

\end{document}



