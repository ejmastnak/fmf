\documentclass[11pt, a4paper]{article}
\usepackage{mwe}
\usepackage{amsmath}
\usepackage{amssymb}
\usepackage{mathtools}
\usepackage{graphicx}
\usepackage{xcolor}
\usepackage{bm} % for bold vectors in math mode
\usepackage{physics} % for differential notation, etc...
\usepackage[separate-uncertainty=true]{siunitx}

\usepackage[normalem]{ulem}  % for underline with line wrapping
\usepackage[margin=3cm]{geometry}
\usepackage[colorlinks = true, allcolors=blue]{hyperref}

\setlength{\parindent}{0pt} % to stop indenting new paragraphs
\newcommand{\diff}{\mathop{}\!\mathrm{d}} % differential
\newcommand{\eqtext}[1]{\qquad \text{#1} \qquad}

\renewcommand{\vec}[1]{\bm{#1}} % for vectors
\newcommand{\uvec}[1]{\hat{\vec{#1}}} % for vectors
\newcommand{\mat}[1]{\mathbf{#1}} % for matrices
\newcommand{\dvec}[1]{\dot{\vec{#1}}} % for dotted vector quantity
\newcommand{\tvec}[1]{\tilde{\vec{#1}}} % for tilde vector quantities
\renewcommand{\t}[1]{\tilde{#1}} % shorthand for tilde

\renewcommand{\r}{\vec{r}}
\newcommand{\E}{\vec{E}} % for electric field; it's used alot!
\newcommand{\B}{\vec{B}} % for magnetic field; it's used alot!
\newcommand{\A}{\vec{A}} % for magnetic vector potential
\newcommand{\e}{\epsilon_{0}} % vacuum permittivity
\newcommand{\pe}{\vec{p}_{e}} % electric dipole moment
\newcommand{\m}{\vec{m}} % magnetic dipole moment


\renewcommand{\div}{\nabla \cdot}
\renewcommand{\curl}{\nabla \cross}
\renewcommand{\grad}{\nabla}
\renewcommand{\laplacian}{\nabla^{2}}

\begin{document}
\title{Electromagnetism Exercises Notes}
\author{Elijan Mastnak}
\date{Winter Semester 2020-2021}
\maketitle

\begin{center}
\textbf{About These Notes}
\end{center}
These are my notes are from the Exercises portion of the class \textit{Elektromagnetno Polje} (Electromagnetic Field), given to third-year physics students at the Faculty of Math and Physics in Ljubljana, Slovenia. The exact problem sets herein are specific to the physics program at the University of Ljubljana, but the content is fairly standard for an late-undergraduate electromagnetism course. I am making the notes publicly available in the hope that they might help others learning the same material.

\vspace{2mm}
\textit{Navigation}: For easier document navigation, the table of contents is ``clickable'', meaning you can jump directly to a section by clicking the colored section names in the table of contents. Unfortunately, \uline{the clickable links do not work in most online or mobile PDF viewers}; you have to download the file first.

\vspace{2mm}
\textit{On Authorship:} 
The exercises are led by Asst. Prof. Martin Klanj\v{s}ek, who has curated the problem sets and guides us through the solutions. Accordingly, credit for the problems in these notes goes to Prof. Klanj\v{s}ek. I have merely typeset the problems and provided additional explanations where I saw fit. The original exercise in Slovene (without solutions) can be found on \href{https://www-f5.ijs.si/emp-2020-2021.html}{the course website}. 

\vspace{2mm}
\textit{Disclaimer:} Mistakes---both trivial typos and legitimate errors---are likely. Keep in mind that these are the notes of an undergraduate student in the process of learning the material himself---take what you read with a grain of salt. If you find mistakes and feel like telling me, by \href{https://github.com/ejmastnak/fmf}{Github} pull request, \href{mailto:ejmastnak@gmail.com}{email} or some other means, I'll be happy to hear from you, even for the most trivial of errors.



\tableofcontents

\newpage

\section{Electrostatics}

\subsection{First Exercise Set}

\subsubsection{Background Theory: Electric Field of a Charge Distribution}
The electric field of a spatial charge distribution with volume charge density $ \rho(\r) $ is
\begin{equation*}
	\vec{E}(\vec{r}) = \frac{1}{4\pi \epsilon_{0}}\int \frac{\rho(\tvec{r}) \diff^{3}\tvec{r} }{\abs{\vec{r} - \tvec{r}}^{2}} \frac{\vec{r} - \tvec{r}}{\abs{\vec{r} - \tvec{r}}}
\end{equation*}
where $ \tvec{r} $ is a dummy variable for integration. Think of $ \rho(\tvec{r}) \diff^{3}\tvec{r} $ as an infinitesimal element of charge at the position $ \tvec{r} $, while $ \vec{r} - \tvec{r} $ represents the position of each charge. The term $ \frac{\vec{r} - \tvec{r}}{\abs{\vec{r} - \tvec{r}}} $ just gives the direction of each charge element.

\subsubsection{Charged Disk}
\textit{A charged disk has surface charge density $ \sigma $ and radius $ a $. What is the disk's electric field $ E(z) $ along an axis through the disk's center and normal to the disk? Investigate the limit behavior of $ E(z) $ for small and large $ z $.}
\begin{itemize}
	\item Break the disk into infinitesimal concentric rings and integrate over the contributions $ \diff E $ of each ring; $ r $ and $ \diff r $ represent the radius and thickness of a ring, respectively.
	
	First break each ring into infinitesimal segments with area $ r \diff \phi \diff r $. Along the perpendicular $ z $ axis through the disk's center each, segment contributes
	\begin{equation*}
		\diff E_{1} = \frac{1}{4\pi \epsilon_{0}} \frac{\sigma r\diff \phi \diff r }{z^{2} + r^{2}}
	\end{equation*}
	Notice the term $ r\diff \phi \diff r $ must have units of area to produce a charge when multiplied by the surface charge density $ \sigma $. The term $ z^{2} + r^{2} $ is just the square of the distance $ \sqrt{z^{2} + r^{2}} $ from the charge element to the $ z $ axis.
	
	\item Recognize the circular symmetry of each ring: both the $ x $ and $ y $ components of the electric field symmetrically cancel along the $ z $ axis, so the electric field only has a $ z $ component. Relate the magnitudes of $ \diff E $ and $ \diff E_{1} $ along the $ z $ axis with similar triangles to get the contribution $ \diff E $ of a ring of radius $ r $ at the point $ (0, 0, z) $:
	\begin{equation*}
		\frac{\diff E}{\diff E_{1}} = \frac{z}{\sqrt{z^{2} + r^{2}}} \implies  \diff E = \frac{z}{(z^{2} + r^{2})^{3/2}} \frac{\sigma r\diff \phi \diff r }{4\pi \epsilon_{0}} 
	\end{equation*}
	
	\item Integrate over the contributions $ \diff E $ to find $ E(z) $:
	\begin{equation*}
		E(z) = \int \diff E = \int_{0}^{2\pi}\int_{0}^{a}\diff \phi \diff r \frac{\sigma z r}{4\pi \epsilon_{0} (z^{2} + r^{2})^{3/2}} = \frac{\sigma z}{2\epsilon_{0}} \int_{0}^{a}\frac{r \diff r}{(z^{2} + r^{2})^{3/2}}
	\end{equation*}
	Solve the integral with the substitution $ u = z^{2} + r^{2} $, $\diff u = 2r \diff r $:
	\begin{align*}
		E(z) &= \int_{z^{2}}^{z^{2} + a^{2}} \frac{\diff u}{u^{3/2}} = -\frac{\sigma z}{2 \epsilon_{0}} \left(\frac{1}{\sqrt{z^{2} + a^{2}}} - \frac{1}{\sqrt{z^{2}}}\right)\\
		&=\frac{\sigma}{2\epsilon_{0}} \left(1 - \frac{z}{\sqrt{z^{2} + a^{2}}}\right) = \frac{\sigma}{2\epsilon_{0}} \left(1 - \frac{1}{1 + \frac{a^{2}}{z^{2}}}\right)
	\end{align*}
	
	\item For $ z \ll a $ (very close to the disk), $ 1 + \frac{a^{2}}{z^{2}} \to \infty $ and $ \frac{1}{1 + \frac{a^{2}}{z^{2}}} \to 0$, leaving \vspace{-2mm}
	\begin{equation*}
		E(z) \to \frac{\sigma}{2\epsilon_{0}} \qquad (z \ll a)
	\end{equation*}
	which is the electric field of an infinite charged plane.
	
	\item For $ z \gg a $ (very far from the disk), $ \frac{a^{2}}{z^{2}} \ll 1 $, and we use the Taylor approximation $ (1 + x)^{p} \approx 1 + px $ for $ x \ll 1 $:
	\begin{equation*}
		E(z) = \frac{\sigma}{2\epsilon_{0}} \left[1 -  \left (1 + \frac{a^{2}}{z^{2}}\right )^{-1/2}\right] \approx \frac{\sigma}{2\epsilon_{0}}\left[ 1 - \left(1 - \frac{a^{2}}{2z^{2}}\right) \right] = \frac{\sigma a^{2}}{4\epsilon_{0}z^{2}}
	\end{equation*}
	Multiply above and below by $ \pi $ to match this to the expression for a point charge:
	\begin{equation*}
		E(z) = \frac{\pi a^{2} \sigma}{4\pi \epsilon_{0} z^{2}} = \frac{\sigma S}{4\pi \epsilon_{0}z^{2}} = \frac{q}{4\pi \epsilon_{0}z^{2}}  \qquad (z \gg a) 
	\end{equation*}
	where $ q = \sigma S $ is the disk's charge. 

\end{itemize}


\subsubsection{Charged Plate with a Slit}
\textit{We take an infinitely large charged rectangular plate with surface charged density $ \sigma $ and remove a slit of width $ a $ from the plate. Determine the electric field $ E $ in the plane perpendicular to the plate and passing through the center of the slit as a function of the vertical distance $ z $ from the plate. Investigate the limit behavior of $ E(z) $ for small and large $ z $.}
\begin{itemize}
	\item Break the plate into thin ribbons and integrate over the contributions $ \diff E $ of each ribbon; $ r $ and $ \diff r $ represent the orthogonal distance from the slit and the thickness of each ribbon, respectively.
	
	\item Find the electric field of a ribbon a distance $ r $ from the slit along the $ z $ axis using Gauss's law with a cylindrical surface. For a cylinder of radius $ R $ and length $ l $, Gauss's law reads
	\begin{equation*}
		\oint \vec{E} \cdot \diff \vec{S} = 2\pi R l E = \frac{q_{\text{enc}}}{\epsilon_{0}} \implies E(R) = \frac{q_{\text{enc}}}{2\pi \epsilon_{0} l R}
	\end{equation*}
	Applied to the ribbon, the enclosed charged $ q_{\text{enc}} $ is the ribbon's infinitesimal charge $ \diff q_{1} = \sigma \diff S_{1} = \sigma l \diff r $, while the cylinder's radius $ R $ is the distance from the ribbon to the $ z $ axis: $ R = \sqrt{z^{2} + r^{2}} $, so the contribution $ \diff E_{1} $ of one ribbon is
	\begin{equation*}
		\diff E_{1} = \frac{\diff q_{1}}{2\pi \epsilon_{0} l \sqrt{z^{2} + r^{2}}} = \frac{\sigma l \diff r}{2\pi \epsilon_{0} l\sqrt{z^{2} + r^{2}} } = \frac{\sigma \diff r}{2\pi \epsilon_{0} \sqrt{z^{2} + r^{2}} }
	\end{equation*}
	
	\item Because of mirror-image symmetry, both the $ x $ and $ y $ components of the electric field cancel, leaving only the $ z $ component $ \diff E_{z} $. Relate $ \diff E_{z} $ and $ \diff E_{1} $ using similar triangles
	\begin{equation*}
		\frac{\diff E_{z}}{\diff E_{1}} = \frac{z}{\sqrt{z^{2} + r^{2}}} \implies  \diff E_{z} = \frac{\sigma z \diff r}{2\pi \epsilon_{0} (z^{2} + r^{2})}
	\end{equation*}
	
	\item Find the total electric field along the $ z $ axis by integrating over the contributions $ \diff E_{z} $ of all the ribbons. Because of mirror symmetry, you can calculate only the contribution of e.g. the right plane and multiple the result by two.
	\begin{align*}
		E(z) &= \int \diff E_{z} = 2 \int_{a/2}^{\infty} \left(\frac{\sigma z \diff r}{2\pi \epsilon_{0} (z^{2} + r^{2})}\right) = \frac{\sigma z}{\pi \epsilon_{0}} \int_{a/2}^{\infty} \frac{\diff r}{z^{2} + r^{2}}\\
		&= \frac{\sigma z}{\pi \epsilon_{0}} \left[\frac{1}{z}\arctan \frac{r}{z}\right]_{a/2}^{\infty} = \frac{\sigma}{\pi \epsilon_{0}} \left[\frac{\pi}{2} - \arctan(\frac{a}{2z})\right]
	\end{align*}
	
	\item In the limit $ z \gg a $ (very far from the slit), $ \arctan\frac{a}{2z} \to 0 $ and the electric field along the $ z $ axis is 
	\begin{equation*}
		E(z) = \frac{\sigma}{2 \epsilon_{0}} \qquad (z \gg a)
	\end{equation*}
	which is the field of an infinite sheet of charge.
	
	\item In the limit $ z \ll a $ (very close to the slit), we have $ \frac{a}{2z} \to \infty $. Use the asymptotic expansion $ \arctan x \approx \frac{\pi}{2} - \frac{1}{x} $ for large $ x $ to get
	\begin{equation*}
		E(z) \approx \frac{\sigma}{\pi \epsilon_{0}} \left[\frac{\pi}{2} - \left(\frac{\pi}{2} - \frac{2z}{a}\right)\right] = \frac{2\sigma}{\pi \epsilon_{0}} \frac{z}{a}
	\end{equation*}
	In this case the electric field scales linearly as $ E \sim z $. 
\end{itemize} 


\subsection{Second Exercise Set}
\subsubsection{Theory: The Poisson Equation and the Fourier Transform}

\begin{itemize}
	\item Start with Gauss's law in differential form and the relationship between $ \E $ and $ U $
	\begin{equation*}
		\div \vec{E} = \frac{\rho}{\epsilon_{0}} \eqtext{and} \vec{E} = - \grad U
	\end{equation*}
	Substitute $ \E = - \grad U $ into Gauss's law to get
	\begin{equation*}
		\div \big[-\grad U\big] = \frac{\rho}{\epsilon_{0}} \implies \nabla^{2}U = -\frac{\rho}{\epsilon_{0}}
	\end{equation*}
	The last equality takes the form of a \textit{Poisson equation}.\footnote{In general any equation of the form $ \nabla^{2}f(\r) = g(\r) $ is called a Poisson equation.} It relates charge density $ \rho $ to electric potential $ U $. In other words, if we know a spatial charge distribution $ \rho $, we can find the electric field potential $ U $ thus the electric field $ \vec{E} $ with $ \E = - \grad U $.
	
	\item As a simple example we start with a point charge. The charge distribution is
	\begin{equation*}
		\rho(\vec{r}) = q \delta(\vec{r}) \implies \nabla^{2}U = -\frac{\rho}{\epsilon_{0}} = - \frac{q}{\epsilon_{0}}\delta(\r)
	\end{equation*}
	This the Poisson equation for a point charge. We will solve the equation with a Fourier transform.
	
	\item First, a review of the Fourier transform. Think of it as an expansion over a basis of plane waves of the form $ e^{i \vec{k} \cdot \r} $. Where $ U(\vec{k}) $ plays the role of a weight function determining how much each wave contributes to the expansion
	\begin{equation*}
		U(\r) = \int \diff^{3}k U(\vec{k}) e^{i \vec{k}\cdot \r}
	\end{equation*}
	where $ U(\vec{k}) $ is the amplitude of the plane wave with wave vector $ \vec{k} $. To find $ U(\vec{k}) $ we take the inner product of both sides of the above equation with the basis function $ e^{-i \tvec{k}\cdot\r} $. 
	\begin{equation*}
		\int U(\r) e^{-i \tvec{k}\cdot \r} \diff^{3}r = \int \int \diff^{3}k \diff^{3}r U(\vec{k}) e^{i (\vec{k} - \tvec{k})\cdot \r}
	\end{equation*}
	The integrals over $ \r $ on the right-hand side is in fact a delta function, because the orthogonal plane waves cancel out over all space except at the origin, where they constructively interfere to infinity. So 
	\begin{equation*}
		\int U(\r) e^{-i \tvec{k}\cdot \r} \diff^{3}r = (2\pi)^{3} \int U(\vec{k}) \delta (\vec{k} - \tvec{k}) \diff^{3}k = (2\pi)^{3} U(\tvec{k})
	\end{equation*}
	The delta function suppresses the integral everywhere besides at $ (\vec{k} - \tvec{k}) $. And that's how you get the amplitude $ U(\tvec{k}) $ for a wave vector $ \tvec{k} $. 
	\begin{equation*}
		U(\tvec{k}) = \frac{1}{(2\pi)^{3}} \int U(\r) e^{-i\vec{k}\cdot \r} \diff^{3}r
	\end{equation*}
	This is also an inverse Fourier transform of sorts, to get $ U(\vec{k}) $ from $ U(\r) $.
	
	\textbf{Recipe:} Fourier transform the Poisson equation from $ \r $ into $ \vec{k} $ space (where the Laplacian operator $ \laplacian $ simplifies to $ -ik^{2} $ under $ \mathcal{F} $) and solve for the amplitude $ U(\vec{k}) $ of each plane wave $ e^{-i\vec{k}\cdot \r} $. Substitute $ U(\vec{k}) $ into the Fourier transform
	\begin{equation*}
		U(\r) = \int \diff^{3}k U(\vec{k}) e^{i \vec{k}\cdot \r}
	\end{equation*}
	and evaluate the integral---usually in spherical coordinates---to find find $ U(\r) $. 
	
	\item Here is the first of two more notes. What happens if we insert gradient into the Fourier transform? Remember the gradient operator acts only on $ \r $. So
	\begin{equation*}
		\grad U(\r) = \int \diff^{3}k U(\vec{k}) \grad e^{i \vec{k}\cdot \r}
	\end{equation*}
	Evaluating the gradient over $ (x, y, z) $ components gives
	\begin{equation*}
		\grad e^{i \vec{k}\cdot \r} = 
		\begin{bmatrix}
			ikx\\
			iky\\
			ikz
		\end{bmatrix}
		e^{i(k_{1}x + k_{2}y + k_{3}z)} = ik e^{i\vec{k}\cdot \r}
	\end{equation*}
	
	So:
	\begin{equation*}
		\grad U(\r) = \int \diff^{3}k U(\vec{k}) \grad e^{i \vec{k}\cdot \r} = \int \diff^{3}k U(\vec{k})i \vec{k} e^{i \vec{k}\cdot \r}
	\end{equation*}
	which has the form of a Fourier transform! This time of the function $ U(\vec{k})i \vec{k} $. 
	
	The interpretation is that the gradient operator $ \grad $ transforms to multiplication by $ i\vec{k} $ under the Fourier transform. Analogously, the Laplacian $ \nabla^{2} $ would transform into multiplication by $ (i\vec{k})^{2} = - k^{2} $. 
	
	
	\item And the second of the two notes: What happens if we insert $ \delta(\r) $ into the Fourier transform? Let $ \delta(\vec{k}) $ denote the amplitude in the expansion of $ \delta(\r) $, analogous to $ U(\vec{k}) $ for $ U(\r) $. Using the inverse Fourier transform and the integral properties of the delta function produces
	\begin{equation*}
		\delta (\vec{k}) = \frac{1}{(2\pi)^{3}} \int \delta(\r) e^{-i\vec{k}\cdot\r}\diff^{3}r = \frac{1}{(2\pi)^{3}} e^{-i\vec{k}\cdot 0} = \frac{1}{(2\pi)^{3}}
	\end{equation*}
\end{itemize}


\subsubsection{Poisson Equation for a Point Particle}
\textit{The Poisson equation for a point particle is}
\begin{equation*}
	\nabla^{2}U(\r) = - \frac{q}{\epsilon_{0}} \delta (\r)
\end{equation*}
\textit{Solve the equation for $ U(\r) $.}
\begin{itemize}
	
	\item The plan is to transform into $ \vec{k} $ space, solve for $ U(\vec{k}) $, then transform back to $ U(\r) $. 
	
	First, take the Fourier transform of both sides---use the Fourier transform identities $ \nabla^{2} \to -k^{2} $ and $ \delta(\r) \to \frac{1}{(2\pi)^{3}} $ from the theory section.
	\begin{equation*}
		- k^{2}U(\vec{k}) = - \frac{q}{\epsilon_{0}} \frac{1}{(2\pi)^{3}} \implies U(\vec{k}) = \frac{q}{(2\pi)^{3} \epsilon_{0} k^{2}}
	\end{equation*}
	
	\item Next, find $ U(\r) $ using a second Fourier transform
	\begin{equation*}
		U(\r) = \int \diff^{3} k \frac{q e^{i\vec{k}\cdot \r}}{k^{2}\epsilon_{0}(2\pi)^{3}} = \frac{q}{(2\pi)^{3}\epsilon_{0}} \int \frac{e^{i\vec{k}\cdot \r}}{k^{2}} \diff^{3}k
	\end{equation*}
	
	\item Now, how to solve the integral? Suppose $ \theta $ is the angle between $ \r $ and $ \vec{k} $. To simplify the dot product. And then integrate in spherical coordinates:
	\begin{equation*}
		U(\r) =  \frac{q}{(2\pi)^{3}\epsilon_{0}} \int_{0}^{2\pi}\diff \phi \int_{-1}^{1} \diff [\cos \theta] \int_{0}^{\infty} \diff k k^{2} \frac{e^{ik r\cos \theta}}{k^{2}} 
	\end{equation*}
	So integrate over $ \phi $, which is simple and gives $ 2\pi $:
	\begin{equation*}
		U(\r) = \frac{q}{(2\pi)^{2}\epsilon_{0}} \int_{-1}^{1}\int_{0}^{\infty} e^{i \cos \theta k r} \diff[\cos \theta] \diff k
	\end{equation*}
	Integrate first over $ \theta $, (to avoid $ e^{i \cos \theta \cdot \infty} $ from the upper $ k $ limit). Recognize the sine function the difference of exponents!
	\begin{equation*}
		U(\r) = \frac{q}{(2\pi)^{2}\epsilon_{0}} \int_{0}^{\infty} \eval{\frac{e^{i \cos \theta k r}}{i k r}}_{\theta = -1}^{1} \diff k = \frac{q}{(2\pi)^{2}\epsilon_{0}} \int_{0}^{\infty}\frac{e^{ikr}-e^{-ikr}}{ikr} \diff k = \frac{q}{(2\pi)^{2}\epsilon_{0}} \int_{0}^{\infty}\frac{2\sin (kr)}{kr} \diff k
	\end{equation*}
	The integral of the $ \operatorname{sinc} $ function is
	\begin{equation*}
		\int_{0}^{\infty} \frac{\sin x}{x}\diff x = \frac{\pi}{2}
	\end{equation*}
	And applying this integral gives
	\begin{equation*}
		U(\r) = \frac{2q}{(2\pi)^{2}\epsilon_{0}}\frac{\pi}{2r} = \frac{q}{4\pi \epsilon_{0}r}
	\end{equation*}
	which is the electric potential of a point charge. Cool! We derived it directly from Maxwell's equations.
	
	\item Finally, substitute our result for $ U(\r) $ into the Poisson equation for a point charge. The result is
	\begin{equation*}
		\laplacian \frac{q}{4\pi \epsilon_{0}r} = -\frac{q}{\e}\delta (\r) \implies \nabla^{2}\frac{1}{r} = - 4\pi \delta(\r),
	\end{equation*}
	which will be useful in the next problems. 
	
\end{itemize}

\subsubsection{Theory Interlude: Electric Field of a General Charge Distribution}
\begin{itemize}
	\item We just solved the Poisson equation for the simple case $ \rho(\r) = \delta(\r) $. Can we use this result to solve the general case $ \rho = \rho(\r) $? The answer is yes, if we expand $ \rho(\r) $ over a basis of delta functions, as follows:
	\begin{equation*}
		\rho(\r) = \int \diff^{3}\tilde{r} \rho(\tvec{r}) \delta(\r - \tvec{r})
	\end{equation*}
	In this case, the solution of $ U(\r) $ to the Poisson equation is
	\begin{equation*}
		U(\r) = \int \diff^{3}\tilde{r}\rho(\tvec{r}) \frac{1}{4\pi \epsilon_{0}\abs{\r - \tvec{r}}} = \frac{1}{4\pi \epsilon_{0}} \int\frac{\diff^{3}\tilde{r}\rho(\tvec{r})}{\abs{\r - \tvec{r}}}
	\end{equation*}
	Well that's cool. By solving the Poisson equation for a delta function and then expanding an arbitrary $ \rho(\r) $ in terms of the delta function, we got the equation to the Poisson equation for any $ \rho(\r) $. And we can then get electric field using
	\begin{equation*}
		\vec{E} = - \grad U = \frac{1}{4\pi \epsilon_{0}} \int\frac{\diff^{3}\tilde{r}\rho(\tvec{r})}{\abs{\r - \tvec{r}}^{2}} \frac{\r - \tvec{r}}{\abs{r - \tvec{r}}}
	\end{equation*}
	which agrees with the equation quoted in the previous exercise set.
\end{itemize}

\subsubsection{Electric Field of a Hydrogen Atom}
\textit{The hydrogen atom has the electric potential}
\begin{equation*}
	U(\r) = \frac{q}{4\pi \epsilon_{0}} \frac{e^{-\alpha r}}{r}\left(1 + \frac{\alpha r}{2}\right), \qquad \alpha = \frac{2}{r_{B}}
\end{equation*}
\textit{Find the charge density $ \rho(\r) $ that generates this potential.}
\begin{itemize}
	\item Use the Poisson equation, which connects $ U $ and $ \rho $
	\begin{equation*}
		\nabla^{2}U(\r) = - \frac{\rho(\r)}{\epsilon_{0}}
	\end{equation*}
	Calculate the Laplacian of our $ U(\r) $ and work in spherical coordinates, since the potential is spherically symmetric (depends only on $ r $). As a review, when acting on a function that depends only on $ r $, $ \nabla^{2} $ in spherical coordinates reads
	\begin{equation*}
		\nabla^{2} = \frac{1}{r^{2}} \pdv{}{r}\left(r^{2} \pdv{}{r}\right)
	\end{equation*}
	
	\item Applying $ \nabla^{2} $ to $ U(r) $, after some straightforward but rather tedious differentiation, leads to
	\begin{equation*}
		\nabla^{2}U(\r) = \frac{q\alpha^{3}}{8 \pi \epsilon_{0}}e^{-\alpha r}
	\end{equation*}
	Rearranging the Poisson equation then gives
	\begin{equation*}
		\rho(\r) = - \frac{q\alpha^{3}}{8 \pi}e^{-\alpha r}
	\end{equation*}
	
	\item Note that the charge density is negative, which corresponds to the negatively charged electron cloud. Inserting the definition of $ \alpha = \frac{2}{r_{B}}$ gives
	\begin{equation*}
		\rho(\r) = -\frac{q}{\pi r_{B}^{3}} e^{-\frac{2r}{r_{B}}}
	\end{equation*}
	
	Another interpretation: $ e^{-\frac{2r}{r_{B}}} $ is equivalent to $ \left(e^{-\frac{r}{r_{B}}}\right)^{2} $, which is the square of the hydrogen atom's ground state wave function. The square of the wave function is probability, and multiplying the probability by $ \frac{q}{r_{B}^{3}} $ gives a charge density. 
	
	\item And why does the charge of the proton not contribute? Because the proton occurs at the origin, which corresponds to a charge density singularity at the origin. Terms in $ U(r) $ with $ \frac{1}{r} $ generate the singularity. And we just glossed over this when applying $ \nabla^{2} $. 
	
	We can resolve this problem with a special case from the limit
	\begin{equation*}
		\lim_{r\to 0}U(\r) = \frac{q}{4\pi \epsilon_{0} r}
	\end{equation*}
	We would then have to solve the Poisson equation for this potential. But we already know this potential: it is the potential for a point charge and corresponds to a charge density
	\begin{equation*}
		\rho(\r) = q \delta(\r)
	\end{equation*}
	The correct total result for the hydrogen atom is the sum of the electron cloud result and the charge density of the nucleus.
	\begin{equation*}
		\rho(\r) = q \delta(\r) - \frac{q\alpha^{3}}{8 \pi}e^{-\alpha r}
	\end{equation*}
	
	\textit{Lesson: Be careful when working with the Poisson equation if $ U(r) $ has singularities!}
	
\end{itemize}

\subsection{Third Exercise Set}

\subsubsection{Theory: Laplace Equation}
Just a quick review from the last exercise set: the Poisson equation used to solve for the electric field potential generated by a charge density $ \rho $ is
\begin{equation*}
	\nabla^{2}U(\r) = - \frac{\rho(\r)}{\e} \eqtext{where} \laplacian = \pdv[2]{}{x} + \pdv[2]{}{y} + \pdv[2]{}{z}
\end{equation*}
Often $ \rho(\r) = 0 $ in places we're solving for the electric potential. In this case $ \laplacian U(\r) = 0 $. This equation is called a \textit{Laplace equation}.

\subsubsection{Perpendicular Ribbon in a Parallel-Plate Capacitor}
\textit{We place a long, thin conducting ribbon between the plates of a large parallel-plate capacitor, perpendicularly to the plates; the ribbon almost touches both plates with a little bit of air/insulation between the ribbon edges and the plates. The ribbon height and distance between the plates is $ a $. We ground both plates and set the ribbon potential to $ U_{0} $. What is the electric potential inside the capacitor?}

\begin{itemize}
	\item First, decide on a coordinate system. Choose Cartesian coordinates since the problem has rectangular symmetry. I will try to describe the axes, but you basically need to see a picture. Here's my attempt: we drew a 2D projection on lined paper where the capacitor plates run from left to right along the page and are separated vertically by the distance $ a $; the ribbon is a vertical line between the top and bottom plates. In this case, the $ y $ axis runs vertically upward along the ribbon (connecting the top and bottom plates), the $ x $ axis runs left to right along the page, and the $ z $ axis points out of the page. 
	
	Note that the problem is independent of $ z $ (by translation symmetry in the $ z $ direction), so we only need $ U(x, y) $. More so, the problem has reflection symmetry, so we can find the solution on only one side of the ribbon (one half of the $ x $ axis) and reflect the solution about the $ y $ axis.
	
	\item The space between the capacitor plates is empty---there is no charge, and we use the Laplace equation for the space between the plates.
	\begin{equation*}
		\laplacian U(x, y) = 0
	\end{equation*}
	Charge can occur only along the ribbon or on the capacitor plates.
	
	\item Next, determine boundary conditions for $ U(x, y) $ so the equation has a unique solution. The problem's boundaries are the ribbon and edges of the capacitor plates.
	
	On the bottom plate, $ U(x, 0) = 0 $. On the upper plate, $ U(x, a) = 0 $. Both are zero because the plates are grounded. For the ribbon $ U(0, y) = U_{0} $. And we need one more boundary---infinity. We require only that $ U(x \to \infty, y) $ is bounded, i.e. that $ U $ does not diverge at $ \infty $. 
	
	\item First, attempt solving the problem with separation of variables: $ U(x, y) = X(x)Y(y) $---this approach tends to work well with symmetric problems. Plugging this ansatz into the Laplace equation and evaluating $ \laplacian $ gives
	\begin{equation*}
		X^{''}Y + XY^{''} = 0 \implies \frac{X''}{X} = - \frac{Y''}{Y}
	\end{equation*}
	The separation of variables is successful---we were able to get only $ x $ and only $ x $ on different sides of the equation. 
	
	\item As usual, set the equation equal to a separation constant $ \kappa^{2} $ and get two equations
	\begin{equation*}
		X'' - \kappa^{2}X = 0 \eqtext{and} Y'' + \kappa^{2} Y = 0
	\end{equation*}
	Both equations have simple solutions! The equations for $ X $ and $ Y $ are solved by exponential and sinusoidal functions, respectively.
	\begin{equation*}
		X(x) = Ae^{\kappa x} + Be^{-\kappa x} \eqtext{and} Y(y) = C\sin(\kappa y) + D\cos (\kappa y)
	\end{equation*}
	We then substitute the expressions for $ X $ and $ Y $ back into the ansatz $ U = XY $:
	\begin{equation*}
		U(x, y)  = X(x)Y(y) = \left(Ae^{\kappa x} + Be^{-\kappa x}\right)\left(C\sin(\kappa y) + D\cos (\kappa y)\right)
	\end{equation*}
	
	
	\item We find the coefficients $ A, B, C $ and $ D $ using the boundary conditions. 
	\begin{itemize}
		\item Start with the most powerful condition, that $ U(x \to \infty, y) $ is bounded. This condition implies $ A = 0 $ to suppress the divergent exponential function $ e^{\kappa } $. 
			
		\item Then, use the next two simplest conditions, the ones requiring $ U(x, y) = 0 $. Starting with $ U(x, 0) = 0 $ gives
		\begin{equation*}
			0 \equiv U(x, 0) = 1 \cdot (0 + D) \implies D = 0
		\end{equation*}
		With both $ A = D = 0 $, we're left at this point with only 
		\begin{equation*}
			U(x, y) = Be^{-\kappa x} \cdot C \sin (\kappa y)
		\end{equation*}
		
		\item Next, applying $ U(x, a) = 0$ gives
		\begin{equation*}
			0 \equiv U(x, a) = Be^{-\kappa x} C \sin (\kappa a) \equiv F e^{-\kappa x} \sin (\kappa a)
		\end{equation*} 
		Note that we've joined the product of two constants into one constant $ F = BC $. 
		
		We have two options: either $ F = 0 $ or $ \sin (\kappa a) = 0$. The option $ F = 0 $ gives the trivial solution $ U(x, y) = 0 $. The non-trivial solution comes from 
		\begin{equation*}
			\sin(\kappa a ) = 0 \implies \kappa a = n \pi, \quad n 1, 2, 3, \ldots
		\end{equation*}
		Note that $ n = 0 $ leads to a trivial solution $ U(x, y) = 0 $, which we reject.
		
		Respecting the quantization of $ \kappa $ and $ F $ by the index $ n $, the general solution at this point is the linear superposition
		\begin{equation*}
			U(x, y) = \sum_{n=1}^{\infty} F_{n} e^{-\kappa_{n}x}\sin(\kappa_{n}y)
		\end{equation*}
		
		\item To find $ F_{n} $, we use the last boundary condition $ U(0, y) = U_{0} $.
		\begin{equation*}
			U_{0} = \sum_{n=1}^{\infty} F_{n} \sin(\kappa_{n}y) = \sum_{n=1}^{\infty} F_{n} \sin(\frac{n\pi y}{a})
		\end{equation*}
		where we've used $ \kappa_{n} = \frac{n\pi}{a} $. This is a Fourier expansion of the constant $ U_{0} $ over sine functions. 
		
		We find the coefficients by taking the inner product of both sides of the equation (I think on the vector space $ L(0, a) $), which amounts to multiplying both sides by $ \sin \frac{m\pi y}{a} $ and integrating both sides over $ y $ from $ 0 $ to $ a $. 
		
		The left side $ U_{0} $ becomes
		\begin{equation*}
			U_{0} \int_{0}^{a}\sin(\frac{m\pi y}{a}) \diff y = -\frac{U_{0}a}{m \pi} \cos(\frac{m\pi y}{a}) \bigg |_{0}^{a} = \frac{U_{0}a}{m \pi}\big[1 - (-1)^{m}\big]
		\end{equation*}
		And the right hand side, with the sum, we switch the sum and integral to get
		\begin{equation*}
			\sum_{n = 1}^{\infty} F_{n} \int_{0}^{a} \sin(\frac{n\pi y}{a}) \sin(\frac{m\pi y}{a} ) \diff y = \sum_{n = 1}^{\infty} F_{n} \delta_{mn} \int_{0}^{a} \sin^{2}\left (\frac{m\pi y}{a} \right ) \diff y = \frac{F_{m}a}{2}
		\end{equation*}
		Because of the orthogonality of the sine functions, the integral is zero for $ m \neq n $. Only the case $ m = n $ gives a non-zero result. 
		
		Equating the two sides gives the desired expression for $ F_{m} $:
		\begin{equation*}
			\frac{U_{0}a}{m \pi}\big[1 - (-1)^{m}\big] = \frac{F_{m}a}{2} \implies F_{m} = \frac{2U_{0}}{m\pi}\big[1 - (-1)^{m}\big]
		\end{equation*}
	\end{itemize}
	
	\item With $ F_{m} $ known, the final result for $ U(x, y) $ is then
	\begin{equation*}
		U(x, y) = \frac{2U_{0}}{\pi} \sum_{n = 1}^{\infty}\frac{1 - (-1)^{n}}{n}\exp(-\frac{n\pi x}{a}) \sin(\frac{n\pi y}{a})
	\end{equation*}
	Some limit cases: for $ x \gg a $, the exponent terms very small, and we can neglect all terms in the series except the first term $ e^{-\frac{\pi x}{a}} $ with $ n = 1 $. The result is
	\begin{equation*}
		U(x, y) = \frac{4U_{0}}{\pi}\exp(-\frac{\pi x}{a}) \sin (\frac{\pi y}{a})
	\end{equation*}
	
	\item A separate case for which we can find a nice analytic solution is in the center of the capacitor at $ y = \frac{a}{2} $. The solution reads
	\begin{equation*}
		U(x, \tfrac{a}{2}) = \frac{2U_{0}}{\pi} \sum_{n = 1}^{\infty}\frac{1 - (-1)^{n}}{n}\exp(-\frac{n\pi x}{a}) \sin(\frac{n\pi}{2})
	\end{equation*}
	Instead of finding $ U(x, \tfrac{a}{2}) $, we'll find the electric field $ \E(x, \tfrac{a}{2}) $. Because of reflection symmetry across the line $ y = \frac{a}{2} $, the electric field cannot have a $ y $ component---$ \E $ only has an $ x $ component. We'll find $ E_{x}(x) $ from the potential:
	\begin{equation*}
		E_{x}(x) = -\pdv{}{x}U(x, \tfrac{a}{2}) = -\frac{2U_{0}}{a} \sum_{n = 1}^{\infty}\big[1 - (-1)^{n}\big]\exp(-\frac{n\pi x}{a}) \sin(\frac{n\pi}{2})
	\end{equation*}
	Next, note that
	\begin{equation*}
		\big[1 - (-1)^{n}\big]\sin(\frac{n\pi}{2}) = 
		\begin{cases}
			0 & n \text{ even}\\
			2 & n = 1, 5, 9, \ldots\\
			- 2 & n = 3, 7, 11, \ldots
		\end{cases}
	\end{equation*}
	The sum simplifies to
	\begin{align*}
		E_{x}(x) = \frac{4U_{0}}{a}\left[e^{-\frac{\pi x}{a}} - e^{-\frac{3\pi x}{a}} + e^{-\frac{5\pi x}{a}} \mp \ldots \right] = \frac{4U_{0}}{a}e^{-\frac{\pi x}{a}}\left[1 - e^{-\frac{2\pi x}{a}} + \left(e^{-\frac{2\pi x}{a}}\right)^{2}\mp \ldots \right]
	\end{align*}
	which is a geometric series in $ e^{-\frac{2\pi x}{a}} $. The result is
	\begin{equation*}
		E_{x}(x) =  \frac{4U_{0}}{a}\frac{e^{-\frac{\pi x}{a}}}{1 + e^{-\frac{2\pi x}{a}}} = \frac{4U_{0}}{a}\frac{1}{e^{\frac{\pi x}{a}} + e^{-\frac{\pi x}{a}}} = \frac{2U_{0}}{a\cosh(\frac{\pi x}{a})}
	\end{equation*}
\end{itemize}

\subsubsection{A Halved Conducting Cylinder}
\textit{Consider a long cylinder of radius $ a $ cut in half along a plane running along the cylinder's longitudinal axis. We separate the two cylinder halves by an arbitrarily small amount (so the halves are insulated) and apply a potential difference $ U_{0} $ between the two halves. The halved cylinder acts as a capacitor. Find the electric potential inside the cylinder.}

\begin{itemize}
	\item Decide on a coordinate system: we use cylindrical coordinates because our problem has cylindrical symmetry. Let the $ z $ axis run along the cylinder's longitudinal axis. Because of translational symmetry along the $ z $ axis, $ U $ is independent of $ z $. 
	
	\item There is no charge inside the cylinder, so we get a Laplace equation
	\begin{equation*}
		\laplacian U(r, \phi) = 0
	\end{equation*}
	In cylindrical coordinates (when acting on a function independent of $ z $), the Laplacian operator reads
	\begin{equation*}
		\laplacian = \frac{1}{r}\pdv{r}\left(r \pdv{r}\right) + \frac{1}{r^{2}} \pdv[2]{}{\phi}
	\end{equation*}
	In our case,
	\begin{equation*}
		\laplacian U(r, \phi) = \frac{1}{r}\pdv{r}\left(r \pdv{U}{r}\right) + \frac{1}{r^{2}} \pdv[2]{U}{\phi} = 0
	\end{equation*}
	
	\item Again, we separate variables with the ansatz $ U(r, \phi) = R(r)\Phi(\phi) $. Plugging this into the Laplace equation gives
	\begin{equation*}
		\Phi \frac{1}{r}(rR')' + \frac{R}{r^{2}}\Phi'' = \Phi \left(\frac{R'}{r} + R''\right) + \frac{R}{r^{2}}\Phi'' = 0
	\end{equation*}
	Note that $ r'  = 1 $. Dividing through by $ \Phi $ and rearranging gives
	\begin{equation*}
		\frac{rR'}{R} + r^{2}\frac{R''}{R} = - \frac{\Phi''}{\Phi}
	\end{equation*}
	
	\item Following the usual separation procedure, we set both sides equal to the separation constant $ m^{2} $. The equations for $ \Phi $ and $ R $ read
	\begin{equation*}
		\Phi'' + m^{2} \Phi = 0 \eqtext{and} r^{2}R'' + rR' - m^{2}R = 0
	\end{equation*}
	The solution for $ \Phi $ is sinusoidal:
	\begin{equation*}
		\Phi(\phi) = A \sin(m\phi) + B\cos(m\phi)
	\end{equation*}
	Now, our cylindrical problem is periodic in $ \phi $ with period $ 2\pi $---this just means the cylinder repeats after one revolution. Periodicity in $ \phi $ is possible only if $ m $ takes on integer values, so we can immediately index the solutions for $ \Phi $ with
	\begin{equation*}
		\Phi_{m}(\phi) = A_{m} \sin(m\phi) + B_{m}\cos(m\phi), \quad m = 1, 2, 3, \ldots 
	\end{equation*}
	We only use positive integers because the odd/even symmetry of $ \sin $ and $ \cos $ means negative integers give the same result as positive one---solving for negative $ m $ would be redundant. We reject $ m = 0 $ because this solution leads to $ \Phi'' = 0 $, meaning $ \Phi $ is a linear function. But a linear function can't be periodic in $ \phi $, so we reject $ m = 0 $.
	
	The second equation for $ R $ is solved with powers of $ r $. The result is
	\begin{equation*}
		R_{m}(r) = C_{m}r^{m} + D_{m}r^{-m}
	\end{equation*}
	
	\item The general solution is the linear superposition
	\begin{equation*}
		U(r, \phi) = \sum_{m = 1}^{\infty}\Phi_{m}(\phi)R_{m}(m) = \sum_{m = 1}^{\infty} \left(A_{m} \sin(m\phi) + B_{m}\cos(m\phi)\right)\left( C_{m}r^{m} + D_{m}r^{-m} \right)
	\end{equation*}
	\textit{Note: We ended the problem at this point (ran out of time) and continued in the fourth exercise set. I'm completing the problem in this section to maintain continuity.}
	
	\item To find a solution specific to our problem, we apply boundary conditions. We already applied the periodic boundary condition $ U(r, \phi) = U(r, \phi + 2\pi) $, which required $ m $ be integer-valued.
	
	
	A second boundary condition requires the capacitor halves have a potential difference $ U_{0} $ between them. It is best to write this potential difference in the symmetric form
	\begin{equation*}
		U(a, \phi) = 
		\begin{cases}
			\frac{U_{0}}{2} & \phi \in (0, \pi)\\
			-\frac{U_{0}}{2} & \phi \in (\pi, 2\pi)
		\end{cases}
	\end{equation*}
	There is another condition---that $ U $ does not diverge at $ r = 0 $. This condition implies the $ D_{m} $ coefficients are zero, because the $ D_{m}r^{-m} $ term diverges at $ r = 0 $. 
	
	Observation: the second boundary condition is an odd function of $ \phi $. This implies that only odd (sine) terms can appear in the final solution. This allows us to set the $ A_{m} $ coefficients equal to zero to eliminate the cosine terms. We are left with
	\begin{equation*}
		U(r, \phi) = \sum_{m=1}^{\infty} F_{m}r^{m} \sin (m\phi)
	\end{equation*}
	where we have defined $ B_{m}C_{m} \equiv F_{m} $. 
	
	\item Applying the second boundary condition at $ r = a $ gives
	\begin{equation*}
		U(a, \phi) = \sum_{m=1}^{\infty} F_{m}a^{m} \sin (m\phi)
	\end{equation*} 
	To solve this, we take the scalar product of the equation with $ \sin(n \phi) $: 
	\begin{equation*}
		\int_{0}^{2\pi} U(a, \phi)\sin(n \phi) \diff \phi =  \int_{0}^{2\pi} \sum_{m=1}^{\infty} F_{m}a^{m} \sin (m\phi) \sin(n \phi) \diff \phi
	\end{equation*}
	Plugging in the step values of $ U(a, \phi) $ gives
	\begin{equation*}
		\frac{U_{0}}{2}\int_{0}^{\pi}\sin(n \phi) \diff \phi - \frac{U_{0}}{2}\int_{\pi}^{2\pi}\sin(n \phi) \diff \phi  =  \int_{0}^{2\pi} \sum_{m=1}^{\infty} F_{m}a^{m} \sin (m\phi) \sin(n \phi) \diff \phi
	\end{equation*}
	First, we solve the left-hand side
	\begin{align*}
		&\frac{U_{0}}{2}\left[-\frac{1}{n}\cos(n\phi)\big |_{0}^{\pi} + \frac{1}{n}\cos (n\phi)\big |_{\pi}^{2\pi} \right] = \frac{U_{0}}{2n}\left[-\cos(\pi n) + 1 + 1 - \cos(n\pi)\right]\\
		&{}\qquad = \frac{U_{0}}{n}\left(1 - (-1)^{n}\right)
	\end{align*}
	where we've used the identity $ \cos(n\pi) = (-1)^{n} $.
	
	\item Next, we solve the right-hand side. Switching the order of integration and summation gives
	\begin{equation*}
		  \sum_{m=1}^{\infty} F_{m}a^{m} \int_{0}^{2\pi}  \sin (m\phi) \sin(n \phi) \diff \phi = \sum_{m=1}^{\infty} F_{m}a^{m}\left (\frac{2\pi}{2}\delta_{mn} \right ) = F_{n}a^{n} \pi
	\end{equation*}
	Combining the left and right sides gives
	\begin{equation*}
		F_{n} = \frac{U_{0}\big[1 - (-1)^{n}\big]}{\pi n a^{n}}
	\end{equation*}
	So, the solution for $ U(r, \phi) $ is
	\begin{equation*}
		U(r, \phi) = \sum_{n = 1}^{\infty} \left[\frac{U_{0}\big[1 - (-1)^{n}\big]}{\pi n a^{n}}\right]r^{n}\sin(n\phi) = \frac{U_{0}}{\pi} \sum_{n = 1}^{\infty} \left(\frac{r}{a}\right)^{n} \frac{\big[1 - (-1)^{n}\big]}{n}\sin(n\phi)
	\end{equation*}
	
	\item Next, some limiting cases. It will be easier to work in terms of electric field instead of potential. We will find the electric field in the two planes parallel and perpendicular to the slit between the capacitor halves. 
	
	First, in the perpendicular (vertical) plane. The field points from high to low potential, so from the top half of the capacitor to the bottom half. In this plane we can work with just one coordinate $ r $, which represents the vertical distance from the cylinder's center. Note that $ \phi = \frac{\pi}{2} $. The component $ E_{r} $ we're after is
	\begin{align*}
		E_{r} &= - \pdv{r}\eval{U(r, \phi)}_{\phi = \frac{\pi}{2}}  = - \frac{U_{0}}{\pi} \sum_{n = 1}^{\infty} n\left(\frac{r}{a}\right)^{n -1}\frac{1}{a}\frac{\big[1 - (-1)^{n}\big]}{n}\sin(n\phi)\bigg|_{\phi = \frac{\pi}{2}}\\
		&=  -\frac{U_{0}}{\pi a} \sum_{n= 1}^{\infty}\left(\frac{r}{a}\right)^{n-1} \frac{\big[1 - (-1)^{n}\big]}{n}\sin(\frac{n\pi}{2})
	\end{align*}
	The sum simplifies considerably when you realize
	\begin{equation*}
		\frac{1 - (-1)^{n}}{n}\sin(\frac{n\pi}{2}) = 
		\begin{cases}
			0 & n \text{ even} \\
			2 & n = 1, 5,\ldots \\
			-2 & n = 3, 7,\ldots 
		\end{cases}
	\end{equation*}
	We can then write the field as a geometric series
	\begin{equation*}
		E_{r} = \frac{-2U_{0}}{\pi a}\left[1 - \left(\frac{r}{a}\right)^{2} + \left(\frac{r}{a}\right)^{4} \mp \cdots  \right] = \frac{-2U_{0}}{\pi a} \frac{1}{1 + \left(\frac{r}{a}\right)^{2}}
	\end{equation*}
	Note that $ E_{r} $ is largest at $ r = 0 $, decreases monotonically with increasing $ r $, and falls to half of its maximum value at $ r = a $. 
	
	\item In a field parallel to the slit, we would set $ \phi = 0 $. This plane is perpendicular to the vertical plane, which used the radial component $ E_{r} $, so for the parallel plane we work with the $ \phi $ component $ E_{\phi} $. 
	\begin{equation*}
		E_{\phi} = - \frac{1}{r}\pdv{\phi}U(r, \phi) \bigg |_{\phi = 0}
	\end{equation*}
	As before, the sum simplifies considerably to a geometric series. The result turns out to be
	\begin{equation*}
		E_{\phi} = - \frac{2U_{0}}{\pi a}\frac{1}{1 - \left(\frac{r}{a}\right)^{2}}
	\end{equation*}
	Note that $ E_{\phi} $ diverges at $ r = a $. This is a consequence of the very small slit spacing between the capacitor halves at $ r = a $; schematically have $ E = \frac{U_{0}}{d} \to \infty $ as $ d \to 0 $. 
	
\end{itemize}

\subsection{Fourth Exercise Set}


\subsubsection{Conducting Sphere in a Uniform Electric Field}
\textit{A conducting sphere of radius $ a $ is placed in a uniform electric field $ E_{0} $ pointing in the $ z $ direction. Find the electric field $ U $ inside and outside the sphere.} 

\begin{itemize}
	\item Use spherical coordinates to take advantage of spherical symmetry. This means we want $ U(r, \phi, \theta) $. Because of the problem's rotational symmetry, the solution will be independent of $ \phi $. We need only $ U(r, \theta) $. 
	
	\item The sphere is at a constant potential because it is a conductor. We'll set $ U = 0 $ inside the sphere. In the space around the sphere, we solve the Laplace equation
	\begin{equation*}
		\laplacian U(r, \theta) = 0
	\end{equation*}
	We then separate variables via $ U = R(r)\Theta(\theta) $, which leads to the general solution
	\begin{equation*}
		U(r, \theta) = \sum_{l = 0}^{\infty} \left[A_{l}r^{l}+B_{l}r^{-(l+1)}\right]P_{l}(\cos \theta)
	\end{equation*}
	where $ P_{l} $ are the Legendre polynomials. 
	
	\item On to the boundary conditions. On the surface we'll set $ U(a, \theta) = 0 $. And at infinity, we use the boundary condition 
	\begin{equation*}
		U(r \to \infty, 0) = - E_{0}z = -E_{0}r \cos \theta
	\end{equation*}
	This is the potential of a uniform electric field (at infinity, the potential from the sphere is negligible). This potential is chosen so that
	\begin{equation*}
		- \pdv{U}{z} = E_{0}
	\end{equation*}
	i.e. that the potential at infinity recovers the uniform electric field $ E_{0} $. 
	
	\item We'll start with the second boundary condition at $ r \to \infty $. Applying the condition to the general solution gives
	\begin{equation*}
		\sum_{l = 0}^{\infty} A_{l}r^{l} P_{l}(\cos \theta) = -E_{0}r \cos \theta
	\end{equation*}
	Note that the $ r^{-(l+1)} $ terms vanish as $ r \to \infty $. The entire series equals a single term proportional to $ \cos \theta $ and the equality holds if only the $ l = 1 $ term in the series is non-zero, which generates a corresponding $ \cos \theta $ term from $ P_{1}(\cos \theta) = \cos \theta $ ie. $ P_{1}(x) = x $. The $ l = 1 $ term is
	\begin{equation*}
		A_{1}r \cos \theta = -E_{0}r \cos \theta \implies A_{1} = - E_{0}
	\end{equation*}
	so we have $ A_{l} = -E_{0}\delta_{l1} $. The solution for $ U(r, \theta) $ simplifies to
	\begin{equation*}
		U(r, \theta) = -E_{0}r \cos \theta + \sum_{l=1}^{\infty} B_{l}r^{-(l+1)}P_{l}(\cos \theta)
	\end{equation*}
	
	\item Next, the second boundary condition: $ U(a, \theta) = 0 $. Substituting the condition into the intermediate solution gives
	\begin{equation*}
		E_{0}a\cos \theta = \sum_{l=1}^{\infty} B_{l}a^{-(l+1)}P_{l}(\cos \theta)
	\end{equation*}
	Again, the entire series equals only a single term. Again, this will be the $ l = 1 $ term corresponding to $ P_{1}(\cos \theta) = \cos \theta $. For $ l \neq 1 $ we have $ B_{l} = 0 $. The $ l = 1 $ term gives
	\begin{equation*}
		E_{0}a\cos \theta = B_{1}a^{-2}\cos \theta \implies B_{1} = E_{0}a^{3}
	\end{equation*}
	With $ A_{l} $ and $ B_{l} $ known for all $ l $, the final result is
	\begin{equation*}
		U(r, \theta) = - E_{0} r \cos \theta + \frac{E_{0}a^{3}}{r^{2}}\cos \theta
	\end{equation*}
	The first term, $ - E_{0} r \cos \theta $, is the potential of the uniform external field $ E_{0} $. The second term comes from the sphere. In fact, this second term has the same form as the potential of an electric dipole!
	
	\textit{Limiting cases are discussed in the next exercise set.}
\end{itemize}


\subsection{Fifth Exercise Set}

\subsubsection{Conducting Sphere in a Uniform Electric Field (continued)}
\begin{itemize}
	\item Where we left off last time, we had found the potential due to the sphere was
	\begin{equation*}
		U(r, \theta) = - E_{0} \cos \theta + \frac{E_{0}a^{3}}{r^{2}}\cos \theta,
	\end{equation*}
	and identified the sphere's contribution $ \frac{E_{0}a^{3}}{r^{2}}\cos \theta $ corresponded to the potential of an electric dipole. Our next step is to explore the sphere's dipole behavior.
	
	\item We consider an infinitesimal element of the sphere's surface at the angle $ \theta $ carrying charge $ \diff q $, on the upper hemisphere with positive charge and $ \theta $. Recall the electric field, as for any conductor, is perpendicular to the surface. 
	
	We write Gauss's law for the small surface element, which is simple because the electric field is perpendicular to the surface
	\begin{equation*}
		E_{\perp} \diff S = \frac{\diff q}{\e} \implies \dv{e}{S} = \sigma = \e E_{\perp}
	\end{equation*}
	This equality gives us an expression for $ \sigma $ in terms of the electric field $ E_{\perp} $ perpendicular to the surface. We can find $ E_{\perp} $ from the potential:
	\begin{equation*}
		E_{\perp} = -\pdv{U}{r}\bigg |_{r = a} = E_{0}\cos \theta + 2E_{0}\cos \theta \implies \sigma = 3\e E_{0} \cos \theta
	\end{equation*}
	The charge density's dependence on $ \theta $ quantitatively demonstrates the sphere's dipole-like charge distribution. 
	
	\item With charge density $ \sigma $ known, we can find the sphere's electric dipole moment via
	\begin{equation*}
		\vec{p}_{e} = \int \tvec{r} \diff q
	\end{equation*}
	We qualitatively expect $ \vec{p}_{e} $ to point upward (from the negative to the positively charged hemisphere), and confirm this analytically. By spherical symmetry, only the $ z $ component of $ \vec{p}_{e} $ is non-zero; this is
	\begin{equation*}
		p_{e_{z}} = \int \tilde{z} \diff q = \int (a \cos \theta) \cdot (\sigma \diff S) = \int( a \cos \theta) \cdot (3 \e E_{0} \cos \theta) \cdot \diff S
	\end{equation*}
	To find $ \diff S $, we find the area of a small band of width $ \diff a $ around the sphere's surface. The band's area is $ 2\pi r \diff a = 2\pi (a \sin \theta ) (a \diff \theta) $. The dipole moment $ p_{e_{z}} $ is then
	\begin{align*}
		p_{e_{z}} &= \int_{0}^{\pi}(3\e E_{0}a\cos^{2}\theta) \cdot 2\pi (a \sin \theta ) (a \diff \theta) = 6a^{3}\pi \e E_{0} \int_{-1}^{1}\cos^{2}\theta \diff [\cos \theta]\\
		& = 6a^{3}\pi \e E_{0} \left[\frac{1}{3}\cos^{3}\theta \right]_{-1}^{1} = 4\pi \e E_{0} a^{3}
	\end{align*}
\end{itemize}

\subsubsection{Electric Dipole in a Conducting Spherical Shell}
\textit{We place an electric dipole with dipole moment $ \pe $ in the center of a conducting spherical shell of radius $ a $. What is the electric potential inside the shell?}

\begin{itemize}
	\item We use spherical coordinates, which are best suited to the problem's spherical symmetry. By rotational symmetry, the potential depends only on the coordinates $ r $ and $ \theta $, not $ \phi $. Besides at the center, the charge density inside the sphere is zero, so we solve the Laplace equation
	\begin{equation*}
		\laplacian U(r, \theta) = 0
	\end{equation*}
	The general solution is
	\begin{equation*}
		U(r, \theta) = \sum_{l = 0}^{\infty}\left(A_{l}r^{l} + B_{l}r^{-(l+1)}\right)P_{l}(\cos \theta)
	\end{equation*}
	where $ P_{l} $ are the Legendre polynomials. 
	
	\item We then apply boundary conditions to find a solution specific to our problem. First, the potential on the shell's surface is constant, since the shell is a conductor. For convenience, we'll set $ U(a, \theta) = 0 $. The second boundary condition concerns the dipole at the sphere's center. Namely, the potential approaches the potential of an electric dipole near the sphere's center. Quantitatively, this condition reads
	\begin{equation*}
		U(r \to 0, \theta) = \frac{p_{e}\cos \theta}{4\pi \e r^{2}}
	\end{equation*}
	
	\item We start with the simpler second boundary condition, (the boundary $ r \to 0 $ eliminates the $ r^{l} $-dependent term). Inserted into the general solution, the second condition reads
	\begin{equation*}
		U(r\to 0, \theta) = \sum_{l = 0}^{\infty} B_{l}r^{-(l+1)}P_{l}(\cos \theta) = \frac{p_{e}\cos \theta}{4\pi \e }r^{-2}
	\end{equation*}
	Note that the entire series sums to only a single term; for this to work, only the $ l = 1 $ term in the series can be non-zero, leaving
	\begin{equation*}
		B_{1}r^{-2}\cos \theta = \frac{p_{e}\cos \theta}{4\pi \e }r^{-2} \implies B_{1} = \frac{p_{e}}{4\pi \e} \eqtext{and} B_{l\neq 1} = 0
	\end{equation*}
	Note the use of $ P_{1}(\cos \theta) = \cos \theta $. The intermediate solution at this stage is
	\begin{equation*}
		U(r, \theta) = \sum_{l = 0}^{\infty} A_{l} r^{l} P_{l}(\cos \theta) + \frac{p_{e}}{4\pi \e}r^{-2} \cos \theta
	\end{equation*}
	We apply the second boundary condition $ U(a, \theta) = 0 $ to get
	\begin{equation*}
		\sum_{l = 0}^{\infty} A_{l} a^{l} P_{l}(\cos \theta) = - \frac{p_{e}}{4\pi \e}a^{-2} \cos \theta
	\end{equation*}
	As before, only the $ l = 1 $ term can be non-zero to satisfy the equality. The result is
	\begin{equation*}
		A_{1}a\cos \theta = - \frac{p_{e}}{4\pi \e}a^{-2} \cos \theta \implies A_{1} = -\frac{p_{e}}{4\pi \e a^{3}} \eqtext{and} A_{l\neq 1} = 0
	\end{equation*}
	With the coefficients $ A_{l} $ and $ B_{l} $ known, the solution for $ U(r, \theta) $ is
	\begin{equation*}
		U(r, \theta) = \left[\frac{p_{e}}{4\pi \e r^{2}} -\frac{p_{e}}{4\pi \e a^{3}}r\right]\cos \theta = \frac{p_{e}\cos \theta}{4\pi \e}\left[\frac{1}{r^{2}} - \frac{r}{a^{3}}\right]
	\end{equation*}
	The $ \frac{1}{r^{2}} $ term is the dipole's contribution. The $ \frac{r}{a^{3}} $ comes from the charge induced on the conducting shell. 
	
	\item The induced term is worth a closer look, noting that $ r\cos \theta = z $. 
	\begin{equation*}
		U_{\text{induced}} = -\frac{p_{e}}{4\pi \e a^{3}}r\cos \theta = -\frac{p_{e}}{4\pi \e a^{3}}z
	\end{equation*}
	In particular, the associated electric field is
	\begin{equation*}
		E_{\text{induced}} = -\pdv{z}U_{\text{induced}} = \frac{p_{e}}{4\pi \e a^{3}}
	\end{equation*}
	In other words, the electric field generated by the induced charge is constant! The uniform field also tells us about the charge distribution on the sphere's surface: to create a uniform field in the $ z $ direction, the shell must have a dipole-like charge distribution, with positive charge on the lower hemisphere and negative charge on the upper hemisphere. 
	
	\item Next, we're interested in the analytic expression for the surface charge density $ \sigma $. We consider a small surface element $ \diff S $, and consider the total electric field at that surface. The electric field must be perpendicular to the surface, since the shell is a conductor. Gauss's law applied to the surface element reads
	\begin{equation*}
		- E_{\perp} \diff S = \frac{\diff q}{\e} \implies \sigma \equiv \dv{q}{S} = -\e E_{\perp}
	\end{equation*}
	Note the minus sign, indicating the field's electric flux leaving the surface element from inside the shell. We find an expression for $ E_{\perp} $ from $ U $:
	\begin{equation*}
		E_{\perp} = -\pdv{U}{r}\bigg|_{r = a} = + \frac{p_{e}\cos \theta}{4\pi \e}\left[\frac{2}{r^{3}} + \frac{1}{a^{3}}\right]_{r = a} = \frac{3p_{e}}{4\pi \e a^{3}} \cos \theta
	\end{equation*}
	The associated surface charge density is
	\begin{equation*}
		\sigma = -\e E_{\perp} = -\frac{3p_{e}}{4\pi a^{3}} \cos \theta
	\end{equation*}
	
	
\end{itemize}

\subsubsection{Point Charge Above a Conducting Plane}
\textit{Consider a positive point charge $ q $ a distance $ d $ above a large, grounded conducting plane. What is the electric potential in space due to the charge-plane system?}
\begin{itemize}
	\item We will use a trick called the \textit{method of images} to solve the problem. 
%	Currently, the electric field below the plane is zero---the field from the upper charge is perpendicularly incident on and vanishes into the grounded plate. 
	Namely, we imagine a negative point charge $ -q $ a distance $ d $ below the plane---a mirror image of the original positive charge. The resulting charge distribution is an electric dipole. 
	
	\textit{Note:} placing an imaginary negative charge a distance $ d $ below the plane does not change the field above the plane due to the positive charge. 
	
	\item Considering both points, the potential at an arbitrary position $ \r $ from the origin is
	\begin{equation*}
		U(\r) = \frac{q}{4\pi \e}\frac{1}{\abs{\r - \vec{d}}} - \frac{q}{4\pi \e}\frac{1}{\abs{\r + \vec{d}}} 
	\end{equation*}
	where the vector $ \vec{d} $ points perpendicularly up from the plane toward the positive charge. Introducing an angle $ \theta $ between $ \vec{d} $ and $ \r $, we have
	\begin{equation*}
		\abs{\r \pm \vec{d}} = \sqrt{r^{2}+d^{2} \pm 2rd\cos \theta}
	\end{equation*}
	The expression for $ U(\r) $ for the two charges is then simply
	\begin{equation*}
		U_{2}(\r) = \frac{q}{4\pi \e}\left[\frac{1}{\sqrt{r^{2}+d^{2} \pm 2rd\cos \theta}} - \frac{1}{\sqrt{r^{2}+d^{2} \pm 2rd\cos \theta}}\right]
	\end{equation*}
	For the original configuration of a single positive charge a distance $ d $ above the plane, the potential above the plane agrees with $ U_{2} $, while the potential below the plane, where there is in reality no charge, is zero. The correct expression for the single charge $ +q $ is then
	\begin{equation*}
		U(\r) = 
		\begin{cases}
			U_{2}(\r) & \text{above the plane}\\
			0 & \text{below the plane}
		\end{cases}
	\end{equation*}
	
	\item Next, we're interested in the surface charge density $ \sigma(\rho) $ on the plane where $ \rho $ is the radial distance in the plane from the origin. As usual, we start with Gauss's law for a small surface element of the plane: 
	\begin{equation*}
		-E_{\perp}\diff S = \frac{\diff q}{\e} \implies \sigma \equiv \frac{\diff q}{\diff S} =  -\sigma E_{\perp}
	\end{equation*}
	To find $ E_{\perp} $, we differentiate $ U $ with respect to the vertical coordinate $ z $. First, we introduce $ z $ into the expression $ \abs{\r \pm \vec{d}} $
	\begin{equation*}
		\abs{\r \pm \vec{d}} = \sqrt{r^{2}+d^{2} \pm 2rd\cos \theta} = \sqrt{\rho^{2} + z^{2} + d^{2} \pm 2dz}
	\end{equation*}
	We then have
	\begin{equation*}
		U(\rho, z) = \frac{q}{4\pi \e}\left[\frac{1}{\sqrt{\rho^{2} + z^{2} + d^{2} - 2dz}} - \frac{1}{\sqrt{\rho^{2} + z^{2} + d^{2} + 2dz}}\right]
	\end{equation*}
	We then find $ E_{\perp} $ and then $ \sigma $ with
	\begin{align*}
		\sigma &= - \e E_{\perp} = -\e \pdv{U}{z}\bigg|_{z=0} = -\frac{q}{4\pi}\left[\frac{d}{(\rho^{2}+d^{2})^{3/2}} +  \frac{d}{(\rho^{2}+d^{2})^{3/2}} \right]\\
		& = -\frac{q}{2\pi} \frac{d}{(\rho^{2}+d^{2})^{3/2}}
	\end{align*}
	
	\item With surface charge density $ \sigma $ known, we then ask what is the total charge on the plane. Integrating the plane over rings with area $ \diff S = 2\pi \rho \diff \rho $, we have
	\begin{equation*}
		q_{\text{plane}} = \int \sigma \diff S = - \int_{0}^{\infty} \frac{qd}{2\pi(\rho^{2}+d^{2})^{3/2}} (2\pi \rho \diff \rho)
	\end{equation*}
	In terms of the new variable $ u = \rho^{2} + d^{2} $, the integral evaluates to
	\begin{equation*}
		q_{\text{plane}} = -\frac{qd}{2}\int_{d^{2}}^{\infty}\frac{\diff u}{u^{3/2}} = qd \left[\frac{1}{u^{1/2}}\right]_{d^{2}}^{\infty} = -q
	\end{equation*}
	
	\item Summary of what we did: recognize that the field above the plane from the positive charge looks like half the field of an electric dipole. Since we know the solution for a dipole, instead of solving the charge-plane system, we solve the (imaginary) two-charge system, which gives the same field above the plane anyway. We then reuse the upper half of the dipole solution for the single-charge plane system, and set the field below the plane equal to zero. The basic idea is: the field above the plane is the same for both the positive-charge plane system and for a dipole system, so we can use either approach to solve for the field above the plane. 
	
	\item Next, we ask what is the electrostatic force on the point charge above the plane? First, some theory:
\end{itemize}  
\textbf{Theory: Electrostatic Force}\\[2mm]
The total electrostatic force $ \vec{F} $ acting on the charges enclosed in a region of space $ V $ permeated with an electric field $ \E $ is
\begin{equation*}
	\vec{F} = \e \oint_{\partial V} \left[\E(\E \cdot \uvec{n}) - \frac{1}{2}E^{2}\uvec{n}\right]\diff S
\end{equation*}
where $ \uvec{n} $ is the normal to the surface $ \partial V $ enclosing the charges. Like with Gauss's law, a good choice of the boundary surface, usually taking advantage of the problem's symmetries, tends to simplify the problem. Alternatively, if the electric field vanishes at infinity, we choose a surface that closes at infinity. 

\begin{itemize}
	\item Back to our problem: we choose an infinite surface whose base runs along the plane, then turns upward and closes at infinity to enclose the upper half of space above the plane. The field in this case is the same $ E_{\perp} $ calculated above:
	\begin{equation*}
		E_{\perp} = \frac{q}{2\pi \e}\frac{d}{(\rho^{2}+d^{2})^{3/2}}
	\end{equation*}
	For the part of the surface running parallel to the plane, the normal to the surface $ \uvec{n} $ points perpendicularly into the plane, parallel to the electric field. The force equation for the bottom half of the surface reads 
	\begin{equation*}
		\vec{F}_{e} = \e \oint \left[E^{2}\uvec{n}- \frac{1}{2}E^{2}\uvec{n}\right]\diff S = \frac{\e}{2}\uvec{n}\int E_{\perp}^{2}\diff S
	\end{equation*}
	In fact, the contribution from the upper half of the surface is zero---the upper half extends to infinity, where the electric field vanishes. We only need to integrate over the bottom of the surface, running parallel to the plane. Writing $ \diff S = 2\pi \rho \diff \rho $ and substituting in the expression for $ E_{\perp} $, the force reads
	\begin{align*}
		\vec{F}_{e} &= \uvec{n}\frac{\e}{2}\int_{0}^{\infty} \frac{q^{2}}{4\pi^{2}\e^{2}}\frac{d^{2}}{(\rho^{2} + d^{2})^{3}} 2\pi \rho \diff \rho = \frac{q^{2}d^{2}}{4\pi \e}\uvec{n}\int_{0}^{\infty}\frac{\rho}{(\rho^{2} + d^{2})^{3}} \diff \rho\\
		&=  \frac{q^{2}d^{2}}{8\pi \e}\uvec{n}\int_{d^{2}}^{\infty}\frac{\diff u}{u^{3}} = - \frac{q^{2}d^{2}}{16\pi \e}\uvec{n} \left[\frac{1}{u^{2}}\right]_{d^{2}}^{\infty} = \frac{q^{2}}{16\pi \e d^{2}}\uvec{n}
	\end{align*}
	The force points in the direction of $ \uvec{n} $---downward into the plane. A final note: if we write
	\begin{equation*}
		\vec{F}_{e} = \frac{q^{2}}{4\pi \e (2d)^{2}}\uvec{n}
	\end{equation*}
	the force takes the form of the electric force between a positive and negative charge separated by a distance $ 2d $---the same situation we used in the method of images above. 
	
\end{itemize}

\subsection{Sixth Exercise Set}

\subsubsection{Theory: Electrostatic Force}
Recall the total electrostatic force $ \vec{F} $ acting on the charges enclosed in a region of space $ V $ permeated with an electric field $ \E $ is
\begin{equation*}
	\vec{F}_{e} = \e \oint_{\partial V}\left[\E(\E\cdot \uvec{n}) - \frac{1}{2}E^{2} \uvec{n}\right] \diff S
\end{equation*}
where $ \uvec{n} $ is the normal to the surface $ \partial V $ enclosing the charges.


\subsubsection{Force on a Conducting Spherical Shell}
\textit{We place a conducting sphere of radius $ a $ in a homogeneous electric field $ E_{0} $. Find the electrostatic force on the upper half of the sphere.}

\begin{itemize}
	\item Suppose the field points in the $ z $ direction. Recall from the previous exercise set that the potential from the sphere and electric field is
	\begin{equation*}
		U(r, \theta) = - E_{0}r \cos \theta + \frac{E_{0}a^{3}}{r^{2}}\cos \theta
	\end{equation*}
	Qualitatively, there are two main contributions to the force on the sphere: an upwards contribution in the positive $ z $ direction from the external electric field, and a downward contribution in the negative $ z $ direction from the negative charge accumulated on the bottom half of the sphere. 
	
	\item We are interested in the force on the upper half of the sphere---the next step is to choose a surface around the sphere's upper half that will simplify the force calculation. Recall the field points perpendicularly out of the conducting sphere's surface at all points. 
	
	With this perpendicular field in mind, choose a surface that tightly hugs the sphere's upper half---in this case, the field and normal to the surface $ \uvec{n} $ are parallel at all points outside the sphere. In the hemisphere plane inside the sphere, there is no field at all. These to facts simplify the dot product $ \E \cdot \uvec{n} $ in the force equation. 
	
	\item We then have $ \E \cdot \uvec{n} = E $ and $ \E( \E \cdot \uvec{n}) = E^{2}\uvec{n} $. The contribution to the force on along the sphere's outside surface is
	\begin{equation*}
		\vec{F}_{e} = \e \int \frac{1}{2} E^{2} \uvec{n} \diff S
	\end{equation*}
	The contribution from the hemisphere plane through the sphere zero, since $ \E = 0 $ inside the sphere. 
	
	\item Next, we find the magnitude $ E $ on the sphere's surface from the potential $ U(r, \theta) $. The field points radially outwards, so we differentiate $ U $ with respect to $ r $ to get
	\begin{equation*}
		E = \pdv{U}{r} \bigg |_{r = a} = E_{0} \cos \theta + 2 E_{0} \cos \theta = 3E_{0} \cos \theta
	\end{equation*}
	Inserting $ E $ into the force equation gives
	\begin{equation*}
		\vec{F}_{e} = \e \int \frac{1}{2}\big(3E_{0} \cos \theta\big)^{2} \uvec{n} \diff S
	\end{equation*}
	In spherical coordinates, the unit normal $ \uvec{n} $ to the sphere's surface is $ \uvec{n} = (\sin \theta \cos \phi, \sin \theta \sin \phi, \cos \theta) $. The surface element $ \diff S $ at the surface $ r = a $ is (just like in the previous exercise sets) $ \diff S = a^{2} \diff \phi \sin \theta \diff \theta $. The force on the sphere's upper half is then
	\begin{equation*}
		\vec{F}_{e} = \frac{9\e E_{0}^{2}}{2} \int_{\theta = 0}^{\pi/2}\int_{\phi = 0}^{2\pi} \cos^{2} \theta 
		\begin{bmatrix}
			\sin \theta \cos \phi\\
			\sin \theta \sin \phi\\
			\cos \theta
		\end{bmatrix}
		(a^{2} \sin \theta \diff \theta \diff \phi) 
	\end{equation*}
	Both the $ x $ and $ y $ components will be zero---integrating $ \cos \phi $ and $ \sin \phi $ over a full period $ 2 \pi $ give zero, while the $ \phi $ contribution to the $ z $ component is $ 2\pi $.  We make this explicit with
	\begin{equation*}
		\vec{F}_{e} = \frac{9\e E_{0}^{2}}{2} \int_{\theta = 0}^{\pi/2}\cos^{2} \theta 
		\begin{bmatrix}
			0\\
			0\\
			2\pi \cos \theta 
		\end{bmatrix}
		(a^{2} \sin \theta \diff \theta) 
	\end{equation*}
	The non-zero $ z $ component $ F_{z} $ is 
	\begin{align*}
		F_{z} &= \frac{9\e E_{0}^{2}}{2} \int_{\theta = 0}^{\pi/2}2\pi \cos^{3} \theta (a^{2} \sin \theta \diff \theta) = 9\pi\e E_{0}^{2}a^{2} \int_{\theta = 0}^{\pi/2} \cos^{3} \theta (\sin \theta \diff \theta)\\
		& =  9\pi\e E_{0}^{2}a^{2} \int_{0}^{1} \cos^{3} \theta \diff [\cos \theta] = \frac{9\pi\e a^{2} E_{0}^{2}}{4}
	\end{align*}
	The vector force can be written simply as
	\begin{equation*}
		\vec{F} = \frac{9\pi\e a^{2} E_{0}^{2}}{4} \uvec{z}
	\end{equation*}
	In other words, the force on the upper half points upward in the positive $ z $ direction.	
	
\end{itemize}

\subsubsection{Point Charge Between Two Conducting Plates}
\textit{We place two large conducting plates at a right angle to each other, so that the plates come close together but just barely do not touch. We then place a point charge $ q $ along the line bisecting the right angle between the plates, at a perpendicular distance $ a $ from each plate. Both plates are grounded. What is the electric potential in the region bounded by the plates at large distances from the plates' intersection?}
\begin{itemize}
	\item Assume $ r = 0 $ along the line connecting the two plates. For a single plate, we could solve the problem with the method of images---see the previous exercise set. With two plates we proceed analogously, with a mirror image for each plate. Because of the two reflections from the two plates, we end up with three imaginary charges plus the one original one in a quadrupole arrangement. (This is hard to describe in words, it is best to see a picture). For large $ r $, the charge arrangement will have the field of an electric quadrupole. Solving the problem thus reduces to a multipole expansion of the electric potential to the quadrupole term.  
\end{itemize}	
	
\textbf{Theory: Multipole Expansion}
\begin{itemize}

	\item The multipole expansion of $ U $ to quadrupole order, using the Einstein summation convention, is
	\begin{equation*}
		U(r) = \frac{1}{4\pi \e} \left[\frac{q}{r} + \frac{p_{i}r_{i}}{r^{3}} + \frac{\mathrm{Q}_{ij}r_{i}r_{j}}{r^{5}} \right]
	\end{equation*}
	where $ \vec{p} $ is the electric dipole moment and $ \mat{Q} $ is the quadrupole moment tensor. \textit{Note}: we could think of the charge $ q $ is a scalar monopole moment, creating a logical progression from scalar monopole moment to vector dipole moment to tensor quadrupole moment. 
	
	\item We find the dipole moment with 
	\begin{equation*}
		\vec{p} = \int  \diff^{3} \tilde{r} \rho(\tvec{r}) \tvec{r}
	\end{equation*}
	We find the quadrupole moment by components:
	\begin{equation*}
		\mathrm{Q}_{ij} = \int  \diff^{3} \tilde{r} \rho(\tvec{r}) \big( 3\tilde{r}_{i}\tilde{r}_{j} - \delta_{ij}\tilde{r}^{2} \big)
	\end{equation*}
	The discrete analog a configuration of $ N $ charges reads
	\begin{equation*}
		\mathrm{Q}_{ij} = \sum_{n=1}^{N} q_{n} \left(3r_{n_{i}}r_{n_{j}} - \delta_{ij}r_{n}^{2}\right)
	\end{equation*}
	Note that both definitions produces a symmetric tensor. Also important: the tensor's trace---the sum of the diagonal elements is zero:
	\begin{equation*}
		\tr\mat{Q} = \sum_{n} q_{n}\left[3x_{n}^{2} - r_{n}^{2} + 3z_{n}^{2} - r_{n}^{2} + 3z_{n}^{2} - r_{n}^{2} \right] = \sum_{n}q_{n}\left[3r_{n}^{2} - 3r_{n}^{2}\right] = 0
	\end{equation*}
\end{itemize}

\textbf{Back to Our Problem}
\begin{itemize}
	\item For our imaginary quadrupole configuration of four charges, the total charge, and thus the monopole moment, is zero. Analogously, the total dipole moment of the arrangement, which consists of two positive and two negative charges, is zero---the two dipoles cancel each other out. 
	
	From the three terms in our multipole expansion of $ U(r) $, only the quadrupole term remains. We just have to calculate the quadrupole tensor $ \mathrm{Q}_{ij} $. We label the four charges in the imaginary quadrupole configuration as 1, 2, 3, and 4, where 1 is the original positive charge in the upper right corner, 2 is the negative image charge in the upper left corner, 3 is the positive image charge in the lower left corner, and 4 is the negative image charge in the lower right corner. 
	
	\item Using the discrete formula for $ \mathrm{Q}_{ij} $, the first component $ \mathrm{Q}_{xx} $ is 
	\begin{align*}
		\mathrm{Q}_{xx} &= \sum_{n=1}^{N} q_{n} \left(3x_{n}^{2} - \delta_{ij}r_{n}^{2}\right) = q\left(3a^{2} - 2a^{2}\right) +  (-q)\left(3a^{2} - 2a^{2}\right)\\
		&= q\left(3a^{2} - 2a^{2}\right) +  (-q)\left(3a^{2} - 2a^{2}\right) = 0
	\end{align*}
	The other diagonal terms $ \mathrm{Q}_{yy} $ and $ \mathrm{Q}_{zz} $ will analogously sum to zero.
	
	All off-diagonal terms with a $ z $ component are also zero, since the charges lie in a plane with $ z = 0 $. We thus have
	$ \mathrm{Q}_{xz} = \mathrm{Q}_{zx} = \mathrm{Q}_{yz} = \mathrm{Q}_{zy} = 0$. We have just two terms left calculate: $ \mathrm{Q}_{xy} $  and $ \mathrm{Q}_{yx} $. By the tensor's symmetry, the two are equal, so we really only have one term:
	\begin{align*}
		\mathrm{Q}_{xy} &= \sum_{n=1}^{N} q_{n} \left(3x_{n}y_{n} - 0\cdot r_{n}^{2}\right) = 3qa^{2} + 3(-q)(-a^{2}) + 3qa^{2} + 3(-q)(-a^{2})\\
		 &= 12qa^{2}
	\end{align*}
	The quadrupole tensor is
	\begin{equation*}
		\mat{Q} =
		\begin{bmatrix}
			0 & 12q a^{2} & 0\\
			12q a^{2} & 0 & 0\\
			0 & 0 & 0
		\end{bmatrix}
	\end{equation*}
	As expected, the tensor is symmetric with trace $ \tr \mat{Q} = 0 $. 
	
	\item Recall the quadrupole expansion of $ U(r) $:
	\begin{equation*}
		U(r) = \frac{1}{4\pi \e} \left[\frac{q}{r} + \frac{p_{i}r_{i}}{r^{3}} + \frac{\mathrm{Q}_{ij}r_{i}r_{j}}{r^{5}} \right]
	\end{equation*}
	In our case, with $ q = 0 $ and $ \vec{p} = 0 $, we have
	\begin{align*}
		U(r) &= \frac{1}{4\pi \e} \left[\frac{12qa^{2}xy}{r^{5}} + \frac{12qa^{2}yx}{r^{5}} + 0 + \cdots + 0\right] = \frac{6qa^{2}}{\pi \e} \frac{xy}{r^{5}}\\
		&=\frac{6qa^{2}}{\pi \e} \frac{\cos \phi \sin \phi \sin^{2}\theta}{r^{3}}
	\end{align*}
	The second line uses the spherical coordinates $ x = r\cos \phi \sin \theta $ and $ y = r\sin \phi \sin \theta $. In fact, the expression for $ U(r) $ takes the exact same form as the wave function of a $ d $ electron orbital (angular momentum quantum number $ l = 2 $) in a hydrogen atom. The equipotential surfaces of $ U(r) $ have the same spatial distribution as the $ d_{xy} $  orbitals for a hydrogen wave function. 
	
\end{itemize}

\newpage

\section{Magnetostatics}

\subsection{Seventh Exercise Set}

\subsubsection{Theory: Magnetic Vector Potential and the Biot-Savart Law}
\begin{itemize}
	\item We will need to use two more Maxwell equations for magnetostatics. The first is
	\begin{equation*}
		\div \B = 0
	\end{equation*}
	This equation rules out the possibility of magnetic monopoles and allows $ \B $ to be written as the curl of a vector potential $ \\A $ as
	\begin{equation*}
		\B = \curl \A
	\end{equation*}
	The second Maxwell equation is 
	\begin{equation*}
		\curl \B = \mu_{0}\left(\vec{j} + \e \pdv{\E}{t}\right)
	\end{equation*}
	For static situations with a constant electric field this simplifies to
	\begin{equation*}
		\curl \B = \mu_{0}\vec{j}
	\end{equation*}
	Substituting the expression $ \B = \curl \A $ into the static second Maxwell equation produces
	\begin{equation*}
		\curl \big(\curl \A\big) = \div \big(\div \A\big) - \laplacian \A = \mu_{0} \vec{j}
	\end{equation*}
	
	\item The magnetic vector potential is defined only up to a constant; we usually choose $ \A $ so that $ \div \A = 0 $. In this convention, we have
	\begin{equation*}
		\laplacian \A = - \mu_{0}\vec{j}
	\end{equation*}
	where $ \vec{j} $ is the current density vector. This equation is a vector analog of the Poisson equation $ \laplacian U = - \frac{\rho}{\e} $ from electrostatics. Similarly to how the electrostatic potential $ U $ at a point $ \r $ in region of space with charge density $ \rho $ is found with
	\begin{equation*}
		U = \frac{1}{4 \pi \e} \int \frac{\rho(\tvec{r})\diff^{3}\tilde{r}}{\abs{\r - \tvec{r}}}
	\end{equation*}
	The magnetic potential at a point $ \r $ in region of space with current density $ \vec{j} $ is
	\begin{equation*}
		\A(\r) = \frac{\mu_{0}}{4 \pi} \int \frac{\vec{j}(\tvec{r})\diff^{3}\tilde{r}}{\abs{\r - \tvec{r}}}
	\end{equation*}
	
	\item In problems involving one-dimensional conductors, where $ \vec{j} $ is non-zero only along the conductor, the expression for $ \vec{j}(\tvec{r})\diff^{3}\tilde{r} $ simplifies to
	\begin{equation*}
		\vec{j}(\tvec{r})\diff^{3}\tilde{r} = \vec{j}(\tvec{r}) \diff \tilde{S} \diff \tilde{l} = I \uvec{t} \diff \tilde{l}
	\end{equation*}
	where $ \uvec{t} $ is the unit normal vector tangent to the conductor, $ I $ is the current through the conductor at the point $ \tvec{r} $ and $ \diff \tilde{l} $ is a small distance element along the conductor's length. The magnetic vector potential for a one-dimensional conductor carrying a current $ I $ then simplifies to
	\begin{equation*}
		\A(\r) = \frac{\mu_{0}I}{4\pi} \int \frac{\uvec{t} \diff l}{\abs{\r - \tvec{r}}}
	\end{equation*}
	
	\item Recall $ \B = \curl \A $. Taking the curl of the general expression for $ \A $ in terms of current density $ \vec{j} $ gives general expression for the magnetic field 
	\begin{equation*}
		\B (\r) = \frac{\mu_{0}}{4\pi}\int \frac{\vec{j}(\tvec{r}) \cross (\r - \tvec{r})}{\abs{\r - \tvec{r}}^{3}} \diff^{3}\tilde{r}
	\end{equation*}
	This is a general form of the Biot-Savart law for the magnetic field $ \B $ of a current distribution.
	
\end{itemize}

\subsubsection{Magnetic Field of a Circular Current Loop}
\textit{A closed conducting loop of radius $ a $ carries current $ I $. What is the magnetic vector potential far from the conducting loop?}
\begin{itemize}
	\item Our starting point is the vector potential of a one-dimensional conductor from the theory section, i.e.
	\begin{equation*}
		\A(\r) = \frac{\mu_{0}I}{4\pi} \int \frac{\uvec{t} \diff l}{\abs{\r - \tvec{r}}}
	\end{equation*}
	We need expressions for $ \r $, $ \tvec{r} $ and $ \uvec{t} $.
	
	Assume the the loop lies in the $ x, y $ plane with the $ z $ axis normal to the loop. For mathematical convenience, we rotate the $ x, y $ plane so $ \r $ lies in the $ x, z $ plane (i.e. $ \phi = 0 $); this just gives us one less non-zero component to work with, since the $ \sin \phi $ term in the $ y $ component of $ \r $ is zero. In polar coordinates $ \r $ reads
	\begin{equation*}
		\r = \big (r \sin \theta, 0, r \cos \theta \big )
	\end{equation*}
	where $ \theta $ is the angle between $ \r $ and the $ z $ axis. 
	
	The integration variable $ \tvec{r} $, which runs over the current loop in the $ x, y $ plane, reads
	\begin{equation*}
		\tvec{r} = \big(a \cos \tilde{\phi}, a \sin \tilde{\phi}, 0 \big)
	\end{equation*}
	Where $ \tilde{\phi} $ is the azimuthal angle between $ \tvec{r} $ and the $ x $ axis. 
	
	Finally, the expression for $ \uvec{t} $, the tangent to the current loop, is
	\begin{equation*}
		\uvec{t} = \big( -\sin \tilde{\phi},  \cos \tilde{\phi}, 0\big)
	\end{equation*}
	The small distance element is $ \diff \tilde{l} = a \tilde{\phi}$ .
	
%	\item We can now put the pieces together in the equation for $ \A $. First,
	\begin{align*}
		\abs{\r - \tvec{r}} &= \sqrt{(r \sin \theta - a \cos \tilde{\phi} )^{2} + a^{2}\sin^{2}\tilde{\phi} + r^{2}\cos^{2} \theta}\\
		& = \sqrt{r^{2} + a^{2} - 2ra \sin \theta \cos \tilde \phi}\\
		&\approx \sqrt{r^{2} - 2ra \sin \theta \cos \tilde \phi} = r \sqrt{1 - \frac{2a}{r}\sin \theta \cos \tilde{\phi}}
	\end{align*}
	where the last lines uses $ r \gg a $ (recall we're interested in the solution far from the conducting loop). We now have, again using $ a \ll r \implies \frac{a}{r} \ll 1 $,
	\begin{equation*}
		\frac{1}{\abs{\r - \tvec{r}}} = \frac{1}{r} \left(1 - \frac{2a}{r}\sin \theta \cos \tilde{\phi}\right)^{-1/2} \approx \frac{1}{r}\left(1 + \frac{1}{2}  \frac{2a}{r}\sin \theta \cos \tilde{\phi}\right)
	\end{equation*}
	Substituting the expressions for $ \frac{1}{\abs{\r - \tvec{r}}} $, $ \uvec{t} $ and $ \diff \tilde{l} $ into expression for $ \A $ gives
	\begin{align*}
		\A(\r) &= \frac{\mu_{0}I}{4\pi} \int \frac{\uvec{t} \diff l}{\abs{\r - \tvec{r}}} = \frac{\mu_{0}I}{4\pi} \frac{a}{r} \int_{0}^{2\pi} \diff \tilde{\phi}  
		\begin{bmatrix}
			- \sin \tilde{\phi}\\
			\cos \tilde{\phi}\\
			0
		\end{bmatrix}
		\left(1 + \frac{1}{2}  \frac{2a}{r}\sin \theta \cos \tilde{\phi}\right)
	\end{align*}
	
	\item The integrals conveniently simplify, since we are integrating sinusoidal terms over an entire period. Only the integral of $ \cos^{2} \tilde{\phi} $ in the $ y $ component is nonzero. We end up with $ A_{x} = A_{z} = 0 $ and
	\begin{equation*}
		A_{y}(r) = \frac{\mu_{0}I}{4\pi} \frac{a}{r} \int_{0}^{2\pi}
		 \frac{a\sin \theta}{r} \cos^{2} \tilde{\phi} \diff \tilde{\phi} = \frac{\mu_{0}I}{4} \frac{a^{2}}{r^{2}}  \sin \theta 
	\end{equation*}
	or, in vector form,
	\begin{equation*}
		\A(\r) = \frac{\mu_{0}I}{4} \frac{a^{2}}{r^{2}}  \sin \theta \uvec{y} = \frac{\mu_{0}I}{4} \frac{a^{2}}{r^{2}} \uvec{z} \cross \uvec{r}
	\end{equation*}
	The second expression is preferable: the first, in terms of $ \uvec{y} $, holds only with $ x, y $ plane rotated so $ \phi = 0 $ and $ \r $ lies in the $ x, z $ plane. The second, which uses $ \sin \theta \uvec{y} = \uvec{z} \cross \uvec{r} $, holds for any orientation of the $ x, y $ plane. 
	
	
	Finally, using $ I \pi a^{2} $ = $ \abs{\vec{m}} $ (where $ \m $ is the loop's magnetic dipole moment, which points in the direction of the loop's normal), the expression for $ \A $ simplifies to
	\begin{equation*}
		\A(\r) = \frac{\mu_{0} \m \cross \uvec{r}}{4\pi r^{2}} = \frac{\mu_{0} \m \cross \r}{4\pi r^{3}}
	\end{equation*}
	This is the same form of magnetic vector potential as for a magnetic dipole. In other words, a circular current loop behaves as magnetic dipole at long distances.
	
\end{itemize}

\subsubsection{Magnetic Field of a Rotating Charged Disk}
\textit{Consider a charged, insulating disk with uniform surface charge density $ \sigma $ and radius $ a $. The disk rotates uniformly about an axis through its center with constant angular speed $ \omega $. Find the magnetic field along the axis of rotation.}
\begin{itemize}
	\item We choose a coordinate system so the disk lies in the $ x, y $ plane and the rotation axis coincides with the $ z $ axis, so $ \bm{\omega} = (0, 0, \omega) $. Start with the general Biot-Savart law
	\begin{equation*}
		\B (\r) = \frac{\mu_{0}}{4\pi}\int \frac{\vec{j}(\tvec{r}) \cross (\r - \tvec{r})}{\abs{\r - \tvec{r}}^{3}} \diff^{3}\tilde{r}
	\end{equation*}
	and, like in the previous problem, find expressions for each vector quantity in the equation. The expression for $ \r $ along the $ z $ axis is simply $ \r = (0, 0, z) $, while the expression for $ \tvec{r} $, which lies in the disk in the $ x, y $ plane, is
	\begin{equation*}
		\tvec{r} = \big(\tilde{r} \cos \tilde{\phi}, \tilde{r} \sin \tilde{\phi}, 0 \big)
	\end{equation*}
	The difference of $ \r $ and $ \tvec{r} $ and its magnitude is
	\begin{equation*}
		\r - \tvec{r} = \big(-\tilde{r} \cos \tilde{\phi}, - \tilde{r} \sin \tilde{\phi}, z \big) \eqtext{and} \abs{\r - \tvec{r}} = \sqrt{\tilde{r}^{2} + z^{2}}
	\end{equation*}
	Finally, the current density $ \vec{j} $, which points tangent to the disk's rotation, is
	\begin{equation*}
		\vec{j} = j(-\sin \t{\phi}, \cos \t{\phi},0 )
	\end{equation*}
%	The volume element $ \diff^{3}\tilde{r} $ is $ \diff^{3}\tilde{r} = \t{r} \diff \t{\phi} \diff \t{r} \diff h$ where $ \diff h $ is the disk's thickness. 
	
	\item Using the expressions for our vector quantities, the cross product $ \vec{j}(\tvec{r}) \cross (\r - \tvec{r}) $ is
	\begin{equation*}
		\vec{j}(\tvec{r}) \cross (\r - \tvec{r}) = j
		\begin{bmatrix}
			-\sin \t{\phi}\\
			\cos \t{\phi}\\
			0
		\end{bmatrix}
		\begin{bmatrix}
			-\tilde{r} \cos \tilde{\phi}\\
			- \tilde{r} \sin \tilde{\phi}\\
			z
		\end{bmatrix}
		= j
		\begin{bmatrix}
			z \cos \t{\phi}\\
			z \sin \t{\phi}\\
			\t{r}
		\end{bmatrix}
	\end{equation*}
	Substituting $ \vec{j}(\tvec{r}) \cross (\r - \tvec{r}) $ and $  \abs{\r - \tvec{r}} $ into the Biot-Savart law gives
	\begin{equation*}
		\B(z) = \frac{\mu_{0}}{4 \pi} \int \frac{j \diff^{3} \t{r}}{\big(\tilde{r}^{2} + z^{2}\big)^{3/2}} 
		\begin{bmatrix}
			z \cos \t{\phi}\\
			z \sin \t{\phi}\\
			\t{r}
		\end{bmatrix}
	\end{equation*}
	Next, we write $ j \diff^{3}\t{r} = j \diff \t{S} \diff \t{l} = j \diff \t{S} (\t{r} \diff \t{\phi} ) = \diff I \t{r} \diff \t{\phi} $. Note that the product $ j \diff \t{S} $ is the current element $ \diff I $ in the surface element $ \diff \tilde{S} $. The current $ \diff I $ at the radius $ \t{r} $ on the disk rotating with period $ t_{0} = \frac{2\pi}{\omega} $ is
	\begin{equation*}
		\diff I = \frac{\diff q}{t_{0}} = \frac{\diff q}{2\pi} \omega  = \frac{(\sigma 2\pi \t{r}\diff \t r)}{2\pi} \omega = \sigma \omega \t{r} \diff \t{r}
	\end{equation*}
	Substituting $ j \diff^{3}\t{r} =  \diff I \t{r} \diff \t{\phi}  =  \sigma \omega \t{r}^{2} \diff \t{r} \diff \t{\phi}  $ into the Biot-Savart law gives
	\begin{equation*}
		\B(z) = \frac{\mu_{0}}{4 \pi} \int_{0}^{a} \diff \t{r} \int_{0}^{2\pi} \diff \t{\phi}	\frac{\sigma \omega \t{r}^{2}}{\big(\tilde{r}^{2} + z^{2}\big)^{3/2}} 
		\begin{bmatrix}
			z \cos \t{\phi}\\
			z \sin \t{\phi}\\
			\t{r}
		\end{bmatrix}
	\end{equation*}
	The first and second components of $ \B $ contain integrals $ \cos $ and $ \sin $ terms over a full period---the result is zero. After integrating over $ \t{\phi} $, the magnetic field simplifies to
	\begin{equation*}
		\B(z) = \frac{\mu_{0}}{4 \pi} \int_{0}^{a} \diff \t{r} \frac{\sigma \omega \t{r}^{2}}{\big(\tilde{r}^{2} + z^{2}\big)^{3/2}} 
		\begin{bmatrix}
			0\\
			0\\
			2\pi \t{r}
		\end{bmatrix}
	\end{equation*}
	Only the $ z $ component of $ \B $ is non-zero; it is
	\begin{equation*}
		B_{z}(z) = \frac{\mu_{0}}{2} \int_{0}^{a} \diff \t{r} \frac{\sigma \omega \t{r}^{3}}{\big(\tilde{r}^{2} + z^{2}\big)^{3/2}} 
	\end{equation*}
	In terms of the new variable $ u = \tilde{r}^{2} + z^{2} $, the integral evaluates to
	\begin{align*}
		B_{z}(z) &= \frac{\mu_{0}}{2} \frac{\sigma \omega}{2}\int_{z^{2}}^{z^{2} + a^{2}}\frac{u-z^{2}}{u^{3/2}} \diff u = \frac{\mu_{0}\sigma \omega}{4} \left[2u^{1/2} + 2z^{2}u^{-1/2}\right]_{a^{2}}^{z^{2}+a^{2}}\\
		 & = \frac{\mu_{0}\sigma \omega}{2} \left(\sqrt{z^{2} + a^{2}} - z + \frac{z^{2}}{\sqrt{z^{2} + a^{2}}} - z\right)\\
		 & = \frac{\mu_{0}\sigma \omega }{2}\left(\frac{2z^{2} + a^{2}}{\sqrt{z^{2} + a^{2}}} - 2z\right)
	\end{align*}
	The magnetic field along the $ z $ axis is thus $ \B = (0, 0, B_{z}) $ with $ B_{z} $ as above.
	
	\item Next, we consider the limit case $ z \gg a $, far from the rotating disk. Expanding the square root to fourth order in the small quantity $ \frac{a}{z} $, multiplying out and simplifying like terms gives
	\begin{align*}
		B_{z} &= \frac{\mu_{0}\sigma \omega}{2} \left(\frac{2z^{2} + a^{2}}{z\sqrt{1 + \frac{a^{2}}{z^{2}}}} - 2z\right) \approx \frac{\mu_{0}\sigma \omega}{2} \left[\left(2z + \frac{a^{2}}{z}\right)\left(1 - \frac{a^{2}}{2z^{2}} + \frac{3}{8}\frac{a^{4}}{z^{4}}\right) - 2z \right]\\
		& = \frac{\mu_{0}\sigma \omega}{2} \left[2z + \frac{a^{2}}{z} - \frac{a^{2}}{z} - \frac{1}{2}\frac{a^{4}}{z^{3}} + \frac{3}{4}\frac{a^{4}}{z^{3}} + \frac{3}{8}\frac{a^{6}}{z^{5}} - 2z\right] = \frac{\mu_{0}\sigma \omega}{2} \left[ \frac{1}{4}\frac{a^{4}}{z^{3}} + \frac{3}{8}\frac{a^{6}}{z^{5}}\right] 
	\end{align*}
	Neglecting the highest-order $ \frac{a^{6}}{z^{5}} $ term gives the simple result
	\begin{equation*}
		B_{z} \approx  \frac{\mu_{0}\sigma \omega}{8} \frac{a^{4}}{z^{3}}, \qquad z \gg a
	\end{equation*}
	In other words, far from the disk, the magnetic field falls off as $ z^{-3} $, just like the field of a magnetic dipole.
	
	\item Next, we will try to write the magnetic field in the form $ B_{z} \propto \frac{\abs{\m}}{z^{3}} $ where $ \m $ is the disk's magnetic dipole moment. Integrating over concentric rings with area $ S $ carrying current $ \diff I $, the disk's magnetic dipole moment is
	\begin{equation*}
		\abs{\m} = \int S \diff I = \int_{0}^{a} (\pi \t{r}^{2}) \cdot (\sigma \omega \t{r}\diff \t{r}) = \pi \sigma \omega \int_{0}^{a} \t{r}^{3} \diff \t{r} = \frac{\pi}{4}\sigma \omega a^{4}
	\end{equation*}
	Comparing this expression for $ \abs{\m} $ to the similar expression for $ B_{z} $ leads to 
	\begin{equation*}
		B_{z} \approx  \frac{\mu_{0}\sigma \omega}{8} \frac{a^{4}}{z^{3}} = \frac{\mu_{0}\abs{\m}}{2\pi z^{3}}
	\end{equation*}	
	which is in the desired form $  B_{z} \propto \frac{\abs{\m}}{z^{3}} $. The general form for the magnetic field of a magnetic dipole is
	\begin{equation*}
		\B(\r) = \frac{\mu_{0}}{4\pi}\frac{3(\m \cdot \r)\r - \m r^{2}}{r^{5}}
	\end{equation*}
	In fact, this expression is equivalent to our result $ B_{z}(z) = \frac{\mu_{0}\abs{\m}}{2\pi z^{3}} $. Since $ \m $ and $ \r $ both point along the $ z $ axis, their dot product is $ \m \cdot \r = \abs{m}r $. Along the $ z $ axis, $ \r = (0, 0, z) $ and the general expression for the dipole magnetic field simplifies to
	\begin{equation*}
		\B(\r) = \frac{\mu_{0}}{4\pi} \frac{3\abs{m}z^{2} - \abs{m}z^{2}}{z^{5}}\uvec{z} = \frac{\mu_{0}\abs{\m}}{2\pi z^{3}} \uvec{z},
	\end{equation*}
	in agreement with our expression for $ B_{z} $ for $ z \gg a $.
	
\end{itemize}

\textbf{Theory: Magnetic Force}\vspace{2mm}

The magnetic force $ \vec{F} $ on the matter contained in the region of space enclosed in the region $ V $ and permeated by the magnetic field $ \B $ is
\begin{equation*}
	\vec{F} = \frac{1}{\mu_{0}}\oint_{\partial V}\big[\B(\B \cdot \uvec{n}) - \frac{1}{2}\B^{2}\uvec{n}\big] \diff S
\end{equation*}
where $ \uvec{n} $ is the normal vector to the region's boundary $ \partial V $.

\iffalse
\subsubsection{Magnetic Force in a Coaxial Cable}
\textit{A long coaxial cable consists of a thin inner wire and outer sheath with radius $ a $. The inner wire carries a current $ I $, and we connect the sheath to the inner wire at the cable's ends so that the current $ I $ returns along the outer sheath in the opposite direction as the current along the inner wire. Find the magnetic force per unit length on the outer sheath.}

\begin{itemize}
	\item There are two contributions to the magnetic force on the sheath: the repulsive, radial force between the sheath and the inner wire and an attractive ``surface tension'' force on sheath's surface, which carries uniformly distributed current $ I $. 
	
\end{itemize}
\fi 

\end{document}





