\documentclass[11pt, a4paper]{article}
\usepackage{mwe}
\usepackage{amsmath}
\usepackage{amssymb}
\usepackage{mathtools}
\usepackage{graphicx}
\usepackage{xcolor}
\usepackage{bm} % for bold vectors in math mode
\usepackage{physics} % for differential notation, etc...
\usepackage[separate-uncertainty=true]{siunitx}

\usepackage{graphicx}  % for figures
\graphicspath{{"figures/"}} 
\usepackage{wrapfig}


\usepackage[normalem]{ulem}  % for underline with line wrapping
\usepackage[margin=3cm]{geometry}
\usepackage[colorlinks = true, allcolors=blue]{hyperref}

\setlength{\parindent}{0pt} % to stop indenting new paragraphs
\newcommand{\diff}{\mathop{}\!\mathrm{d}} % differential
\newcommand{\eqtext}[1]{\qquad \text{#1} \qquad}

\renewcommand{\vec}[1]{\bm{#1}} % for vectors
\newcommand{\uvec}[1]{\hat{\vec{#1}}} % for vectors
\newcommand{\mat}[1]{\mathbf{#1}} % for matrices
\newcommand{\dvec}[1]{\dot{\vec{#1}}} % for dotted vector quantity
\newcommand{\tvec}[1]{\tilde{\vec{#1}}} % for tilde vector quantities
\renewcommand{\t}[1]{\tilde{#1}} % shorthand for tilde

\renewcommand{\r}{\vec{r}}
\newcommand{\rh}{\vec{\rho}}
\newcommand{\E}{\vec{E}}  % for electric field
\newcommand{\D}{\vec{D}}  % for electric D-field
\newcommand{\B}{\vec{B}}  % for magnetic field
\renewcommand{\H}{\vec{H}}  % for magnetic H-field
\newcommand{\A}{\vec{A}}  % for magnetic vector potential
\renewcommand{\S}{\mathbf{S}}  % Poynting vector
\renewcommand{\SS}{\mathrm{S}}  % Poynting vector
\newcommand{\e}{\epsilon}
\newcommand{\ee}{\epsilon_{0}}  % vacuum permittivity
\newcommand{\eee}{\bm{\chi}}  % permittivity tensor
\newcommand{\mm}{\mu_{0}}  % vacuum permeability (\m used for magnetic dipole moment)
\newcommand{\pe}{\vec{p}}  % electric dipole moment
\newcommand{\m}{\vec{m}}  % magnetic dipole moment
\renewcommand{\P}{\vec{P}}  % electric polarization

\newcommand{\Poy}{Poynting\xspace}

\renewcommand{\div}{\nabla \cdot}
\renewcommand{\curl}{\nabla \cross}
\renewcommand{\grad}{\nabla}
\renewcommand{\laplacian}{\nabla^{2}}

\begin{document}
\title{Electromagnetism Exercise Notes}
\author{Elijan Mastnak}
\date{Winter Semester 2020-2021}
\maketitle

\begin{center}
\textbf{About These Notes}
\end{center}
These are my notes are from the Exercises portion of the class \textit{Elektromagnetno Polje} (Electromagnetic Field), given to third-year physics students at the Faculty of Math and Physics in Ljubljana, Slovenia. The exact problem sets herein are specific to the physics program at the University of Ljubljana, but the content is fairly standard for a late-undergraduate electromagnetism course. I am making the notes publicly available in the hope that they might help others learning the same material.

\vspace{2mm}
\textit{Navigation}: For easier document navigation, the table of contents is ``clickable'', meaning you can jump directly to a section by clicking the colored section names in the table of contents. Unfortunately, \textit{the clickable links do not work in most online or mobile PDF viewers}; you have to download the file first.

\vspace{2mm}
\textit{On Authorship:} 
The exercises are led by Asst. Prof. Martin Klanj\v{s}ek, who has curated the problem sets and guides us through the solutions. Accordingly, credit for the problems in these notes goes to Prof. Klanj\v{s}ek. I take credit for nothing beyond typesetting the notes, translating to English, and perhaps an additional explanation or two where I saw fit. The original exercise in Slovene (without solutions) can be found on the \href{https://www-f5.ijs.si/emp-2020-2021.html}{\underline{course website}}. 

\vspace{2mm}
\textit{Disclaimer:} Mistakes---both trivial typos and legitimate errors---are likely. Keep in mind that these are the notes of an undergraduate student in the process of learning the material himself---take what you read with a grain of salt. If you find mistakes and feel like telling me, by \href{https://github.com/ejmastnak/fmf}{\underline{GitHub}} pull request, \href{mailto:ejmastnak@gmail.com}{\underline{email}} or some other means, I'll be happy to hear from you, even for the most trivial of errors.

\newpage

\tableofcontents

\newpage

\section{Electrostatics}

\subsection{First Exercise Set}

\subsubsection{Background Theory: Electric Field of a Charge Distribution}
The electric field of a spatial charge distribution with volume charge density $ \rho(\r) $ is
\begin{equation*}
	\vec{E}(\vec{r}) = \frac{1}{4\pi \ee}\int \frac{\rho(\tvec{r}) \diff^{3}\tvec{r} }{\abs{\vec{r} - \tvec{r}}^{2}} \frac{\vec{r} - \tvec{r}}{\abs{\vec{r} - \tvec{r}}}
\end{equation*}
where $ \tvec{r} $ is a dummy variable for integration. Think of $ \rho(\tvec{r}) \diff^{3}\tvec{r} $ as an infinitesimal element of charge at the position $ \tvec{r} $, while $ \vec{r} - \tvec{r} $ represents the position of each charge. The term $ \frac{\vec{r} - \tvec{r}}{\abs{\vec{r} - \tvec{r}}} $ just gives the direction of each charge element.

\subsubsection{Charged Disk}
\textit{A charged disk has surface charge density $ \sigma $ and radius $ a $. What is the disk's electric field $ E(z) $ along an axis through the disk's center and normal to the disk? Investigate the limit behavior of $ E(z) $ for small and large $ z $.}
\begin{itemize}
	\item Break the disk into infinitesimal concentric rings and integrate over the contributions $ \diff E $ of each ring; $ r $ and $ \diff r $ represent the radius and thickness of a ring, respectively.
	
	First break each ring into infinitesimal segments with area $ r \diff \phi \diff r $. Along the perpendicular $ z $ axis through the disk's center each, segment contributes
	\begin{equation*}
		\diff E_{1} = \frac{1}{4\pi \ee} \frac{\sigma r\diff \phi \diff r }{z^{2} + r^{2}}
	\end{equation*}
	Notice the term $ r\diff \phi \diff r $ must have units of area to produce a charge when multiplied by the surface charge density $ \sigma $. The term $ z^{2} + r^{2} $ is just the square of the distance $ \sqrt{z^{2} + r^{2}} $ from the charge element to the $ z $ axis.
	
	\item Recognize the circular symmetry of each ring: both the $ x $ and $ y $ components of the electric field symmetrically cancel along the $ z $ axis, so the electric field only has a $ z $ component. Relate the magnitudes of $ \diff E $ and $ \diff E_{1} $ along the $ z $ axis with similar triangles to get the contribution $ \diff E $ of a ring of radius $ r $ at the point $ (0, 0, z) $:
	\begin{equation*}
		\frac{\diff E}{\diff E_{1}} = \frac{z}{\sqrt{z^{2} + r^{2}}} \implies  \diff E = \frac{z}{(z^{2} + r^{2})^{3/2}} \frac{\sigma r\diff \phi \diff r }{4\pi \ee} 
	\end{equation*}
	
	\item Integrate over the contributions $ \diff E $ to find $ E(z) $:
	\begin{equation*}
		E(z) = \int \diff E = \int_{0}^{2\pi}\int_{0}^{a}\diff \phi \diff r \frac{\sigma z r}{4\pi \ee (z^{2} + r^{2})^{3/2}} = \frac{\sigma z}{2\ee} \int_{0}^{a}\frac{r \diff r}{(z^{2} + r^{2})^{3/2}}
	\end{equation*}
	Solve the integral with the substitution $ u = z^{2} + r^{2} $, $\diff u = 2r \diff r $:
	\begin{align*}
		E(z) &= \int_{z^{2}}^{z^{2} + a^{2}} \frac{\diff u}{u^{3/2}} = -\frac{\sigma z}{2 \ee} \left(\frac{1}{\sqrt{z^{2} + a^{2}}} - \frac{1}{\sqrt{z^{2}}}\right)\\
		&=\frac{\sigma}{2\ee} \left(1 - \frac{z}{\sqrt{z^{2} + a^{2}}}\right) = \frac{\sigma}{2\ee} \left(1 - \frac{1}{1 + \frac{a^{2}}{z^{2}}}\right)
	\end{align*}
	
	\item For $ z \ll a $ (very close to the disk), $ 1 + \frac{a^{2}}{z^{2}} \to \infty $ and $ \frac{1}{1 + \frac{a^{2}}{z^{2}}} \to 0$, leaving \vspace{-2mm}
	\begin{equation*}
		E(z) \to \frac{\sigma}{2\ee} \qquad (z \ll a)
	\end{equation*}
	which is the electric field of an infinite charged plane.
	
	\item For $ z \gg a $ (very far from the disk), $ \frac{a^{2}}{z^{2}} \ll 1 $, and we use the Taylor approximation $ (1 + x)^{p} \approx 1 + px $ for $ x \ll 1 $:
	\begin{equation*}
		E(z) = \frac{\sigma}{2\ee} \left[1 -  \left (1 + \frac{a^{2}}{z^{2}}\right )^{-1/2}\right] \approx \frac{\sigma}{2\ee}\left[ 1 - \left(1 - \frac{a^{2}}{2z^{2}}\right) \right] = \frac{\sigma a^{2}}{4\ee z^{2}}
	\end{equation*}
	Multiply above and below by $ \pi $ to match this to the expression for a point charge:
	\begin{equation*}
		E(z) = \frac{\pi a^{2} \sigma}{4\pi \ee z^{2}} = \frac{\sigma S}{4\pi \ee z^{2}} = \frac{q}{4\pi \ee z^{2}}  \qquad (z \gg a) 
	\end{equation*}
	where $ q = \sigma S $ is the disk's charge. 

\end{itemize}


\subsubsection{Charged Plate with a Slit}
\textit{We take an infinitely large charged rectangular plate with surface charged density $ \sigma $ and remove a slit of width $ a $ from the plate. Determine the electric field $ E $ in the plane perpendicular to the plate and passing through the center of the slit as a function of the vertical distance $ z $ from the plate. Investigate the limit behavior of $ E(z) $ for small and large $ z $.}
\begin{itemize}
	\item Break the plate into thin ribbons and integrate over the contributions $ \diff E $ of each ribbon; $ r $ and $ \diff r $ represent the orthogonal distance from the slit and the thickness of each ribbon, respectively.
	
	\item Find the electric field of a ribbon a distance $ r $ from the slit along the $ z $ axis using Gauss's law with a cylindrical surface. For a cylinder of radius $ R $ and length $ l $, Gauss's law reads
	\begin{equation*}
		\oint \vec{E} \cdot \diff \vec{S} = 2\pi R l E = \frac{q_{\text{enc}}}{\ee} \implies E(R) = \frac{q_{\text{enc}}}{2\pi \ee l R}
	\end{equation*}
	Applied to the ribbon, the enclosed charged $ q_{\text{enc}} $ is the ribbon's infinitesimal charge $ \diff q_{1} = \sigma \diff S_{1} = \sigma l \diff r $, while the cylinder's radius $ R $ is the distance from the ribbon to the $ z $ axis: $ R = \sqrt{z^{2} + r^{2}} $, so the contribution $ \diff E_{1} $ of one ribbon is
	\begin{equation*}
		\diff E_{1} = \frac{\diff q_{1}}{2\pi \ee l \sqrt{z^{2} + r^{2}}} = \frac{\sigma l \diff r}{2\pi \ee l\sqrt{z^{2} + r^{2}} } = \frac{\sigma \diff r}{2\pi \ee \sqrt{z^{2} + r^{2}} }
	\end{equation*}
	
	\item Because of mirror-image symmetry, both the $ x $ and $ y $ components of the electric field cancel, leaving only the $ z $ component $ \diff E_{z} $. Relate $ \diff E_{z} $ and $ \diff E_{1} $ using similar triangles
	\begin{equation*}
		\frac{\diff E_{z}}{\diff E_{1}} = \frac{z}{\sqrt{z^{2} + r^{2}}} \implies  \diff E_{z} = \frac{\sigma z \diff r}{2\pi \ee (z^{2} + r^{2})}
	\end{equation*}
	
	\item Find the total electric field along the $ z $ axis by integrating over the contributions $ \diff E_{z} $ of all the ribbons. Because of mirror symmetry, you can calculate only the contribution of e.g. the right plane and multiple the result by two.
	\begin{align*}
		E(z) &= \int \diff E_{z} = 2 \int_{a/2}^{\infty} \left(\frac{\sigma z \diff r}{2\pi \ee (z^{2} + r^{2})}\right) = \frac{\sigma z}{\pi \ee} \int_{a/2}^{\infty} \frac{\diff r}{z^{2} + r^{2}}\\
		&= \frac{\sigma z}{\pi \ee} \left[\frac{1}{z}\arctan \frac{r}{z}\right]_{a/2}^{\infty} = \frac{\sigma}{\pi \ee} \left[\frac{\pi}{2} - \arctan(\frac{a}{2z})\right]
	\end{align*}
	
	\item In the limit $ z \gg a $ (very far from the slit), $ \arctan\frac{a}{2z} \to 0 $ and the electric field along the $ z $ axis is 
	\begin{equation*}
		E(z) = \frac{\sigma}{2 \ee} \qquad (z \gg a)
	\end{equation*}
	which is the field of an infinite sheet of charge.
	
	\item In the limit $ z \ll a $ (very close to the slit), we have $ \frac{a}{2z} \to \infty $. Use the asymptotic expansion $ \arctan x \approx \frac{\pi}{2} - \frac{1}{x} $ for large $ x $ to get
	\begin{equation*}
		E(z) \approx \frac{\sigma}{\pi \ee} \left[\frac{\pi}{2} - \left(\frac{\pi}{2} - \frac{2z}{a}\right)\right] = \frac{2\sigma}{\pi \ee} \frac{z}{a}
	\end{equation*}
	In this case the electric field scales linearly as $ E \sim z $. 
\end{itemize} 


\subsection{Second Exercise Set}
\subsubsection{Theory: The Poisson Equation and the Fourier Transform}

\begin{itemize}
	\item Start with Gauss's law in differential form and the relationship between $ \E $ and $ U $
	\begin{equation*}
		\div \vec{E} = \frac{\rho}{\ee} \eqtext{and} \vec{E} = - \grad U
	\end{equation*}
	Substitute $ \E = - \grad U $ into Gauss's law to get
	\begin{equation*}
		\div \big[-\grad U\big] = \frac{\rho}{\ee} \implies \nabla^{2}U = -\frac{\rho}{\ee}
	\end{equation*}
	The last equality takes the form of a \textit{Poisson equation}.\footnote{In general any equation of the form $ \nabla^{2}f(\r) = g(\r) $ is called a Poisson equation.} It relates charge density $ \rho $ to electric potential $ U $. In other words, if we know a spatial charge distribution $ \rho $, we can find the electric field potential $ U $ thus the electric field $ \vec{E} $ with $ \E = - \grad U $.
	
	\item As a simple example we start with a point charge. The charge distribution is
	\begin{equation*}
		\rho(\vec{r}) = q \delta(\vec{r}) \implies \nabla^{2}U = -\frac{\rho}{\ee} = - \frac{q}{\ee}\delta(\r)
	\end{equation*}
	This the Poisson equation for a point charge. We will solve the equation with a Fourier transform.
	
	\item First, a review of the Fourier transform. Think of it as an expansion over a basis of plane waves of the form $ e^{i \vec{k} \cdot \r} $. Where $ U(\vec{k}) $ plays the role of a weight function determining how much each wave contributes to the expansion
	\begin{equation*}
		U(\r) = \int \diff^{3}k U(\vec{k}) e^{i \vec{k}\cdot \r}
	\end{equation*}
	where $ U(\vec{k}) $ is the amplitude of the plane wave with wave vector $ \vec{k} $. To find $ U(\vec{k}) $ we take the inner product of both sides of the above equation with the basis function $ e^{-i \tvec{k}\cdot\r} $. 
	\begin{equation*}
		\int U(\r) e^{-i \tvec{k}\cdot \r} \diff^{3}r = \int \int \diff^{3}k \diff^{3}r U(\vec{k}) e^{i (\vec{k} - \tvec{k})\cdot \r}
	\end{equation*}
	The integrals over $ \r $ on the right-hand side is in fact a delta function, because the orthogonal plane waves cancel out over all space except at the origin, where they constructively interfere to infinity. So 
	\begin{equation*}
		\int U(\r) e^{-i \tvec{k}\cdot \r} \diff^{3}r = (2\pi)^{3} \int U(\vec{k}) \delta (\vec{k} - \tvec{k}) \diff^{3}k = (2\pi)^{3} U(\tvec{k})
	\end{equation*}
	The delta function suppresses the integral everywhere besides at $ (\vec{k} - \tvec{k}) $. And that's how you get the amplitude $ U(\tvec{k}) $ for a wave vector $ \tvec{k} $. 
	\begin{equation*}
		U(\tvec{k}) = \frac{1}{(2\pi)^{3}} \int U(\r) e^{-i\vec{k}\cdot \r} \diff^{3}r
	\end{equation*}
	This is also an inverse Fourier transform of sorts, to get $ U(\vec{k}) $ from $ U(\r) $.
	
	\textbf{Recipe:} Fourier transform the Poisson equation from $ \r $ into $ \vec{k} $ space (where the Laplacian operator $ \laplacian $ simplifies to $ -ik^{2} $ under $ \mathcal{F} $) and solve for the amplitude $ U(\vec{k}) $ of each plane wave $ e^{-i\vec{k}\cdot \r} $. Substitute $ U(\vec{k}) $ into the Fourier transform
	\begin{equation*}
		U(\r) = \int \diff^{3}k U(\vec{k}) e^{i \vec{k}\cdot \r}
	\end{equation*}
	and evaluate the integral---usually in spherical coordinates---to find find $ U(\r) $. 
	
	\item Here is the first of two more notes. What happens if we insert gradient into the Fourier transform? Remember the gradient operator acts only on $ \r $. So
	\begin{equation*}
		\grad U(\r) = \int \diff^{3}k U(\vec{k}) \grad e^{i \vec{k}\cdot \r}
	\end{equation*}
	Evaluating the gradient over $ (x, y, z) $ components gives
	\begin{equation*}
		\grad e^{i \vec{k}\cdot \r} = 
		\begin{bmatrix}
			ikx\\
			iky\\
			ikz
		\end{bmatrix}
		e^{i(k_{1}x + k_{2}y + k_{3}z)} = ik e^{i\vec{k}\cdot \r}
	\end{equation*}
	
	So:
	\begin{equation*}
		\grad U(\r) = \int \diff^{3}k U(\vec{k}) \grad e^{i \vec{k}\cdot \r} = \int \diff^{3}k U(\vec{k})i \vec{k} e^{i \vec{k}\cdot \r}
	\end{equation*}
	which has the form of a Fourier transform! This time of the function $ U(\vec{k})i \vec{k} $. 
	
	The interpretation is that the gradient operator $ \grad $ transforms to multiplication by $ i\vec{k} $ under the Fourier transform. Analogously, the Laplacian $ \nabla^{2} $ would transform into multiplication by $ (i\vec{k})^{2} = - k^{2} $. 
	
	
	\item And the second of the two notes: What happens if we insert $ \delta(\r) $ into the Fourier transform? Let $ \delta(\vec{k}) $ denote the amplitude in the expansion of $ \delta(\r) $, analogous to $ U(\vec{k}) $ for $ U(\r) $. Using the inverse Fourier transform and the integral properties of the delta function produces
	\begin{equation*}
		\delta (\vec{k}) = \frac{1}{(2\pi)^{3}} \int \delta(\r) e^{-i\vec{k}\cdot\r}\diff^{3}r = \frac{1}{(2\pi)^{3}} e^{-i\vec{k}\cdot 0} = \frac{1}{(2\pi)^{3}}
	\end{equation*}
\end{itemize}


\subsubsection{Poisson Equation for a Point Particle}
\textit{The Poisson equation for a point particle is}
\begin{equation*}
	\nabla^{2}U(\r) = - \frac{q}{\ee} \delta (\r)
\end{equation*}
\textit{Solve the equation for $ U(\r) $.}
\begin{itemize}
	
	\item The plan is to transform into $ \vec{k} $ space, solve for $ U(\vec{k}) $, then transform back to $ U(\r) $. 
	
	First, take the Fourier transform of both sides---use the Fourier transform identities $ \nabla^{2} \to -k^{2} $ and $ \delta(\r) \to \frac{1}{(2\pi)^{3}} $ from the theory section.
	\begin{equation*}
		- k^{2}U(\vec{k}) = - \frac{q}{\ee} \frac{1}{(2\pi)^{3}} \implies U(\vec{k}) = \frac{q}{(2\pi)^{3} \ee k^{2}}
	\end{equation*}
	
	\item Next, find $ U(\r) $ using a second Fourier transform
	\begin{equation*}
		U(\r) = \int \diff^{3} k \frac{q e^{i\vec{k}\cdot \r}}{k^{2}\ee(2\pi)^{3}} = \frac{q}{(2\pi)^{3}\ee} \int \frac{e^{i\vec{k}\cdot \r}}{k^{2}} \diff^{3}k
	\end{equation*}
	
	\item Now, how to solve the integral? Suppose $ \theta $ is the angle between $ \r $ and $ \vec{k} $. To simplify the dot product. And then integrate in spherical coordinates:
	\begin{equation*}
		U(\r) =  \frac{q}{(2\pi)^{3}\ee} \int_{0}^{2\pi}\diff \phi \int_{-1}^{1} \diff [\cos \theta] \int_{0}^{\infty} \diff k k^{2} \frac{e^{ik r\cos \theta}}{k^{2}} 
	\end{equation*}
	So integrate over $ \phi $, which is simple and gives $ 2\pi $:
	\begin{equation*}
		U(\r) = \frac{q}{(2\pi)^{2}\ee} \int_{-1}^{1}\int_{0}^{\infty} e^{i \cos \theta k r} \diff[\cos \theta] \diff k
	\end{equation*}
	Integrate first over $ \theta $, (to avoid $ e^{i \cos \theta \cdot \infty} $ from the upper $ k $ limit). Recognize the sine function the difference of exponents!
	\begin{equation*}
		U(\r) = \frac{q}{(2\pi)^{2}\ee} \int_{0}^{\infty} \eval{\frac{e^{i \cos \theta k r}}{i k r}}_{\theta = -1}^{1} \diff k = \frac{q}{(2\pi)^{2}\ee} \int_{0}^{\infty}\frac{e^{ikr}-e^{-ikr}}{ikr} \diff k = \frac{q}{(2\pi)^{2}\ee} \int_{0}^{\infty}\frac{2\sin (kr)}{kr} \diff k
	\end{equation*}
	The integral of the $ \operatorname{sinc} $ function is
	\begin{equation*}
		\int_{0}^{\infty} \frac{\sin x}{x}\diff x = \frac{\pi}{2}
	\end{equation*}
	And applying this integral gives
	\begin{equation*}
		U(\r) = \frac{2q}{(2\pi)^{2}\ee}\frac{\pi}{2r} = \frac{q}{4\pi \ee r}
	\end{equation*}
	which is the electric potential of a point charge. Cool! We derived it directly from Maxwell's equations.
	
	\item Finally, substitute our result for $ U(\r) $ into the Poisson equation for a point charge. The result is
	\begin{equation*}
		\laplacian \frac{q}{4\pi \ee r} = -\frac{q}{\ee}\delta (\r) \implies \nabla^{2}\frac{1}{r} = - 4\pi \delta(\r),
	\end{equation*}
	which will be useful in the next problems. 
	
\end{itemize}

\subsubsection{Theory Interlude: Electric Field of a General Charge Distribution}
\begin{itemize}
	\item We just solved the Poisson equation for the simple case $ \rho(\r) = \delta(\r) $. Can we use this result to solve the general case $ \rho = \rho(\r) $? The answer is yes, if we expand $ \rho(\r) $ over a basis of delta functions, as follows:
	\begin{equation*}
		\rho(\r) = \int \diff^{3}\tilde{r} \rho(\tvec{r}) \delta(\r - \tvec{r})
	\end{equation*}
	In this case, the solution of $ U(\r) $ to the Poisson equation is
	\begin{equation*}
		U(\r) = \int \diff^{3}\tilde{r}\rho(\tvec{r}) \frac{1}{4\pi \ee\abs{\r - \tvec{r}}} = \frac{1}{4\pi \ee} \int\frac{\diff^{3}\tilde{r}\rho(\tvec{r})}{\abs{\r - \tvec{r}}}
	\end{equation*}
	Well that's cool. By solving the Poisson equation for a delta function and then expanding an arbitrary $ \rho(\r) $ in terms of the delta function, we got the equation to the Poisson equation for any $ \rho(\r) $. And we can then get electric field using
	\begin{equation*}
		\vec{E} = - \grad U = \frac{1}{4\pi \ee} \int\frac{\diff^{3}\tilde{r}\rho(\tvec{r})}{\abs{\r - \tvec{r}}^{2}} \frac{\r - \tvec{r}}{\abs{r - \tvec{r}}}
	\end{equation*}
	which agrees with the equation quoted in the previous exercise set.
\end{itemize}

\subsubsection{Electric Field of a Hydrogen Atom}
\textit{The hydrogen atom has the electric potential}
\begin{equation*}
	U(\r) = \frac{q}{4\pi \ee} \frac{e^{-\alpha r}}{r}\left(1 + \frac{\alpha r}{2}\right), \qquad \alpha = \frac{2}{r_{B}}
\end{equation*}
\textit{Find the charge density $ \rho(\r) $ that generates this potential.}
\begin{itemize}
	\item Use the Poisson equation, which connects $ U $ and $ \rho $
	\begin{equation*}
		\nabla^{2}U(\r) = - \frac{\rho(\r)}{\ee}
	\end{equation*}
	Calculate the Laplacian of our $ U(\r) $ and work in spherical coordinates, since the potential is spherically symmetric (depends only on $ r $). As a review, when acting on a function that depends only on $ r $, $ \nabla^{2} $ in spherical coordinates reads
	\begin{equation*}
		\nabla^{2} = \frac{1}{r^{2}} \pdv{}{r}\left(r^{2} \pdv{}{r}\right)
	\end{equation*}
	
	\item Applying $ \nabla^{2} $ to $ U(r) $, after some straightforward but rather tedious differentiation, leads to
	\begin{equation*}
		\nabla^{2}U(\r) = \frac{q\alpha^{3}}{8 \pi \ee}e^{-\alpha r}
	\end{equation*}
	Rearranging the Poisson equation then gives
	\begin{equation*}
		\rho(\r) = - \frac{q\alpha^{3}}{8 \pi}e^{-\alpha r}
	\end{equation*}
	
	\item Note that the charge density is negative, which corresponds to the negatively charged electron cloud. Inserting the definition of $ \alpha = \frac{2}{r_{B}}$ gives
	\begin{equation*}
		\rho(\r) = -\frac{q}{\pi r_{B}^{3}} e^{-\frac{2r}{r_{B}}}
	\end{equation*}
	
	Another interpretation: $ e^{-\frac{2r}{r_{B}}} $ is equivalent to $ \left(e^{-\frac{r}{r_{B}}}\right)^{2} $, which is the square of the hydrogen atom's ground state wave function. The square of the wave function is probability, and multiplying the probability by $ \frac{q}{r_{B}^{3}} $ gives a charge density. 
	
	\item And why does the charge of the proton not contribute? Because the proton occurs at the origin, which corresponds to a charge density singularity at the origin. Terms in $ U(r) $ with $ \frac{1}{r} $ generate the singularity. And we just glossed over this when applying $ \nabla^{2} $. 
	
	We can resolve this problem with a special case from the limit
	\begin{equation*}
		\lim_{r\to 0}U(\r) = \frac{q}{4\pi \ee r}
	\end{equation*}
	We would then have to solve the Poisson equation for this potential. But we already know this potential: it is the potential for a point charge and corresponds to a charge density
	\begin{equation*}
		\rho(\r) = q \delta(\r)
	\end{equation*}
	The correct total result for the hydrogen atom is the sum of the electron cloud result and the charge density of the nucleus.
	\begin{equation*}
		\rho(\r) = q \delta(\r) - \frac{q\alpha^{3}}{8 \pi}e^{-\alpha r}
	\end{equation*}
	
	\textit{Lesson: Be careful when working with the Poisson equation if $ U(r) $ has singularities!}
	
\end{itemize}

\subsection{Third Exercise Set}

\subsubsection{Theory: Laplace Equation}
Just a quick review from the last exercise set: the Poisson equation used to solve for the electric field potential generated by a charge density $ \rho $ is
\begin{equation*}
	\nabla^{2}U(\r) = - \frac{\rho(\r)}{\ee} \eqtext{where} \laplacian = \pdv[2]{}{x} + \pdv[2]{}{y} + \pdv[2]{}{z}
\end{equation*}
Often $ \rho(\r) = 0 $ in places we're solving for the electric potential. In this case $ \laplacian U(\r) = 0 $. This equation is called a \textit{Laplace equation}.

\subsubsection{Perpendicular Ribbon in a Parallel-Plate Capacitor}
\textit{We place a long, thin conducting ribbon between the plates of a large parallel-plate capacitor, perpendicularly to the plates; the ribbon almost touches both plates with a little bit of air/insulation between the ribbon edges and the plates. The ribbon height and distance between the plates is $ a $. We ground both plates and set the ribbon potential to $ U_{0} $. What is the electric potential inside the capacitor?}

\begin{itemize}
	\item First, decide on a coordinate system. Choose Cartesian coordinates since the problem has rectangular symmetry. I will try to describe the axes, but you basically need to see a picture. Here's my attempt: we drew a 2D projection on lined paper where the capacitor plates run from left to right along the page and are separated vertically by the distance $ a $; the ribbon is a vertical line between the top and bottom plates. In this case, the $ y $ axis runs vertically upward along the ribbon (connecting the top and bottom plates), the $ x $ axis runs left to right along the page, and the $ z $ axis points out of the page. 
	
	Note that the problem is independent of $ z $ (by translation symmetry in the $ z $ direction), so we only need $ U(x, y) $. More so, the problem has reflection symmetry, so we can find the solution on only one side of the ribbon (one half of the $ x $ axis) and reflect the solution about the $ y $ axis.
	
	\item The space between the capacitor plates is empty---there is no charge, and we use the Laplace equation for the space between the plates.
	\begin{equation*}
		\laplacian U(x, y) = 0
	\end{equation*}
	Charge can occur only along the ribbon or on the capacitor plates.
	
	\item Next, determine boundary conditions for $ U(x, y) $ so the equation has a unique solution. The problem's boundaries are the ribbon and edges of the capacitor plates.
	
	On the bottom plate, $ U(x, 0) = 0 $. On the upper plate, $ U(x, a) = 0 $. Both are zero because the plates are grounded. For the ribbon $ U(0, y) = U_{0} $. And we need one more boundary---infinity. We require only that $ U(x \to \infty, y) $ is bounded, i.e. that $ U $ does not diverge at $ \infty $. 
	
	\item First, attempt solving the problem with separation of variables: $ U(x, y) = X(x)Y(y) $---this approach tends to work well with symmetric problems. Plugging this ansatz into the Laplace equation and evaluating $ \laplacian $ gives
	\begin{equation*}
		X^{''}Y + XY^{''} = 0 \implies \frac{X''}{X} = - \frac{Y''}{Y}
	\end{equation*}
	The separation of variables is successful---we were able to get only $ x $ and only $ x $ on different sides of the equation. 
	
	\item As usual, set the equation equal to a separation constant $ \kappa^{2} $ and get two equations
	\begin{equation*}
		X'' - \kappa^{2}X = 0 \eqtext{and} Y'' + \kappa^{2} Y = 0
	\end{equation*}
	Both equations have simple solutions! The equations for $ X $ and $ Y $ are solved by exponential and sinusoidal functions, respectively.
	\begin{equation*}
		X(x) = Ae^{\kappa x} + Be^{-\kappa x} \eqtext{and} Y(y) = C\sin(\kappa y) + D\cos (\kappa y)
	\end{equation*}
	We then substitute the expressions for $ X $ and $ Y $ back into the ansatz $ U = XY $:
	\begin{equation*}
		U(x, y)  = X(x)Y(y) = \left(Ae^{\kappa x} + Be^{-\kappa x}\right)\left(C\sin(\kappa y) + D\cos (\kappa y)\right)
	\end{equation*}
	
	
	\item We find the coefficients $ A, B, C $ and $ D $ using the boundary conditions. 
	\begin{itemize}
		\item Start with the most powerful condition, that $ U(x \to \infty, y) $ is bounded. This condition implies $ A = 0 $ to suppress the divergent exponential function $ e^{\kappa } $. 
			
		\item Then, use the next two simplest conditions, the ones requiring $ U(x, y) = 0 $. Starting with $ U(x, 0) = 0 $ gives
		\begin{equation*}
			0 \equiv U(x, 0) = 1 \cdot (0 + D) \implies D = 0
		\end{equation*}
		With both $ A = D = 0 $, we're left at this point with only 
		\begin{equation*}
			U(x, y) = Be^{-\kappa x} \cdot C \sin (\kappa y)
		\end{equation*}
		
		\item Next, applying $ U(x, a) = 0$ gives
		\begin{equation*}
			0 \equiv U(x, a) = Be^{-\kappa x} C \sin (\kappa a) \equiv F e^{-\kappa x} \sin (\kappa a)
		\end{equation*} 
		Note that we've joined the product of two constants into one constant $ F = BC $. 
		
		We have two options: either $ F = 0 $ or $ \sin (\kappa a) = 0$. The option $ F = 0 $ gives the trivial solution $ U(x, y) = 0 $. The non-trivial solution comes from 
		\begin{equation*}
			\sin(\kappa a ) = 0 \implies \kappa a = n \pi, \quad n 1, 2, 3, \ldots
		\end{equation*}
		Note that $ n = 0 $ leads to a trivial solution $ U(x, y) = 0 $, which we reject.
		
		Respecting the quantization of $ \kappa $ and $ F $ by the index $ n $, the general solution at this point is the linear superposition
		\begin{equation*}
			U(x, y) = \sum_{n=1}^{\infty} F_{n} e^{-\kappa_{n}x}\sin(\kappa_{n}y)
		\end{equation*}
		
		\item To find $ F_{n} $, we use the last boundary condition $ U(0, y) = U_{0} $.
		\begin{equation*}
			U_{0} = \sum_{n=1}^{\infty} F_{n} \sin(\kappa_{n}y) = \sum_{n=1}^{\infty} F_{n} \sin(\frac{n\pi y}{a})
		\end{equation*}
		where we've used $ \kappa_{n} = \frac{n\pi}{a} $. This is a Fourier expansion of the constant $ U_{0} $ over sine functions. 
		
		We find the coefficients by taking the inner product of both sides of the equation (I think on the vector space $ L(0, a) $), which amounts to multiplying both sides by $ \sin \frac{m\pi y}{a} $ and integrating both sides over $ y $ from $ 0 $ to $ a $. 
		
		The left side $ U_{0} $ becomes
		\begin{equation*}
			U_{0} \int_{0}^{a}\sin(\frac{m\pi y}{a}) \diff y = -\frac{U_{0}a}{m \pi} \cos(\frac{m\pi y}{a}) \bigg |_{0}^{a} = \frac{U_{0}a}{m \pi}\big[1 - (-1)^{m}\big]
		\end{equation*}
		And the right hand side, with the sum, we switch the sum and integral to get
		\begin{equation*}
			\sum_{n = 1}^{\infty} F_{n} \int_{0}^{a} \sin(\frac{n\pi y}{a}) \sin(\frac{m\pi y}{a} ) \diff y = \sum_{n = 1}^{\infty} F_{n} \delta_{mn} \int_{0}^{a} \sin^{2}\left (\frac{m\pi y}{a} \right ) \diff y = \frac{F_{m}a}{2}
		\end{equation*}
		Because of the orthogonality of the sine functions, the integral is zero for $ m \neq n $. Only the case $ m = n $ gives a non-zero result. 
		
		Equating the two sides gives the desired expression for $ F_{m} $:
		\begin{equation*}
			\frac{U_{0}a}{m \pi}\big[1 - (-1)^{m}\big] = \frac{F_{m}a}{2} \implies F_{m} = \frac{2U_{0}}{m\pi}\big[1 - (-1)^{m}\big]
		\end{equation*}
	\end{itemize}
	
	\item With $ F_{m} $ known, the final result for $ U(x, y) $ is then
	\begin{equation*}
		U(x, y) = \frac{2U_{0}}{\pi} \sum_{n = 1}^{\infty}\frac{1 - (-1)^{n}}{n}\exp(-\frac{n\pi x}{a}) \sin(\frac{n\pi y}{a})
	\end{equation*}
	Some limit cases: for $ x \gg a $, the exponent terms very small, and we can neglect all terms in the series except the first term $ e^{-\frac{\pi x}{a}} $ with $ n = 1 $. The result is
	\begin{equation*}
		U(x, y) = \frac{4U_{0}}{\pi}\exp(-\frac{\pi x}{a}) \sin (\frac{\pi y}{a})
	\end{equation*}
	
	\item A separate case for which we can find a nice analytic solution is in the center of the capacitor at $ y = \frac{a}{2} $. The solution reads
	\begin{equation*}
		U(x, \tfrac{a}{2}) = \frac{2U_{0}}{\pi} \sum_{n = 1}^{\infty}\frac{1 - (-1)^{n}}{n}\exp(-\frac{n\pi x}{a}) \sin(\frac{n\pi}{2})
	\end{equation*}
	Instead of finding $ U(x, \tfrac{a}{2}) $, we'll find the electric field $ \E(x, \tfrac{a}{2}) $. Because of reflection symmetry across the line $ y = \frac{a}{2} $, the electric field cannot have a $ y $ component---$ \E $ only has an $ x $ component. We'll find $ E_{x}(x) $ from the potential:
	\begin{equation*}
		E_{x}(x) = -\pdv{}{x}U(x, \tfrac{a}{2}) = -\frac{2U_{0}}{a} \sum_{n = 1}^{\infty}\big[1 - (-1)^{n}\big]\exp(-\frac{n\pi x}{a}) \sin(\frac{n\pi}{2})
	\end{equation*}
	Next, note that
	\begin{equation*}
		\big[1 - (-1)^{n}\big]\sin(\frac{n\pi}{2}) = 
		\begin{cases}
			0 & n \text{ even}\\
			2 & n = 1, 5, 9, \ldots\\
			- 2 & n = 3, 7, 11, \ldots
		\end{cases}
	\end{equation*}
	The sum simplifies to
	\begin{align*}
		E_{x}(x) = \frac{4U_{0}}{a}\left[e^{-\frac{\pi x}{a}} - e^{-\frac{3\pi x}{a}} + e^{-\frac{5\pi x}{a}} \mp \ldots \right] = \frac{4U_{0}}{a}e^{-\frac{\pi x}{a}}\left[1 - e^{-\frac{2\pi x}{a}} + \left(e^{-\frac{2\pi x}{a}}\right)^{2}\mp \ldots \right]
	\end{align*}
	which is a geometric series in $ e^{-\frac{2\pi x}{a}} $. The result is
	\begin{equation*}
		E_{x}(x) =  \frac{4U_{0}}{a}\frac{e^{-\frac{\pi x}{a}}}{1 + e^{-\frac{2\pi x}{a}}} = \frac{4U_{0}}{a}\frac{1}{e^{\frac{\pi x}{a}} + e^{-\frac{\pi x}{a}}} = \frac{2U_{0}}{a\cosh(\frac{\pi x}{a})}
	\end{equation*}
\end{itemize}

\subsubsection{A Halved Conducting Cylinder}
\textit{Consider a long cylinder of radius $ a $ cut in half along a plane running along the cylinder's longitudinal axis. We separate the two cylinder halves by an arbitrarily small amount (so the halves are insulated) and apply a potential difference $ U_{0} $ between the two halves. The halved cylinder acts as a capacitor. Find the electric potential inside the cylinder.}

\begin{itemize}
	\item Decide on a coordinate system: we use cylindrical coordinates because our problem has cylindrical symmetry. Let the $ z $ axis run along the cylinder's longitudinal axis. Because of translational symmetry along the $ z $ axis, $ U $ is independent of $ z $. 
	
	\item There is no charge inside the cylinder, so we get a Laplace equation
	\begin{equation*}
		\laplacian U(r, \phi) = 0
	\end{equation*}
	In cylindrical coordinates (when acting on a function independent of $ z $), the Laplacian operator reads
	\begin{equation*}
		\laplacian = \frac{1}{r}\pdv{r}\left(r \pdv{r}\right) + \frac{1}{r^{2}} \pdv[2]{}{\phi}
	\end{equation*}
	In our case,
	\begin{equation*}
		\laplacian U(r, \phi) = \frac{1}{r}\pdv{r}\left(r \pdv{U}{r}\right) + \frac{1}{r^{2}} \pdv[2]{U}{\phi} = 0
	\end{equation*}
	
	\item Again, we separate variables with the ansatz $ U(r, \phi) = R(r)\Phi(\phi) $. Plugging this into the Laplace equation gives
	\begin{equation*}
		\Phi \frac{1}{r}(rR')' + \frac{R}{r^{2}}\Phi'' = \Phi \left(\frac{R'}{r} + R''\right) + \frac{R}{r^{2}}\Phi'' = 0
	\end{equation*}
	Note that $ r'  = 1 $. Dividing through by $ \Phi $ and rearranging gives
	\begin{equation*}
		\frac{rR'}{R} + r^{2}\frac{R''}{R} = - \frac{\Phi''}{\Phi}
	\end{equation*}
	
	\item Following the usual separation procedure, we set both sides equal to the separation constant $ m^{2} $. The equations for $ \Phi $ and $ R $ read
	\begin{equation*}
		\Phi'' + m^{2} \Phi = 0 \eqtext{and} r^{2}R'' + rR' - m^{2}R = 0
	\end{equation*}
	The solution for $ \Phi $ is sinusoidal:
	\begin{equation*}
		\Phi(\phi) = A \sin(m\phi) + B\cos(m\phi)
	\end{equation*}
	Now, our cylindrical problem is periodic in $ \phi $ with period $ 2\pi $---this just means the cylinder repeats after one revolution. Periodicity in $ \phi $ is possible only if $ m $ takes on integer values, so we can immediately index the solutions for $ \Phi $ with
	\begin{equation*}
		\Phi_{m}(\phi) = A_{m} \sin(m\phi) + B_{m}\cos(m\phi), \quad m = 1, 2, 3, \ldots 
	\end{equation*}
	We only use positive integers because the odd/even symmetry of $ \sin $ and $ \cos $ means negative integers give the same result as positive one---solving for negative $ m $ would be redundant. We reject $ m = 0 $ because this solution leads to $ \Phi'' = 0 $, meaning $ \Phi $ is a linear function. But a linear function can't be periodic in $ \phi $, so we reject $ m = 0 $.
	
	The second equation for $ R $ is solved with powers of $ r $. The result is
	\begin{equation*}
		R_{m}(r) = C_{m}r^{m} + D_{m}r^{-m}
	\end{equation*}
	
	\item The general solution is the linear superposition
	\begin{equation*}
		U(r, \phi) = \sum_{m = 1}^{\infty}\Phi_{m}(\phi)R_{m}(m) = \sum_{m = 1}^{\infty} \left(A_{m} \sin(m\phi) + B_{m}\cos(m\phi)\right)\left( C_{m}r^{m} + D_{m}r^{-m} \right)
	\end{equation*}
	\textit{Note: We ended the problem at this point (ran out of time) and continued in the fourth exercise set.}

	\end{itemize}
	
\subsection{Fourth Exercise Set}

\subsubsection{A Halved Conducting Cylinder (continued)}
\begin{itemize}	
		
	\item We left off last time with the general solution
	\begin{equation*}
		U(r, \phi) = \sum_{m = 1}^{\infty}\Phi_{m}(\phi)R_{m}(m) = \sum_{m = 1}^{\infty} \left(A_{m} \sin(m\phi) + B_{m}\cos(m\phi)\right)\left( C_{m}r^{m} + D_{m}r^{-m} \right)
	\end{equation*}
	for the potential inside the cylinder. To find a solution specific to our problem, we apply boundary conditions. We already applied the periodic boundary condition $ U(r, \phi) = U(r, \phi + 2\pi) $, which required $ m $ be integer-valued.
	
	
	A second boundary condition requires the capacitor halves have a potential difference $ U_{0} $ between them. It is best to write this potential difference in the symmetric form
	\begin{equation*}
		U(a, \phi) = 
		\begin{cases}
			\frac{U_{0}}{2} & \phi \in (0, \pi)\\
			-\frac{U_{0}}{2} & \phi \in (\pi, 2\pi)
		\end{cases}
	\end{equation*}
	There is another condition---that $ U $ does not diverge at $ r = 0 $. This condition implies the $ D_{m} $ coefficients are zero, because the $ D_{m}r^{-m} $ term diverges at $ r = 0 $. 
	
	Observation: the second boundary condition is an odd function of $ \phi $. This implies that only odd (sine) terms can appear in the final solution. This allows us to set the $ A_{m} $ coefficients equal to zero to eliminate the cosine terms. We are left with
	\begin{equation*}
		U(r, \phi) = \sum_{m=1}^{\infty} F_{m}r^{m} \sin (m\phi)
	\end{equation*}
	where we have defined $ B_{m}C_{m} \equiv F_{m} $. 
	
	\item Applying the second boundary condition at $ r = a $ gives
	\begin{equation*}
		U(a, \phi) = \sum_{m=1}^{\infty} F_{m}a^{m} \sin (m\phi)
	\end{equation*} 
	To solve this, we take the scalar product of the equation with $ \sin(n \phi) $: 
	\begin{equation*}
		\int_{0}^{2\pi} U(a, \phi)\sin(n \phi) \diff \phi =  \int_{0}^{2\pi} \sum_{m=1}^{\infty} F_{m}a^{m} \sin (m\phi) \sin(n \phi) \diff \phi
	\end{equation*}
	Plugging in the step values of $ U(a, \phi) $ gives
	\begin{equation*}
		\frac{U_{0}}{2}\int_{0}^{\pi}\sin(n \phi) \diff \phi - \frac{U_{0}}{2}\int_{\pi}^{2\pi}\sin(n \phi) \diff \phi  =  \int_{0}^{2\pi} \sum_{m=1}^{\infty} F_{m}a^{m} \sin (m\phi) \sin(n \phi) \diff \phi
	\end{equation*}
	First, we solve the left-hand side
	\begin{align*}
		&\frac{U_{0}}{2}\left[-\frac{1}{n}\cos(n\phi)\big |_{0}^{\pi} + \frac{1}{n}\cos (n\phi)\big |_{\pi}^{2\pi} \right] = \frac{U_{0}}{2n}\left[-\cos(\pi n) + 1 + 1 - \cos(n\pi)\right]\\
		&{}\qquad = \frac{U_{0}}{n}\left(1 - (-1)^{n}\right)
	\end{align*}
	where we've used the identity $ \cos(n\pi) = (-1)^{n} $.
	
	\item Next, we solve the right-hand side. Switching the order of integration and summation gives
	\begin{equation*}
		  \sum_{m=1}^{\infty} F_{m}a^{m} \int_{0}^{2\pi}  \sin (m\phi) \sin(n \phi) \diff \phi = \sum_{m=1}^{\infty} F_{m}a^{m}\left (\frac{2\pi}{2}\delta_{mn} \right ) = F_{n}a^{n} \pi
	\end{equation*}
	Combining the left and right sides gives
	\begin{equation*}
		F_{n} = \frac{U_{0}\big[1 - (-1)^{n}\big]}{\pi n a^{n}}
	\end{equation*}
	So, the solution for $ U(r, \phi) $ is
	\begin{equation*}
		U(r, \phi) = \sum_{n = 1}^{\infty} \left[\frac{U_{0}\big[1 - (-1)^{n}\big]}{\pi n a^{n}}\right]r^{n}\sin(n\phi) = \frac{U_{0}}{\pi} \sum_{n = 1}^{\infty} \left(\frac{r}{a}\right)^{n} \frac{\big[1 - (-1)^{n}\big]}{n}\sin(n\phi)
	\end{equation*}
	
	\item Next, some limiting cases. It will be easier to work in terms of electric field instead of potential. We will find the electric field in the two planes parallel and perpendicular to the slit between the capacitor halves. 
	
	First, in the perpendicular (vertical) plane. The field points from high to low potential, so from the top half of the capacitor to the bottom half. In this plane we can work with just one coordinate $ r $, which represents the vertical distance from the cylinder's center. Note that $ \phi = \frac{\pi}{2} $. The component $ E_{r} $ we're after is
	\begin{align*}
		E_{r} &= - \pdv{r}\eval{U(r, \phi)}_{\phi = \frac{\pi}{2}}  = - \frac{U_{0}}{\pi} \sum_{n = 1}^{\infty} n\left(\frac{r}{a}\right)^{n -1}\frac{1}{a}\frac{\big[1 - (-1)^{n}\big]}{n}\sin(n\phi)\bigg|_{\phi = \frac{\pi}{2}}\\
		&=  -\frac{U_{0}}{\pi a} \sum_{n= 1}^{\infty}\left(\frac{r}{a}\right)^{n-1} \frac{\big[1 - (-1)^{n}\big]}{n}\sin(\frac{n\pi}{2})
	\end{align*}
	The sum simplifies considerably when you realize
	\begin{equation*}
		\frac{1 - (-1)^{n}}{n}\sin(\frac{n\pi}{2}) = 
		\begin{cases}
			0 & n \text{ even} \\
			2 & n = 1, 5,\ldots \\
			-2 & n = 3, 7,\ldots 
		\end{cases}
	\end{equation*}
	We can then write the field as a geometric series
	\begin{equation*}
		E_{r} = \frac{-2U_{0}}{\pi a}\left[1 - \left(\frac{r}{a}\right)^{2} + \left(\frac{r}{a}\right)^{4} \mp \cdots  \right] = \frac{-2U_{0}}{\pi a} \frac{1}{1 + \left(\frac{r}{a}\right)^{2}}
	\end{equation*}
	Note that $ E_{r} $ is largest at $ r = 0 $, decreases monotonically with increasing $ r $, and falls to half of its maximum value at $ r = a $. 
	
	\item In a field parallel to the slit, we would set $ \phi = 0 $. This plane is perpendicular to the vertical plane, which used the radial component $ E_{r} $, so for the parallel plane we work with the $ \phi $ component $ E_{\phi} $. 
	\begin{equation*}
		E_{\phi} = - \frac{1}{r}\pdv{\phi}U(r, \phi) \bigg |_{\phi = 0}
	\end{equation*}
	As before, the sum simplifies considerably to a geometric series. The result turns out to be
	\begin{equation*}
		E_{\phi} = - \frac{2U_{0}}{\pi a}\frac{1}{1 - \left(\frac{r}{a}\right)^{2}}
	\end{equation*}
	Note that $ E_{\phi} $ diverges at $ r = a $. This is a consequence of the very small slit spacing between the capacitor halves at $ r = a $; schematically have $ E = \frac{U_{0}}{d} \to \infty $ as $ d \to 0 $. 
	
\end{itemize}




\subsubsection{Conducting Sphere in a Uniform Electric Field}
\textit{A conducting sphere of radius $ a $ is placed in a uniform electric field $ E_{0} $ pointing in the $ z $ direction. Find the electric field $ U $ inside and outside the sphere.} 

\begin{itemize}
	\item Use spherical coordinates to take advantage of spherical symmetry. This means we want $ U(r, \phi, \theta) $. Because of the problem's rotational symmetry, the solution will be independent of $ \phi $. We need only $ U(r, \theta) $. 
	
	\item The sphere is at a constant potential because it is a conductor. We'll set $ U = 0 $ inside the sphere. In the space around the sphere, we solve the Laplace equation
	\begin{equation*}
		\laplacian U(r, \theta) = 0
	\end{equation*}
	We then separate variables via $ U = R(r)\Theta(\theta) $, which leads to the general solution
	\begin{equation*}
		U(r, \theta) = \sum_{l = 0}^{\infty} \left[A_{l}r^{l}+B_{l}r^{-(l+1)}\right]P_{l}(\cos \theta)
	\end{equation*}
	where $ P_{l} $ are the Legendre polynomials. 
	
	\item On to the boundary conditions. On the surface we'll set $ U(a, \theta) = 0 $. And at infinity, we use the boundary condition 
	\begin{equation*}
		U(r \to \infty, 0) = - E_{0}z = -E_{0}r \cos \theta
	\end{equation*}
	This is the potential of a uniform electric field (at infinity, the potential from the sphere is negligible). This potential is chosen so that
	\begin{equation*}
		- \pdv{U}{z} = E_{0}
	\end{equation*}
	i.e. that the potential at infinity recovers the uniform electric field $ E_{0} $. 
	
	\item We'll start with the second boundary condition at $ r \to \infty $. Applying the condition to the general solution gives
	\begin{equation*}
		\sum_{l = 0}^{\infty} A_{l}r^{l} P_{l}(\cos \theta) = -E_{0}r \cos \theta
	\end{equation*}
	Note that the $ r^{-(l+1)} $ terms vanish as $ r \to \infty $. The entire series equals a single term proportional to $ \cos \theta $ and the equality holds if only the $ l = 1 $ term in the series is non-zero, which generates a corresponding $ \cos \theta $ term from $ P_{1}(\cos \theta) = \cos \theta $ ie. $ P_{1}(x) = x $. The $ l = 1 $ term is
	\begin{equation*}
		A_{1}r \cos \theta = -E_{0}r \cos \theta \implies A_{1} = - E_{0}
	\end{equation*}
	so we have $ A_{l} = -E_{0}\delta_{l1} $. The solution for $ U(r, \theta) $ simplifies to
	\begin{equation*}
		U(r, \theta) = -E_{0}r \cos \theta + \sum_{l=1}^{\infty} B_{l}r^{-(l+1)}P_{l}(\cos \theta)
	\end{equation*}
	
	\item Next, the second boundary condition: $ U(a, \theta) = 0 $. Substituting the condition into the intermediate solution gives
	\begin{equation*}
		E_{0}a\cos \theta = \sum_{l=1}^{\infty} B_{l}a^{-(l+1)}P_{l}(\cos \theta)
	\end{equation*}
	Again, the entire series equals only a single term. Again, this will be the $ l = 1 $ term corresponding to $ P_{1}(\cos \theta) = \cos \theta $. For $ l \neq 1 $ we have $ B_{l} = 0 $. The $ l = 1 $ term gives
	\begin{equation*}
		E_{0}a\cos \theta = B_{1}a^{-2}\cos \theta \implies B_{1} = E_{0}a^{3}
	\end{equation*}
	With $ A_{l} $ and $ B_{l} $ known for all $ l $, the final result is
	\begin{equation*}
		U(r, \theta) = - E_{0} r \cos \theta + \frac{E_{0}a^{3}}{r^{2}}\cos \theta
	\end{equation*}
	The first term, $ - E_{0} r \cos \theta $, is the potential of the uniform external field $ E_{0} $. The second term comes from the sphere. In fact, this second term has the same form as the potential of an electric dipole!
	
	\textit{Limiting cases are discussed in the next exercise set.}
\end{itemize}


\subsection{Fifth Exercise Set}

\subsubsection{Conducting Sphere in a Uniform Electric Field (continued)}
\begin{itemize}
	\item Where we left off last time, we had found the potential due to the sphere was
	\begin{equation*}
		U(r, \theta) = - E_{0} \cos \theta + \frac{E_{0}a^{3}}{r^{2}}\cos \theta,
	\end{equation*}
	and identified the sphere's contribution $ \frac{E_{0}a^{3}}{r^{2}}\cos \theta $ corresponded to the potential of an electric dipole. Our next step is to explore the sphere's dipole behavior.
	
	\item We consider an infinitesimal element of the sphere's surface at the angle $ \theta $ carrying charge $ \diff q $, on the upper hemisphere with positive charge and $ \theta $. Recall the electric field, as for any conductor, is perpendicular to the surface. 
	
	We write Gauss's law for the small surface element, which is simple because the electric field is perpendicular to the surface
	\begin{equation*}
		E_{\perp} \diff S = \frac{\diff q}{\ee} \implies \dv{e}{S} = \sigma = \ee E_{\perp}
	\end{equation*}
	This equality gives us an expression for $ \sigma $ in terms of the electric field $ E_{\perp} $ perpendicular to the surface. We can find $ E_{\perp} $ from the potential:
	\begin{equation*}
		E_{\perp} = -\pdv{U}{r}\bigg |_{r = a} = E_{0}\cos \theta + 2E_{0}\cos \theta \implies \sigma = 3\ee E_{0} \cos \theta
	\end{equation*}
	The charge density's dependence on $ \theta $ quantitatively demonstrates the sphere's dipole-like charge distribution. 
	
	\item With charge density $ \sigma $ known, we can find the sphere's electric dipole moment via
	\begin{equation*}
		\vec{p}_{e} = \int \tvec{r} \diff q
	\end{equation*}
	We qualitatively expect $ \vec{p}_{e} $ to point upward (from the negative to the positively charged hemisphere), and confirm this analytically. By spherical symmetry, only the $ z $ component of $ \vec{p}_{e} $ is non-zero; this is
	\begin{equation*}
		p_{e_{z}} = \int \tilde{z} \diff q = \int (a \cos \theta) \cdot (\sigma \diff S) = \int( a \cos \theta) \cdot (3 \ee E_{0} \cos \theta) \cdot \diff S
	\end{equation*}
	To find $ \diff S $, we find the area of a small band of width $ \diff a $ around the sphere's surface. The band's area is $ 2\pi r \diff a = 2\pi (a \sin \theta ) (a \diff \theta) $. The dipole moment $ p_{e_{z}} $ is then
	\begin{align*}
		p_{e_{z}} &= \int_{0}^{\pi}(3\ee E_{0}a\cos^{2}\theta) \cdot 2\pi (a \sin \theta ) (a \diff \theta) = 6a^{3}\pi \ee E_{0} \int_{-1}^{1}\cos^{2}\theta \diff [\cos \theta]\\
		& = 6a^{3}\pi \ee E_{0} \left[\frac{1}{3}\cos^{3}\theta \right]_{-1}^{1} = 4\pi \ee E_{0} a^{3}
	\end{align*}
\end{itemize}

\subsubsection{Electric Dipole in a Conducting Spherical Shell}
\textit{We place an electric dipole with dipole moment $ \pe $ in the center of a conducting spherical shell of radius $ a $. What is the electric potential inside the shell?}

\begin{itemize}
	\item We use spherical coordinates, which are best suited to the problem's spherical symmetry. By rotational symmetry, the potential depends only on the coordinates $ r $ and $ \theta $, not $ \phi $. Besides at the center, the charge density inside the sphere is zero, so we solve the Laplace equation
	\begin{equation*}
		\laplacian U(r, \theta) = 0
	\end{equation*}
	The general solution is
	\begin{equation*}
		U(r, \theta) = \sum_{l = 0}^{\infty}\left(A_{l}r^{l} + B_{l}r^{-(l+1)}\right)P_{l}(\cos \theta)
	\end{equation*}
	where $ P_{l} $ are the Legendre polynomials. 
	
	\item We then apply boundary conditions to find a solution specific to our problem. First, the potential on the shell's surface is constant, since the shell is a conductor. For convenience, we'll set $ U(a, \theta) = 0 $. The second boundary condition concerns the dipole at the sphere's center. Namely, the potential approaches the potential of an electric dipole near the sphere's center. Quantitatively, this condition reads
	\begin{equation*}
		U(r \to 0, \theta) = \frac{p_{e}\cos \theta}{4\pi \ee r^{2}}
	\end{equation*}
	
	\item We start with the simpler second boundary condition, (the boundary $ r \to 0 $ eliminates the $ r^{l} $-dependent term). Inserted into the general solution, the second condition reads
	\begin{equation*}
		U(r\to 0, \theta) = \sum_{l = 0}^{\infty} B_{l}r^{-(l+1)}P_{l}(\cos \theta) = \frac{p_{e}\cos \theta}{4\pi \ee }r^{-2}
	\end{equation*}
	Note that the entire series sums to only a single term; for this to work, only the $ l = 1 $ term in the series can be non-zero, leaving
	\begin{equation*}
		B_{1}r^{-2}\cos \theta = \frac{p_{e}\cos \theta}{4\pi \ee }r^{-2} \implies B_{1} = \frac{p_{e}}{4\pi \ee} \eqtext{and} B_{l\neq 1} = 0
	\end{equation*}
	Note the use of $ P_{1}(\cos \theta) = \cos \theta $. The intermediate solution at this stage is
	\begin{equation*}
		U(r, \theta) = \sum_{l = 0}^{\infty} A_{l} r^{l} P_{l}(\cos \theta) + \frac{p_{e}}{4\pi \ee}r^{-2} \cos \theta
	\end{equation*}
	We apply the second boundary condition $ U(a, \theta) = 0 $ to get
	\begin{equation*}
		\sum_{l = 0}^{\infty} A_{l} a^{l} P_{l}(\cos \theta) = - \frac{p_{e}}{4\pi \ee}a^{-2} \cos \theta
	\end{equation*}
	As before, only the $ l = 1 $ term can be non-zero to satisfy the equality. The result is
	\begin{equation*}
		A_{1}a\cos \theta = - \frac{p_{e}}{4\pi \ee}a^{-2} \cos \theta \implies A_{1} = -\frac{p_{e}}{4\pi \ee a^{3}} \eqtext{and} A_{l\neq 1} = 0
	\end{equation*}
	With the coefficients $ A_{l} $ and $ B_{l} $ known, the solution for $ U(r, \theta) $ is
	\begin{equation*}
		U(r, \theta) = \left[\frac{p_{e}}{4\pi \ee r^{2}} -\frac{p_{e}}{4\pi \ee a^{3}}r\right]\cos \theta = \frac{p_{e}\cos \theta}{4\pi \ee}\left[\frac{1}{r^{2}} - \frac{r}{a^{3}}\right]
	\end{equation*}
	The $ \frac{1}{r^{2}} $ term is the dipole's contribution. The $ \frac{r}{a^{3}} $ comes from the charge induced on the conducting shell. 
	
	\item The induced term is worth a closer look, noting that $ r\cos \theta = z $. 
	\begin{equation*}
		U_{\text{induced}} = -\frac{p_{e}}{4\pi \ee a^{3}}r\cos \theta = -\frac{p_{e}}{4\pi \ee a^{3}}z
	\end{equation*}
	In particular, the associated electric field is
	\begin{equation*}
		E_{\text{induced}} = -\pdv{z}U_{\text{induced}} = \frac{p_{e}}{4\pi \ee a^{3}}
	\end{equation*}
	In other words, the electric field generated by the induced charge is constant! The uniform field also tells us about the charge distribution on the sphere's surface: to create a uniform field in the $ z $ direction, the shell must have a dipole-like charge distribution, with positive charge on the lower hemisphere and negative charge on the upper hemisphere. 
	
	\item Next, we're interested in the analytic expression for the surface charge density $ \sigma $. We consider a small surface element $ \diff S $, and consider the total electric field at that surface. The electric field must be perpendicular to the surface, since the shell is a conductor. Gauss's law applied to the surface element reads
	\begin{equation*}
		- E_{\perp} \diff S = \frac{\diff q}{\ee} \implies \sigma \equiv \dv{q}{S} = -\ee E_{\perp}
	\end{equation*}
	Note the minus sign, indicating the field's electric flux leaving the surface element from inside the shell. We find an expression for $ E_{\perp} $ from $ U $:
	\begin{equation*}
		E_{\perp} = -\pdv{U}{r}\bigg|_{r = a} = + \frac{p_{e}\cos \theta}{4\pi \ee}\left[\frac{2}{r^{3}} + \frac{1}{a^{3}}\right]_{r = a} = \frac{3p_{e}}{4\pi \ee a^{3}} \cos \theta
	\end{equation*}
	The associated surface charge density is
	\begin{equation*}
		\sigma = -\ee E_{\perp} = -\frac{3p_{e}}{4\pi a^{3}} \cos \theta
	\end{equation*}
	
	
\end{itemize}

\subsubsection{Point Charge Above a Conducting Plane}
\textit{Consider a positive point charge $ q $ a distance $ d $ above a large, grounded conducting plane. What is the electric potential in space due to the charge-plane system?}
\begin{itemize}
	\item We will use a trick called the \textit{method of images} to solve the problem. 
%	Currently, the electric field below the plane is zero---the field from the upper charge is perpendicularly incident on and vanishes into the grounded plate. 
	Namely, we imagine a negative point charge $ -q $ a distance $ d $ below the plane---a mirror image of the original positive charge. The resulting charge distribution is an electric dipole. 
	
	\textit{Note:} placing an imaginary negative charge a distance $ d $ below the plane does not change the field above the plane due to the positive charge. 
	
	\item Considering both points, the potential at an arbitrary position $ \r $ from the origin is
	\begin{equation*}
		U(\r) = \frac{q}{4\pi \ee}\frac{1}{\abs{\r - \vec{d}}} - \frac{q}{4\pi \ee}\frac{1}{\abs{\r + \vec{d}}} 
	\end{equation*}
	where the vector $ \vec{d} $ points perpendicularly up from the plane toward the positive charge. Introducing an angle $ \theta $ between $ \vec{d} $ and $ \r $, we have
	\begin{equation*}
		\abs{\r \pm \vec{d}} = \sqrt{r^{2}+d^{2} \pm 2rd\cos \theta}
	\end{equation*}
	The expression for $ U(\r) $ for the two charges is then simply
	\begin{equation*}
		U_{2}(\r) = \frac{q}{4\pi \ee}\left[\frac{1}{\sqrt{r^{2}+d^{2} \pm 2rd\cos \theta}} - \frac{1}{\sqrt{r^{2}+d^{2} \pm 2rd\cos \theta}}\right]
	\end{equation*}
	For the original configuration of a single positive charge a distance $ d $ above the plane, the potential above the plane agrees with $ U_{2} $, while the potential below the plane, where there is in reality no charge, is zero. The correct expression for the single charge $ +q $ is then
	\begin{equation*}
		U(\r) = 
		\begin{cases}
			U_{2}(\r) & \text{above the plane}\\
			0 & \text{below the plane}
		\end{cases}
	\end{equation*}
	
	\item Next, we're interested in the surface charge density $ \sigma(\rho) $ on the plane where $ \rho $ is the radial distance in the plane from the origin. As usual, we start with Gauss's law for a small surface element of the plane: 
	\begin{equation*}
		-E_{\perp}\diff S = \frac{\diff q}{\ee} \implies \sigma \equiv \frac{\diff q}{\diff S} =  -\sigma E_{\perp}
	\end{equation*}
	To find $ E_{\perp} $, we differentiate $ U $ with respect to the vertical coordinate $ z $. First, we introduce $ z $ into the expression $ \abs{\r \pm \vec{d}} $
	\begin{equation*}
		\abs{\r \pm \vec{d}} = \sqrt{r^{2}+d^{2} \pm 2rd\cos \theta} = \sqrt{\rho^{2} + z^{2} + d^{2} \pm 2dz}
	\end{equation*}
	We then have
	\begin{equation*}
		U(\rho, z) = \frac{q}{4\pi \ee}\left[\frac{1}{\sqrt{\rho^{2} + z^{2} + d^{2} - 2dz}} - \frac{1}{\sqrt{\rho^{2} + z^{2} + d^{2} + 2dz}}\right]
	\end{equation*}
	We then find $ E_{\perp} $ and then $ \sigma $ with
	\begin{align*}
		\sigma &= - \ee E_{\perp} = -\ee \pdv{U}{z}\bigg|_{z=0} = -\frac{q}{4\pi}\left[\frac{d}{(\rho^{2}+d^{2})^{3/2}} +  \frac{d}{(\rho^{2}+d^{2})^{3/2}} \right]\\
		& = -\frac{q}{2\pi} \frac{d}{(\rho^{2}+d^{2})^{3/2}}
	\end{align*}
	
	\item With surface charge density $ \sigma $ known, we then ask what is the total charge on the plane. Integrating the plane over rings with area $ \diff S = 2\pi \rho \diff \rho $, we have
	\begin{equation*}
		q_{\text{plane}} = \int \sigma \diff S = - \int_{0}^{\infty} \frac{qd}{2\pi(\rho^{2}+d^{2})^{3/2}} (2\pi \rho \diff \rho)
	\end{equation*}
	In terms of the new variable $ u = \rho^{2} + d^{2} $, the integral evaluates to
	\begin{equation*}
		q_{\text{plane}} = -\frac{qd}{2}\int_{d^{2}}^{\infty}\frac{\diff u}{u^{3/2}} = qd \left[\frac{1}{u^{1/2}}\right]_{d^{2}}^{\infty} = -q
	\end{equation*}
	
	\item Summary of what we did: recognize that the field above the plane from the positive charge looks like half the field of an electric dipole. Since we know the solution for a dipole, instead of solving the charge-plane system, we solve the (imaginary) two-charge system, which gives the same field above the plane anyway. We then reuse the upper half of the dipole solution for the single-charge plane system, and set the field below the plane equal to zero. The basic idea is: the field above the plane is the same for both the positive-charge plane system and for a dipole system, so we can use either approach to solve for the field above the plane. 
	
	\item Next, we ask what is the electrostatic force on the point charge above the plane? First, some theory:
\end{itemize}  
\textbf{Theory: Electrostatic Force}\\[2mm]
The total electrostatic force $ \vec{F} $ acting on the charges enclosed in a region of space $ V $ permeated with an electric field $ \E $ is
\begin{equation*}
	\vec{F} = \ee \oint_{\partial V} \left[\E(\E \cdot \uvec{n}) - \frac{1}{2}E^{2}\uvec{n}\right]\diff S
\end{equation*}
where $ \uvec{n} $ is the normal to the surface $ \partial V $ enclosing the charges. Like with Gauss's law, a good choice of the boundary surface, usually taking advantage of the problem's symmetries, tends to simplify the problem. Alternatively, if the electric field vanishes at infinity, we choose a surface that closes at infinity. 

\begin{itemize}
	\item Back to our problem: we choose an infinite surface whose base runs along the plane, then turns upward and closes at infinity to enclose the upper half of space above the plane. The field in this case is the same $ E_{\perp} $ calculated above:
	\begin{equation*}
		E_{\perp} = \frac{q}{2\pi \ee}\frac{d}{(\rho^{2}+d^{2})^{3/2}}
	\end{equation*}
	For the part of the surface running parallel to the plane, the normal to the surface $ \uvec{n} $ points perpendicularly into the plane, parallel to the electric field. The force equation for the bottom half of the surface reads 
	\begin{equation*}
		\vec{F}_{e} = \ee \oint \left[E^{2}\uvec{n}- \frac{1}{2}E^{2}\uvec{n}\right]\diff S = \frac{\ee}{2}\uvec{n}\int E_{\perp}^{2}\diff S
	\end{equation*}
	In fact, the contribution from the upper half of the surface is zero---the upper half extends to infinity, where the electric field vanishes. We only need to integrate over the bottom of the surface, running parallel to the plane. Writing $ \diff S = 2\pi \rho \diff \rho $ and substituting in the expression for $ E_{\perp} $, the force reads
	\begin{align*}
		\vec{F}_{e} &= \uvec{n}\frac{\ee}{2}\int_{0}^{\infty} \frac{q^{2}}{4\pi^{2}\ee^{2}}\frac{d^{2}}{(\rho^{2} + d^{2})^{3}} 2\pi \rho \diff \rho = \frac{q^{2}d^{2}}{4\pi \ee}\uvec{n}\int_{0}^{\infty}\frac{\rho}{(\rho^{2} + d^{2})^{3}} \diff \rho\\
		&=  \frac{q^{2}d^{2}}{8\pi \ee}\uvec{n}\int_{d^{2}}^{\infty}\frac{\diff u}{u^{3}} = - \frac{q^{2}d^{2}}{16\pi \ee}\uvec{n} \left[\frac{1}{u^{2}}\right]_{d^{2}}^{\infty} = \frac{q^{2}}{16\pi \ee d^{2}}\uvec{n}
	\end{align*}
	The force points in the direction of $ \uvec{n} $---downward into the plane. A final note: if we write
	\begin{equation*}
		\vec{F}_{e} = \frac{q^{2}}{4\pi \ee (2d)^{2}}\uvec{n}
	\end{equation*}
	the force takes the form of the electric force between a positive and negative charge separated by a distance $ 2d $---the same situation we used in the method of images above. 
	
\end{itemize}

\subsection{Sixth Exercise Set}

\subsubsection{Theory: Electrostatic Force}
Recall the total electrostatic force $ \vec{F} $ acting on the charges enclosed in a region of space $ V $ permeated with an electric field $ \E $ is
\begin{equation*}
	\vec{F}_{e} = \ee \oint_{\partial V}\left[\E(\E\cdot \uvec{n}) - \frac{1}{2}E^{2} \uvec{n}\right] \diff S
\end{equation*}
where $ \uvec{n} $ is the normal to the surface $ \partial V $ enclosing the charges.


\subsubsection{Force on a Conducting Spherical Shell}
\textit{We place a conducting sphere of radius $ a $ in a homogeneous electric field $ E_{0} $. Find the electrostatic force on the upper half of the sphere.}

\begin{itemize}
	\item Suppose the field points in the $ z $ direction. Recall from the previous exercise set that the potential from the sphere and electric field is
	\begin{equation*}
		U(r, \theta) = - E_{0}r \cos \theta + \frac{E_{0}a^{3}}{r^{2}}\cos \theta
	\end{equation*}
	Qualitatively, there are two main contributions to the force on the sphere: an upwards contribution in the positive $ z $ direction from the external electric field, and a downward contribution in the negative $ z $ direction from the negative charge accumulated on the bottom half of the sphere. 
	
	\item We are interested in the force on the upper half of the sphere---the next step is to choose a surface around the sphere's upper half that will simplify the force calculation. Recall the field points perpendicularly out of the conducting sphere's surface at all points. 
	
	With this perpendicular field in mind, choose a surface that tightly hugs the sphere's upper half---in this case, the field and normal to the surface $ \uvec{n} $ are parallel at all points outside the sphere. In the hemisphere plane inside the sphere, there is no field at all. These to facts simplify the dot product $ \E \cdot \uvec{n} $ in the force equation. 
	
	\item We then have $ \E \cdot \uvec{n} = E $ and $ \E( \E \cdot \uvec{n}) = E^{2}\uvec{n} $. The contribution to the force on along the sphere's outside surface is
	\begin{equation*}
		\vec{F}_{e} = \ee \int \frac{1}{2} E^{2} \uvec{n} \diff S
	\end{equation*}
	The contribution from the hemisphere plane through the sphere zero, since $ \E = 0 $ inside the sphere. 
	
	\item Next, we find the magnitude $ E $ on the sphere's surface from the potential $ U(r, \theta) $. The field points radially outwards, so we differentiate $ U $ with respect to $ r $ to get
	\begin{equation*}
		E = \pdv{U}{r} \bigg |_{r = a} = E_{0} \cos \theta + 2 E_{0} \cos \theta = 3E_{0} \cos \theta
	\end{equation*}
	Inserting $ E $ into the force equation gives
	\begin{equation*}
		\vec{F}_{e} = \ee \int \frac{1}{2}\big(3E_{0} \cos \theta\big)^{2} \uvec{n} \diff S
	\end{equation*}
	In spherical coordinates, the unit normal $ \uvec{n} $ to the sphere's surface is $ \uvec{n} = (\sin \theta \cos \phi, \sin \theta \sin \phi, \cos \theta) $. The surface element $ \diff S $ at the surface $ r = a $ is (just like in the previous exercise sets) $ \diff S = a^{2} \diff \phi \sin \theta \diff \theta $. The force on the sphere's upper half is then
	\begin{equation*}
		\vec{F}_{e} = \frac{9\ee E_{0}^{2}}{2} \int_{\theta = 0}^{\pi/2}\int_{\phi = 0}^{2\pi} \cos^{2} \theta 
		\begin{bmatrix}
			\sin \theta \cos \phi\\
			\sin \theta \sin \phi\\
			\cos \theta
		\end{bmatrix}
		(a^{2} \sin \theta \diff \theta \diff \phi) 
	\end{equation*}
	Both the $ x $ and $ y $ components will be zero---integrating $ \cos \phi $ and $ \sin \phi $ over a full period $ 2 \pi $ give zero, while the $ \phi $ contribution to the $ z $ component is $ 2\pi $.  We make this explicit with
	\begin{equation*}
		\vec{F}_{e} = \frac{9\ee E_{0}^{2}}{2} \int_{\theta = 0}^{\pi/2}\cos^{2} \theta 
		\begin{bmatrix}
			0\\
			0\\
			2\pi \cos \theta 
		\end{bmatrix}
		(a^{2} \sin \theta \diff \theta) 
	\end{equation*}
	The non-zero $ z $ component $ F_{z} $ is 
	\begin{align*}
		F_{z} &= \frac{9\ee E_{0}^{2}}{2} \int_{\theta = 0}^{\pi/2}2\pi \cos^{3} \theta (a^{2} \sin \theta \diff \theta) = 9\pi\ee E_{0}^{2}a^{2} \int_{\theta = 0}^{\pi/2} \cos^{3} \theta (\sin \theta \diff \theta)\\
		& =  9\pi\ee E_{0}^{2}a^{2} \int_{0}^{1} \cos^{3} \theta \diff [\cos \theta] = \frac{9\pi\ee a^{2} E_{0}^{2}}{4}
	\end{align*}
	The vector force can be written simply as
	\begin{equation*}
		\vec{F} = \frac{9\pi\ee a^{2} E_{0}^{2}}{4} \uvec{z}
	\end{equation*}
	In other words, the force on the upper half points upward in the positive $ z $ direction.	
	
\end{itemize}

\subsubsection{Point Charge Between Two Conducting Plates}
\textit{We place two large conducting plates at a right angle to each other, so that the plates come close together but just barely do not touch. We then place a point charge $ q $ along the line bisecting the right angle between the plates, at a perpendicular distance $ a $ from each plate. Both plates are grounded. What is the electric potential in the region bounded by the plates at large distances from the plates' intersection?}
\begin{itemize}
	\item Assume $ r = 0 $ along the line connecting the two plates. For a single plate, we could solve the problem with the method of images---see the previous exercise set. With two plates we proceed analogously, with a mirror image for each plate. Because of the two reflections from the two plates, we end up with three imaginary charges plus the one original one in a quadrupole arrangement. (This is hard to describe in words, it is best to see a picture). For large $ r $, the charge arrangement will have the field of an electric quadrupole. Solving the problem thus reduces to a multipole expansion of the electric potential to the quadrupole term.  
\end{itemize}	
	
\textbf{Theory: Multipole Expansion}
\begin{itemize}

	\item The multipole expansion of $ U $ to quadrupole order, using the Einstein summation convention, is
	\begin{equation*}
		U(r) = \frac{1}{4\pi \ee} \left[\frac{q}{r} + \frac{p_{i}r_{i}}{r^{3}} + \frac{\mathrm{Q}_{ij}r_{i}r_{j}}{r^{5}} \right]
	\end{equation*}
	where $ \vec{p} $ is the electric dipole moment and $ \mat{Q} $ is the quadrupole moment tensor. \textit{Note}: we could think of the charge $ q $ is a scalar monopole moment, creating a logical progression from scalar monopole moment to vector dipole moment to tensor quadrupole moment. 
	
	\item We find the dipole moment with 
	\begin{equation*}
		\vec{p} = \int  \diff^{3} \tilde{r} \rho(\tvec{r}) \tvec{r}
	\end{equation*}
	We find the quadrupole moment by components:
	\begin{equation*}
		\mathrm{Q}_{ij} = \int  \diff^{3} \tilde{r} \rho(\tvec{r}) \big( 3\tilde{r}_{i}\tilde{r}_{j} - \delta_{ij}\tilde{r}^{2} \big)
	\end{equation*}
	The discrete analog a configuration of $ N $ charges reads
	\begin{equation*}
		\mathrm{Q}_{ij} = \sum_{n=1}^{N} q_{n} \left(3r_{n_{i}}r_{n_{j}} - \delta_{ij}r_{n}^{2}\right)
	\end{equation*}
	Note that both definitions produces a symmetric tensor. Also important: the tensor's trace---the sum of the diagonal elements is zero:
	\begin{equation*}
		\tr\mat{Q} = \sum_{n} q_{n}\left[3x_{n}^{2} - r_{n}^{2} + 3z_{n}^{2} - r_{n}^{2} + 3z_{n}^{2} - r_{n}^{2} \right] = \sum_{n}q_{n}\left[3r_{n}^{2} - 3r_{n}^{2}\right] = 0
	\end{equation*}
\end{itemize}

\textbf{Back to Our Problem}
\begin{itemize}
	\item For our imaginary quadrupole configuration of four charges, the total charge, and thus the monopole moment, is zero. Analogously, the total dipole moment of the arrangement, which consists of two positive and two negative charges, is zero---the two dipoles cancel each other out. 
	
	From the three terms in our multipole expansion of $ U(r) $, only the quadrupole term remains. We just have to calculate the quadrupole tensor $ \mathrm{Q}_{ij} $. We label the four charges in the imaginary quadrupole configuration as 1, 2, 3, and 4, where 1 is the original positive charge in the upper right corner, 2 is the negative image charge in the upper left corner, 3 is the positive image charge in the lower left corner, and 4 is the negative image charge in the lower right corner. 
	
	\item Using the discrete formula for $ \mathrm{Q}_{ij} $, the first component $ \mathrm{Q}_{xx} $ is 
	\begin{align*}
		\mathrm{Q}_{xx} &= \sum_{n=1}^{N} q_{n} \left(3x_{n}^{2} - \delta_{ij}r_{n}^{2}\right) = q\left(3a^{2} - 2a^{2}\right) +  (-q)\left(3a^{2} - 2a^{2}\right)\\
		&= q\left(3a^{2} - 2a^{2}\right) +  (-q)\left(3a^{2} - 2a^{2}\right) = 0
	\end{align*}
	The other diagonal terms $ \mathrm{Q}_{yy} $ and $ \mathrm{Q}_{zz} $ will analogously sum to zero.
	
	All off-diagonal terms with a $ z $ component are also zero, since the charges lie in a plane with $ z = 0 $. We thus have
	$ \mathrm{Q}_{xz} = \mathrm{Q}_{zx} = \mathrm{Q}_{yz} = \mathrm{Q}_{zy} = 0$. We have just two terms left calculate: $ \mathrm{Q}_{xy} $  and $ \mathrm{Q}_{yx} $. By the tensor's symmetry, the two are equal, so we really only have one term:
	\begin{align*}
		\mathrm{Q}_{xy} &= \sum_{n=1}^{N} q_{n} \left(3x_{n}y_{n} - 0\cdot r_{n}^{2}\right) = 3qa^{2} + 3(-q)(-a^{2}) + 3qa^{2} + 3(-q)(-a^{2})\\
		 &= 12qa^{2}
	\end{align*}
	The quadrupole tensor is
	\begin{equation*}
		\mat{Q} =
		\begin{bmatrix}
			0 & 12q a^{2} & 0\\
			12q a^{2} & 0 & 0\\
			0 & 0 & 0
		\end{bmatrix}
	\end{equation*}
	As expected, the tensor is symmetric with trace $ \tr \mat{Q} = 0 $. 
	
	\item Recall the quadrupole expansion of $ U(r) $:
	\begin{equation*}
		U(r) = \frac{1}{4\pi \ee} \left[\frac{q}{r} + \frac{p_{i}r_{i}}{r^{3}} + \frac{\mathrm{Q}_{ij}r_{i}r_{j}}{r^{5}} \right]
	\end{equation*}
	In our case, with $ q = 0 $ and $ \vec{p} = 0 $, we have
	\begin{align*}
		U(r) &= \frac{1}{4\pi \ee} \left[\frac{12qa^{2}xy}{r^{5}} + \frac{12qa^{2}yx}{r^{5}} + 0 + \cdots + 0\right] = \frac{6qa^{2}}{\pi \ee} \frac{xy}{r^{5}}\\
		&=\frac{6qa^{2}}{\pi \ee} \frac{\cos \phi \sin \phi \sin^{2}\theta}{r^{3}}
	\end{align*}
	The second line uses the spherical coordinates $ x = r\cos \phi \sin \theta $ and $ y = r\sin \phi \sin \theta $. In fact, the expression for $ U(r) $ takes the exact same form as the wave function of a $ d $ electron orbital (angular momentum quantum number $ l = 2 $) in a hydrogen atom. The equipotential surfaces of $ U(r) $ have the same spatial distribution as the $ d_{xy} $  orbitals for a hydrogen wave function. 
	
\end{itemize}

\newpage

\section{Magnetostatics and Electrodynamics}

\subsection{Seventh Exercise Set}

\subsubsection{Theory: Magnetic Vector Potential and the Biot-Savart Law}
\begin{itemize}
	\item We will need to use two more Maxwell equations for magnetostatics. The first is
	\begin{equation*}
		\div \B = 0
	\end{equation*}
	This equation rules out the possibility of magnetic monopoles and allows $ \B $ to be written as the curl of a vector potential $ \\A $ as
	\begin{equation*}
		\B = \curl \A
	\end{equation*}
	The second Maxwell equation is 
	\begin{equation*}
		\curl \B = \mu_{0}\left(\vec{j} + \ee \pdv{\E}{t}\right)
	\end{equation*}
	For static situations with a constant electric field this simplifies to
	\begin{equation*}
		\curl \B = \mu_{0}\vec{j}
	\end{equation*}
	Substituting the expression $ \B = \curl \A $ into the static second Maxwell equation produces
	\begin{equation*}
		\curl \big(\curl \A\big) = \div \big(\div \A\big) - \laplacian \A = \mu_{0} \vec{j}
	\end{equation*}
	
	\item The magnetic vector potential is defined only up to a constant; we usually choose $ \A $ so that $ \div \A = 0 $. In this convention, we have
	\begin{equation*}
		\laplacian \A = - \mu_{0}\vec{j}
	\end{equation*}
	where $ \vec{j} $ is the current density vector. This equation is a vector analog of the Poisson equation $ \laplacian U = - \frac{\rho}{\ee} $ from electrostatics. Similarly to how the electrostatic potential $ U $ at a point $ \r $ in region of space with charge density $ \rho $ is found with
	\begin{equation*}
		U = \frac{1}{4 \pi \ee} \int \frac{\rho(\tvec{r})\diff^{3}\tilde{r}}{\abs{\r - \tvec{r}}}
	\end{equation*}
	The magnetic potential at a point $ \r $ in region of space with current density $ \vec{j} $ is
	\begin{equation*}
		\A(\r) = \frac{\mu_{0}}{4 \pi} \int \frac{\vec{j}(\tvec{r})\diff^{3}\tilde{r}}{\abs{\r - \tvec{r}}}
	\end{equation*}
	
	\item In problems involving one-dimensional conductors, where $ \vec{j} $ is non-zero only along the conductor, the expression for $ \vec{j}(\tvec{r})\diff^{3}\tilde{r} $ simplifies to
	\begin{equation*}
		\vec{j}(\tvec{r})\diff^{3}\tilde{r} = \vec{j}(\tvec{r}) \diff \tilde{S} \diff \tilde{l} = I \uvec{t} \diff \tilde{l}
	\end{equation*}
	where $ \uvec{t} $ is the unit normal vector tangent to the conductor, $ I $ is the current through the conductor at the point $ \tvec{r} $ and $ \diff \tilde{l} $ is a small distance element along the conductor's length. The magnetic vector potential for a one-dimensional conductor carrying a current $ I $ then simplifies to
	\begin{equation*}
		\A(\r) = \frac{\mu_{0}I}{4\pi} \int \frac{\uvec{t} \diff l}{\abs{\r - \tvec{r}}}
	\end{equation*}
	
	\item Recall $ \B = \curl \A $. Taking the curl of the general expression for $ \A $ in terms of current density $ \vec{j} $ gives general expression for the magnetic field 
	\begin{equation*}
		\B (\r) = \frac{\mu_{0}}{4\pi}\int \frac{\vec{j}(\tvec{r}) \cross (\r - \tvec{r})}{\abs{\r - \tvec{r}}^{3}} \diff^{3}\tilde{r}
	\end{equation*}
	This is a general form of the Biot-Savart law for the magnetic field $ \B $ of a current distribution.
	
\end{itemize}

\subsubsection{Magnetic Field of a Circular Current Loop}
\textit{A closed conducting loop of radius $ a $ carries current $ I $. What is the magnetic vector potential far from the conducting loop?}
\begin{itemize}
	\item Our starting point is the vector potential of a one-dimensional conductor from the theory section, i.e.
	\begin{equation*}
		\A(\r) = \frac{\mu_{0}I}{4\pi} \int \frac{\uvec{t} \diff l}{\abs{\r - \tvec{r}}}
	\end{equation*}
	We need expressions for $ \r $, $ \tvec{r} $ and $ \uvec{t} $.
	
	Assume the loop lies in the $ x, y $ plane with the $ z $ axis normal to the loop. For mathematical convenience, we rotate the $ x, y $ plane so $ \r $ lies in the $ x, z $ plane (i.e. $ \phi = 0 $); this just gives us one less non-zero component to work with, since the $ \sin \phi $ term in the $ y $ component of $ \r $ is zero. In polar coordinates $ \r $ reads
	\begin{equation*}
		\r = \big (r \sin \theta, 0, r \cos \theta \big )
	\end{equation*}
	where $ \theta $ is the angle between $ \r $ and the $ z $ axis. 
	
	The integration variable $ \tvec{r} $, which runs over the current loop in the $ x, y $ plane, reads
	\begin{equation*}
		\tvec{r} = \big(a \cos \tilde{\phi}, a \sin \tilde{\phi}, 0 \big)
	\end{equation*}
	Where $ \tilde{\phi} $ is the azimuthal angle between $ \tvec{r} $ and the $ x $ axis. 
	
	Finally, the expression for $ \uvec{t} $, the tangent to the current loop, is
	\begin{equation*}
		\uvec{t} = \big( -\sin \tilde{\phi},  \cos \tilde{\phi}, 0\big)
	\end{equation*}
	The small distance element is $ \diff \tilde{l} = a \tilde{\phi}$ .
	
%	\item We can now put the pieces together in the equation for $ \A $. First,
	\begin{align*}
		\abs{\r - \tvec{r}} &= \sqrt{(r \sin \theta - a \cos \tilde{\phi} )^{2} + a^{2}\sin^{2}\tilde{\phi} + r^{2}\cos^{2} \theta}\\
		& = \sqrt{r^{2} + a^{2} - 2ra \sin \theta \cos \tilde \phi}\\
		&\approx \sqrt{r^{2} - 2ra \sin \theta \cos \tilde \phi} = r \sqrt{1 - \frac{2a}{r}\sin \theta \cos \tilde{\phi}}
	\end{align*}
	where the last lines uses $ r \gg a $ (recall we're interested in the solution far from the conducting loop). We now have, again using $ a \ll r \implies \frac{a}{r} \ll 1 $,
	\begin{equation*}
		\frac{1}{\abs{\r - \tvec{r}}} = \frac{1}{r} \left(1 - \frac{2a}{r}\sin \theta \cos \tilde{\phi}\right)^{-1/2} \approx \frac{1}{r}\left(1 + \frac{1}{2}  \frac{2a}{r}\sin \theta \cos \tilde{\phi}\right)
	\end{equation*}
	Substituting the expressions for $ \frac{1}{\abs{\r - \tvec{r}}} $, $ \uvec{t} $ and $ \diff \tilde{l} $ into expression for $ \A $ gives
	\begin{align*}
		\A(\r) &= \frac{\mu_{0}I}{4\pi} \int \frac{\uvec{t} \diff l}{\abs{\r - \tvec{r}}} = \frac{\mu_{0}I}{4\pi} \frac{a}{r} \int_{0}^{2\pi} \diff \tilde{\phi}  
		\begin{bmatrix}
			- \sin \tilde{\phi}\\
			\cos \tilde{\phi}\\
			0
		\end{bmatrix}
		\left(1 + \frac{1}{2}  \frac{2a}{r}\sin \theta \cos \tilde{\phi}\right)
	\end{align*}
	
	\item The integrals conveniently simplify, since we are integrating sinusoidal terms over an entire period. Only the integral of $ \cos^{2} \tilde{\phi} $ in the $ y $ component is nonzero. We end up with $ A_{x} = A_{z} = 0 $ and
	\begin{equation*}
		A_{y}(r) = \frac{\mu_{0}I}{4\pi} \frac{a}{r} \int_{0}^{2\pi}
		 \frac{a\sin \theta}{r} \cos^{2} \tilde{\phi} \diff \tilde{\phi} = \frac{\mu_{0}I}{4} \frac{a^{2}}{r^{2}}  \sin \theta 
	\end{equation*}
	or, in vector form,
	\begin{equation*}
		\A(\r) = \frac{\mu_{0}I}{4} \frac{a^{2}}{r^{2}}  \sin \theta \uvec{y} = \frac{\mu_{0}I}{4} \frac{a^{2}}{r^{2}} \uvec{z} \cross \uvec{r}
	\end{equation*}
	The second expression is preferable: the first, in terms of $ \uvec{y} $, holds only with $ x, y $ plane rotated so $ \phi = 0 $ and $ \r $ lies in the $ x, z $ plane. The second, which uses $ \sin \theta \uvec{y} = \uvec{z} \cross \uvec{r} $, holds for any orientation of the $ x, y $ plane. 
	
	
	Finally, using $ I \pi a^{2} $ = $ \abs{\vec{m}} $ (where $ \m $ is the loop's magnetic dipole moment, which points in the direction of the loop's normal), the expression for $ \A $ simplifies to
	\begin{equation*}
		\A(\r) = \frac{\mu_{0} \m \cross \uvec{r}}{4\pi r^{2}} = \frac{\mu_{0} \m \cross \r}{4\pi r^{3}}
	\end{equation*}
	This is the same form of magnetic vector potential as for a magnetic dipole. In other words, a circular current loop behaves as magnetic dipole at long distances.
	
\end{itemize}

\subsubsection{Magnetic Field of a Rotating Charged Disk}
\textit{Consider a charged, insulating disk with uniform surface charge density $ \sigma $ and radius $ a $. The disk rotates uniformly about an axis through its center with constant angular speed $ \omega $. Find the magnetic field along the axis of rotation.}
\begin{itemize}
	\item We choose a coordinate system so the disk lies in the $ x, y $ plane and the rotation axis coincides with the $ z $ axis, so $ \bm{\omega} = (0, 0, \omega) $. Start with the general Biot-Savart law
	\begin{equation*}
		\B (\r) = \frac{\mu_{0}}{4\pi}\int \frac{\vec{j}(\tvec{r}) \cross (\r - \tvec{r})}{\abs{\r - \tvec{r}}^{3}} \diff^{3}\tilde{r}
	\end{equation*}
	and, like in the previous problem, find expressions for each vector quantity in the equation. The expression for $ \r $ along the $ z $ axis is simply $ \r = (0, 0, z) $, while the expression for $ \tvec{r} $, which lies in the disk in the $ x, y $ plane, is
	\begin{equation*}
		\tvec{r} = \big(\tilde{r} \cos \tilde{\phi}, \tilde{r} \sin \tilde{\phi}, 0 \big)
	\end{equation*}
	The difference of $ \r $ and $ \tvec{r} $ and its magnitude is
	\begin{equation*}
		\r - \tvec{r} = \big(-\tilde{r} \cos \tilde{\phi}, - \tilde{r} \sin \tilde{\phi}, z \big) \eqtext{and} \abs{\r - \tvec{r}} = \sqrt{\tilde{r}^{2} + z^{2}}
	\end{equation*}
	Finally, the current density $ \vec{j} $, which points tangent to the disk's rotation, is
	\begin{equation*}
		\vec{j} = j(-\sin \t{\phi}, \cos \t{\phi},0 )
	\end{equation*}
%	The volume element $ \diff^{3}\tilde{r} $ is $ \diff^{3}\tilde{r} = \t{r} \diff \t{\phi} \diff \t{r} \diff h$ where $ \diff h $ is the disk's thickness. 
	
	\item Using the expressions for our vector quantities, the cross product $ \vec{j}(\tvec{r}) \cross (\r - \tvec{r}) $ is
	\begin{equation*}
		\vec{j}(\tvec{r}) \cross (\r - \tvec{r}) = j
		\begin{bmatrix}
			-\sin \t{\phi}\\
			\cos \t{\phi}\\
			0
		\end{bmatrix}
		\begin{bmatrix}
			-\tilde{r} \cos \tilde{\phi}\\
			- \tilde{r} \sin \tilde{\phi}\\
			z
		\end{bmatrix}
		= j
		\begin{bmatrix}
			z \cos \t{\phi}\\
			z \sin \t{\phi}\\
			\t{r}
		\end{bmatrix}
	\end{equation*}
	Substituting $ \vec{j}(\tvec{r}) \cross (\r - \tvec{r}) $ and $  \abs{\r - \tvec{r}} $ into the Biot-Savart law gives
	\begin{equation*}
		\B(z) = \frac{\mu_{0}}{4 \pi} \int \frac{j \diff^{3} \t{r}}{\big(\tilde{r}^{2} + z^{2}\big)^{3/2}} 
		\begin{bmatrix}
			z \cos \t{\phi}\\
			z \sin \t{\phi}\\
			\t{r}
		\end{bmatrix}
	\end{equation*}
	Next, we write $ j \diff^{3}\t{r} = j \diff \t{S} \diff \t{l} = j \diff \t{S} (\t{r} \diff \t{\phi} ) = \diff I \t{r} \diff \t{\phi} $. Note that the product $ j \diff \t{S} $ is the current element $ \diff I $ in the surface element $ \diff \tilde{S} $. The current $ \diff I $ at the radius $ \t{r} $ on the disk rotating with period $ t_{0} = \frac{2\pi}{\omega} $ is
	\begin{equation*}
		\diff I = \frac{\diff q}{t_{0}} = \frac{\diff q}{2\pi} \omega  = \frac{(\sigma 2\pi \t{r}\diff \t r)}{2\pi} \omega = \sigma \omega \t{r} \diff \t{r}
	\end{equation*}
	Substituting $ j \diff^{3}\t{r} =  \diff I \t{r} \diff \t{\phi}  =  \sigma \omega \t{r}^{2} \diff \t{r} \diff \t{\phi}  $ into the Biot-Savart law gives
	\begin{equation*}
		\B(z) = \frac{\mu_{0}}{4 \pi} \int_{0}^{a} \diff \t{r} \int_{0}^{2\pi} \diff \t{\phi}	\frac{\sigma \omega \t{r}^{2}}{\big(\tilde{r}^{2} + z^{2}\big)^{3/2}} 
		\begin{bmatrix}
			z \cos \t{\phi}\\
			z \sin \t{\phi}\\
			\t{r}
		\end{bmatrix}
	\end{equation*}
	The first and second components of $ \B $ contain integrals $ \cos $ and $ \sin $ terms over a full period---the result is zero. After integrating over $ \t{\phi} $, the magnetic field simplifies to
	\begin{equation*}
		\B(z) = \frac{\mu_{0}}{4 \pi} \int_{0}^{a} \diff \t{r} \frac{\sigma \omega \t{r}^{2}}{\big(\tilde{r}^{2} + z^{2}\big)^{3/2}} 
		\begin{bmatrix}
			0\\
			0\\
			2\pi \t{r}
		\end{bmatrix}
	\end{equation*}
	Only the $ z $ component of $ \B $ is non-zero; it is
	\begin{equation*}
		B_{z}(z) = \frac{\mu_{0}}{2} \int_{0}^{a} \diff \t{r} \frac{\sigma \omega \t{r}^{3}}{\big(\tilde{r}^{2} + z^{2}\big)^{3/2}} 
	\end{equation*}
	In terms of the new variable $ u = \tilde{r}^{2} + z^{2} $, the integral evaluates to
	\begin{align*}
		B_{z}(z) &= \frac{\mu_{0}}{2} \frac{\sigma \omega}{2}\int_{z^{2}}^{z^{2} + a^{2}}\frac{u-z^{2}}{u^{3/2}} \diff u = \frac{\mu_{0}\sigma \omega}{4} \left[2u^{1/2} + 2z^{2}u^{-1/2}\right]_{a^{2}}^{z^{2}+a^{2}}\\
		 & = \frac{\mu_{0}\sigma \omega}{2} \left(\sqrt{z^{2} + a^{2}} - z + \frac{z^{2}}{\sqrt{z^{2} + a^{2}}} - z\right)\\
		 & = \frac{\mu_{0}\sigma \omega }{2}\left(\frac{2z^{2} + a^{2}}{\sqrt{z^{2} + a^{2}}} - 2z\right)
	\end{align*}
	The magnetic field along the $ z $ axis is thus $ \B = (0, 0, B_{z}) $ with $ B_{z} $ as above.
	
	\item Next, we consider the limit case $ z \gg a $, far from the rotating disk. Expanding the square root to fourth order in the small quantity $ \frac{a}{z} $, multiplying out and simplifying like terms gives
	\begin{align*}
		B_{z} &= \frac{\mu_{0}\sigma \omega}{2} \left(\frac{2z^{2} + a^{2}}{z\sqrt{1 + \frac{a^{2}}{z^{2}}}} - 2z\right) \approx \frac{\mu_{0}\sigma \omega}{2} \left[\left(2z + \frac{a^{2}}{z}\right)\left(1 - \frac{a^{2}}{2z^{2}} + \frac{3}{8}\frac{a^{4}}{z^{4}}\right) - 2z \right]\\
		& = \frac{\mu_{0}\sigma \omega}{2} \left[2z + \frac{a^{2}}{z} - \frac{a^{2}}{z} - \frac{1}{2}\frac{a^{4}}{z^{3}} + \frac{3}{4}\frac{a^{4}}{z^{3}} + \frac{3}{8}\frac{a^{6}}{z^{5}} - 2z\right] = \frac{\mu_{0}\sigma \omega}{2} \left[ \frac{1}{4}\frac{a^{4}}{z^{3}} + \frac{3}{8}\frac{a^{6}}{z^{5}}\right] 
	\end{align*}
	Neglecting the highest-order $ \frac{a^{6}}{z^{5}} $ term gives the simple result
	\begin{equation*}
		B_{z} \approx  \frac{\mu_{0}\sigma \omega}{8} \frac{a^{4}}{z^{3}}, \qquad z \gg a
	\end{equation*}
	In other words, far from the disk, the magnetic field falls off as $ z^{-3} $, just like the field of a magnetic dipole.
	
	\item Next, we will try to write the magnetic field in the form $ B_{z} \propto \frac{\abs{\m}}{z^{3}} $ where $ \m $ is the disk's magnetic dipole moment. Integrating over concentric rings with area $ S $ carrying current $ \diff I $, the disk's magnetic dipole moment is
	\begin{equation*}
		\abs{\m} = \int S \diff I = \int_{0}^{a} (\pi \t{r}^{2}) \cdot (\sigma \omega \t{r}\diff \t{r}) = \pi \sigma \omega \int_{0}^{a} \t{r}^{3} \diff \t{r} = \frac{\pi}{4}\sigma \omega a^{4}
	\end{equation*}
	Comparing this expression for $ \abs{\m} $ to the similar expression for $ B_{z} $ leads to 
	\begin{equation*}
		B_{z} \approx  \frac{\mu_{0}\sigma \omega}{8} \frac{a^{4}}{z^{3}} = \frac{\mu_{0}\abs{\m}}{2\pi z^{3}}
	\end{equation*}	
	which is in the desired form $  B_{z} \propto \frac{\abs{\m}}{z^{3}} $. The general form for the magnetic field of a magnetic dipole is
	\begin{equation*}
		\B(\r) = \frac{\mu_{0}}{4\pi}\frac{3(\m \cdot \r)\r - \m r^{2}}{r^{5}}
	\end{equation*}
	In fact, this expression is equivalent to our result $ B_{z}(z) = \frac{\mu_{0}\abs{\m}}{2\pi z^{3}} $. Since $ \m $ and $ \r $ both point along the $ z $ axis, their dot product is $ \m \cdot \r = \abs{m}r $. Along the $ z $ axis, $ \r = (0, 0, z) $ and the general expression for the dipole magnetic field simplifies to
	\begin{equation*}
		\B(\r) = \frac{\mu_{0}}{4\pi} \frac{3\abs{m}z^{2} - \abs{m}z^{2}}{z^{5}}\uvec{z} = \frac{\mu_{0}\abs{\m}}{2\pi z^{3}} \uvec{z},
	\end{equation*}
	in agreement with our expression for $ B_{z} $ for $ z \gg a $.
	
\end{itemize}

\subsection{Eighth Exercise Set}

\textbf{Theory: Magnetic Force}\vspace{2mm}

The magnetic force $ \vec{F} $ on the matter contained in the region of space enclosed in the region $ V $ and permeated by the magnetic field $ \B $ is
\begin{equation*}
	\vec{F} = \frac{1}{\mu_{0}}\oint_{\partial V}\left [\B(\B \cdot \uvec{n}) - \frac{1}{2}\B^{2}\uvec{n}\right ] \diff S
\end{equation*}
where $ \uvec{n} $ is the normal vector to the region's boundary $ \partial V $.

\subsubsection{Magnetic Force in a Coaxial Cable}
\textit{A long coaxial cable consists of a thin inner wire and outer sheath with radius $ a $. The inner wire carries a current $ I $, and we connect the sheath to the inner wire at the cable's ends so that the current $ I $ returns along the outer sheath in the opposite direction as the current along the inner wire. Find the magnetic force per unit length on the outer sheath.}

\begin{itemize}
	\item There are two contributions to the magnetic force on the sheath: the repulsive, radially outward force between the sheath and the inner wire and an attractive ``surface tension'' force distributed across the sheath's surface, which carries uniformly distributed current $ I $ (think of the sheath as a collection of parallel conducting wires, which attract each other).
	
	\item Consider one-half of the sheath, which forms a semicircular cross section. There are downward forces $ F_{1} $ at end of the semicircle and an upward magnetic force $ F_{m} $ acting on the top of the sheath. In equilibrium, the forces are related by $ F_{m} = 2F_{1} $.
	
	\item First, we find the magnetic field in the conductor. Inside the sheath, Ampere's law with a circular path encircling the inner wire reads
	\begin{equation*}
		\oint \vec{B} \cdot \diff \bm{l} = \mu_{0} I \implies B (2\pi r) = \mu_{0}I \implies B = \frac{\mu_{0}I}{2\pi r}
	\end{equation*}
	Outside the conductor, the net current enclosed by a circular path encircling both the outer sheath and inner wire is zero, and the Ampere's law reads
	\begin{equation*}
		\int \vec{B} \cdot \diff \bm{l} = \mu_{0} (+I - I) = 0 \implies B = 0
	\end{equation*}
	In other words, there is no magnetic field outside the conductor.
	
	\item With the coaxial cable's magnetic field known, we now find the magnetic force $ \vec{F}_{m} $ on a semi-circular half of the outer sheath.
	\begin{equation*}
		\vec{F}_{m} = \frac{1}{\mu_{0}}\oint_{\partial V}\left [\B(\B \cdot \uvec{n}) - \frac{1}{2}B^{2}\uvec{n}\right ] \diff S
	\end{equation*}
	We choose an integration surface tightly hugging the half-sheath and work in cylindrical coordinates $ r, \phi $. Outside the sheath, the magnetic field is zero and there is no contribution to $ \bm{F}_{m} $. Inside the sheath, the magnetic field is tangent to the semicircle, so $ \vec{B}\cdot \uvec{n} = 0 $, and the force integral reads
	\begin{equation*}
		\vec{F}_{m} = \frac{1}{\mu_{0}} \int \big[0 - \frac{1}{2}B^{2}\uvec{n}\big] \diff S 
	\end{equation*}
	The magnetic field at along the sheath (where $ r = a $) is constant and equal to $ B = \frac{\mu_{0}I}{2\pi a} $ and can be moved outside the integral. The normal vector in cylindrical coordinates reads 
	\begin{equation*}
		\uvec{n} = (- \cos \phi, -\sin \phi, 0)
	\end{equation*} 
	while the surface element is $ \diff S = a l \diff \phi $. The force integral reads
	\begin{equation*}
		\vec{F}_{m} = - \frac{1}{2}\frac{\mu_{0}I^{2}}{4\pi^{2}a^{2}} \int_{0}^{\pi}
		\begin{bmatrix}
			- \cos \phi\\
			 -\sin \phi\\
			  0
		\end{bmatrix}
		l a \diff \phi
	\end{equation*}
	The $ x $ component with $ \cos \phi $ integrates to zero over $ \phi \in [0, \pi] $. Only the $ y $ component is nonzero, and the vector force reads
	\begin{equation*}
	 	\vec{F}_{m} = +\frac{1}{2}\frac{\mu_{0}I^{2}}{4\pi^{2}a^{2}} (2a l)\uvec{y} = \frac{\mu_{0}I^{2}l}{4\pi^{2}a} \uvec{y}
	\end{equation*}
	
	\item The forces $ F_{1} $ at the two ends of the semicircle are then
	\begin{equation*}
		F_{1} = \frac{F_{m}}{2} =  \frac{\mu_{0}I^{2}l}{8\pi^{2}a} \implies \frac{F}{l} = \frac{\mu_{0}I^{2}}{8\pi^{2}a} 
	\end{equation*}
\end{itemize}


\subsubsection{Tension in a Toroidal Inductor}
\textit{A toroidal inductor with $ N $ coils radius $ r_{1} $ and cross-sectional radius $ r_{2} $ carries a current $ I $. Find the tension force on a single coil if $ r_{1} \gg r_{2} $.}
\begin{itemize}
	\item First, we find the magnetic field inside the inductor with Ampere's law using a closed circular path of radius $ r \in (r_{1}, r_{1} + r_{2}) $ in the inductor's equatorial plane. Ampere's law reads
	\begin{equation*}
		B_{\text{in}} (2\pi r) = \mu_{0} (NI) \implies B_{\text{in}} = \frac{\mu_{0}NI}{2\pi r}
	\end{equation*}
	Outside the inductor, with a circular path of radius $ r > r_{1} + r_{2} $ in the inductor's equatorial plane, Ampere's law reads
	\begin{equation*}
		B_{\text{out}} (2\pi r) = \mu_{0} (NI + N(-I)) = 0 \implies B_{\text{out}} = 0
	\end{equation*}
	
	\item Consider a semicircular half of a single inductor coil. As in the previous problem, the magnetic force with magnitude $ F_{m} $ acts upwards on the top of the semicircle, while two ``surface tension'' forces $ F_{1} = \frac{F_{m}}{2} $ act downward at the semicircle's two ends. As usual, we find the magnetic force $ \vec{F}_{m} $ using
	\begin{equation*}
		\vec{F}_{m} = \frac{1}{\mu_{0}}\oint_{\partial V}\left [\B(\B \cdot \uvec{n}) - \frac{1}{2}B^{2}\uvec{n}\right ] \diff S
	\end{equation*}
	For the integration surface (awkward to describe, best to see a picture) we choose a surface that looks like a a coin cut in half, basically a thin three-dimensional extension of a semicircle enclosing the coil's semicircular upper half. We split the surface into four parts: the left and right semicircular faces, the circular ribbon along the surface's outer radius, and the thin rectangular plane along the surface's bottom. 
	
	The circular upper ribbon occurs just outside the inductor coil (where $ B = 0 $) and thus does not contribute the magnetic force. The left and right semicircular faces have equal magnitude and opposite-sign contributions, since the normal to the surface changes sign for each face. 
	
	\item Only the rectangular plane has a nonzero contribution to $ \vec{F}_{m} $. Under the assumption $ r_{1} \gg r_{2} $, the magnetic field along the plane simplifies to
	\begin{equation*}
		B = \int_{r_{1}}^{r_{1} + r_{2}} \frac{\mu_{0}NI}{2\pi \tilde{r}} \diff \tilde{r} \approx \frac{\mu_{0}NI}{2\pi r_{1}}
	\end{equation*}
	$ \vec{B} $ points along the toroid's longitudinal axis (into the plane of a cross-sectional coil) and is perpendicular to the normal $ \uvec{n} $ to the planar integration surface, so $ \uvec{B} \cdot \uvec{n} = 0 $. The magnetic force simplifies to
	\begin{align*}
		\vec{F}_{m} &= \frac{1}{\mu_{0}}\left [0 - \frac{1}{2}B^{2}\uvec{n}\right ] \diff S = - \frac{1}{2\mu_{0}} \left(\frac{\mu_{0}NI}{2\pi r_{1}}\right)^{2} \uvec{n} \int \diff S \\
		&= - \frac{1}{2\mu_{0}} \left(\frac{\mu_{0}NI}{2\pi r_{1}}\right)^{2} \uvec{n} \left (r_{2} \frac{2\pi r_{1}}{N}\right ) = - \frac{\mu_{0}NI^{2}}{2\pi}\frac{r_{2}}{r_{1}} \uvec{n}
	\end{align*}
	$ \uvec{n} $ points downward, so $ -\uvec{n} $ and thus $ \vec{F}_{m} $ point upward and pull the coil apart. The magnitude of the tension on the coil is then
	\begin{equation*}
		F_{1} = \frac{F_{m}}{2} =  \frac{\mu_{0}NI^{2}}{4\pi}\frac{r_{2}}{r_{1}}
	\end{equation*}
	
\end{itemize}

\subsubsection{Resistance of a Thin Conducting Plate}
\textit{The conducting plate is half of a thin annulus with inner radius $ r_{1} $, outer radius $ r_{2} $, thickness $ h $ and resistivity $ \sigma $ with electrodes at each end.  Find the plates electric resistance $ R $ when we establish a potential difference $ U_{0}$ between the electrodes at the plate's ends.}

\begin{itemize}
	\item We first find the current density in the conducting plate using Ohm's law
	\begin{equation*}
		\vec{j} = \sigma \E 
	\end{equation*}
	where $ \vec{j} $ and $ \E $ are the current density and electric field in the conductor. We write $ \vec{j} $ in terms of the continuity equation
	\begin{equation*}
		\div \vec{j} + \pdv{\rho}{t} = 0
	\end{equation*}
	For a stationary charge distribution, we have $ \div \vec{j} = 0 $. We then take the divergence of both sides of the equation $ \vec{j} = \sigma \E = \rho  $ and apply $ \E = - \grad U $ to get
	\begin{equation*}
		\div \vec{j} = \sigma \div \E = - \sigma \laplacian U = \div \vec{j} = 0 \implies \laplacian U = 0
	\end{equation*}
	We end up with a Laplace equation for $ U $ in the conductor. The plan is to find $ U(\r) $, then $ \E(r) $, then $ \vec{j} $ via $ \vec{j} = \sigma \E $, then $ I $ and finally resistance with $ R = \frac{U}{I} $. 
	
	\item The general solution of the Laplace equation in cylindrical coordinates (including the $ m = 0 $ term) is
	\begin{align*}
		U(r, \phi) &= \sum_{m=1}^{\infty} \big[A_{m}\cos(m\phi) + B_{m}\sin(m \phi)\big] \cdot \big[C_{m}r^{m} + D_{m}r^{-m}\big]\\
		& \qquad \qquad  + (a \phi + b)(c \ln r + d)
	\end{align*}
	To find a unique solution, we need boundary conditions for our particular problem. At the first electrode at $ \phi = 0 $, the electric potential is constant; we'll set $ U(r, 0) = 0 $  for convenience. At the second electrode, the potential is $ U(r, \pi)  = -U_{0}$ to make a potential difference $ U_{0} $ between electrodes (we choose $ -U_{0} $ so the current runs in the direction of increasing $ \phi $). 
	
	Along the annulus's semicircular boundaries the electric field must be tangent to the surface to satisfy $ \div \vec{j} = 0 $. Along these surfaces the radial component of both $ \vec{j} $ and $ \E $ is zero; $ E_{r} = 0 $ gives the boundary condition
	\begin{equation*}
		E_{r} = - \eval{\pdv{U}{r}}_{r_{1}, r_{2}} = 0
	\end{equation*}
	
	\item The general solution for $ U(r, \phi) $ satisfies the boundary condition $ \eval{\pdv{U}{r}}_{r_{1}, r_{2}} = 0 $ for all $ \phi $ only if $ C_{m} = D_{m} = c = 0 $ (try finding $ \eval{\pdv{U}{r}}_{r_{1}, r_{2}}  $ to see for yourself). With $  C_{m} = D_{m} = c = 0 $, the expression for $ U(r, \phi) $ simplifies to
	\begin{equation*}
		U(r, \phi) = (a \phi + b)d \equiv \t{a}\phi + \t{b}
	\end{equation*}
	The boundary condition $ U(r, 0) = 0 $ gives
	\begin{equation*}
		U(r, 0) \equiv = 0 + \t{b} \implies \t{b} = 0
	\end{equation*}
	The final boundary condition $ U(r, \pi) = -U_{0} $ gives
	\begin{equation*}
		U(r, \pi) \equiv = - U_{0} \implies \t{a} = -\frac{U_{0}}{\pi} \implies U(r, \phi) = - \frac{U_{0}}{\pi} \phi
	\end{equation*}
	In other words, the electric potential is a linear function of $ \phi $. 
	
	\item With $ U(r, \phi) $ known, we find the tangential electric field $ E_{t} $ with
	\begin{equation*}
		E_{t} = - \frac{1}{r}\pdv{U}{\phi} = \frac{U_{0}}{\pi r},
	\end{equation*}
	from which we find the tangential current density $ j_{t} $ with
	\begin{equation*}
		j_{t} = \sigma E_{t} = \frac{\sigma U_{0}}{\pi r}
	\end{equation*}
	We find the total current $ I $ from the current density via
	\begin{equation*}
		I = \int j_{t} \diff S = \int_{r_{1}}^{r_{2}} \left(\frac{\sigma U_{0}}{\pi r}\right) \cdot h \diff r = \frac{\sigma U_{0} h}{\pi} \ln \frac{r_{2}}{r_{1}}
	\end{equation*}
	where $ h $ is the plate's thickness. The electrical resistance is then
	\begin{equation*}
		R = \frac{U_{0}}{I} = \frac{\pi}{\sigma h \ln \frac{r_{2}}{r_{1}}}
	\end{equation*}
\end{itemize}

\subsection{Ninth Exercise Set}

\textbf{Theory: Inductance}
\begin{itemize}
	\item Inductance is the proportionality constant between magnetic flux $ \Phi $ and current $ I $. For a single object, e.g. a current loop, the relationship between $ \Phi $ and $ I $ reads
	\begin{equation*}
		\Phi_{1} = L_{11} I_{1}
	\end{equation*}
	where the quantity $ L_{11} $ is called self-inductance. 
	
	\item For a system of two current-carrying loops, where the current through one loop generates a magnetic field and thus magnetic flux through the second loop, and vice versa, the relationships between $ \Phi $ and $ I $ read
	\begin{equation*}
		\Phi_{1} = L_{12}I_{2} \eqtext{and}  \Phi_{2} = L_{21}I_{1}
	\end{equation*}
	where the quantities $ L_{12} $ and $ L_{21} $ are the loops' mutual inductances. Without derivation, we state that $ L_{12} = L_{21} $ for reasons of symmetry.
	
\end{itemize}





\subsubsection{Mutual Inductance}


\textit{Consider two long, parallel wires of length $ l $ separated by a distance $ d \ll l $ and connected at their endpoints to form a long conducting loop. We place a square frame with side length $ \sqrt{2}d $ between the wires and supply the frame with an alternating current source $ I_{1} = I_{1_{0}}\sin \omega t $. Find the system's mutual inductivity and the induced current in the wires. Neglect Ohmic losses, and assume $ a \ll d $ where $ a $ is the wires' radius.}



\vspace{2mm}
\textbf{Finding Mutual Inductance}
\begin{itemize}
	\item We label the frame as object 1 and the wires as object 2 and introduce a coordinate $ y $ separating the parallel wires so that the bottom wire occurs at $ y = 0 $ and the top wire at $ y = d $. To find the system's mutual inductance $ L_{12} $, we consider the hypothetical magnetic flux $ \Phi_{1} $ through the frame due to a current $ I_{2} $ in the parallel wires. We then find $ L_{12} $ with
	\begin{equation*}
		L_{12} = \frac{\Phi_{1}}{I_{2}}
	\end{equation*}
	
	\item If the parallel wires carry a current $ I_{2} $ (in opposite directions, since they form a closed loop), the corresponding magnetic field is
	\begin{equation*}
		B = \frac{\mm I_{2}}{2\pi} \left(\frac{1}{y} + \frac{1}{d-y}\right) =  \frac{\mm I_{2}}{2\pi}  \frac{d}{y(d-y)}
	\end{equation*}

	
	\item We find magnetic flux with $ \Phi = \int \B \cdot \diff \vec{S} $; the dot product drops because $ \B $ is parallel to the frame's cross section $ \diff \vec{S} $. The square frame's surface element is $ \diff S = 2y\diff y $ (the frame's width is 2y), and the magnetic flux through the square frame is
	\begin{align*}
		\Phi_{1} &= \int B \diff S = 2\int_{0}^{d/2} \frac{\mm I_{2}d}{2\pi}  \frac{2y\diff y}{y(d-y)} = -\frac{2\mm I_{2}d}{\pi} \ln(d-y)\big |_{0}^{d/2}\\
		& =  \frac{2\mm I_{2}d}{\pi} \left[\ln d - \ln\frac{d}{2}\right] =  \frac{2\mm d \ln 2}{\pi}  I_{2}
	\end{align*}
	Note the use of symmetry---we integrate only from $ 0 $ to $ d/2 $ and multiply by two.
	
	The mutual inductance---the proportionality between $ \Phi $ and $ I_{2} $ is 
	\begin{equation*}
		L_{12} = \frac{\Phi_{1}}{I_{2}} = \frac{2\ln 2}{\pi} \mm d
	\end{equation*}
	Note that $ L_{12} $ depends only on system's geometry. 
\end{itemize}

\textbf{Induced Current in the Parallel Wires}
\begin{itemize}
	\item Recall the square frame carries an alternating current
	\begin{equation*}
		I_{1}(t) = I_{1_{0}}\sin \omega t,
	\end{equation*}
	so we expect the induced current $ I_{2}(t) $ in the parallel wires to alternate with the same frequency $ \omega $ and a general form
	\begin{equation*}
		I_{2}(t) = I_{2_{0}}\sin(\omega t)
	\end{equation*}
	We will find the ratio of current amplitudes $ I_{2_{0}} / I_{1_{0}} $.
	
	\item We'll solve the problem as follows: use the frame current $ I_{1} $ to find the magnetic flux $ \Phi_{2} $ through the parallel wires, then use $ \Phi_{2} $ to find induced voltage $ U_{2} $ in the wires, and finally use $ U_{2} $ to find the induced current $ I_{2} $.
	
	
	We find $ \Phi_{2} $ from $ \Phi_{2} = L_{21}I_{1} $ and the symmetry relation $ L_{12} = L_{21} $, i.e.
	\begin{equation*}
		\Phi_{2} = L_{21}I_{1} = L_{12} I_{1}
	\end{equation*}
	where $ L_{12} $ was found the first part of the problem. We then find $ U_{2} $ with
	\begin{equation*}
		U_{12} = - \dot{\Phi}_{2} = - L_{12}\dot{I}_{1}
	\end{equation*}
	
	\item Next, we find $ I_{2} $ from $ U_{2} $ from the general circuit equation
	\begin{equation*}
		U = RI + L_{\text{s}}\dot{I},
	\end{equation*}
	which relates the voltage in a loop of resistance $ R $ and self-inductance $ L_{\text{s}} $ to the current $ I $ through the loop. In our case, neglecting resistance, we have
	\begin{equation*}
		U_{12} \approx L_{22} \dot{I}_{2}
	\end{equation*}
	where $ L_{22} $ is the parallel wire's self-inductance. Substituting $ U_{12} $ into the earlier expression $ U_{12} = -L_{12}\dot{I}_{1} $ gives
	\begin{equation*}
		-L_{12} \dot{I}_{1} = L_{22}\dot{I}_{2} 
	\end{equation*}
	If we assume both $ I_{1} $ and $ I_{2} $ are sinusoidal with amplitudes $ I_{1_{0}} $ and $ I_{2_{0}} $, the above reduces to
	\begin{equation*}
		\frac{I_{2_{0}}}{I_{1_{0}}} = \frac{L_{12}}{L_{22}}
	\end{equation*}
	\textit{What about the minus sign?}
	
	\item To find the current amplitude ratio, we just need to find the parallel wires self-inductance $ L_{22} $ using $ \Phi_{2} = L_{22}I_{2} $
	
	If we send a hypothetical current current $ I_{2} $ through the wires, the corresponding magnetic field is, as before,
	\begin{equation*}
		B = \frac{\mm I_{2}}{2\pi} \left(\frac{1}{y} + \frac{1}{d-y}\right) 
	\end{equation*}
	The field and surface are parallel, so $ \vec{B} \cdot \diff \vec{S} = B \diff S $, and the magnetic flux is
	\begin{equation*}
		\Phi_{2} = \int B \diff S = \frac{\mm I_{2} l d}{2 \pi}\int_{a}^{d-a}\left(\frac{1}{y} + \frac{1}{d-y}\right)\diff y
	\end{equation*}
	where $ a $ is the wire's radius. The integral evaluates to
	\begin{equation*}
		\Phi_{2} = \frac{\mm I_{2} l}{2 \pi}\left[\ln \frac{d-a}{a} - \ln \frac{a}{d-a}\right] = \frac{\mm I_{2} l}{\pi} \ln \frac{d-a}{a}
	\end{equation*}
	Applying $ d \gg a $ we have
	\begin{equation*}
		\Phi_{2} = \frac{\mm I_{2} l}{\pi} \ln \frac{d}{a}
	\end{equation*}
	The self-inductance---the proportionality constant between $ \Phi_{2} $ and $ I_{2} $---is thus
	\begin{equation*}
		L_{22} = \frac{\mm l}{\pi}\ln \frac{d}{a}
	\end{equation*}
	
	\item With $ L_{22} $ known, we return to the current amplitude ratio to get
	\begin{equation*}
		\frac{I_{2_{0}}}{I_{1_{0}}} = \frac{\frac{2\ln 2}{\pi} \mm d}{\frac{\mm l}{\pi}\ln \frac{d}{a}} = \frac{2\ln 2}{\frac{l}{d} \ln \frac{d}{a}}
	\end{equation*}
	To get a better feel for the numbers involved, if we assume $ l/d = d/a = 10 $, we have 
	\begin{equation*}
		\frac{I_{2_{0}}}{I_{1_{0}}} \approx 0.06
	\end{equation*}
\end{itemize}

\subsubsection{Cabrera Experiment---Magnetic Monopoles}
\textit{Consider a superconducting current loop with non-zero inductance $ L$, radius $ a $ and a built-in ammeter. Assume a magnetic monopole passes through the loop along the axis of symmetry. Find the resulting current pulse in the loop.}
\begin{itemize}
	\item First, the magnetic field of a hypothetical monopole is
	\begin{equation*}
		\B(\r) = \frac{\mm g}{4\pi} \frac{\r}{r^{3}}
	\end{equation*}
	where $ g $ is the ``magnetic charge'', with units $ \si{\ampere \meter} $. Our plan is to find the magnetic flux through the loop, use this to find voltage induced in the loop, use the induced voltage to find current.
	
	\item Let $ d(t) $ be the monopole's perpendicular distance from the loop's center, and let $ \rho $ denote radial distance from the current loop's center (in the plane of the loop).
	
	The magnetic field magnitude a distance $ r $ from the monopole is
	\begin{equation*}
		B = \frac{\mm}{4\pi} \frac{g}{r^{2}} = \frac{\mm g}{4\pi}\frac{1}{d^{2} + \rho^{2}}
	\end{equation*}
	We find the magnetic field component $ B_{\perp} $ perpendicular to the current loop with similar triangles:
	\begin{equation*}
		B_{\perp} = \frac{d}{\sqrt{d^{2} + \rho^{2}}} B = \frac{\mm g d}{4\pi}\frac{1}{(\rho^{2} + d^{2})^{3/2}}
	\end{equation*}
	
	\item The magnetic flux $ \Phi $ through the loop, using $  \diff S = 2\pi \rho \diff \rho $, is
	\begin{align*}
		\Phi &= \int B_{\perp} \diff S = \frac{\mm g d}{2} \int_{0}^{a} \frac{\rho \diff \rho}{(\rho^{2} + d^{2})^{3/2}} = \frac{\mm g d}{4}(-2)\frac{1}{\sqrt{\rho^{2} + d^{2}}}\bigg |_{0}^{a}\\
		& = \frac{\mm g}{2}\left(1 - \frac{d}{\sqrt{a^{2} + d^{2}}}\right)
	\end{align*}
	
	\item We assume the monopole moves with constant speed $ v $ and passes through the loop's center at $ t = 0 $. On the left of the loop (and thus for negative time), the monopole's perpendicular distance from the loop is
	\begin{equation*}
		d(t) = - vt
	\end{equation*}
	The time-dependent flux on the left of the loop is thus
	\begin{equation*}
		\Phi(t) = \frac{\mm g}{2}\left(1 + \frac{vt}{\sqrt{a^{2} +v^{2}t^{2}}}\right)
	\end{equation*}
	On the right of loop (and for positive time), the distance from the loop is $ d = vt $, and thus the magnetic flux through the loop is
	\begin{equation*}
		\Phi(t) = -\frac{\mm g}{2}\left(1 - \frac{vt}{\sqrt{a^{2} +v^{2}t^{2}}}\right)
	\end{equation*}
	Note the additional minus sign, since on the right of the loop, once the particle passes through, the magnetic flux points in the opposite direction.
	
	Note that $ \Phi(t) $ increases from $ 0 $ at $ t \to -\infty $ to a maximum value of $ \frac{\mm g}{2} $ as $ t \to 0 $ from the left. As the particle passes through the loop at $ t = 0 $, $ \Phi $ jumps discontinuously to $ - \frac{\mm g}{2}  $, and then decreases back to 0 as $ t \to \infty $. 
	
	\item Next, we find the induced voltage in the loop. We start with a modified Maxwell equation:
	\begin{equation*}
		\curl \E = - \pdv{\B}{t} - \mm \vec{j}_{\text{m}}
	\end{equation*}
	where $ \vec{j}_{\text{m}} $ is magnetic current density and accounts for the possible existence of magnetic monopoles. We then use this modified Maxwell equation to re-derive the law of induction in the presence of magnetic monopoles. We integrate the equation over the loop's surface to get
	\begin{equation*}
		\iint_{S} (\curl \E)\diff \vec{S} = - \pdv{t}\iint_{S} \B \cdot \diff \vec{S} - \mm \iint_{S} \vec{j}_{\text{m}}\diff S 
	\end{equation*}
	and apply Stokes' law to get
	\begin{equation*}
		\oint \E \cdot \diff \vec{l} = U_{i} = - \dot{\Phi} - \mm I_{\text{m}}
	\end{equation*}
	where $ I_{\text{m}} $ is magnetic current through the loop's cross section and $ U_{i} $ is the induced voltage in the loop. The magnetic current through the loop is nonzero only at the singular instant when the magnetic monopole passes through, which we model with the delta function:
	\begin{equation*}
		I_{\text{m}} = g \delta (t)
	\end{equation*}
	where $ \delta(t) $ has units $ \si{\second^{-1}} $. 
	
	\item If we neglect resistive and capacitive effects, the loop's circuit equation reads $ U_{i} = L \dot{I} $. Substituting in $  U_{i} = - \dot{\Phi} - \mm I_{\text{m}} $ gives
	\begin{equation*}
		U_{i} =  U_{i} = - \dot{\Phi} - \mm I_{\text{m}} = - \dot{\Phi} - \mm g \delta (t) = L \dot{I}
	\end{equation*}
	We then integrate this equation over time to get
	\begin{align*}
		&L \int_{-\infty}^{t}\dot{I}\diff \t{t} = - \int_{-\infty}^{t} \dot{\Phi} \diff \t{t} - \mm g \int_{-\infty}^{t} \delta(\t{t})\diff \t{t}\\
		&LI(\t{t}) \Big |_{-\infty}^{t} = - \Phi(\t{t}) \Big|_{-\infty}^{t} - \mm g H(t)
	\end{align*}
	where $ H(t) $ is the Heaviside step function. Since $ I(t \to -\infty) = \Phi(t \to -\infty) = 0 $ (when the monopole is infinitely far from the loop), the equation reduces to
	\begin{equation*}
		LI(t) = - \Phi(t)  - \mm gH(t)
	\end{equation*}
	
	\item Remember that $ \Phi $ changes from $ \frac{\mm g}{2} $ to $ -\frac{\mm g}{2} $ as the monopole passes through the loop at $ t = 0 $. The discontinuity in $ \Phi $ is exactly balanced by the Heaviside step function activating with magnitude $ \mm g $ at $ t = 0 $, and the effect is that $ LI(t) $ and thus the current through the loop is as a continuous, measurable quantity. 
	
	\item If we use the Dirac quantization of magnetic charge, which reads
	\begin{equation*}
		\frac{\mm g e_{0}}{2} = h \implies \mm g = \frac{2h}{e_{0}},
	\end{equation*}
	we get a numerical result for the quantity $ \mm g $, and a theoretically expected value value of $ LI(t) $.
\end{itemize}


\subsection{Tenth Exercise Set}

\subsubsection{Skin Effect in a Ribbon-Like Conductor}
\textit{Consider a long conducting ribbon of conductivity $ \sigma $, width $ a $, height $ b $ and length $ l $ where $ l \gg b \gg a $. We place ideal electrodes at each end of the ribbon, and connect the electrodes to an alternating voltage source with frequency $ \omega $. Find the dependence of the conductor's impedance on frequency and investigate the high and low-frequency limits.}

\begin{itemize}
	\item Our plan is to solve for the electric field $ \E(t) $ in the ribbon, use $ \E $ to find the potential difference $ U $, and finally use $ U $ to find the ribbon's impedance. We start with the relevant Maxwell equations, which are
	\begin{align*}
		& \curl \E = - \pdv{\B}{t} && \div \E = 0\\
		& \curl \B = \mm \vec{j} + \ee\mm \pdv{\E}{t} && \div \B = 0
	\end{align*}
	Note that $ \div \E = 0 $ because the conductor is neutral---no electric field escapes.
	
	We make a quasi-static approximation to simplify the equations, neglecting the displacement current $  \ee\mm \pdv{\E}{t} $ to get
	\begin{equation*}
		 \curl \B \approx \mm \vec{j}
	\end{equation*}
	We take the curl of the first Maxwell equation to get 
	\begin{equation*}
		\curl (\curl \E) = \grad (\div \E) - \laplacian \E  = - \pdv{t} \curl \B
	\end{equation*}
	We then apply $ \div \E = 0 $ and $ \curl \B = \mm \j = \mm (\sigma \E) $ to get
	\begin{equation*}
		- \laplacian \E = - \pdv{t} \mm \sigma \E
	\end{equation*}
	
	\item Since the ribbon is attached to an alternating voltage source, the electric field $ \E = \E(t) $ reads
	\begin{equation*}
		\E(t) = \E_{0} e^{i\omega t} \implies \pdv{t}\E(t) = i \omega \E(t)
	\end{equation*}
	Finally, we substitute the $ \pdv{E}{t} $ into the earlier equation $ - \laplacian \E = - \pdv{t} \mm \sigma \E $ to get
	\begin{equation*}
		\laplacian \E - i \omega \mm \sigma \E \equiv \laplacian \E - k^{2}\E = 0
	\end{equation*}
	where we have defined the amplitude
	\begin{equation*}
		k^{2} \equiv i \omega \mm \sigma \implies k = \sqrt{i \omega \mm \sigma} = \frac{1 + i}{\sqrt{2}}\sqrt{\omega \mm \sigma}
	\end{equation*}
	
	
	\item Next, we define coordinate system whose $ x $ axis aligns with the ribbon's width $ a $ and whose $ y $ axis aligns with the ribbon's height. The conductor's length corresponds to the $ z $ axis. We choose the origin so that $ x \in [-a/2, a/2] $, meaning the ribbon's center occurs at $ x = 0 $. 
	
	With respect to this coordinate system, we can then simplify the Laplacian $ \laplacian \E $. Since the ribbon's width $ a $ is much smaller than the height and length, the Laplacian's derivatives with respect to $ y $ and $ z $ are negligible, i.e.
	\begin{equation*}
		\laplacian \E = \left(\pdv[2]{}{x} + \pdv[2]{}{y} + \pdv[2]{}{z}\right)\E \approx \pdv[2]{\E}{x}
	\end{equation*}
	The alternating voltage is applied along the conductor's length---along the $ z $ axis---so the electric field reads $ \E = E_{z}(x) \uvec{z} $. Plugging all of these simplifications into the amplitude equation gives
	\begin{equation*}
		\laplacian \E - k^{2}\E \approx \pdv[2]{}{x} \big[E_{z}(x) \uvec{z}\big] - k^{2}E_{z}(x) \uvec{z} =   E_{z}''(x) - k^{2}E_{z}(x) = 0
	\end{equation*}
	\item The solutions to the equation 
	\begin{equation*}
		E_{z}'' - k^{2}E_{z} = 0
	\end{equation*}
	can be written either as exponents or hyperbolic functions; we will use hyperbolic functions, which are best suited to the problem's reflection symmetry about the $ y $ axis. The general solution is
	\begin{equation*}
		E_{z}(x) = A\cosh kx + B\sinh kx
	\end{equation*}
	However, the problem's reflection symmetry means $ E_{z} $ will have only the even component $ \cosh $, and the solution simplifies to
	\begin{equation*}
		E_{z}(x) = A\cosh kx
	\end{equation*}
	
	\item We find $ A $ from the problem's boundary conditions. Assuming $ E_{z}(\pm a/2) = E_{0} $, we have
	\begin{equation*}
		A \cosh \left(\pm \frac{ka}{2}\right) = A \cosh \frac{ka}{2} = E_{0} \implies A = \frac{E_{0}}{\cosh \frac{ka}{2}}
	\end{equation*} 
	With the constant $ A $ known, the solution for $ E_{z}(x) $ is then
	\begin{equation*}
		E_{z}(x) = A\cosh kx = E_{0}\frac{\cosh kx}{\cosh \frac{ka}{2}}
	\end{equation*}
	Next, we draw some qualitative sketches of $ E(x) $ with $ k $ as a parameter. The resulting curves show that as $ k $ (and thus frequency $ \omega $) increases, $ E(x) $ becomes concentrated near the ribbon's outer surfaces  $ x = \pm a/2 $. This is a qualitative demonstration of the skin effect, where electric field and current become concentrated along a conductor's surface at high frequencies.
\end{itemize}	
	
\textbf{Potential Difference, Current and Impedance}
\begin{itemize}
	
	\item Next, with $ E_{z}(x) $ known, the potential difference across the ribbon's is simply
	\begin{equation*}
		U_{z}(x) = E_{z}(x) l
	\end{equation*}
	where $ l $ is the conductor length. 
	
	\item With potential difference known, we find the current through the conductor with
	\begin{equation*}
		I = \int j \diff S = \int (\sigma E_{z}) \diff S = \frac{\sigma E_{0}}{\cosh \frac{ka}{2}} \int_{-a/2}^{a/2} \cosh(kx) (b \diff x)
	\end{equation*}
	where we have written the surface element $ \diff S = b \diff x $ in terms of the conductor's height $ b $. The integral evaluates to
	\begin{align*}
		I &= \frac{\sigma E_{0} b}{\cosh \frac{ka}{2}} \int_{-a/2}^{a/2} \cosh(kx) \diff x =  \frac{\sigma E_{0} b}{\cosh \frac{ka}{2}}  \cdot \frac{2}{k} \cdot \sinh kx \Big |_{x = 0}^{a/2} \\
		& =  \frac{2\sigma E_{0} b}{k} \tanh \frac{ka}{2}
	\end{align*} 
	Next, we introduce the dimensionless quantity $ \kappa = \frac{ka}{2} $ and substitute in $ U_{0} = E_{0} l$ to get
	\begin{equation*}
		I = \frac{\sigma U_{0} b a}{l} \frac{2}{ka}\tanh \frac{ka}{2} = \frac{\sigma U_{0} b a}{l} \frac{\tanh \kappa}{\kappa} = \frac{U_{0}}{R_{0}} \frac{\tanh \kappa}{\kappa}
	\end{equation*}
	where, in the last equality, we have substituted in the conductor's static resistance
	\begin{equation*}
		R_{0} = \frac{l}{\sigma S}
	\end{equation*}
	In terms of $ R_{0} $, the conductor's impedance is then
	\begin{equation*}
		Z \equiv \frac{U_{0}}{I} = R_{0}\frac{\kappa}{\tanh \kappa}
	\end{equation*}
\end{itemize}	

\textbf{High and Low Frequency Limits}
\begin{itemize}

	\item In the low frequency limit, $ k $ and thus $ \kappa $ are small ($ \kappa \ll 1 $) and we expand the $ \tanh x \approx  $ function to get
	\begin{equation*}
		Z \approx R_{0} \frac{\kappa}{\kappa} = R_{0}
	\end{equation*}
	In other words, the ribbon's impedance approaches the static resistance $ R_{0} $.
	
	\item Finally, for large frequencies and thus $ \kappa \gg 1 $, we use the asymptotic expansion $ \tanh x \approx 1 $ to get
	\begin{equation*}
		Z \approx R_{0}\kappa = \frac{R_{0}a}{2}k = \frac{R_{0}a}{2} \sqrt{\frac{\sigma \mm \omega}{2}}(1 + i)
	\end{equation*}
	The real component $ \Re Z $ corresponds to resistance $ R(\omega) $, while the imaginary component $ \Im Z $ corresponds to reactance. Reactance is out of phase with resistance by $ \pi/2 $. At high frequencies, the resistance is
	\begin{equation*}
		R(\omega) = \frac{r_{0}a}{2}\sqrt{\frac{\sigma \mm \omega}{2}}
	\end{equation*}
	Note the relationship $ R \propto \sqrt{\omega} $, meaning resistance increases with frequency, since the electric field is more concentrated at the conductor edges (the ``skin'') and the current has a smaller effective cross section and thus larger resistance.
	
	
\end{itemize}

\subsubsection{Theory: Conservation of Electromagnetic Energy}

\begin{itemize}
	\item Conservation of electromagnetic energy is written as the energy balance
	\begin{equation*}
		\pdv{w}{t} + \div \S + \vec{j}\cdot \E = 0
	\end{equation*}
	where $ w $ is electromagnetic energy density
	\begin{equation*}
		w = \frac{1}{2}\left(\ee E^{2} + \frac{1}{\mm}B^{2}\right)
	\end{equation*}
	and $ \S $ is the \Poy vector, defined as
	\begin{equation*}
		\S = \frac{1}{\mm}\E \cross \B = \E \cross \H
	\end{equation*}
	
	\item In integral form for a region of space $ V $, the energy equation reads
	\begin{equation*}
		\pdv{t}\int_{V}w \diff V + \int_{\partial V} \S \cdot  \diff \vec{S} + \int_{V} \vec{j}\cdot \E \diff V = 0
	\end{equation*}
	where we have used the divergence theorem to convert the Poynting vector term to a surface integral.
	\begin{itemize}
		\item The $ w $ term corresponds to the changing electromagnetic field energy within the region $ V $
		
		\item The $ \S $ term encodes energy flow (power) through the surface
		
		\item The $ \vec{j} \cdot \E $ term corresponds to Ohmic energy losses within the region.
	\end{itemize}
	
\end{itemize}

\subsubsection{Power in a Coaxial and Cylindrical Conductor}
\textit{Find the electromagnetic energy flow (power) through:}
\begin{enumerate}
	\item \textit{The cross section of a coaxial cable carrying current $ I $, of length $ l $, with potential difference $ U $ between the inner and outer conductor. Neglect resistive losses.}
	
	\item \textit{The lateral surface long, straight cylindrical conductor of radius $ a $, length $ l $, conductivity $ \sigma $ and carrying a current $ I $ with a potential difference $ U $ between the cylinder ends.}
\end{enumerate}

\textbf{Lossless Coaxial Conductor}
\begin{itemize}
	\item In a coaxial cable, the magnetic field $ \B $ points tangent to the circular cross section (perpendicular to the radial direction), while the electric field $ \E $ points radially outward. We assume positive charge accumulates on the inner conductor and negative charge on the outer conductor.
	
	We find the magnetic field with Ampere's law using a loop around the inner conductor. The result is
	\begin{equation*}
		\mm I = B \cdot 2\pi r \implies B = \frac{\mm I}{2\pi r}
	\end{equation*}
	Meanwhile, we find electric field with Gauss's law, using a cylinder enclosing the inner conductor:
	\begin{equation*}
		Q = \ee E \cdot 2\pi r l \implies E = \frac{Q}{2\pi \ee l r}
	\end{equation*}
	
	
	\item The induced charge $ Q $ and potential difference between the inner and outer conductor are related by
	\begin{equation*}
		U = \int_{a}^{b} E \diff r = \frac{Q}{2\pi \ee l}\ln \frac{b}{a}
	\end{equation*}
	In terms of $ U $, the electric field is thus
	\begin{equation*}
		E = \frac{Q}{2\pi \ee l r} = \frac{U}{r \ln \frac{b}{a}}
	\end{equation*}
	
	\item The \Poy vector points in the direction $ \E \cross \B $, along the conductor's longitudinal axis. In our case we have $ \E \perp \B $, so the \Poy vector magnitude $ \SS $ is
	\begin{equation*}
		\SS = \frac{1}{\mm}EB = \frac{1}{\mm} \left (\frac{\mm I}{2\pi r}\right )\left(\frac{U}{r \ln \frac{b}{a}}\right) = \frac{UI}{2 \pi \ln \frac{b}{a}r^{2}}
	\end{equation*}
	
	\item The electromagnetic power through the cross section is thus
	\begin{align*}
		P &= \int \S \cdot \diff \vec{S} = \int \SS \diff S = \int_{a}^{b}\left(\frac{UI}{2 \pi \ln \frac{b}{a}r^{2}}\right)(2\pi r \diff r)\\
		& = \frac{UI}{\ln \frac{b}{a}} \int_{a}^{b} \frac{\diff r}{r} =	\frac{UI}{\ln \frac{b}{a}}\ln \frac{b}{a} = UI
	\end{align*}
	where $ \S \cdot \diff \vec{S} = \SS \diff S $ because $ \S $ is parallel to coaxial cable's cross section.
	
%	\item Note the power entering one endpoint exits the other, and energy is conserved. 
\end{itemize}

\textbf{Cylindrical Conductor}
\begin{itemize}
	\item As for the coaxial cable, the cylindrical conductor's magnetic field is tangent to the conductor's circular cross section. Meanwhile, the electric field points in the direction of the potential difference---along the conductor's longitudinal axis. The \Poy vector $ \S = \E \cross \B $ thus along the radial direction. We choose $ I $ to flow so that $ \S $ points radially inward towards the conductor's center (as opposed to radially outward for an opposite current).
	
	\item The electric field in the conductor, which arises from the potential difference $ U $ between the conductor's ends, is simply
	\begin{equation*}
		E = \frac{U}{l}
	\end{equation*}
	where $ l $ is the conductor's length.
	
	We find the conductor's magnetic field with Ampere's law, using a circular loop centered along the conductor's longitudinal axis. Assuming current is uniformly distributed across the cross section, the magnetic field is
	\begin{equation*}
		B \cdot (2\pi r) = \mm I \left(\frac{r}{a}\right)^{2} \implies B =  \frac{\mm I}{2\pi a^{2}}r
	\end{equation*}
	
	\item Since $ \E $ and $ \B $ is perpendicular, the \Poy vector is
	\begin{equation*}
		\SS = \frac{1}{\mm}E B = \frac{1}{\mm} \left(\frac{U}{L}\right)\left( \frac{\mm I}{2\pi a^{2}}r\right) = \frac{UI}{2\pi a^{2}l}r
	\end{equation*}
	Note that $ \SS $ increases linearly with $ r $. 
	
	\item We find electromagnetic power by integrating $ \SS $ over the cylinder's lateral surface (not the cross section!). The power through a lateral surface at radius $ r $ is
	\begin{equation*}
		P(r) = \int_{S_{\text{lat}}} \S \cdot \diff \vec{S} = \int_{S_{\text{lat}}} \SS \diff S =  \left(\frac{UI}{2\pi a^{2}l}r\right) (2\pi r l) = \frac{UIr^{2}}{a^{2}}
	\end{equation*}
	The power through the entire conductor's lateral surface occurs at $ r = a $, which produces the familiar result
	\begin{equation*}
		P = \frac{UIa^{2}}{a^{2}} = UI
	\end{equation*}
	
	\item Next, we consider the conductor's energy balance
	\begin{equation*}
		- \pdv{t}\int_{V}w \diff V = \int_{\partial V} \S \cdot \diff \vec{S} + \int_{V} \vec{j} \cdot \E \diff V =  P + \int_{V} \vec{j} \cdot \E \diff V
	\end{equation*}
	Because the situation is stationary we have $ \pdv{w}{t} = 0$, and thus
	\begin{equation*}
		P + \int_{V} \vec{j} \cdot \E \diff V \equiv P + P_{\text{loss}} = 0 \implies P_{\text{loss}} = -P = -UI
	\end{equation*}
	Note that Ohmic losses amount to $ P_{\text{loss}} = -UI $, which is the familiar expression from e.g. high school physics. The energy balance reads
	\begin{equation*}
		P + \int \vec{j} \cdot \E \diff V =  P + P_{\text{loss}} = UI - UI = 0
	\end{equation*}
	and energy is conserved, at it must be. 
	
	
\end{itemize}

\subsubsection{Cylindrical Conductor with a Slit}
\textit{Consider a long cylindrical conductor with radius $ a $ carrying a current $ I $. We make a narrow slit of width $ d \ll r $ through the conductor's cross section cross section. Find the energy flow through the slit's lateral surface.}

\begin{itemize}
	\item Because of the current $ I $, positive charge accumulates on one surface of the slit and negative charge on the other. Because $ d \ll r $, we can thus treat the two surfaces of the slit as parallel-plate capacitor. 
	
	\item The charge accumulating on the slit is $ q(t) = It $, and, using the model of a parallel-plate capacitor, the corresponding electric field is
	\begin{equation*}
		E(t) = \frac{\sigma(t)}{\ee} = \frac{1}{\ee}\frac{q(t)}{S} = \frac{I}{\ee S}t
	\end{equation*}
	
	\item Next, we find the magnetic field in the slit. The time-varying changing electric field $ E(t) $ creates displacement current, and the Maxwell equation for $ \curl \B $ reads
	\begin{equation*}
		\curl \B  = \mm \vec{j} + \ee\mm \pdv{\E}{t} = \ee\mm \pdv{\E}{t}
	\end{equation*}
	Note that $ \mm \vec{j} = 0 $ inside the slit, since there's no free current. We integrate the equation over the conductor's cross-sectional surface $ S $ to get
	\begin{equation*}
		\int_{S} \curl \B \cdot \diff \vec{S} = \oint \B \cdot \diff \vec{l}  = \ee \mm \pdv{t}\int_{S}\E\cdot \diff S =  \ee \mm \pdv{\E}{t} \cdot (\pi r^{2})
	\end{equation*}
	We evaluate the line integral over the surface's boundary and substitute in electric field to get
	\begin{equation*}
		\oint \B \cdot \diff \vec{l} = B \cdot (2\pi r) = \ee \mm \pdv{E(t)}{t} \implies B(r) = \frac{\mm I}{2S} r
	\end{equation*}
	where $ S = \pi r^{2} $ is the cross-sectional surface area. The magnetic field points tangent to the cylindrical lateral, just like the magnetic field of a conductor, except that free current is replaced by displacement current. 
	
%	Since free current and displacement current have the same direction, $ B $ still points tangent to the surface
	
	\item Since $ \E $ points along the conductor's longitudinal axis and $ \B $ is tangent to the lateral surface, $ \S = \E \cross \B $ points along the radial direction. Because $ \B \perp \S$, the power through the lateral surface is simply
	\begin{equation*}
		P = \int_{S_{\text{lat}}} \SS \diff S = \frac{1}{\mm}\int_{S_\text{lat}} EB \diff S = \frac{1}{\mm} \int_{S_{\text{lat}}}  \left(\frac{I}{\ee S}t\right)\left(\frac{\mm I}{2S} r\right) \diff S
	\end{equation*}
	We set $ r = a $ at the lateral surface to get
	\begin{equation*}
		P = \frac{I^{2}a}{2\ee S^{2}}t (2\pi a d) = \frac{I^{2}d}{\ee S} t
	\end{equation*}
	where $ S = \pi a^{2} $ is cross sectional area at $ r = a $.
	
	\item  Finally we consider the conductor's energy balance. If we neglect Ohmic losses, the energy balance reads
	\begin{equation*}
		\pdv{t}\int_{V}w \diff V + \int_{\partial V} \S \cdot \diff \vec{S} = \pdv{W_{\text{EM}}}{t} + P = 0
	\end{equation*}
	where $ W_{\text{EM}} = W_{\text{E}} + W_{\text{B}} $ is the sum of magnetic and electric field energy. Note that $ B = B(r)$  and thus magnetic energy $ W_{\text{B}} $ are independent of time. It follows that
	\begin{equation*}
		\pdv{W_{\text{EM}}}{t} = \pdv{W_{\text{E}}}{t} + 0
	\end{equation*}
	Using $ W_{\text{E}} = w_{\text{E}}V $, the electric field energy in the slit is
	\begin{equation*}
		W_{\text{E}} = w_{\text{E}}V = \left(\frac{1}{2} \ee E^{2}\right) \cdot \left(Sd\right) = \frac{\ee}{2}\frac{I^{2}}{\ee^{2}S^{2}}Sdt^{2} = \frac{I^{2}d}{2\ee S}t^{2}
	\end{equation*}
	In terms of $ W_{\text{E}} $, the time derivative of total magnetic field energy is thus
	\begin{equation*}
		\pdv{W_{\text{EM}}}{t} = \pdv{W_{\text{E}}}{t} = \pdv{t}\frac{I^{2}d}{2\ee S}t^{2} = \frac{I^{2}d}{\ee S} t = P
	\end{equation*}
	Up to a negative sign (depending on the direction of current), $ \pdv{W_{\text{EM}}}{t} $ and $ P $ are equal, indicating that energy is conserved, as it must be.
\end{itemize}

\subsection{Eleventh Exercise Set}

\subsubsection{A Radially Polarized Sphere}
\textit{Consider a radially polarized sphere, in which electric polarization points in the radial direction and grows with increasing radius as $ \P(\r) = k \r $ where $ k $ is a constant. Find the volume density of bound charges, the surface density of bound charges, and the total bound charge. Finally, find the electric field due to the sphere.}

\begin{itemize}
	\item Polarization and bound charge density $ \rho_{\text{b}} $ are related by
	\begin{equation*}
		\rho_{\text{b}} = - \div \P
	\end{equation*}
	while electric field and total charge density $ \rho $ are related by Gauss's law
	\begin{equation*}
		\rho = \ee \div \E
	\end{equation*}
	
	\item We find volume charge density with $ \rho_{\text{b}} = - \div \P $ and the known expression for $ \P $:
	\begin{equation*}
		\rho_{\text{b}} = - \div \P = - \div (k \r) = -k \div \r = - 3k
	\end{equation*}
	where we've used $ \div \r = 3 $. Note the bound volume charge density is negative, since the electric dipoles in the sphere have their positive poles radially outward and negative pole radially inward, so negative charge is concentrated toward the sphere's center.

% The ``in-between'' dipoles cancel, leaving negative charge within the sphere and positive charge at the surface
	
	\item To relate surface and volume charge density, we integrate $ \rho_{\text{b}} = - \div \P $ over the sphere's volume and apply the divergence theorem
	\begin{equation*}
		\int_{V} \rho_{\text{b}}\diff V = q_{\text{b}} = - \int_{V}\div \P \diff V = - \int_{S}\P \cdot \diff \vec{S} = - \P \cdot \vec{S} \big |_{\text{in}}^{\text{out}}
	\end{equation*}
	Where the last step writes $ \vec{S} = \uvec{n}S $ and notes that polarization is zero outside the sphere. The bound surface charge density is then
	\begin{equation*}
		q_{\text{b}} = \P \cdot \uvec{n} S \implies \sigma_{\text{B}} = \frac{q_{\text{b}}}{S} = \P \cdot \uvec{n}
	\end{equation*}
	At the surface $ r = a $, charge density $ \sigma_{\text{b}} $ evaluates to
	\begin{equation*}
		\sigma_{\text{b}} = \P \cdot \uvec{n} = (k \r) \cdot \uvec{n} = k r \big |_{r = a}  = k a
	\end{equation*}
	Note that surface charge density is positive, since the electric dipoles have their positive pole oriented radially outward. 
	
	\item We find total bound charge from $ \rho_{\text{b}} $ and $ \sigma_{\text{b}} $ by integrating both volume and surface charge densities:
	\begin{equation*}
		q_{\text{b}} = \int_{V} \rho_{\text{b}} \diff V + \int_{S} \sigma_{\text{b}} \diff S = - 3k \frac{4}{3}\pi a^{3} + ka 4\pi a^{2} = 0
	\end{equation*} 
	We should expect total charge to be zero, since all dipoles within the sphere should cancel out.
\end{itemize}	
	
\textbf{Electric Field}
\begin{itemize}
	\item By Gauss's law, the electric field outside the sphere is zero, since the sphere's total charge is zero.

	\item Inside the sphere, using the problem's spherical symmetry, we use Gauss's law with a spherical surface to get
	\begin{equation*}
		\ee E 4\pi r^{2} = \rho_{\text{b}} V = \rho_{\text{b}} \frac{4}{3}\pi r^{3} \implies E(r) = \frac{\rho_{\text{b}}}{3 \ee} r
	\end{equation*} 
	The electric field points radially outwards:
	\begin{equation*}
		\E(r) = \frac{\rho_{\text{b}}}{3 \ee} \r
	\end{equation*}
	We then substitute in the charge density $ \rho_{\text{b}} = -3k $ to get
	\begin{equation*}
		\E = \frac{-k}{\ee}\r \implies \E = - \frac{\P}{\ee}
	\end{equation*}
	Note the linear relationship between electric field and polarization.
	
	The linear relationship arises because the sphere contains only bound charge, meaning total charge and bound charge are equal, i.e. $ \rho = \rho_{\text{b}} $. We could then combine $ \rho_{\text{b}} = \rho = - \div \P $ and Gauss's law $ \rho = \div (\ee \E) $ to get  $ \E = - \frac{\P}{\ee} $.
	
\end{itemize}

\subsubsection{A Halved Polarized Sphere}
\textit{Consider a sphere of radius $ a $ with homogeneous (uniform) polarization $ \P $ pointing in the $ z $ direction. Find the electric field inside and outside the sphere. }

\textit{We then cut the sphere in half in a plane perpendicular to the direction of polarization and slightly separate the halves to form a capacitor. Find the electric field in the slit between the halves.}

\begin{itemize}
	\item First, we find volume density of bound charges, which is
	\begin{equation*}
		\rho_{\text{b}} = - \div \P = 0,
	\end{equation*}
	since the divergence of $ \P $, which is homogeneous, is zero by definition. In other words, there are no bound charges in the sphere's volume. 
	
	\item Next, we find bound surface charge density $ \sigma_{\text{b}} = \P \cdot \uvec{n} $ where $ \uvec{n} $ is the normal to the surface. In terms of an $ \theta $  between the normal $ \uvec{n} $ and polarization $ \P $, surface charge density is 
	\begin{equation*}
		\sigma_{\text{b}} = \P \cdot \uvec{n} = P \cos \theta
	\end{equation*}
% Note that the surface charge density is nonzero.
	
	\item We proceed with the Poisson equation, which simplifies to the Laplace equation within the sphere were $ \rho_{\text{b}} = 0 $. 
	\begin{equation*}
		\laplacian U(\r) = - \frac{\rho_{\text{b}}}{\ee} = 0
	\end{equation*}
	We account for the non-zero surface charge density with boundary conditions. 
	
	The general solution for $ U(\r) $ is 
	\begin{equation*}
		U(\r) = \sum_{l = 0}^{\infty}(A_{l}r^{l} + B_{l}r^{-(l+1)})P_{l}(\cos \theta)
	\end{equation*}
	Since the surface charge distribution $ \sigma_{\text{b}} $ depends only on $ \cos \theta $, our solution can include only $ \cos \theta $-dependent terms, which occurs for $ l = 1 $. The general solution reduces to
	\begin{equation*}
		U(\r) = (A_{1}r + B_{1}r^{-2})\cos \theta
	\end{equation*}
	To avoid divergence at $ r \to 0 $ and $ r \to \infty $, we separate $ U(\r) $ according to
	\begin{equation*}
		U(\r) \equiv 
		\begin{cases}
			U_{\text{in}} = A_{1}r \cos \theta & r < a\\
			U_{\text{out}} = B_{1}r^{-2} \cos \theta & r > a
		\end{cases}
	\end{equation*}
	We find $ A_{1} $ and $ B_{1} $ with  boundary conditions. 
	
	\item First boundary condition: the electric potential must be continuous at the boundary (a discontinuous potential would imply infinite charge, which is nonphysical). The continuity condition at $ r = a $ requires
	\begin{equation*}
		A_{1}a = \frac{B_{1}}{a^{2}} \implies B_{1} = A_{1}a^{3}
	\end{equation*}
	
	\item For the second boundary condition we require electric field is perpendicular to the surface. We then apply Gauss's law near the sphere's surface to get
	\begin{equation*}
		q_{\text{b}} = \ee S (E_{\text{out}}^{\perp} - E_{\text{in}}^{\perp})  \implies \sigma_{\text{b}} = \ee (E_{\text{out}}^{\perp} - E_{\text{in}}^{\perp})
	\end{equation*}
	We know $ \sigma_{\text{b}} = P \cos \theta $ and use $ E_{\text{in/out}}^{\perp} = - \pdv{U_{\text{in/out}}}{r}\big |_{r = a} $ to get
	\begin{equation*}
		\sigma_{\text{b}} = P\cos \theta = \ee (E_{\text{out}}^{\perp} - E_{\text{in}}^{\perp}) = \ee \left[  \frac{2B_{1}}{a^{3}} + A_{1}\right]\cos \theta 
	\end{equation*}
	We substitute the first boundary condition $ B_{1} = A_{1}a^{3} $ into the second to get
	\begin{equation*}
		P \cos \theta = \ee \left(2A_{1} + A_{1}\right)\cos \theta \implies P = 3 \ee A_{1} 
	\end{equation*}
	and thus
	\begin{equation*}
		A_{1} = \frac{P}{3\ee} \eqtext{and} B_{1} = \frac{Pa^{3}}{3\ee}
	\end{equation*}
	The electric potential is then
	\begin{equation*}
		U(\r) \equiv 
		\begin{cases}
			U_{\text{in}} = \dfrac{P}{3\ee} r \cos \theta & r < a\\[3mm]
			U_{\text{out}} = \dfrac{Pa^{3}}{3\ee}\dfrac{1}{r^{2}}\cos \theta & r > a
		\end{cases}
	\end{equation*}
	
	\item We now discuss the solution. The 3 in the denominators is called the depolarization factor (it was 1 in the previous solution where we had $ U = \frac{P}{\ee}\cdots $). Next, we note that $ r \cos \theta = z $---since $ U_{\text{in}} $ depends only on $ z $, the field inside the sphere is homogeneous:
	\begin{equation*}
		E_{z} = - \pdv{U_{\text{in}}}{z} = -\frac{P}{3\ee} \implies \E = - \frac{\P}{3\ee}
	\end{equation*}
	where we note that $ \P $ points in the $ z $ direction. The field is negative because with our choice of polarization, positive charges are in the positive $ z $ direction and negative charges in the negative z direction. 
	
	\item The field outside the sphere is field of an electric dipole; the cosine term is analogous to the dipole dot product term $ \pe \cdot \r $. A dipole field is expected, since the sphere is polarized like an electric dipole. 
	
\end{itemize}

\textbf{Halved Sphere; Electric Field in the Slit}
\begin{itemize}
	\item We now consider the electric field in the slit when the sphere is cut in half in a plane perpendicular to the polarization. 
	
	Because of the slit through the middle, bound charges accumulates on the cut surfaces as well as the outer surface. Since polarization points ``upwards'' in the positive $ z $ direction, the bound charges on cut surface of the upper hemisphere are negative, while the bound charges on the cut surface of the lower sphere are positive. The relationship $ \sigma_{\text{b}} = \P \cdot \uvec{n} $ is preserved. 
	
	\item To find electric field because of the slit, we model the slit as a parallel-plate capacitor, for which the electric field reads
	\begin{equation*}
		\t{\E} = \frac{\t{\sigma_{\text{b}}}}{\ee}\uvec{n} = \frac{(\P \cdot \uvec{n})}{\ee} \uvec{n} = \frac{\P}{\ee}
	\end{equation*}
	where tilde accounts for the flat surface and $ \uvec{n} $ is the normal to the cut surface.
	
	We must also consider the electric field contribution from the bound charge on the hemispherical surfaces. Reusing the already derived electric field
	\begin{equation*}
		\E_{\text{in}} = - \frac{\P}{3\ee}
	\end{equation*}
	The total field in the slit is
	\begin{equation*}
		\E = \t{\E} + \E_{\text{in}} = \frac{\P}{\ee} - \frac{\P}{3\ee} = \frac{2\P}{3\ee}
	\end{equation*}
\end{itemize}


\subsubsection{Theory: Dielectric and Displacement Field}
\begin{itemize}
		\item The previous two problems involved ferro-electric materials in which the polarization was built in to the material, even in the absence of an external electric field. We now consider dielectric materials, in which polarization occurs only in the presence of an external field. 
		
		\item We recall the equations $ \rho_{\text{b}} = - \div \P $ and Gauss's law $ \rho = \ee \div \E $. We subtract the equations to get
		\begin{equation*}
			\rho_{\text{f}} = \rho - \rho_{\text{b}} = \div (\ee \E + \P) \equiv \div \D
		\end{equation*}
		where the free charge density $ \rho_{\text{f}} = \rho - \rho_{\text{b}} $ is difference between the total and bound charge densities. The electric displacement field is defined as
		\begin{equation*}
			\D = \ee \E + \P
		\end{equation*}
		The $ \D $ field depends on free charge and is useful when analyzing dielectrics.
		
		\item For small fields, we make the approximation $ \D \propto \E $ where $ \D $ and $ \E $ are linearly dependent, which results in the approximate linear relationship 
		\begin{equation*}
			\D = \ee \eee \E
		\end{equation*}
		where $ \eee $ is the permittivity tensor.
\end{itemize}

\subsubsection{Parallel-Plate Capacitor with an Anisotropic Dielectric}
\textit{Consider a parallel-plate capacitor with plate separation $ d $, plate surface area $ S $, and the intra-plate space filled with a dielectric insulator. whose dielectric tensor has components $ \e_{1} $, $ \e_{2} $ and $ \e_{2} $. The principle axes corresponding to $ \e_{1} $ and $ \e_{3} $ are parallel to the capacitor plates, but the principle axis corresponding to $ \e_{2} $ makes an angle $ \phi $ with the normal to the plates. Find the capacitor's capacitance $ C $.}

% Work in a two-dimensional coordinate system in which the $ y $ axis coincides with the vertical separation between the capacitor plates.

\begin{itemize}
	\item First, recall that capacitance is in general defined as
	\begin{equation*}
		C = \frac{q}{U}
	\end{equation*}
	where $ q $ is the charge on the capacitor plates and $ U $ is the potential difference between the plates.
	
	\item Because $ \eee $ is angled (equivalently, because $ \eee $'s second principle axis does not align with the normal to the capacitor plates), $ \E $ and $ \D $ are not parallel. 
	
	From the boundary conditions for Maxwell's equations, we know $ \E $ must be perpendicular to the capacitor plates. To show this, for review, we consider the Maxwell equation
	\begin{equation*}
		\curl \E = - \pdv{\B}{t} 
	\end{equation*}
	In our static case we have $ \pdv{\B}{t} = 0 $. We integrate the equation over a surface hugging the boundary and apply Stokes' law to get 
	\begin{equation*}
		\int_{S}\curl \E \cdot \div \vec{S} = \oint \E \cdot \diff \vec{l} = 0
	\end{equation*}
	where the line integral runs over a rectangular path thinly hugging a capacitor plate. Because the path thinly hugs the plate, the integral registers only the component of $ \E $ parallel to the plates:
	\begin{equation*}
		E_{\text{out}}^{(\parallel)}\cdot l - E_{\text{in}}^{(\parallel)} \cdot l = 0 \implies E_{\text{out}}^{(\parallel)} = E_{\text{in}}^{(\parallel)}
	\end{equation*}
	Because the electric field outside the capacitor is zero, we have
	\begin{equation*}
		E_{\text{out}}^{(\parallel)} = 0 \implies E_{\text{in}}^{(\parallel)} = 0
	\end{equation*}
	The result $ E_{\text{in}}^{(\parallel)} = 0 $ implies the electric field has only a component perpendicular to the capacitor plates. 
	
	\item Working in the two-dimensional $ x, y $ plane (since the problem is invariant to translation in the $ z $ direction along the plates) the dielectric tensor in the principle axis system, which we denote $ \t{\eee} $, reads
	\begin{equation*}
		\t{\eee} = 
		\begin{pmatrix}
			\e_{1} & 0\\
			0 & \e_{2}
		\end{pmatrix}
	\end{equation*}
	The capacitor's coordinate system is rotated by $ \phi $ relative to the principle axis system. To transform from $ \t{\eee} $ to $ \eee $, we rotate the principle axes tensor by rotation matrices:
	\begin{align*}
		\eee &= \mat{R} \t{\eee} \mat{R} = 
		\begin{pmatrix}
			\cos \phi & \sin \phi\\
			- \sin \phi & \cos \phi
		\end{pmatrix}
		\begin{pmatrix}
			\e_{1} & 0\\
			0 & \e_{2}
		\end{pmatrix}
		\begin{pmatrix}
			\cos \phi & - \sin \phi\\
			\sin \phi & \cos \phi
		\end{pmatrix}\\
		& = 
		\begin{bmatrix}
			\e_{1} \cos^{2}\phi + \e_{2} \sin^{2}\phi & (\e_{2} - \e_{1})\sin \phi \cos \phi \\
			(\e_{2} - \e_{1})\sin \phi \cos \phi & \e_{1} \sin^{2}\phi + \e_{2} \cos^{2}\phi &
		\end{bmatrix}
	\end{align*}
	Note that $ \eee $ is symmetric, as is expected for the dielectric tensor.
	
	\item Because of the boundary condition, $ \E_{\parallel} = 0 $, if follows that $ \E $ has only a $ y $ component (where the $ y $ axis is normal to the plates). The electric field then reads $ \E = (E_{x}, E_{y}) \equiv (0, E) $, which we combine with $ \D = \ee \eee \E $ to get
	\begin{equation*}
		\begin{pmatrix}
			D_{x}\\
			D_{y}
		\end{pmatrix}
		 = 
		 \ee 
 		\begin{bmatrix}
 			\e_{1} \cos^{2}\phi + \e_{2} \sin^{2}\phi & (\e_{2} - \e_{1})\sin \phi \cos \phi \\
 			(\e_{2} - \e_{1})\sin \phi \cos \phi & \e_{1} \sin^{2}\phi + \e_{2} \cos^{2}\phi &
 		\end{bmatrix}
		\begin{pmatrix}
			0\\
			E
		\end{pmatrix}
	\end{equation*}
	
	\item In terms of $ E $, the electric field between the capacitor plates is simply
	\begin{equation*}
		U = Ed
	\end{equation*}
	where $ d $ is the distance between the plates. 
			
	\item Next, we use the boundary condition on the $ \D $ field to find the charge $ q $ on the plates. We derive the relevant boundary condition, we apply Gauss's law to a thin region tightly enclosing a capacitor plate. For the $ \D $ field, which obeys $ \div \D = \rho_{\text{f}} $, Gauss's law reads
	\begin{equation*}
		\int_{V} \rho_{\text{f}} \diff V = \int_{S} \D \cdot \diff \vec{S} \implies q_{\text{f}} = \left[D_{\text{out}}^{(\perp)} - D_{\text{in}}^{(\perp)} \right] S
	\end{equation*}
	In our case, where $ D_{\text{out}}^{(\perp)} = 0 $ and $ D_{\text{out}}^{(\perp)} = D_{y} $ the charge on the capacitor plates (up to a minus sign depending on the definition of the surface normal) is
	\begin{equation*}
		q_{\text{f}} = D_{y} S
	\end{equation*}	
	We find $ D_{y} $ from the earlier matrix equation:
	\begin{equation*}
		D_{y} = \ee \left( \e_{1} \sin^{2}\phi + \e_{2} \cos^{2}\phi \right) E = \ee \left( \e_{1} \sin^{2}\phi + \e_{2} \cos^{2}\phi \right) \frac{U}{d}
	\end{equation*}
	The charge on the capacitor plates is then
	\begin{equation*}
		q_{\text{f}} = D_{y} S = \frac{\ee SU}{d} \left( \e_{1} \sin^{2}\phi + \e_{2} \cos^{2}\phi \right) 
	\end{equation*}
	
	\item In terms of $ q_{\text{f}} $, the capacitor's capacitance $ C $ is
	\begin{equation*}
		C = \frac{q_{\text{f}}}{U} = \frac{\ee S}{d}\left(\e_{1} \sin^{2}\phi + \e_{2} \cos^{2}\phi\right) = C_{0}\left(\e_{1} \sin^{2}\phi + \e_{2} \cos^{2}\phi\right)
	\end{equation*}
	where $ C_{0} =  \frac{\ee S}{d} $ is the capacitance of an empty capacitor. 
	
	Note that for $ \phi = 0 $, corresponding to an isotropic dielectric in which the dielectric tensor's second principle axes \textit{does} align with normal to the capacitor plates, the capacitor's capacitance reduces to $ C = C_{0}\e_{2} $. 
\end{itemize}



\subsection{Twelfth Exercise Set (rough draft)}
\subsubsection{Point Dipole in a Spherical Dielectric Cavity}
\textit{Spherical cavity of radius $ a $ inside a dielectric material with permittivity $ \e $. Place a dipole with dipole momentum $ \pe $ in the cavity. What is electric potential inside and outside the cavity? And find the resulting effective dipole moment $ \pe' $ in the dielectric.}
\begin{itemize}
	\item  Assume $ \pe $ points in the $ z $ direction.
	
	
	\item We solve the Laplace equation
	\begin{equation*}
		\laplacian U(\r) = 0
	\end{equation*}
	in spherical coordinates. The general solution is
	\begin{equation*}
		U(r) = \sum_{l = 0}^{\infty}(A_{l}r^{l} + B_{l}r^{-(l+1)})P_{l}(\cos \theta)
	\end{equation*}
	Boundary conditions: we consider the spherical cavity's boundary, and the center of the cavity. The potential at the center should approach the dipole potential:
	\begin{equation*}
		U(r \to 0) = U_{\mathrm{dipole}} = \frac{p_{e}\cos \theta}{4 \pi \ee r^{2}}
	\end{equation*}
	This implies only the $ l = 1 $ term with $ \cos \theta $ can exist in the general solution for $ U $. 
	
	Outside only terms with $ A $  survive to avoid divergence at small $ r $.
	
	Outside only terms with $ B $ terms survive to avoid divergence at large $ r $.
	
	\item We have (reusing earlier results from similar problems) a general solution
	\begin{equation*}
		U(r, \theta) = 
		\begin{cases}
			\dfrac{p_{e}\cos \theta}{4 \pi \ee r^{2}} + A_{1}r \cos \theta & r < a\\[2mm]
			\dfrac{B_{1}}{r^{2}}\cos \theta & r > a
		\end{cases}
	\end{equation*}
	
	\item More boundary conditions: $ U $ is continuous at $ r = a $. We get (canceling $ \cos \theta $)
	\begin{equation*}
		\frac{p_{e}}{4\pi \ee a^{2}} + A_{1}a = \frac{B_{1}}{a^{2}}
	\end{equation*}
	
	\item And a final boundary condition using the electric field: we work with the electric field tangent to the cavity's surface. From Maxwell:
	\begin{equation*}
		\curl \E = 0 \to \oint \E \cdot \diff \vec{l} = 0
	\end{equation*}
	Which implies $ \E_{\parallel} $ is equal on each side of the cavity. 
	
	We differentiate with respect to $ \theta $ to get $ E_{\parallel} $
	\begin{equation*}
		E_{\parallel} = -\frac{1}{r}\pdv{U}{\theta}
	\end{equation*}
	Applying the continuity condition of $ \E_{\parallel} $ at the boundary gives
	\begin{equation*}
		- \frac{p_{e}}{4\pi \ee a^{2}} \sin \theta - A_{1} \sin \theta = \frac{B_{1}}{a^{2}} \sin \theta 
	\end{equation*}
	This is the same result as the previous boundary condition requiring continuity of $ U $ at the boundary. 
	
	\item We need another boundary condition: We use the $ \D $ field and thus consider only free charges. We use Gauss's law for $ \D $, which requires that $ D_{\perp} $ is the same at both sides of the boundary. In general
	\begin{equation*}
		D_{\perp} = - \ee \e \pdv{U}{r}
	\end{equation*}
	and we evaluate this at $ a $ and apply continuity at the boundary to get
	\begin{equation*}
		+2 \frac{p_{e}}{4\pi \ee a^{3}} - A_{1} = 2 \e \frac{B_{1}}{a^{3}}
	\end{equation*}
	
	\item Continuity of $ U $ gives
	\begin{equation*}
		\frac{p_{e}}{4\pi \ee a^{3}} + A_{1} = \frac{B_{1}}{a^{3}}
	\end{equation*}
	Adding this to the $ D_{\perp} $ equation gives
	\begin{equation*}
		3 \frac{p_{e}}{4 \pi \ee a^{3}} = \frac{B_{1}}{a^{3}}(1 + 2 \e)
	\end{equation*}
	Which gives us
	\begin{equation*}
		B_{1} = \frac{3}{1 + 2\e} \frac{p_{e}}{4\pi \ee}
	\end{equation*}
	
	\item From which we can find $ A_{1} $ with
	\begin{equation*}
		A_{1} = \frac{B_{1}}{a^{3}} - \frac{p_{e}}{4\pi \ee a^{3}} = \frac{p_{e}}{4\pi \ee a^{3}}\left(\frac{3}{1 + 2\e} - 1\right) = \frac{2(1 - \e)}{1 + 2\e} \frac{p_{e}}{4 \pi \ee a^{3}}
	\end{equation*}
	
	\item Our specific solution for $ U(r, \theta) $ is then
	\begin{equation*}
		U(r, \theta) = \frac{p_{e}}{4\pi \ee}\cos \theta
		\begin{cases}
			\dfrac{1}{r^{2}} - \dfrac{2(r-1)}{1 + 2\e}\dfrac{r}{a^{3}} & r < a\\[2mm]
			\dfrac{3}{1 + 2\e} \dfrac{1}{r^{2}} & r > a
		\end{cases}
	\end{equation*}
	For $ r < a $ the first $ 1/r^{2} $ term is the dipole term. The second $ r < a $ contains a $ z = r \cos \theta $ term, which corresponds to a homogeneous field inside the cavity. 
	
	For $ r > a $ we have another $ 1/r^{2} $ term---again like a dipole. But note the larger factor $ \frac{3}{1 + 2\e} $, which would reduce to the usual $ 1 $ for relative permittivity $ \e = 1 $ This is the so-called effective dipole term in the problem instructions. Outside we thus have the effective dipole moment
	\begin{equation*}
		p_{e}' = \frac{3}{1 + 2\e}p_{e}
	\end{equation*}
	
	\item We could find surface density of bound charges with Gauss's law and
	\begin{equation*}
		\sigma_{\mathrm{b}} = \ee \left(E_{\mathrm{\perp}}^{\text{out}} - E_{\mathrm{\perp}}^{\text{in}}\right)
	\end{equation*}
	Alternatively, we could us
	\begin{equation*}
		\sigma_{\mathrm{b}} = \P \cdot \uvec{n}
	\end{equation*}
	where $ \P $ is electric polarization. Note that $ \uvec{n} $ points from the dielectric material into the cavity, and $ \P $ points out of the cavity. Thus
	\begin{equation*}
		 \P \cdot \uvec{n} = - P_{\perp}
	\end{equation*}
	We could then use
	\begin{equation*}
		\D = \ee \e \E = \ee \E + \P 
	\end{equation*}
	And thus 
	\begin{equation*}
		\P = \ee (\e - 1)\E
	\end{equation*}
	which gives
	\begin{equation*}
		\sigma_{\mathrm{b}} = -\ee(\e - 1)\E_{\perp}\big |_{r = a}
	\end{equation*}
\end{itemize}


\subsubsection{Dielectric Constant of Cold Plasma}
\textit{Find the relative permittivity $ \e $ of a cold plasma.}
\begin{itemize}
	\item Recall plasma is ionized gas---electrons leave their atoms, and we end up with positive ions from atomic cores and negative free electrons. Since atoms are much more massive than electrons, we assume the atoms are at rest as a first approximation. 
	
	\item We assume an oscillatory electric field inside the plasma:
	\begin{equation*}
		\E = \E_{0} e^{i(kz - \omega t)}
	\end{equation*}
	Assume atoms have charge $ q $ and electrons have charge $ -q $. The force on a representative free electron is then
	\begin{equation*}
		\vec{F} = - q \E = - q \E_{0} e^{i(kz - \omega t)} \equiv m \vec{a} = m \ddot{\r}   
	\end{equation*}
	where we have introduced the coordinate $ \r $ to measure the electron's displacement.
	
	Since force is oscillatory, we have
	\begin{equation*}
		\r = \r_{0}e^{i(kz - \omega t)}
	\end{equation*}
	and thus
	\begin{equation*}
		\vec{F} = - m \omega^{2}\r_{0}e^{i(kz - \omega t)} =  - q \E_{0} e^{i(kz - \omega t)}
	\end{equation*}
	which implies
	\begin{equation*}
		m \omega^{2} \r_{0} = q \E_{0} \implies \r_{0} = \frac{q}{m \omega^{2}}\E_{0}
	\end{equation*}
	This amplitude of displacement of the negative electron from the positive atom creates a dipole moment with amplitude
	\begin{equation*}
		\pe_{0} = - q \r_{0} = - \frac{q^{2}}{m \omega^{2}} \E_{0}
	\end{equation*}
	The total polarization in the substance is then
	\begin{equation*}
		\P_{0} = n \pe_{0} =  - \frac{nq^{2}}{m \omega^{2}} \E_{0}
	\end{equation*}
	where $ n $ is the number density of electric dipoles in the material. 
	
	We now have a relationship between $ \P $ and $ \E $, which we can connect with permittivity:
	\begin{equation*}
		\P_{0} = \ee (\e - 1)\E_{0}
	\end{equation*}
	From which we find the dielectric constant (relative permittivity)
	\begin{equation*}
		\e = 1 - \frac{nq^{2}}{m \ee \omega^{2}}
	\end{equation*}
	We note that $ \frac{nq^{2}}{m \ee} $ has units of frequency, and we define
	\begin{equation*}
		\omega_{\mathrm{p}} = \sqrt{\frac{nq^{2}}{m \ee}}
	\end{equation*}
	which describes the oscillation of electrons if displaced from the positive atomic ions. We then have
	\begin{equation*}
		\e = 1 - \frac{\omega_{\mathrm{p}}^{2}}{\omega^{2}}
	\end{equation*}
	
	
	
	\item Note that for field frequencies $ \omega > \omega_{\mathrm{p}} $ we have $ \e > 0 $ and for $ \omega < \omega_{\mathrm{p}} $ we have $ \e < 0 $. For material with $ \e < 0 $, EM radiation cannot propagate through the material---discussed more shortly.
\end{itemize}
\textbf{Maxwell Equation}
\begin{itemize}
	\item We want to find the wave equation. In material, we have
	\begin{equation*}
		\div \E = 0 \eqtext{and} \curl \E = - \mm \pdv{\H}{t}
	\end{equation*}
	and also 
	\begin{equation*}
		\div \H = 0 \eqtext{and} \curl \H = \ee \e \pdv{\E}{t}
	\end{equation*}
	note that there is no current density $ \vec{j} $ in our material, which simplifies the last equation---there is only displacement current. 
	
	\item Apply curl to the first lines to get
	\begin{equation*}
		\curl (\curl \E) = - mm \pdv{t} \left(\ee \e \pdv{\E}{t}\right)
	\end{equation*}
	\begin{equation*}
		\grad (\div \E) - \laplacian \E = - \mm \e \ee \pdv[2]{\E}{t}
	\end{equation*}
	Because $ \div \E = 0 $ we get
	\begin{equation*}
		\laplacian \E - \mm \ee \e \pdv[2]{\E}{t} = 0
	\end{equation*}
	If we insert the earlier oscillatory ansatz $ \E = \E_{0}e^{i(kz - \omega t)} $ we get
	\begin{equation*}
		- k^{2} \E - \mm \ee \e (- \omega^{2})\E = 0 \implies \E (\mm \ee \e \omega^{2} - k^{2}) = 0
	\end{equation*}
	Which gives us
	\begin{equation*}
		\mm \ee \e \omega^{2} - k^{2} = 0 \implies \frac{\omega^{2}}{k^{2}} = \frac{1}{\ee \mm} \frac{1}{\e} = \frac{c_{0}^{2}}{\e}
	\end{equation*}
	And next, substituting in $ \e = 1 - \frac{\omega_{\mathrm{p}}^{2}}{\omega^{2}} $, we have
	\begin{equation*}
		c_{0}^{2} = \frac{\omega^{2}}{k^{2}}\left( 1 - \frac{\omega_{\mathrm{p}}^{2}}{\omega^{2}}\right) = \frac{\omega^{2}}{k^{2}} - \frac{\omega_{\mathrm{p}}^{2}}{k^{2}}
	\end{equation*}
	And then rearrange to get
	\begin{equation*}
		k^{2}c_{0}^{2} = \omega^{2} - \omega_{\mathrm{p}}^{2} \implies \omega = \sqrt{\omega_{\mathrm{p}}^{2} + c_{0}^{2}k^{2}}
	\end{equation*}
	This is a dispersion relation for electric field in a material---it relates the frequency $ \omega $ and wave-number $ k $ of the electric field. Note that the relation is nonlinear.
	
	\item For electric field in a vacuum, we have $ \e = 1 $; in this case we have the linear dispersion relation
	\begin{equation*}
		\frac{\omega^{2}}{k^{2}} = \frac{c_{0}^{2}}{\e} = c_{0}^{2} \implies \omega = c_{0}k
	\end{equation*}
	
	\item Back to the dispersion relation in material:
	\begin{equation*}
		 \omega = \sqrt{\omega_{\mathrm{p}}^{2} + c_{0}^{2}k^{2}}
	\end{equation*}
	For large $ k $ we get roughly $ \omega = c_{0}k $. 
	
	For small $ k $ we get $ \omega \to \omega_{\mathrm{p}} $. 
	
	A more accurate expansion for small $ \omega $ would give
	\begin{equation*}
		\omega = \omega_{\mathrm{p}}\left(1 + \frac{c_{0}^{2}}{\omega_{\mathrm{p}}^{2}} k^{2}\right)^{1/2} \approx \omega_{\mathrm{p}}\left[1 + \frac{c_{0}^{2}}{2 \omega_{\mathrm{p}}^{2}}k^{2}\right]
	\end{equation*}
	
	\item Note that for small $ k $ and small frequencies, we have $ \omega \sim k^{2} $, which means the photons making up the electric field behave like particles with mass, which leads to the concept of effective mass, like in solid state physics.
	
	\item Note also that the dispersion relation prohibits $ \omega < \omega_{\mathrm{p}} $ since at the minimum possible $ k $, i.e. $ k = 0 $, frequency is $ \omega = \omega_{\mathrm{p}} $. The prohibited region of $ \omega < \omega_{\mathrm{p}} $ corresponds to $ \e < 0 $---this connection  validates the previous assertion that EM radiation cannot exist in materials with $ \e < 0 $. 
	
	\item We more thoroughly consider $ \e < 0 $. Rearranging the equation 
	\begin{equation*}
		\frac{\omega^{2}}{k^{2}} = \frac{c_{0}^{2}}{\e}
	\end{equation*}
	leads to 
	\begin{equation*}
		k = \frac{\omega}{c_{0}}\sqrt{\e}  = \frac{\omega}{c_{0}}i \sqrt{\e_{R}}
	\end{equation*}
	Note that $ k $ is imaginary because $ \e < 0 $. Substituting this $ k $ into our wave ansatz for $ \E $ gives
	\begin{equation*}
		\E = \E_{0} e^{i(kz - \omega t)} = \E_{0}e^{-\frac{\omega}{c_{0}}\sqrt{\e_{R}}z}e^{-i\omega t}
	\end{equation*}
	The position-dependent term is $ \propto e^{-z} $, which means that an electromagnetic wave will exponentially decay with position in a material with $ \e < 0 $. 
	
	\item Next, we use the dispersion relation to find the group and phase velocity of the electromagnetic waves in the plasma. The phase velocity is 
	\begin{equation*}
		v_{\mathrm{p}} = \frac{\omega_{\mathrm{p}}}{k} = \frac{\sqrt{\omega_{\mathrm{p}}^{2} + c_{0}^{2}k^{2}}}{k} = \sqrt{c_{0}^{2} + \frac{\omega_{\mathrm{p}}^{2}}{k^{2}}}
	\end{equation*}
	Note that $ v_{p} > c_{0} $---the phase velocity is greater than the speed of light in vacuum. 
	
	\item The group velocity---the quantity relevant to the universal speed limit of $ c_{0} $---is 
	\begin{equation*}
		v_{\mathrm{g}} = \pdv{\omega}{k} = \frac{2 c_{0}^{2}k}{2 \sqrt{\omega_{\mathrm{p}}^{2} + c_{0}^{2}k^{2}}} = \frac{c_{0}}{\sqrt{1 + \frac{\omega_{\mathrm{p}}^{2}}{c_{0}^{2}k^{2}}}}
	\end{equation*}
	As expected, $ v_{\mathrm{g}} < c_{0} $, in agreement with special relativity. Note also $ v_{\mathrm{p}}v_{\mathrm{g}} = c_{0}^{2} $.
\end{itemize}

\section{Electromagnetic Wave Propagation}

\subsection{Thirteenth Exercise Set (rough draft)}
\subsubsection{Theory: Electromagnetic Wave Propagation in Waveguides}
\begin{itemize}
	\item Assume empty waveguides with $ \e = 1 $---we can use the wave equation
	\begin{equation*}
		\laplacian \E - \frac{1}{c_{0}^{2}}\pdv[2]{\E}{t} = 0
	\end{equation*}
	
	\item We assume bounded waveguide geometry, e.g. a pipe-like waveguide along the $ z $ axis. We use the following ansatz for electric field:
	\begin{equation*}
		\E(\r, t) = \E(\rh) e^{i(kz - \omega t)}
	\end{equation*}
	where $ \rh $ is the position within the waveguide's planar cross section, i.e. the $ xy $ plane perpendicular to $ \uvec{z} $ along the direction of EM propagation. 
	
	\item We write the Laplacian in two components to get
	\begin{equation*}
		\laplacian = \pdv[2]{}{z} + \nabla_{\perp}^{2} = -k^{2}  + \nabla_{\perp}^{2}
	\end{equation*}
	where $ \nabla_{\perp}^{2} = \pdv[2]{}{x} + \pdv[2]{}{y} $ corresponds to differentiation in the $ xy $ plane/the waveguide's cross section.
	
	We write the time derivative as
	\begin{equation*}
		\pdv[2]{}{t} = - \omega^{2}
	\end{equation*}
	The wave equation in a waveguide then simplifies to 
	\begin{equation*}
		 \left[\nabla_{\perp}^{2} + \left(\frac{\omega^{2}}{c_{0}^{2}} - k^{2}\right)\right]\E = 0
	\end{equation*}
	where $ \E = \E(\rho) $. However, because their is no time derivative, we can use the same equation for $ \E = \E(\r, t) $.
	
	\item $ \E $ has three components, so the wave equation in vector form really has 3 equations for each component. 
	
	Analogously, $ \H $ has three components, and thus three associated equations. 
	
	However, we can express $ \H $ in terms of $ \E $, which simplifies the equations involved. 
	
\end{itemize}

\textbf{Relationship Between $ \E $ and $ \H $}
\begin{itemize}
	\item We start with the Maxwell equations
	\begin{equation*}
		\curl \E = - \mm \pdv{\H}{t} \eqtext{and} \curl \H = \ee \pdv{\E}{t}
	\end{equation*}
	In components, the equations read
	\begin{equation*}
		\begin{bmatrix}
		\pdv{x}\\
		\pdv{y}\\
		\pdv{z}
		\end{bmatrix}
		\cross
		\begin{bmatrix}
		E_{x}\\
		E_{y}\\
		E_{z}
		\end{bmatrix}
		= 
		\begin{bmatrix}
		\pdv{E_{z}}{y} - ik E_{y}
		ik E_{x} - \pdv{E_{z}}{x}\\
		\pdv{E_{y}}{x} - \pdv{E_{x}}{y}
		\end{bmatrix} 
		= i \mm \omega
		\begin{bmatrix}
		H_{x}\\
		H_{y}\\
		H_{z}
		\end{bmatrix}
	\end{equation*}
	where we have used $ \pdv{z} = ik $ and $ \pdv{t} = -i\omega $
	
	\item The second equation reads
	\begin{equation*}
		\begin{bmatrix}
		\pdv{x}\\
		\pdv{y}\\
		\pdv{z}
		\end{bmatrix}
		\cross
		\begin{bmatrix}
		H_{x}\\
		H_{y}\\
		H_{z}
		\end{bmatrix}
		= 
		\begin{bmatrix}
		\pdv{H_{z}}{y} - ik H_{y}
		ik H_{x} - \pdv{H_{z}}{x}\\
		\pdv{H_{y}}{x} - \pdv{H_{x}}{y}
		\end{bmatrix} 
		= -i \ee \omega
		\begin{bmatrix}
		E_{x}\\
		E_{y}\\
		E_{z}
		\end{bmatrix}
	\end{equation*}
	We have six component equations. 
	
	\item The first and fifth equations both contain $ H_{x} $ and $ E_{y} $.
	
	They read
	\begin{equation*}
		\pdv{E_{z}}{y} - ikE_{y} = i \mm \omega H_{x} \eqtext{and} ikH_{x}  - \pdv{H_{y}}{x}   = -i\ee \omega E_{y}
	\end{equation*}
	We combine to get
	\begin{equation*}
		k \pdv{E_{x}}{y} - i k^{2}E_{y}= \mm \omega \left(\pdv{H_{z}}{x} - i \ee \omega E_{y}\right)
	\end{equation*}
	then
	\begin{equation*}
		k \pdv{E_{z}}{y} - \mm \omega \pdv{H_{z}}{x} = iE_{y}\left(k^{2} - \omega^{2}\mm \ee\right) =  iE_{y}\left(k^{2} - \frac{\omega^{2}}{c^{2}}\right) 
	\end{equation*}
	We find $ E_{y} $ in terms of the $ z $ components with
	\begin{equation*}
		E_{y} = i \frac{k \pdv{E_{z}}{y} - \mm \omega \pdv{H_{z}}{x}}{\frac{\omega^{2}}{c^{2}} - k^{2}}
	\end{equation*}
	
	\item Similarly, the second and fourth component equations contain $ E_{x} $ and $ H_{y} $ in terms of $ E_{z} $ and $ H_{z} $. 
	
	Without derivation (analogous to the above) the other equations for the components $ E_{x}, E_{y} $ and $ H_{x}, H_{y} $ are
	\begin{align*}
		&H_{x} = i \frac{k\pdv{H_{z}}{x} - \omega \ee \pdv{E_{z}}{y}}{\frac{\omega^{2}}{c^{2}} - k^{2}}\\
		&E_{x} = i \frac{k \pdv{E_{z}}{x} + \omega \mm \pdv{H_{z}}{y}}{\frac{\omega^{2}}{c^{2}} - k^{2}}\\
		&H_{y} = i \frac{k\pdv{H_{z}}{y} + \omega \ee \pdv{E_{z}}{x}}{\frac{\omega^{2}}{c^{2}} - k^{2}}\\
	\end{align*}
	Note that these equations apply only in Cartesian coordinates.
	
	\item Next, we consider boundary conditions. Consider a boundary wall of the waveguide. 
	
	Encircle a small portion of the boundary wall with a surface to use Stoke's law.
	
	The wall contains $ \rho(z, t) $ and $ \j(z, t) $. 
	
	We use the other two equations:
	\begin{equation*}
		\curl \E = - \mm \pdv{\H}{t} \eqtext{and} \div \H = 0
	\end{equation*}
	When we send the loop to zero thickness, the first equation gives
	\begin{equation*}
		\curl \E \to 0 \implies E_{\parallel} = 0
	\end{equation*}
	while the second equation gives
	\begin{equation*}
		H_{\perp} \to 0
	\end{equation*}
	
	\item Finally, we find $ E_{z} $ and $ H_{z} $ from the waveguide wave equation
	\begin{equation*}
		\left[\laplacian_{\perp} + \left(\frac{w^{2}}{c^{2}} - k^{2}\right)\right]
		\begin{Bmatrix}
		E_{z}\\
		H_{z}
		\end{Bmatrix}
		= 0
	\end{equation*}
	We consider solutions with two orthogonal components: one with $ E_{z} = 0 $ and $ H_{z} \neq 0 $ and the second with $ H_{z} = 0 $ and $ E_{z} \neq 0 $. The general solution will be a linear combination of the two solutions.
	
	The first method is called transverse electric (TE), since the electric field is perpendicular (transverse) to the direction of EM wave propagation (and $ E_{z} = 0 $). 
	
	The first method is called transverse magnetic (TM), since the magnetic field is perpendicular (transverse) to the direction of EM wave propagation (and $ H_{z} = 0 $). 
	
%	Note that the opposite E or H component has a component in the direction $ z $ of wave propagation---which doesn't hold in free space.
	
\end{itemize}

\subsubsection{Waveguide of Two Parallel Plates}
\textit{Consider a waveguide consisting to two large parallel conducting plates separated by a distance $ a $}
\begin{itemize}
	\item Let $ x $ be vertical (up down in paper), $ z $ left right in paper (direction of wave propagation) and $ y $ out of the page. 
	
	In this case $ \pdv{y} = 0 $ because of translational symmetry along $ y $. 
	
	\item For TM we have $ E_{z} \neq 0 $ and $ H_{z} = 0 $. 
	
	First what remaining components are non-zero. $ H_{y} \neq 0 $. $ E_{y} = 0 $. $ H_{x} = 0 $ and $ E_{x} \neq 0 $. Noting that $ \pdv{y} = 0 $. 
	
	$ \H $ points out of the page along $ y $, while $ \E $ points in the $ xz $ plane.
	
	\item Next, we find $ E_{z} $ using
	\begin{equation*}
		\left[\laplacian_{\perp} + \left(\frac{w^{2}}{c^{2}} + k^{2}\right)\right]E_{z} \equiv \left[\laplacian_{\perp} + \kappa^{2}\right]E_{z}
	\end{equation*}
	Since $ \pdv{y} = 0 $ this becomes
	\begin{equation*}
		\left(\pdv[2]{}{x} + \kappa^{2}\right)E_{z} = 0
	\end{equation*}
	The solution is oscillation. We can write the general solution as
	\begin{equation*}
		E_{z}(x) = A\sin \kappa x + B \cos \kappa x
	\end{equation*}
	
	\item Boundary conditions $ E_{\parallel} = 0 $. The parallel components to waveguide surface are $ z $ and $ y $. We already know $ E_{y} = 0 $, so at the boundary surfaces we have
	\begin{equation*}
		E_{\perp} = E_{z}(0) = E_{z}(a) = 0
	\end{equation*}
	The condition $ E_{z}(0) = 0$ implies  $ B = 0 $ and thus
	\begin{equation*}
		E_{z} = A\sin \kappa x
	\end{equation*}
	While $ E_{z}(a) = 0 $ implies $ \kappa a = n \pi $, so 
	\begin{equation*}
		\kappa = \frac{n\pi}{a}, \qquad n = 1, 2, 3, \ldots
	\end{equation*}
	Note that $ n = 0 $ gives a trivial solution
	
	The TM solution for $ E_{z} $ is then
	\begin{equation*}
		E_{z}(x) = A \sin \frac{n\pi x}{a}
	\end{equation*}
\end{itemize}
\textbf{Dispersion Relation}
\begin{itemize}
	\item We want to find the relationship between $ \omega $ and $ k $.
	
	We use the $ E_{z} $ wave equation and the definition of $ \kappa $ to get
	\begin{equation*}
		\kappa^{2} = \frac{\omega^{2}}{c^{2}} - k^{2} = \left(\frac{n\pi}{a}\right)^{2}
	\end{equation*}
	which implies
	\begin{equation*}
		\omega = c_{0}\sqrt{k^{2} + \left(\frac{n\pi}{a}\right)^{2}}
	\end{equation*}
	At large $ k $, this approaches the free space relation $ \omega = c_{0}k $
	
	\item Note also that frequencies below $ \frac{n\pi}{a} $ are unattainable---in other words, there is a minimum possible frequency at which EM waves can propagate through the waveguide.
	
	\item We call the frequency band spanning the frequencies between $ n = 1 $ and $ n = 2 $ the wave band. The bandwidth is the difference between the $ n = 1 $ and $ n = 2 $ frequencies.
	\begin{equation*}
		\Delta \omega = \frac{c_{0}\pi}{a}
	\end{equation*}
\end{itemize}

\textbf{Impedance}
\begin{itemize}
	\item The impedance for TM waves is defined in terms of the fields as
	\begin{equation*}
		Z \equiv \frac{E_{\perp}}{H_{\parallel}}
	\end{equation*}
	In our case for the parallel-plate waveguide this reads
	\begin{equation*}
		Z = \frac{E_{x}}{H_{y}}
	\end{equation*}
	Using the equations for the components $ E_{x} $ and $ H_{y} $, we have
	\begin{equation*}
		Z = \frac{ik \pdv{E_{z}}{x}}{i \omega \ee \pdv{E_{z}}{x}} = \frac{k}{\omega \ee}
	\end{equation*}
	We use the dispersion relation to get
	\begin{equation*}
		Z = \frac{\sqrt{\frac{\omega^{2}}{c_{0}^{2}} - \left(\frac{n\pi}{a}\right)^{2}}}{\omega \ee} = \frac{1}{c_{0}\ee}\sqrt{1 - \left(\frac{n\pi c_{0}}{a}\right)^{2}\frac{1}{\omega^{2}}}
	\end{equation*}
	The coefficient is
	\begin{equation*}
		\frac{1}{c_{0}\ee} = \frac{\sqrt{\ee \mm}}{\ee} = \sqrt{\frac{\mm}{\ee}} \equiv Z_{0}
	\end{equation*}
	which is the impedance of free space.
	
	\item We also define the frequency
	\begin{equation*}
		\frac{n\pi c_{0}}{a} \equiv \omega_{n}
	\end{equation*} 
	which is the frequency of each brach in the dispersion relation.
	
	In terms of $ \omega_{n} $ then the impedance is
	\begin{equation*}
		Z(\omega) = Z_{0}\sqrt{1 - \left(\frac{\omega_{n}}{\omega}\right)^{2}}
	\end{equation*}
	For large $ \omega $, the impedance approaches $ Z_{0} $. 
\end{itemize}

\textbf{Ratio of Electric Field Amplitudes}
\begin{itemize}
	\item Find the ratio of amplitudes $ \frac{E_{z_{0}}}{E_{x_{0}}} $.
	
	First
	\begin{equation*}
		\frac{E_{z}}{E_{x}} = \frac{E_{z}}{ik\pdv{E_{z}}{x}} \kappa^{2}
	\end{equation*}
	We know $ E_{z} = A \sin \kappa x $ from earlier, and get
	\begin{equation*}
		\frac{E_{z_{0}}}{E_{x_{0}}} = \frac{\kappa}{k} = 
	\end{equation*}
	We can interpret $ \kappa $ as a wave vector in the $ x $ direction, while $ k $ is the wave vector along the direction of propagation $ z $. 
	
%	\item With similar triangles involving $ E_{z}, E_{x}, \E, k, \kappa $ and so on we can show $ \E $ is perpendicular to a vector with magnitude $ \sqrt{\kappa^{2} + k^{2}} $---the hypotenuse of $ \kappa $ and $ k $.
\end{itemize}

\textbf{TE Propagation Mode}
\begin{itemize}
	\item This is analogous to TM, so only sketched. In this case $ E_{z} = 0 $ and we find $ H_{z} $. 
	
	\item The wave equation is solved with
	\begin{equation*}
		H_{z} = A \sin \kappa x + B \cos \kappa x
	\end{equation*}
	
	\item Boundary equations are now different though---$ H_{\perp} = 0 $. The perpendicular component is $ H_{x} $. We turn to the component equations to get
	\begin{equation*}
		H_{x} = i \frac{k \pdv{H_{z}}{x}}{\kappa^{2}}
	\end{equation*}
	Since $ H_{x} = 0 $ at the boundary it follows that
	\begin{equation*}
		\pdv{H_{z}}{x}\Big|_{x = 0, a} = 0
	\end{equation*}
	The boundary condition is now in terms of a derivative---note that $ \pdv{H_{z}}{x} $ is a ``normal'' derivative; since in points in the direction $ x $, which is normal to the waveguide's boundary. 
	
	Note that in general, the boundary condition in TE mode involves the normal derivative of $ H_{z} $ with respect to the waveguide boundary surface. 
	
	\item Taking the derivative of $ H_{z} $ and evaluating the boundary condition gives
	\begin{equation*}
		H_{z}(x) = B\cos \frac{n\pi x}{a}, \qquad n = 1, 2, 3, \ldots
	\end{equation*}
	Note that $ n = 0 $ produces $ H_{z} = B $ which is not at trivial solution  in itself, but the other field components involve derivatives of $ H_{z} $ and are trivial, so we again reject $ n = 0 $. 
	
	\item Since $ \kappa $ is the same as in TM mode, the dispersion relation in TE is the same. 
	
	The only notable difference in TE mode is the boundary condition involving the normal derivative of $ H_{z} $. 
	
	And also impedance turns out to be
	\begin{equation*}
		Z(\omega) = \frac{Z_{0}}{\sqrt{1 - \left(\frac{\omega_{n}}{\omega}\right)^{2}}}
	\end{equation*}
	
\end{itemize}

\textbf{Generalization: waveguide with a rectangular cross section}
\begin{itemize}
	\item In this case,
	\begin{equation*}
		\laplacian_{\perp} = \pdv[2]{}{x} + \pdv[2]{}{y}
	\end{equation*}
	and $ \pdv{y} $ is not zero. We then have
	\begin{equation*}
		\left(\pdv[2]{}{x} + \pdv[2]{}{y} + \kappa^{2}\right)E_{x}(x, y) = 0
	\end{equation*}
	We use separation of variables: $ E_{x}(x, y) = X(x)Y(y) $. We get
	\begin{equation*}
		X''Y + XY'' + \kappa^{2}XY = 0 \implies \frac{X''}{X} + \frac{Y''}{Y} + \kappa^{2} = 0
	\end{equation*}
	The $ x $ and $ y $ terms must be constant for the equation to hold, so we define
	\begin{equation*}
		\frac{X''}{X} = - \kappa_{x}^{2} \eqtext{and} \frac{Y''}{Y} = - \kappa_{y}^{2}
	\end{equation*}
	
	\item The equations then read
	\begin{equation*}
		X'' + \kappa_{x}^{2}X = 0 \eqtext{and} 	Y'' + \kappa_{y}^{2}Y = 0
	\end{equation*}
	We solve the equations like before and get sinusoidal solutions. 
	
	\item Boundary conditions now involve $ a $ and $ b $
	\begin{equation*}
		\kappa_{x} = \frac{n\pi }{a}\eqtext{and} \kappa_{y} = \frac{m \pi}{b}, \qquad n, m = 1, 2, 3, \ldots
	\end{equation*}
	Note $ \kappa^{2} = \kappa_{x}^{2} + \kappa_{y}^{2} $.

	\item  Dispersion relation is
	\begin{equation*}
		\omega = c_{0} \sqrt{k^{2} + \left(\frac{n\pi}{a}\right)^{2} +  \left(\frac{m\pi}{b}\right)^{2}}
	\end{equation*}
	The lowest branch now depends on the ratio of $ a $ and $ b $. 
	
	\item Finally, note that because of both two indices $ n $ and $ m $, in TE mode with cosine solutions one of either $ n $ or $ m $ can be zero.
	
	In TM mode both indices are $ n, m = 1, 2, 3, \ldots $.
\end{itemize}



\end{document}





