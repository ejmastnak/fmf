\documentclass[11pt, a4paper]{article}
\usepackage{mwe}
\usepackage{amsmath}
\usepackage{amssymb}
\usepackage{mathtools}
\usepackage{esint}
\usepackage{graphicx}
\usepackage{xcolor}
\usepackage{bm} % for bold vectors in math mode
\usepackage{physics} % for differential notation, etc...
\usepackage[separate-uncertainty=true]{siunitx}

\usepackage[normalem]{ulem}  % for underline with line wrapping
\usepackage[margin=3cm]{geometry}
\usepackage{fancyhdr}
\usepackage{truncate}
\usepackage[colorlinks = true, allcolors=blue]{hyperref}

\setlength{\parindent}{0pt} % to stop indenting new paragraphs
\newcommand{\diff}{\mathop{}\!\mathrm{d}} % differential
\newcommand{\dr}{\diff^{3} \r}  % d^3 r
\newcommand{\dtr}{\diff^{3} \tilde{\r}}  % d^3 r
\newcommand{\dk}{\diff^{3} \vec{k}}  % d^3 k

\newcommand{\Poy}{Poynting\xspace} 
\newcommand{\eqtext}[1]{\qquad \text{#1} \qquad}

\renewcommand{\L}{\mathcal{L}}  % Lagrange density

\renewcommand{\vec}[1]{\bm{#1}} % for vectors
\newcommand{\uvec}[1]{\hat{\vec{#1}}} % for vectors
\newcommand{\mat}[1]{\mathbf{#1}} % for matrices
\newcommand{\dvec}[1]{\dot{\vec{#1}}} % for dotted vector quantity
\newcommand{\tvec}[1]{\tilde{\vec{#1}}} % for tilde vector quantities
\renewcommand{\t}[1]{\tilde{#1}} % shorthand for tilde

\newcommand{\bdot}[1]{\dot{\vec{#1}}}
\newcommand{\bddot}[1]{\ddot{\vec{#1}}}

\renewcommand{\r}{\vec{r}}
\renewcommand{\k}{\vec{k}}
\newcommand{\E}{\vec{E}} % for electric field; it's used alot!
\newcommand{\D}{\vec{D}}  % electric displacement field
\newcommand{\B}{\vec{B}} % for magnetic field; it's used alot!
\renewcommand{\H}{\vec{H}}  % magnetic field strength
\newcommand{\A}{\vec{A}} % for magnetic vector potential
\renewcommand{\P}{\vec{P}}  % electric polarization
\newcommand{\M}{\vec{M}}  % magnetization
\renewcommand{\S}{\mathbf{S}}  % Poynting vector---conflicts somewhat with slanted \bm{S} for vector surface element, but is better than \bm{P}, which directly conflicts with polarization
\newcommand{\T}{\mathbf{T}}  % electric stress-energy tensor
\newcommand{\TT}{\mathrm{T}}  % electric stress-energy tensor
\newcommand{\e}{\epsilon}
\newcommand{\ee}{\epsilon_{0}}  % vacuum permittivity
\newcommand{\mm}{\mu_{0}}  % vacuum permeability
\newcommand{\pe}{\vec{p}_{e}}  % electric dipole moment
\newcommand{\m}{\vec{m}}  % magnetic dipole moment
\renewcommand{\j}{\vec{j}}  % electric current density

\newcommand{\s}{\vec{s}}  % for multipole expansion
\newcommand{\ds}{\diff^{3} \s}  % for multipole expansion


\renewcommand{\div}{\nabla \mspace{-2mu} \cdot \mspace{-2mu}}
\renewcommand{\curl}{\nabla \cross}
\renewcommand{\grad}{\nabla}
\renewcommand{\laplacian}{\nabla^{2}}

\newcommand{\divp}{\nabla' \mspace{-1mu} \cdot}  % for use in system S' for special relativity
\newcommand{\curlp}{\nabla' \mspace{-2mu} \cross}
\newcommand{\gradp}{\nabla'}
\newcommand{\laplacianp}{\nabla^{2}}


% begin header configuration
\pagestyle{fancy}

% Header and footer on non-section pages (default style)
\fancyhf{}
\fancyhead[R]{\href{https://github.com/ejmastnak/fmf}{\small{\texttt{github.com/ejmastnak/fmf}}}} 
\fancyhead[L]{\textit{\truncate{0.65\headwidth}{\rightmark}}}
\fancyfoot[C]{\thepage} 
\renewcommand{\headrulewidth}{0.1pt}

% Header and footer on section pages---identical to default style, but must be explicitly included
\fancypagestyle{plain}{
    \fancyhf{}  
    \fancyhead[R]{\href{https://github.com/ejmastnak/fmf}{\small{\texttt{github.com/ejmastnak/fmf}}}}  
    \fancyhead[L]{\textit{\truncate{0.65\headwidth}{\rightmark}}} 
    \fancyfoot[C]{\thepage}  % centered page number in footer
    \renewcommand{\headrulewidth}{0.1pt}
}

\renewcommand{\sectionmark}[1]{%
  \markboth{\sectionname \thesection}
  {\noexpand\firstsubsectiontitle}}
\renewcommand{\sectionmark}[1]{}

\renewcommand{\subsectionmark}[1]{%
  \markright{\thesubsection\ \, #1}\gdef\firstsubsectiontitle{#1}}

\newcommand\firstsubsectiontitle{}

% end header configuration

\pdfinfo{
	/Title (Electromagnetic Field Oral Exam Study Guide)
	/Author (Elijan Mastnak)
	/Subject (Physics)
}

\begin{document}
\title{Electromagnetic Field Oral Exam Notes}
\author{Elijan Mastnak}
\date{Winter Semester 2020-2021}
\maketitle

\thispagestyle{empty}  % remove headers from introductory page

\begin{center}
\textbf{About These Notes}
\end{center}
These notes provide answers to typical questions\footnote{The oral exam questions are taken from a list of roughly 100 typical exam questions compiled in the year 2006. The original notes, as well as my English interpretations, are included in the \href{https://github.com/ejmastnak/fmf/tree/main/electromagnetic-field}{\underline{GitHub directory}} containing these notes.} from the oral exam required to pass the course \textit{Elektromagnetno polje} (Electromagnetic Field), a required course for third-year physics students at the Faculty of Math and Physics in Ljubljana, Slovenia. I wrote the notes when studying for the exam and am making them publicly available in the hope that they might help others learning the same material. Although the exact oral exam questions are specific to the physics program at the University of Ljubljana, the content is fairly standard for an undergraduate electromagnetism course and might be useful to others learning similar material. The most up-to-date version can be found on \href{https://github.com/ejmastnak/fmf/tree/main/electromagnetic-field}{\underline{GitHub}}.

\vspace{2mm}
\textit{Navigation}: For easier document navigation, the table of contents is ``clickable'', meaning you can jump directly to a section by clicking the colored section names in the table of contents. Unfortunately, the \textit{clickable links do not work in most online or mobile PDF viewers}; you have to download the file first.

\vspace{2mm}
\textit{On Authorship}: The material in these notes comes entirely from Professor Miha Ravnik's lectures at the University of Ljubljana. Accordingly, credit for the content of these notes goes to Professor Ravnik. I take credit only for compiling the common exam material in one place, typesetting the notes, and translating to English.

\vspace{2mm}
\textit{Disclaimer:} Mistakes---both trivial typos and legitimate errors---are likely. Keep in mind that these are the notes of an undergraduate student in the process of learning the material himself---take what you read with a grain of salt. If you find mistakes and feel like telling me, by \href{https://github.com/ejmastnak/fmf}{Github} pull request, \href{mailto:ejmastnak@gmail.com}{email} or some other means, I'll be happy to hear from you, even for the most trivial of errors.


\vspace{5mm}
\textbf{How to use these notes}
\begin{itemize}
    \item The sections follow the order in which the course material was taught.

    \item Each typical exam question is answered in a dedicated subsection, where the question is written in italics at the beginning of the subsection.
    
    \item Each question subsection may potentially be further divided into subsubsections to better organize the answer material.

\end{itemize}


\newpage

\pagestyle{empty}  % no header in table of contents
\tableofcontents

\newpage

\pagestyle{fancy}  % turn on headers for remainder of document

\section{Electrostatics}

\subsection{Maxwell Equations for Electrostatics}
\textit{Write the first two Maxwell equations for a static electric field. State the equation for electric field lines and derive the Poisson equation.}
    
\subsubsection{Electric Field Lines}
\begin{itemize}
    \item Electric field lines follow the direction of the electric field, and are defined in terms of the arc length parameter $ s $ via
    \begin{equation*}
        \dot{\vec{r}}(s) = \dv{\r}{s} = \frac{\E(\r(s))}{\abs{\E(\r(s))}}
    \end{equation*}

    \item Electric field lines obey the following important properties:
    \begin{enumerate}
        \item Electric field lines point in the direction of the electric field $ \E $ and do not cross. If electric field lines could cross, the electrostatic force on a point charge would not be well-defined in terms of the electric field.

        \item The density of electric field lines in a region of space is proportional to the magnitude of the electric field in that region. 

        \item Electric field lines arise from positive charges and disappear into negative charges (recall that the electric field of a point charge points radially outward for a positive charge and radially inward for a negative charge).

        \item Electric field lines are not closed curves.
    \end{enumerate}
\end{itemize}

\subsubsection{Electric Circulation and the First Maxwell Equation}
\begin{itemize}
    \item The first of the Maxwell equations in electrostatics is
    \begin{equation*}
        \curl \E = 0.
    \end{equation*}
    We derive this equation from the electric circulation around a closed curve $ C $, which is defined as the line integral
    \begin{equation*}
        \Gamma_{\text{E}} = \oint_{C} \E \cdot \diff \r,
    \end{equation*}
    which can be thought of a ``sum'' of the tangentional component of the electric field around the loop. 

    \item Because the electrostatic field is conservative, the electric circulation of an arbitrary closed loop in a static electric field is zero. As a result, we have
    \begin{equation*}
        \Gamma_{\text{E}} = \oint_{C} \E \cdot \diff \r = 0.
    \end{equation*}
    We then apply Stokes' theorem to this equation, and the result $ \curl \E = 0 $:
    \begin{equation*}
        0 \equiv \Gamma_{\text{E}} = \oint_{C}\E \cdot \diff \r = \iint_{S} \curl \E \cdot \diff \vec{S} \implies \curl \E = 0.
    \end{equation*}
    In other words, the curl of any static electric field is zero. Note that this result does not hold in general for time-varying electric fields.

\end{itemize}

\subsubsection{Gauss's Law, the Second Maxwell Equation and the Poisson Equation}
\begin{itemize}
    \item The second Maxwell equation for electrostatics reads
    \begin{equation*}
        \div \E = \frac{\rho}{\ee}.
    \end{equation*}
    This equation is also the differential form of Gauss's law. To derive these results, we start with the integral form of Gauss's law for the electric flux through a closed surface $ S $
    \begin{equation*}
        \Phi_{\text{E}} = \oiint \E \cdot \diff \vec{S} = \frac{Q_{\text{enc}}}{\ee},
    \end{equation*}
    where $ Q_{\text{enc}} $ is the total electric charge enclosed within the surface $ S $. 
    
	\item To find the differential form of Gauss's law, we first write $ Q_{\text{enc}} $ in terms of charge density to get
	\begin{equation*}
        \oiint_{\text{S}} \E \cdot \diff \vec{S} = \frac{Q_{\text{enc}}}{\ee} = \frac{1}{\ee}\iiint_{V}\rho(\r) \dr.
	\end{equation*}
	We then recall the divergence theorem, which reads
	\begin{equation*}
		\oiint_{\partial V} \E \cdot \diff \vec{S} = \iiint_{V}\div \E \dr.
	\end{equation*}
	Applied to Gauss's law, the divergence theorem gives
	\begin{equation*}
		\oiint_{\partial V} \E \cdot \diff \vec{S} = \iiint_{V}\div \E \dr = \frac{1}{\ee} \iiint_{V} \rho(\r) \dr.
	\end{equation*}
	We then equate the integrands in the last equality to get the desired differential form of Gauss's law and the second electrostatic Maxwell equation
	\begin{equation*}
		\div \E = \frac{\rho}{\ee}.
	\end{equation*}

    \item The Poisson equation for the electrostatic potential $ \phi(\r) $ generated by a charge distribution $ \rho(\r) $ is
    \begin{equation*}
        \laplacian \phi = - \frac{\rho}{\ee}.
    \end{equation*}
    
    \item To derive the Poisson equation, we substitute $ E = - \grad \phi $ into the differential form of Gauss's law to get
	\begin{equation*}
		\frac{\rho}{\ee} \equiv \div \E = \div \big[- \grad \phi \big]= - \laplacian \phi.
	\end{equation*}
	The result is the Poisson equation
	\begin{equation*}
		\laplacian \phi = - \frac{\rho}{\ee}.
	\end{equation*}
	In empty space (i.e. in the absence of electric charge) charge density is simply $ \rho(\r) = 0 $, and the Poisson equation reduces to the Laplace equation
	\begin{equation*}
		\laplacian \phi = 0.
	\end{equation*}
    
\end{itemize}
    
\subsubsection{Derivation: Integral Form of Gauss's Law}
\begin{itemize}
	\item Consider a distribution of point charges $ \{q_{i}\} $ at the positions $ \{\r_{i}\} $. We enclose the charges with a sphere of radius $ r $ with normal vector $ \uvec{n} $. In general, the electric flux through a closed surface containing the charges is
	\begin{equation*}
		\Phi_{\text{E}} = \oiint_{S} \E \cdot \diff \vec{S} = \oiint_{S} \E \cdot \uvec{n} \diff S = \oiint_{S} E \diff S \cos \theta,
	\end{equation*}
	where $ \theta $ is the angle between the electric field and normal vector to the surface.
	
    \item In our case, for the collection of point charges $ \{q_{i}\} $, which have electric fields of the form $ \E_{i} = \frac{q_{i}}{4\pi \ee}\frac{\r - \r_{i}}{\abs{\r - \r_{i}}^{3}} $, the electric flux is
	\begin{align*}
		\Phi_{\text{E}} &= \oiint_{S} \E \cdot \diff \vec{S} = \sum_{i} \frac{q_{i}}{4\pi \ee} \oiint_{S} \frac{\cos \theta_{i}}{\abs{\r - \r_{i}}^{2}} \diff S \\
		& = \frac{1}{4\pi \ee} \left[ \oiint_{S} \frac{ q_{1} \cos \theta_{1}}{\abs{\r - \r_{2}}^{2}} \diff S + \oiint_{S} \frac{ q_{2} \cos \theta_{2}}{\abs{\r - \r_{2}}^{2}} \diff S  + \cdots \right],
	\end{align*}
	where $ \theta_{i} $ is the angle between each particle's $ \E $ field and the normal vector $ \uvec{n} $. 
	
	\item Next, note that for charges at the origin, the angle $ \theta_{i} $  is zero. We choose the origin so that $ \r_{1} = 0 $. Then $ \theta_{1} = 0 $, $ \cos \theta_{1} = 1 $ and the first integral is (using $ \diff S = \sin \theta \diff \theta \diff \phi $)
	\begin{equation*}
		\oiint_{S} \frac{q_{1}}{r^{2}} \diff S = \int_{\phi = 0}^{2\pi} \int_{\theta = 0}^{\pi} \frac{q_{1}}{r^{2}}r^{2} \sin \theta \diff \theta \diff \phi = 4\pi q_{1}.
	\end{equation*}
	The idea is to play the same game for each integral, switching the origin one at a time and arguing that the value of the integral is the same regardless of the choice of origin. The end result is Gauss's law:
	\begin{equation*}
		\Phi_{\text{E}} = \oiint_{S} \E \cdot \diff \S = \frac{1}{4\pi \ee} \big[4\pi q_{1} + 4\pi q_{2} + \cdots \big] = \frac{Q_{\text{enc}}}{\ee},
	\end{equation*}
	where $ Q_{\text{enc}} = \sum_{i} q_{i} $. In terms of charge density, this relationship reads
	\begin{equation*}
		\Phi_{\text{E}} = \oiint_{\partial V} \E \cdot \diff \vec{S} = \frac{1}{\ee}\iiint_{V} \rho(\r) \dr,
	\end{equation*}
	where $ \partial V $ denotes the boundary surface enclosing the charges in the volume $ V $.
	
\end{itemize}






\subsection{Charge Distributions}
\textit{Provide examples of a few common charge distributions. Derive the magnitude of the electric field's perpendicular component for a surface charge distribution.}
    
\vspace{2mm}
\textbf{Point Charge} 
\begin{itemize}
	\item The charge density of a point charge located at the position $ \r_{0} $ is
	\begin{equation*}
		\rho (\r) = q \delta^{3}(\r - \r_{0}).
	\end{equation*}
\end{itemize}	
	

\textbf{Electric Dipole} 
\begin{itemize}
	\item Using the charge density of a point charge, the charge density of an electric dipole dipole with positive and negative charges at $ \r_{+} $ and $ \r_{-} $, respectively, is
	\begin{equation*}
		\rho(\r) = q \delta^{3}(\r - \r_{+}) - q \delta^{3}(\r - \r_{-}).
	\end{equation*}
	Alternatively, if the dipole's center occurs at $ \r_{0} $ and the positive and negative charges occur at $ \r_{0} + \delta \r$ and $ \r_{0} - \delta \r $, the dipole's charge density is
	\begin{equation*}
		\rho(\r) = q \delta^{3}(\r - \r_{0} - \delta \r) - q\delta^{3}(\r - \r_{0} + \delta \r).
	\end{equation*}
	Next, we consider the limit case $ \abs{\delta \r} \ll \abs{\r_{0}} $, i.e. when the distance $ \delta r $ between the dipole charges is small compared to the distance $ r_{0} $ to the dipole. We expand the charge density to get
	\begin{equation*}
		\rho(\r) \approx - q \delta \r \cdot \grad \big(\delta^{3}(\r - \r_{0})\big) - q \delta \r \cdot \grad \big(\delta^{3}(\r - \r_{0})\big) = -2 q \delta \r \cdot \grad \big(\delta^{3}(\r - \r_{0})\big).
	\end{equation*}
	The term $ 2q\delta \r $ is the dipole moment $ \pe $, and the dipole's charge density is then
	\begin{equation*}
		\rho(\r) = -\pe \cdot \grad \big(\delta^{3}(\r - \r_{0})\big).
	\end{equation*}

\end{itemize}

\textbf{Surface Charge Distribution}
\begin{itemize}
	\item We analyze a surface charge distribution in terms of surface charge density $ \sigma(\r) $. The corresponding volume charge density $ \rho $ is
	\begin{equation*}
		\rho(\r) = \sigma(\r) \delta(z - z_{0}),
	\end{equation*}
	where we assume the surface occurs at $ z = z_{0} $. Recall the delta function has units $ \si{length}^{-1} $, so units are consistent.
\end{itemize}

\textbf{Spherical Charge Distribution}
\begin{itemize}	
	\item The charge density of a uniform spherical charge distribution of radius $ a $ is
	\begin{equation*}
		\rho(\r) = \rho_{0}H(a - r)
		\begin{cases}
			\rho_{0} & r < a\\
			0 & r > a,
		\end{cases}
	\end{equation*}
	where $ H $ is the Heaviside function.
\end{itemize}
	
\textbf{Spherical Dipole}
\begin{itemize}
	\item The polarization of a spherically-distributed dipole within a sphere of radius $ a $ is 
	\begin{equation*}
		\vec{P}(\r) = \dv{\pe}{V} = 
		\begin{cases}
			\vec{P}_{0} & r < a\\
			0 & r > a
		\end{cases}
		= \vec{P}_{0}H(a - r).
	\end{equation*}
	The corresponding charge density is
	\begin{align*}
		\rho(\r) &= - \div \vec{P} = - \div \big[\vec{P}_{0}H(a - r)\big] = - (\div \vec{P}_{0})H(a - r) - \vec{P}_{0} \grad H(a -r)\\
		&= 0 - \vec{P}_{0} \delta(a-r) \cdot \left (-\frac{\r}{r}\right) = \vec{P}_{0}\frac{\r}{r} \delta(a-r).
	\end{align*}
	Note that the charge density occurs only at $ r = a $ (at the sphere's surface). This is because internal charges from internal electric dipoles within the sphere cancel each other out in the sphere's volume.
\end{itemize}


\subsection{The Poisson Equation for the Electrostatic Potential}
\textit{State the solution to the Poisson equation for a point particle and for a generalized charge distribution. State and derive the Green's function for the Poisson equation.}

\begin{itemize}
    \item The solution to the Poisson equation for a point charge $ q $ is
    \begin{equation*}
        \phi(\r) = \frac{q}{4\pi \ee r}.
    \end{equation*}
    The general solution to the Poisson equation for spatial charge distribution $ \rho(\r) $ is
    \begin{equation*}
        \phi(\r) = \iiint_{V} \frac{\rho(\tilde{\r})}{4 \pi \ee \abs{\r - \tilde{\r}}} \dtr.
    \end{equation*}

    \item The Green's function for the Poisson equation is
    \begin{equation*}
        G(\r- \r_{0}) = \frac{1}{4\pi \ee \abs{\r - \tilde{\r}}}.
    \end{equation*}
    
\end{itemize}

\textbf{Derivation: Green's Function for the Poisson Equation}
\begin{itemize}
	\item The Poisson equation for a point charge is
	\begin{equation*}
		\laplacian \phi = - \frac{\rho}{\ee}.
	\end{equation*}
	We find the solution using the convolution
	\begin{equation*}
		\phi(\r) = \iiint_{V} G(\r - \t{\r})\rho(\t{\r}) \dtr,
	\end{equation*}
	where $ G $ is the Green's function for the equation. Solving the Poisson equation thus reduces to finding the Green's function $ G $.
	
% Interpretation: imagine multiplying $ \rho $ by a constant factor---in this case $ \phi $ should increase by the same constant factor. As a result, $ \rho $ and $ \psi $ are related, and this relationship is encoded in the convolution.
	
	\item We first substitute the convolution ansatz for $ \phi $ into the Poisson equation to get
	\begin{equation*}
		\laplacian \phi(\r) = \laplacian \left[\iiint_{V}  G(\r - \t{\r})\rho(\t{\r}) \dtr \right] = \iiint_{V} \laplacian G(\r - \t{\r})\rho(\t{\r}) \dtr = - \frac{\rho(\r)}{\ee}.
	\end{equation*}
	Note that the Laplacian $ \laplacian $ applies only to the coordinates of $ \r $ and not to $ \t{\r} $. For the equality to hold, the Green's function must obey
	\begin{equation*}
		\laplacian G(\r - \t{\r}) = -\frac{\delta^{3}(\r - \t{\r})}{\ee},
	\end{equation*}
	which reproduces the desired equality
	\begin{equation*}
		\iiint_{V} \laplacian G(\r - \t{\r})\rho(\t{\r}) \dtr = \iiint_{V} \left(-\frac{\delta^{3}(\r - \t{\r})}{\ee}\right)\rho(\t{\r}) \dtr =  - \frac{\rho(\r)}{\ee}.
	\end{equation*}
	Note the similarity of the expression
	\begin{equation*}
		\laplacian G(\r - \t{\r}) = -\frac{\delta^{3}(\r - \t{\r})}{\ee}
	\end{equation*}
	to the charge density of a point charge. In fact, it turns out that the Green's function (up to the charge factor $ q $) is the solution to the Poisson equation for a point charge. 
	
	\item We find the Green's function for an arbitrary charge distribution by transforming to Fourier space, which simplifies the Laplace operator to multiplication by $ -k^{2} $.
	
	As an auxiliary calculation, the Green's function in Fourier space is
	\begin{equation*}
		G(\r - \t{\r}) = \iiint \frac{\dk}{(2\pi)^{3}}e^{i\k(\r - \t{\r})}G(\k),
	\end{equation*}
	while the delta function in Fourier space is
	\begin{equation*}
		\delta(\r - \t{\r}) = \iiint \frac{\dk}{(2\pi)^{3}}e^{i\k(\r - \t{\r})}\cdot 1.
	\end{equation*}
	
    \item Substituting the above two results into the Poisson equation gives
	\begin{equation*}
		\laplacian \left(\iiint \frac{\dk}{(2\pi)^{3}} e^{i\k(\r - \t{\r})} G(\k)\right) = -\frac{1}{\ee} \left(\iiint \frac{\dk}{(2\pi)^{3}}e^{i\k(\r - \t{\r})} \right).
	\end{equation*}
	We write the equation in terms of a single integral to get
	\begin{equation*}
		\iiint \frac{\dk}{(2\pi)^{3}} \left[\laplacian e^{i\k(\r - \t{\r})}G(\k) + \frac{1}{\ee}e^{i\k(\r - \t{\r})}\right] = 0,
	\end{equation*}
	and then factor out the exponential term to get
	\begin{equation*}
		\iiint \frac{\dk}{(2\pi)^{3}} \left[-k^{2} G(\k) + \frac{1}{\ee}\right]e^{i\k(\r - \t{\r})} = 0.
	\end{equation*}
	For this equality to hold for all $ \k $ and $ \r $, the quantity in brackets must be zero, which gives us for the expression for the Green's function in $ \k $ space:
	\begin{equation*}
		-k^{2} G(\k) + \frac{1}{\ee} \equiv 0 \implies G(\k) = \frac{1}{\ee k^{2}}.
	\end{equation*}
	
	\item We now aim to transform $ G(\k) $ from $ \k $ space back into position space via
	\begin{equation*}
		G(\r - \t{\r}) = \iiint \frac{\dk}{(2\pi)^{3}}e^{i\k(\r - \t{\r})}G(\k) = \iiint \frac{\dk}{(2\pi)^{3}}e^{i\k(\r - \t{\r})}\left(\frac{1}{\ee k^{2}}\right).
	\end{equation*}
	We perform the integral over $ \k $ by converting to spherical coordinates
	\begin{align*}
		G(\r - \t{\r}) &= \frac{1}{\ee}\int_{0}^{\infty} \int_{0}^{\pi} \int_{0}^{2\pi} \diff \phi \sin \theta \diff \theta k^{2} \diff k \frac{1}{(2\pi)^{3}}e^{i\k(\r - \t{\r})} \frac{1}{k^{2}} \\
		& = \frac{1}{\ee}\frac{1}{(2\pi)^{2}}\int_{0}^{\infty} \int_{-1}^{1} \diff [\cos \theta]  e^{ik\abs{\r - \t{\r}}\cos \theta} \diff k\\
		& = \frac{1}{4\pi \ee \abs{\r - \t{\r}}},
	\end{align*}
	where the last equality relies on the integral $ \int_{0}^{\infty}\frac{\sin x}{x} = \frac{\pi}{2} $. Note that the $ k^{2} $ terms in the first line cancel, which allows the simple conversion back to position space.
	
	\item  Using the just-derived Green's function, the general solution to the Poisson equation for electric potential $ \phi(\r) $  is
	\begin{equation*}
		\phi(\r) = \phi(\r) = \iiint_{V} G(\r - \t{\r})\rho(\t{\r}) \dtr = \iiint_{V} \frac{\rho(\t{\r})}{4\pi \ee \abs{\r - \t{\r}}}\dtr.
	\end{equation*}
    The charge distribution's associated electric field, using $ \E = - \grad \phi $, is
	\begin{equation*}
		\E = - \grad \phi(\r) = - \grad \iiint \frac{\rho(\t{\r})}{4\pi \ee \abs{\r - \t{\r}}}\dtr = \iiint \frac{\rho(\t{\r})\dtr}{4\pi \ee} \frac{\r - \t{\r}}{\abs{\r - \t{\r}}^{3}}.
	\end{equation*}
    These results are important: they means that as long as we know a charge distribution charge density $ \rho(\r) $, we can find the corresponding electric potential $ \phi(\r) $ and electric field $ \E(\r) $. 
	
	% \item Note that the general solution for $ \phi(\r) $ is a generalized form of the electric potential of a point particle of charge $ q $, for which the electric potential is
	% \begin{equation*}
	% 	\phi(\r) = \frac{q}{4\pi \ee r}.
	% \end{equation*}
	% This similarity indicates the Poisson equation and its general solution is consistent with Coulomb's law and the potential of a point particle.
	
\end{itemize}

\subsection{Electrostatic Energy in an External Electric Field}
\textit{Stat and derive the electrostatic energy of a charge distribution in an external electric field.}

\begin{itemize}
    \item The electrostatic energy $ W_{\text{E}} $ associated with a charge distribution $ \rho(\r) $ contained in the volume $ V $ placed in an external electric field $ \E(\r) $ is
    \begin{equation*}
        W_{\text{E}} = \iiint_{V} \rho(\r) \phi(\r) \dr,
    \end{equation*}
    where $ \phi $ is the electric potential associated with the external field $ \E $, and is entirely unrelated to the charge distribution $ \rho $.
    
\end{itemize}

\textbf{Derivation: Electrostatic Energy in an External Field}
\begin{itemize}
	\item Consider a charge $ q $ in a known external electric field $ \E $. Naturally, we cannot create an electric field out of thin air, but for this section we ignore the charges responsible for the electric field and take the field's existence for granted.
	
	\item In the electric field, the particle experiences a force $ \vec{F} = q \E $. We interpret electrostatic energy as the potential energy needed to keep the particle at rest, or alternatively, the work against the electrostatic force needed to bring the particle to its position from infinity. The differential of work is
	\begin{equation*}
		\diff W = - \vec{F} \cdot \diff \r = - q \E \cdot  \diff \r = q \grad \phi \cdot \diff \r.
	\end{equation*}
	By the work-energy theorem, the total work $ W $ and electrostatic energy $ W_{\text{E}} $ are related by
	 \begin{equation*}
		W = \int_{(1)}^{(2)} \diff W = W_{\text{E}}^{(2)} - W_{\text{E}}^{(1)} = q \int_{(1)}^{(2)}\grad \phi \diff \r = q\big(\phi^{(2) }- \phi^{(1)}\big).
	\end{equation*}
	From the equality $ W_{\text{E}}^{(2)} - W_{\text{E}}^{(1)} = q\big(\phi^{(2) }- \phi^{(1)}\big) $, the electrostatic energy of a point charge in an external electric field is thus
	\begin{equation*}
		W_{\text{E}} = q \phi,
	\end{equation*}
	where $ \phi $ is the electric potential arising from the \textit{external} field. 
	
	\item For a continuous charge distribution $ \rho(\r) $, the appropriate generalization is
	\begin{equation*}
		W_{\text{E}} = \iiint_{V} \rho(\r) \phi (\r) \dr,
	\end{equation*}
	where $ \phi $ is again the electric potential arising from the external field only (and has nothing to do with the charge distribution $ \rho(\r) $), while $ V $ is the region of space containing $ \rho $.
\end{itemize}

    
\subsection{Total Electrostatic Field Energy} \label{ss:total-E-energy}
\textit{State and derive the total electrostatic energy in an electric field. Give the expression for total electoric field energy in terms of both the scalar potential $ \phi $ and the electric field $ \E $.}

\begin{itemize}
    \item The total electrostatic energy $ W_{\text{E}} $ contained in the electric field $ \E $ asociated with the electric potential $ \phi $ and generated by a charge distribution $ \rho $ is
    \begin{equation*}
        W_{\text{E}} = \frac{1}{2}\iiint_{V} \rho(\r)\phi(\r)\dr = \frac{\ee}{2} \iiint_{V} E^{2} \dr,
    \end{equation*}
    where $ V $ is the region of space containing the charge distribution $ \rho $.
    
\end{itemize}
\subsubsection{Derivation: Electric Field Energy In Terms of Potential}
\begin{itemize}

    % \item We now consider the total electrostatic energy of charge distribution in an electric field with respect to both the charge distribution and the charges creating the external field. 

	\item To derive the expression for total electric field energy, we consider the work involved in assembling the charge distribution $ \rho $ generating the electric potential $ \rho $ from previously empty space. To model the assembly process, we introduce parameter $ \alpha \in [0, 1] $, which continuously ``turns on'' the charge distribution. 

	We then consider the change in energy $ \diff W_{\text{E}} $ from adding additional charge density $ \diff \rho = \rho(\r) \diff \alpha $ to the charge we have already assembled. The existing charges create an external field like in the previous problem, where the associated electrostatic energy is $ W_{\text{E}} = \int \t{\rho}(\r) \t{\phi} (\r) \dr $. 	The corresponding differential $ \diff W_{\text{E}} $ is
	\begin{align*}
		\diff W_{\text{E}} &= \iiint_{V} \diff \t{\rho}(\r)\t{\phi}(\r) \dr = \iiint_{V} \rho(\r) \diff\alpha \big[\alpha \phi (\r)\big ] \dr \\
		& = \alpha \diff \alpha \iiint_{V} \rho(\r) \phi(\r) \dr.
	\end{align*}
	where we have applied the linearity of the Poisson equation, i.e. $ \laplacian [\alpha \phi] = - \frac{\alpha \rho }{\ee} $. 
	
	\item We then find the total field energy by integrating over $ \alpha $ from 0 to 1:
	\begin{equation*}
		W_{\text{E}} = \diff W_{\text{E}} = \int_{0}^{1} \left[ \alpha \iiint_{V} \rho(\r) \phi(\r) \dr \right] \diff \alpha = \frac{1}{2} \iiint_{V} \rho(\r) \phi(\r) \dr.
	\end{equation*}
	This result refers to the electrostatic energy of the charge distribution $ \rho $ creating the electric potential $ \phi $.
	
	\item Note the difference from the previous section: In this section, the charge distribution $ \rho(\r) $ \textit{creates} the electric potential $ \phi $, and in the previous section, the charge density $ \rho $ was entirely unrelated to the electric potential $ \phi $, whose existence we took for granted and otherwise ignored.
	
	\item To write the just-derived electric field energy in terms of only the charge density creating the field, we substitute the Green's function representation of electric potential, i.e.
	\begin{equation*}
		\phi(\r) = \frac{1}{4\pi \ee}\iint_{V} \frac{\rho(\t{\r})}{\abs{\r - \t{\r}}} \dtr,
	\end{equation*}
	into the electric field energy $ W_{\text{E}} $ to get
	\begin{equation*}
		W_{\text{E}} = \frac{1}{2} \iiint_{V} \rho(\r) \phi(\r) \dr = \frac{1}{8 \pi \ee} \iiint_{V} \left[\iiint_{V} \frac{\rho(\r) \rho(\t{\r})}{\abs{\r - \t{\r}}} \dtr\right] \dr,
	\end{equation*}
	where we stress that $ \rho $ is the charge distribution creating the electric field.
\end{itemize}


\subsubsection{Derivation: Electric Field Energy In Terms of Electric Field}
\begin{itemize}
	\item To write electric field energy in terms of only electric field $ \E $, we begin with
	\begin{equation*}
		W_{\text{E}} = \frac{1}{2} \iiint_{V} \rho(\r) \phi(\r) \dr
	\end{equation*}
	and use Gauss's law $ \E = \div \frac{\rho}{\ee} $ to get
	\begin{equation*}
		W_{\text{E}}  = \frac{1}{2} \iiint_{V} \ee (\div \E) \phi(\r) \dr.
	\end{equation*}
	
    \item Next, we use a reverse-engineered form of the vector calculus identity 
	\begin{equation*}
		\div (f \vec{g}) = \grad f \cdot \vec{g} + f (\div \vec{g}) \implies (\div \E) \phi(\r) = \div (\phi \E) - \grad \phi \cdot \E,
	\end{equation*}
	followed by the divergence theorem, to get
	\begin{align*}
		W_{\text{E}} &= \frac{\ee}{2} \iiint_{V}\left[\div (\phi \E) - \grad \phi \cdot \E\right] \dr = \frac{\ee}{2} \oiint_{\partial V} \phi \E \diff \vec{S} - \frac{\ee}{2} \iiint_{V} (\grad \phi) \E \dr\\
		& =  \frac{\ee}{2} \oiint_{\partial V} \phi \E \diff \vec{S} - \frac{\ee}{2} \iiint_{V} (- \E) \E \dr,
	\end{align*}
	where we've used $ \grad \phi = - \E $ in the last equality. 
	
	\item We then assume the integration volume $ V $ enclosing the charge distribution $ \rho $ is large, in which case the integral of $ \phi \E $ over the surface $ \partial V $ is zero. Here's why: The electric potential falls as $ \phi \sim r^{-1} $ with increasing distance $ r $, and the electric field falls as $ \E \sim r^{-2} $, while surface area grows as $ \vec{S} \sim r^{-2} $. The integrand  $ \phi \E \cdot \diff \vec{S} $ thus falls as $ r^{-1} $ and vanishes as $ r \to \infty $. What remains is the familiar formula
	\begin{equation*}
		W_{\text{E}} = \frac{\ee}{2} \iiint_{V} E^{2} \dr.
	\end{equation*} 
	\textit{Interpretation}: This is the energy associated with the electric field $ \E $ resulting from assembling a charge distribution $ \rho(\r) $ in the originally empty region $ V $.

\end{itemize}

    
\subsection{Electrostatic Force}
\textit{State and derive the force on a charge distribution in an external electric field in terms of the electrostatic stress tensor. Use the result to calculate the electric force between two point charges of a.) equal and b.) opposite charge.}

\begin{itemize}
    \item The force on a charge distribution $ \rho $ contained in the volume $ V $ and permeated by an electric field $ \E $ is
    \begin{equation*}
        \vec{F} = \ee \oiint_{\partial V} \left[ \E (\E \cdot \uvec{n}) - \frac{1}{2} E^{2}\uvec{n} \right] \diff S.
    \end{equation*}
    Note that the electric field accounts for both a potential external field $ \E_{\text{ext}} $ and the field $ \E_{\rho} $ generated by the charge distribution itself.

    \item In terms of the electrostatic stress tensor, the force, in component form, reads
    \begin{equation*}
        F_{i} = \oiint_{\partial V} \TT_{ik}\hat{n}_{k}\diff S = \iiint_{V}\pdv{\TT_{ik}}{r_{k}}\dr,
    \end{equation*}
    where the stress tensor's components are given by
    \begin{equation*}
        \TT_{ik} = \ee \left( E_{i}E_{k} - \frac{1}{2}E^{2}\delta_{ik} \right).
    \end{equation*}
    
\end{itemize}


\subsubsection{Derivation: Electric Force in Terms of of Electric Field}
\begin{itemize}

    \item Consider a body with charge distribution $ \rho(\r) $ generating an electric field $ \E_{\rho} $ and exposed to an external electric field $ \E_{\text{ext}} $. We are interested in the force acting on the charge distribution due to the total electric field $ \E = \E_{\rho} + \E_{\text{ext}}$, and we want to write this force in terms of only the total electric field $ \E $.

	We find the electric force on the charge distribution by generalizing $ \vec{F} = q \E_{\text{ext}} $. 

    \textit{Important}: The field $ \E_{\rho} $ created by the charge distribution $ \rho(\r) $ does not contribute to the electric force on the charge distribution---this makes sense, since a body cannot create a net force on itself. As a result, we can write $ \vec{F} = q \E_{\text{ext}} \to q \E $, or, in terms of the continuous charge distribution $ \rho $,
	\begin{equation*}
		\vec{F} = \iiint_{V} \rho(\r) \E_{\text{ext}}(\r) \dr = \iiint_{V} \rho(\r) \E(\r) \dr,
	\end{equation*}
	where $ V $ is the region of space containing the charge distribution.
	
    \label{ss:e-force}
	\item Next, we use Gauss's law $ \rho(\r) = \ee \div \E_{\rho} $, to write the electric force in the form
	\begin{equation*}
		\vec{F} = \ee \iiint_{V} (\div \E_{\rho}) \E(\r) \dr \to \ee \iiint_{V} (\div \E) \E(\r) \dr 
	\end{equation*}
    \textit{Technicality}: The electric field in the $ \div \E_{\rho} $ term corresponds only to the charge distribution $ \rho(\r) $ and is different from $ \E $, which also includes the external field $ \E_{\text{ext}} $.

    However, we can safely make the assumption $ \rho(\r) = \ee \div \E_{\rho} \to \ee \div \E $. Here's why: assuming the charge distribution $ \rho $---associated with our charged body in the external field $ \E_{\text{ext}} $---is localized in space (as makes sense for a body), then near the body itself, $ \rho $ will dominate the external charge distribution $ \rho_{\text{ext}} $ creating the external field $ \E_{\text{ext}} $, which we assume is far away. Since $ \rho \gg \rho_{\text{ext}} $ near the charged body, we can safely incorporate $ \rho_{\text{ext}} $ into $ \rho $, which allows us to write $ \rho(\r) = \ee \div \E_{\rho} \to \ee \div \E $.

    \item We then proceed with the expression
	\begin{equation*}
		\vec{F} = \ee \iiint_{V} (\div \E) \E(\r) \dr,
	\end{equation*}
	where $ \E $ is the total electric field. The integral currently runs over the entire body's volume, and our next step is to convert to an integral over the body's surface. To do this, we apply the vector calculus identity
	\begin{equation*}
		\div (\E \otimes \E) = \E \cdot (\grad \E) + (\E \cdot \grad) \E,
	\end{equation*}
	in terms of which the electric force reads
	\begin{align*}
		\vec{F} &= \ee \iiint_{V} \big[\div (\E \otimes \E) - (\E \cdot \grad) \E \big] \dr \\
		& = \ee \oiint_{\partial V} \E(\E \cdot
	\end{align*}
	where the last equality uses the divergence theorem. 
	
	\item We then introduce a normal vector $ \uvec{n} $ to the body's surface to get
	\begin{equation*}
		\vec{F} = \ee \oiint_{\partial V} \E (\E \cdot \uvec{n}) \diff S - \ee \iiint_{V} (\E \cdot \grad)\E \dr.
	\end{equation*}
	Next, we use the vector identity $ \frac{1}{2} \grad (E^{2}) = (\E \cdot \grad)\E + \E \cross (\curl \E) $, combined with the electrostatic Maxwell equation $ \curl \E = 0 $ (which doesn't hold in the presence of a time-varying magnetic field) to get
	\begin{equation*}
		\vec{F} = \ee \oiint_{\partial V} \E (\E \cdot \uvec{n}) \diff S - \frac{\ee}{2} \iiint_{V} \grad E^{2}\dr.
	\end{equation*}
	Finally, we apply the scalar field corollary of the divergence theorem to the integral of $ \grad E^{2} $ over the volume $ V $ to get
	\begin{equation*}
		\vec{F} = \ee \oiint_{\partial V} \E (\E \cdot \uvec{n}) \diff S - \frac{\ee}{2} \oiint_{\partial V}E^{2}\uvec{n} \diff S = \ee \oiint_{\partial V} \left[\E(\E\cdot \uvec{n}) - \frac{E^{2}}{2}\uvec{n} \right] \diff S.
	\end{equation*}
	As desired, the integral runs only over the body's surface $ \partial V $, and the electric force is expressed only in terms of the total electric field $ \E $. Note that the information about the body's charge distribution $ \rho $ is hidden in the total electric field, since we converted $ \rho(\r) $ to $ \E $ using Gauss's law.
	
\end{itemize}


\subsubsection{The Electrostatic Stress-Energy Tensor}
\begin{itemize}
	\item We can write the above expression for electrostatic force using a rank-two tensor called the electrostatic stress-energy tensor. By components, in terms of the stress tensor $ \TT_{ik} $, the force is
	\begin{equation*}
		F_{i} = \oiint_{\partial V} \TT_{ik} \hat{n}_{k} \diff S,
	\end{equation*}
	where the components of the stress tensor are defined as
	\begin{equation*}
		\TT_{ik} = \ee \left(E_{i}E_{k} - \frac{1}{2}E^{2}\delta_{ik} \right).
	\end{equation*}
	
	\item We can also write the electrostatic force as a volume integral:
	\begin{equation*}
		F_{i} = \oiint_{\partial V} \TT_{ik} \hat{n}_{k} \diff S = \iiint_{V} \pdv{\TT_{ik}}{r_{k}} \dr.
	\end{equation*}
	The quantity $ \pdv{\TT_{ik}}{r_{k}} $ represents electrostatic force density $ f(\r) $ in the direction $ \uvec{r}_{i} $:
	\begin{equation*}
		f_{i}(\r) = \pdv{\TT_{ik}}{r_{k}}.
	\end{equation*}
	
\end{itemize}
    

\subsection{Multipole Expansion of Electric Potential}
\textit{State and derive the multipole expansion of the electric potential up to the first order term, and use the result to define electric dipole moment, and the electric potential and electric field of an electric dipole.}

\begin{itemize}
    \item The electric dipole moment of a (localized) charge distribution $ \rho $ is defined as
    \begin{equation*}
        \pe = \iiint_{V} \r \rho(\r) \dr,
    \end{equation*}
    where $ V $ is a region enclosing the charge distribution.

    \item The electric potential and electric field of an electric dipole are defined as
    \begin{equation*}
        \phi(r) = \frac{\r \cdot \pe}{4\pi \ee r^{3}} \qquad \text{and} \qquad \E(\r) = \frac{3(\pe \cdot \r)\r - \pe r^{2}}{4 \pi \ee r^{5}},
    \end{equation*}
    where $ \pe $ is the dipole's electric dipole moment.
    
\end{itemize}


\textbf{Derivation Multipole Expansion of Electric Potential}
\begin{itemize}
	\item Consider a charge distribution with density $ \rho(\s) $ localized in the region of space $ V $. Our goal is to find an expression for the electric potential $ \phi(\r) $ far from the charge distribution, written in terms of a multipole expansion of $ \rho(\s) $.
	
    Note that the charge distribution $ \rho(\s) $ must be localized in space for the idea of ``far away'' to make sense.

    \item We begin by writing the electric potential $ \phi(\r) $ in the form
	\begin{equation*}
		\phi(\r) = \frac{1}{4\pi \ee}\iiint_{V}\frac{\rho(\s)}{\abs{\r - \s}} \ds.
	\end{equation*}
    We then expand the integrand to first order in the limit $ \abs{\r} \gg \abs{\s}  $, which produces
	\begin{equation*}
        \frac{1}{\abs{\r - \s}} = \frac{1}{\abs{\r}} - \s \cdot \grad \frac{1}{\abs{\r}} + \cdots = \frac{1}{\abs{\r}} + \frac{\s \cdot \r}{\abs{\r}^{3}} + \cdots,
	\end{equation*}
    where we have used the identity $ \grad \frac{1}{\abs{\r}} = \frac{\r}{\abs{\r^{3}}} $.

    \item We then substitute the above expansion into the expression for $ \phi(\r) $ to get
	\begin{align*}
		\phi(\r) &= \frac{1}{4\pi \ee}\iiint_{V}\frac{\rho(\s)}{\abs{\r - \s}} \ds \\
        & \approx \frac{1}{4\pi \ee \abs{\r}}\iiint_{V} \rho(\s) \ds + \frac{1}{4\pi \ee}\frac{\r}{\abs{\r}^{3}}\iiint_{V} \s \rho(\s)\ds\\
		& = \frac{q}{4\pi \ee r} + \frac{\r \cdot \pe}{4\pi \ee r^{3}},
	\end{align*}
    where we have written the integrals in terms of the charge $ q $ and dipole moment $ \pe $, which are defined as
	\begin{equation*}
		q = \iiint_{V} \rho(\s) \ds \eqtext{and} \pe = \iiint_{V} \s \rho(\s) \ds.
	\end{equation*}
    The electric field and electric potential of an electric dipole are thus
    \begin{equation*}
        \phi(\r) = \frac{\r \cdot \pe}{4\pi \ee r^{3}} \qquad \text{and} \qquad \E(\r) = \frac{3 (\pe \cdot \r) \r - \pe r^{2}}{4 \pi \ee r^{5}}.
    \end{equation*}
    
    
	
	\item Without derivation, a higher-order expansion of the electrostatic potential reads
	\begin{equation*}
		\phi(\r) = \frac{1}{4\pi \ee} \sum_{l=0}^{\infty}\sum_{m=-l}^{l}\frac{4\pi}{(2l+1)}\frac{Q_{lm}}{r^{l+1}}Y_{l}^{m}(\theta, \phi).
	\end{equation*}
	In other words, the multipole expansion is an expansion of $ \phi(\r) $ in the basis of the spherical harmonics $ Y_{l}^{m} $. We find the multipole moments $ Q_{lm} $ with
	\begin{equation*}
		Q_{lm} = \iiint_{V} \rho(\s)s^{l}Y_{l}^{m}(\theta, \phi) \ds.
	\end{equation*}
\end{itemize}

    
\subsection{Multipole Expansion of Electrostatic Energy, Force and Torque}
\textit{State and derive the multipole expansion of electrostatic energy up to the dipole term for a charge distribution in an external electric field. Use the result to derive the force and torque on an electric dipole in an external electric field.}

\begin{itemize}
    \item The multipole expansion of the electrostatic energy of a charge distribution $ \rho(\r) $ localized around the position $ \r_{0} $ and placed in an external electric field $ \E $ is
    \begin{equation*}
        W_{\text{E}} = q \phi(\r_{0}) - \E(\r_{0}) \cdot \pe + \cdots,
    \end{equation*}
    where $ q $ and $ \pe $ are the distribution's monopole and dipole moments, respectively. Note that the dipole energy is minimized when the dipole is parallel to the electric field.

    \item The electrostatic force $ \vec{F} $ and torque $ \vec{M} $ on an electric dipole localized at the position $ \r_{0} $ and placed in an external electric field $ \E $ are
    \begin{equation*}
        \vec{F} = (\pe \cdot \grad)\E(\r_{0}) \qquad \text{and} \qquad \M = \pe \cross \E(\r_{0}).
    \end{equation*}
    
\end{itemize}


\subsubsection{Derivation: Multipole Expansion of Electrostatic Energy}
\begin{itemize}
    \item We consider a charge distribution localized in the region $ V $, centered around the position $ \r_{0} $, which we then place in an \textit{external} potential $ \phi(\r) $. Keep in mind that the external potential $ \phi(\r) $ is unrelated to the charge distribution $ \rho $.
	
	\item In general, the electrostatic energy for a charge distribution in an external field is
	\begin{equation*}
		W_{\text{E}} = \iiint_{V} \rho(\r) \phi(\r) \dr.
	\end{equation*}
	For our localized distribution, we assume the majority of the charge is concentrated around the distribution's ``center'' $ \r_{0} $, in which case the majority of the distribution's electrostatic energy is also concentrated around $ \r_{0} $. We then expand the electric potential about $ \r_{0} $ according to
	\begin{equation*}
		\phi(\r) = \phi(\r_{0}) + (\r - \r_{0}) \grad \phi(\r_{0}) + \cdots,
	\end{equation*}
    where the gradient operator $ \grad $ acts on $ \r $ and not $ r_{0} $. Substituting this expression for $ \phi(\r) $ into the general expression for electrostatic energy $ W_{\text{E}} $ gives
	\begin{align*}
		W_{\text{E}} &= \iiint_{V}\dr \rho(\r)\big[\phi(\r_{0}) + (\r - \r_{0})\grad \phi(\r_{0}) + \cdots \big]\\
		& = \phi(\r_{0}) \iiint_{V} \dr \rho(\r) + \grad \phi(\r_{0}) \iiint_{V} \dr \rho(\r)(\r - \r_{0})\\
		& = q \phi(\r_{0}) - \E(\r_{0}) \cdot \pe + \cdots,
	\end{align*}
    where we have substituted in the charge $ q $ and electric dipole $ \pe $ defined in the previous section, and used $ \grad\phi(\r_{0}) = - \E(\r_{0}) $. The two terms in the above result are the desired expansion of $ W_{\text{E}} $ up to the dipole term.
	
\end{itemize}

\subsubsection{Multipole Expansion of Force in an External Field}

\begin{itemize}
    \item We continue with the charge distribution $ \rho $ localized in the region $ V $, centered around the position $ \r_{0} $, and placed in an external potential $ \phi(\r) $. We begin with the charge distribution's just-derived electrostatic potential energy, which reads
    \begin{equation*}
        W_{\text{E}} = q \phi(\r_{0}) - \E(\r_{0}) \cdot \pe + \cdots.
    \end{equation*}
    We then find the force on the charge distribution from the gradient of the potential energy via $ \vec{F} = - \grad W_{\text{E}} \iff \diff W_{\text{E}} = - \vec{F} \cdot \diff \r $. 

    % The plan is to find the differential $ \diff W_{\text{E}} $, and then identify the force by comparing the expression for $ \diff W_{\text{E}} $ to the general relationship $ \diff W_{\text{E}} = - \vec{F} \cdot \diff \r $.

	\item We begin by differentiating $ W_{\text{E}} $ with respect to $ \r $ to get $ \diff W_{\text{E}} $, which reads
	\begin{equation*}
		\diff W_{\text{E}} = q \grad \phi(\r_{0}) \cdot \diff \r - \grad \big[\pe \cdot \E(\r_{0})\big] \cdot \diff \r \equiv - \vec{F} \cdot \diff \r.
	\end{equation*}
    Next, we cancel the common factor $ \diff \r $ and substitute in $ \grad \phi(\r_{0}) = \E(\r_{0}) $ to get
    \begin{equation*}
        q \E(\r_{0}) - \grad \big[\pe \cdot \E(\r_{0})\big] = - \vec{F}.
    \end{equation*}
    We then substitute in the vector identity 
	\begin{equation*}
		\grad\big[\pe \cdot \E(\r_{0})\big] = \pe \cross (\curl \E(\r_{0})) + (\pe \cdot \grad)\E(\r_{0}),
	\end{equation*}
    and apply the electrostatics relationship $ \curl \E = 0 $, in which case the electrostatic force simplifies to
    \begin{align*}
        \vec{F} &= q \E(\r_{0}) + \pe \cross (\curl \E(\r_{0})) + (\pe \cdot \grad)\E(\r_{0})\\
        & = q \E(\r_{0}) + (\pe \cdot \grad)\E(\r_{0}).
    \end{align*}
    The two terms in this expression are the electrostatic force on a monopole (point charge) and a dipole, respectively.
	
\end{itemize}


\subsubsection{Multipole Expansion of Torque in an External Field}

\begin{itemize}

	\item To find the electrostatic torque on the charge distribution, we consider a small rotation $ \diff \vec{\phi} $ of the charge distribution in space. We begin with the dipole expansion of the charge distribution's electrostatic energy, which reads
	\begin{equation*}
		W_{\text{E}} = q \phi(\r_{0}) - \E (\r_{0}) \cdot \pe + \cdots.
	\end{equation*}
    We will then find the torque on the charge distribution from $ \diff W_{\text{E}} = \M \cdot \diff \vec{\phi} $.

    \item We begin by taking the total derivative of $ W_{\text{E}} $ to get $ \diff W_{\text{E}} $. The term $ q \phi(\r_{0}) $ is a constant quantity and vanishes, and the result is
	\begin{equation*}
		\diff W_{\text{E}} = - \diff \pe \cdot \E(\r_{0}).
	\end{equation*}
    The differential $ \diff \pe $ can be written as the infinitesimal rotation $ \diff \pe = \diff \vec{\phi} \cross \pe $, in which case $ \diff W_{\text{E}} $ becomes
	\begin{equation*}
		\diff W_{\text{E}} = - (\diff \vec{\phi} \cross \pe)\cdot \E(\r_{0}).
	\end{equation*}
    This expression for $ \diff W_{\text{E}} $ contains a scalar triple product, which we rearrange to get
    \begin{equation*}
        \diff W_{\text{E}} = - \diff \vec{\phi} \cdot (\pe \cross \E.)
    \end{equation*}

    \item We then compare the expression $ \diff W_{\text{E}} = - \diff \vec{\phi} \cdot (\pe \cross \E) $ to the general relationship
	\begin{equation*}
        \diff W_{\text{E}} = - \vec{M} \cdot \diff \vec{\phi} = - \diff \vec{\phi} \cdot \M.
	\end{equation*}
    The resulting torque on the charge distribution, up to the dipole term, is thus 
    \begin{equation*}
       \vec{M} = \pe \cross \E(\r_{0}).
    \end{equation*}
	
\end{itemize}
    


\newpage

\section{Magnetostatics}

\subsection{Electric Current Density}
\textit{Define electric current density. State the electric current densities for current along a current-carrying wire and for a continuous charge distribution moving through space.}

\begin{itemize}
	
    \item Electric current density $ \j $ is a vector quantity defined in terms of the relationship
	\begin{equation*}
		I = \iint_{S} \vec{j} \cdot \diff \vec{S} = \iint_{S} \vec{j} \cdot \uvec{n} \diff S,
	\end{equation*}
    where $ I $ is the current through the planar cross-sectional surface $ S $. Since current density is a vector quantity, it can have arbitrary direction in space and is not tied to a conductor. 
	
\end{itemize}

\textbf{Current Density of a Continuous Charge Distribution}
\begin{itemize}
    \item We begin by considering the current $ \diff I = \j \cdot \diff \vec{S} $ through an arbitrary surface element $ \diff \vec{S} $. By definition of current, $ I = \dot{q} $, we have
	\begin{equation*}
		\diff I = \diff \left[\dv{q}{t}\right] = \diff \left(\frac{\rho(\r)\diff V}{\diff t}\right) = \diff \left(\frac{\rho(\r)\diff S \diff x}{\diff t}\right)  = \rho \frac{(\vec{v} \cdot \uvec{n}) \diff S \diff t}{\diff t} = \rho v \diff S,
	\end{equation*}
	where $ (\vec{v} \cdot \uvec{n}) $ is the charge carrier velocity normal to $ \diff \vec{S} $. The above implies
	\begin{equation*}
		\vec{j}(\r) = \rho(\r) \vec{v}(\r),
	\end{equation*}
	where $ \vec{v} $ is the velocity field of the charge distribution at position $ \r $---which we have assumed is uniformly distributed.
\end{itemize}
	
\textbf{Current Density of a One-Dimensional Conductor}
\begin{itemize}
	\item The current density of a one-dimensional conductor embedded in a plane at the origin and carrying current $ I $ is
	\begin{equation*}
		\j(\r) = I \delta^{2}(\vec{\rho}) \uvec{l},
	\end{equation*}
	where $ \uvec{l} $ is the unit vector tangent to the conductor and parallel to the current.
\end{itemize}

\textbf{Current Density of a Moving Point Charge}
\begin{itemize}
	\item The current density of a point charge moving with velocity $ \vec{v} $ is
	\begin{equation*}
		\j(\r) = q \delta^{3}(\r - \r(t))\vec{v}(t).
	\end{equation*}
\end{itemize}
	
\textbf{Current Density of a Surface Charge Distribution}
\begin{itemize}
    \item The current density of charge distributed through a planar surface at $ z = z_{0} $ and moving with velocity $ \vec{v} $ is
	\begin{equation*}
		\j(\r) = \sigma \delta (z - z_{0})\vec{v},
	\end{equation*}
	where $ \sigma $ is surface charge density. 
	
	% \textit{Note}: We can also introduce a surface current density
	% \begin{equation*}
	% 	\vec{j}_{S} = \sigma \vec{v}
	% \end{equation*}
	% This will come in handy later in Maxwell's equations.
\end{itemize}

\subsection{Magnetic Field and Current Density}
\textit{State the relationship between magnetic flux density (i.e. the $ \B $ field) and electric current density. Define and discuss magnetic field lines.}

\begin{itemize}
    \item Current density and magnetic field are related by Ampere's law
    \begin{equation*}
        \oint_{\partial S} \B \cdot \diff \vec{S} = \mm \iint_{S} \j \cdot \diff \vec{S} \implies \curl \B = \mm \j.
    \end{equation*}
    
    \item In terms of the arc length parameter $ s $, magnetic field lines $ \r(s) $ are closed curves related to magnetic field $ \B $ by
	\begin{equation*}
		\dot{\r}(s) = \dv{\r}{s} = \frac{\B(\dot{\r}(s))}{\abs{\B(\r(s))}}.
	\end{equation*}
    Note that all magnetic field lines are closed curves---which corresponds to the fact that magnetic monopoles do not exist (have not been observed) in nature.
    
\end{itemize}

\textbf{Derivation: Ampere's Law}
\begin{itemize}
	\item We begin by considering a current loop $ C' $ parameterized by the following quantities:
    \begin{itemize}
        \item the arc length $ l' $,

        \item the infinitesimal arc segment $ \diff \vec{l}' $ pointing tangent to the conductor in the direction of the electric current,

        \item and the position $ \r(l')  $ along the conductor.
    \end{itemize}

	\item We then imagine enclosing a portion of the loop $ C' $ with another  loop $ C $ parameterized by the analogously defined quantities $ l $, $ \diff \vec{l} $ and $ \r(l) $.
	
	% We are interested in the relationship between $ \curl \B $ and $ \j $. 
	
	We begin by finding the magnetic circulation along the curve $ C $. Using the Biot-Savart to express $ \B $, the circulation $ \Gamma_{\text{M}} $ is
	\begin{equation*}
		\Gamma_{\text{M}} \equiv \oint_{C} \B \cdot \diff \vec{l} = \oint_{C}\left[ - \frac{\mm I}{4 \pi} \oint_{C'}\diff \vec{l}' \cross \grad \left(\frac{1}{\abs{\r(l) - \r(l')}}\right) \right] \cdot \diff \vec{l},
	\end{equation*}
	where we've written the integrand in the second equality with a reverse-engineered gradient. The integrand is a scalar triple product, which we can rearrange to get
	\begin{equation*}
		\Gamma_{\text{M}} =  - \frac{\mm I}{4\pi} \oint_{C} \oint_{C'} (\diff \vec{l} \cross \diff \vec{l}') \cdot \grad \left(\frac{1}{\abs{\r(l) - \r(l')}}\right).
	\end{equation*}
	
    \item Next, we note that $ \diff \vec{l} \cross \diff \vec{l}' $ corresponds to a surface element---we'll call it $ \diff \vec{S} $. We then calculate the gradient and evaluate the dot product to get
	\begin{align*}
		\Gamma_{\text{M}} =  - \frac{\mm I}{4\pi} \iint_{S} \diff \vec{S}\cdot \grad \left(\frac{1}{\abs{\r(l) - \r(l')}}\right) \\
		= +  \frac{\mm I}{4\pi} \iint_{S}  \frac{\diff S \cos \theta}{\abs{\r(l) - \r(l')}^{2}},
	\end{align*}
	where $ \theta $ is the angle between $ \diff \vec{S} $ and the gradient. The integrand is precisely the solid angle element $ \diff \Omega $, in terms of which the magnetic circulation becomes
	\begin{equation*}
		\Gamma_{\text{M}} = \frac{\mm I}{4\pi} \iint_{S} \diff \Omega =  \frac{\mm I}{4\pi}  \cdot 4 \pi = \mm I.
	\end{equation*}
	This result is Ampere's law, which, in terms of magnetic circulation, reads
	\begin{equation*}
		\Gamma_{\text{M}} = \mm I.
	\end{equation*}
    In words, Ampere's law states that the magnetic circulation in a loop (our $ C $) enclosing a current-carrying loop carrying current $ I $ (our $ C' $) is proportional to the current $ I $ through the $ C' $.
	
	\item Finally, we will combine Ampere's law with the definition of magnetic circulation. To review, the two relationships are
	\begin{equation*}
		\Gamma_{\text{M}} = \oint_{\partial S} \B \cdot \diff \vec{S} = \iint_{S} \curl \B \cdot \diff \vec{S} \eqtext{and} \Gamma_{\text{M}} = \mm I = \mm \iint_{S} \j \cdot \diff \vec{S}.
	\end{equation*}
    Combining the two relationships produces the important result
	\begin{equation*}
		\curl \B = \mm \j.
	\end{equation*}
	This is a Maxwell equation for magnetostatics and is an equivalent, often more common expression of Ampere's law.
	
\end{itemize}

\subsection{The Magnetic Vector Potential} \label{ss:magnetic-potential}
\textit{State and derive the expression for the magnetic vector potential. How is magnetic flux written in terms of vector potential?}

\begin{itemize}
    \item The magnetic vector potential $ \A $ associated with a magnetic field $ \B $ is defined by the relationship
    \begin{equation*}
        \B = \curl \A,
    \end{equation*}
    which rests on the Maxwell equation $ \div \B = 0 $.
    
	\item Magnetic flux is written in terms of $ \A $ using Stokes' theorem according to
	\begin{equation*}
		\Phi_{\text{M}} \equiv \iint_{S}\B \cdot \diff \vec{S} = \iint_{S} (\curl \A) \cdot \diff \vec{S} = \oint_{\partial S} \A \cdot \diff \r.
	\end{equation*}
	In other words, the magnetic flux through a surface equals the circulation of the magnetic potential $ \A $ around the surface's boundary $ \partial S $.

\end{itemize}

\textbf{Derivation: The Magnetic Vector Potential}
\begin{itemize}
	
    \item First, we note that we cannot define a magnetic potential via $ \B = - \grad \phi_{\text{M}} $, as for electric field $ \E = - \grad \phi $. This is because magnetic field lines are closed, which results in $ \curl \B \neq 0 $. The equality $ \curl \B \neq 0 $ prohibits the introduction of a scalar potential in the style $ \B = - \grad \phi $.

    The reasoning is straightforward: the curl of a gradient is always zero. If we were to take the curl of a hypothetical equality $ \B = - \grad \phi_{\text{M}} $, we would get $ \curl \B = - \curl \big[ \grad \phi_{\text{M}} \big] = 0 $, which contradicts $ \curl \B \neq 0 $. 
	
	\item As a solution, we use the relationship $ \div \B = 0 $ (which corresponds to the fact that magnetic field lines are closed). From vector calculus, the divergence of the curl of a vector field is always zero, which allows us to write $ \B $ as the curl of a vector field:
	\begin{equation*}
		\div \B \equiv \div \left(\curl \A\right) = 0 \implies \B = \curl \A. 
	\end{equation*}
    The relationship $ \B = \curl \A $ gives us an implicit definition of the magnetic vector potential $ \A $. Again, we stress that this definition rests on the identity $ \div \B = 0 $. 
	
\end{itemize}


\subsection{Magnetic Field and the Biot-Savart Law}
\textit{State and derive the Biot-Savart law for the magnetic field of a continuous charge distribution. Use the result to derive the magnetic field of a straight conducting wire.}

\begin{itemize}
    \item The Biot-Savart law for the magntetic field produced by a wire-like conductor $ C $ carrying current $ I $ reads
    \begin{equation*}
        \B(\r) = \frac{\mm}{4 \pi} \int_{C} \frac{I \diff \vec{\tilde{l}} \cross (\r - \tilde{\r})}{\abs{\r - \r'}^{3}},
    \end{equation*}
    where $ \diff \vec{\tilde{l}} $ is the tangent to the conductor and points in the direction of current.
    

    \item The Biot-Savart law for the magnetic field of a continuous current distribution $ \j $ is
    \begin{equation*}
        \B(\r) = \frac{\mm}{4\pi} \iiint_{V} \frac{\j \cross (\r - \tilde{\r})}{\abs{\r - \tilde{\r}}^{3}} \dtr.
    \end{equation*}
    
    
\end{itemize}

\textbf{Derivation: The Biot-Savart Law for a Wire-Like Conductor}
\begin{itemize}
	
	\item We begin with the vector-product form of the force between two conductors $ C_{1} $ and $ C_{2} $ carrying currents $ I_{1} $ and $ I_{2} $, which reads
	\begin{equation*}
		\vec{F} = - \frac{\mm I_{1}I_{2}}{4\pi}\oint_{C_{1}}\oint_{C_{2}} \frac{\diff \vec{l}_{1}\cross \big[\diff \vec{l}_{2} \cross (\r(l_{2}) - \r(l_{1}) )\big]}{\abs{\r(l_{2}) - \r(l_{1})}^{3}},
	\end{equation*}
    where $ l_{1} $ and $ l_{2} $ are arc length parameters and $ \diff \vec{l}_{1} $ and $ \diff \vec{l}_{2} $ are the tangents to each conductor, in the direction of current. 

    \item Next, we rearrange the above expression for force into the form
	\begin{equation*}
		\vec{F} = \oint_{C_{1}}I_{1}\diff \vec{l}_{1}\cross \left(\frac{\mm I_{2}}{4\pi} \oint_{C_{2}}\frac{\diff \vec{l}_{2} \cross \big(\r(l_{1}) - \r(\l_{2})\big)}{\abs{\r(l_{1}) - \r(l_{2})}^{3}}\right).
	\end{equation*}
	Comparing this expression for force to $ \vec{F} = \oint_{C_{1}} I_{1} \diff \vec{l}_{1} \cross \B $ motivates the definition
	\begin{equation*}
		\B(\r(l_{1})) \equiv \frac{\mm I_{2}}{4\pi} \oint_{C_{2}}\frac{\diff \vec{l}_{2} \cross \big(\r(l_{1}) - \r(\l_{2})\big)}{\abs{\r(l_{1}) - \r(l_{2})}^{3}}.
	\end{equation*}
    This is the magnetic field at a point $ \r(l_{1}) $ along the conductor $ C_{1} $ generated by the current in the conductor $ C_{2} $. Generalizing the above expression to an arbitrary point in space gives the Biot-Savart law for the magnetic field of a wire-like conductor:
    \begin{equation*}
        \B(\r) = \frac{\mm}{4\pi} \int_{C} \frac{I \diff \tilde{\vec{l}} \cross (\r - \tilde{\r})}{\abs{\r - \tilde{\r}}^{3}}.
    \end{equation*}
    By writing a generic line integral $ \int $ instead of a closed line integral $ \oint $, we have allowed the possibility that the conductor $ C $ is not closed, for example as in a long, straight wire.

    \item Finally, generalization to a spatial current distribution is simply a matter of replacing $ I \diff \tilde{\vec{l}} $ with $ \j \dtr $ and integrating over the current distribution's volume $ V $ instead of over the conductor's curve $ C $. The result is
    \begin{equation*}
        \B(\r) = \frac{\mm}{4 \pi} \iiint_{V} \frac{\j \cross (\r - \tilde{\r})}{\abs{\r - \tilde{\r}}^{3}} \dtr.
    \end{equation*}

\end{itemize}
	

\subsection{The Magnetic Poisson Equation}
\textit{State and derive the magnetic analog of the Poisson equation relating magnetic potential and current density. State the general solution, and discuss its relationship to the Biot-Savart law.}

\begin{itemize}

    \item The magnetic analog of the Poisson equation is
    \begin{equation*}
        \laplacian \A = - \mm \j,
    \end{equation*}
    and the general solution is
    \begin{equation*}
        \A(\r) = \frac{\mm}{4\pi}\iiint_{V} \frac{\j(\tilde{\r})}{\abs{\r - \tilde{\r}}} \dtr.
    \end{equation*}
    
	\item The above general solution for $ \A $ is closely related to the Biot-Savart law. To show this, we take the curl of $ \A(\r) $ with respect to $ \r $ to get
	\begin{align*}
		\B(\r) &= \curl \A(\r) = \curl \left(\frac{\mm}{4\pi} \iiint_{V} \frac{\j(\t{\r})}{\abs{\r - \t{\r}}} \dtr\right)\\
		& = \frac{\mm}{4\pi} \iiint_{V} \frac{\j \cross (\r - \t{\r})}{\abs{\r - \t{\r}}^{3}} \dtr,
	\end{align*}
    which is the Biot-Savart law for a current distribution $ \j $. 

\end{itemize}

\textbf{Derivation: The Magnetic Analog of the Poisson Equation} \label{ss:magnetic-poisson}
\begin{itemize}

    \item We begin by writing Ampere's law in terms of $ \A $, which leads to the relationship
	\begin{equation*}
		\mm \j = \curl \B = \curl (\curl \A) = \grad (\div \A) - \laplacian \A.
	\end{equation*}
	It is possible to make the $ \div \A $ term vanish. We do this with Helmholtz's theorem, which tells us that we can write an arbitrary vector field in the form
	\begin{equation*}
		\A = \A_{1} + \A_{2},
	\end{equation*}
	where $ \div \A_{1} = 0 $ and $ \curl \A_{2} = 0 $. In words, Helmholtz's decomposition theorem states that a vector field can be written as the sum of a solenoidal and irrotational vector field.
	
    \item We then introduce a gauge transformation of $ \A $ such that $ \A_{2} = 0 $. We are free to set $ \A_{2} = 0 $ because $ \curl \A_{2} = 0 $ and thus has zero contribution to $ \B = \curl \A $, regardless of what value we choose for $ \A_{2} $. Substituting $ \A_{2} = 0 $ into the Helmholtz composition gives $ \A = \A_{1} $. We then take the divergence of this equality and apply $ \div \A_{1} = 0 $ to get
    \begin{equation*}
        \A = \A_{1} \implies \div \A = \div \A_{1} \equiv 0 \implies \div \A = 0.
    \end{equation*}
    Finally, we substitute $ \div \A = 0 $ into Ampere's law to get the desired equation
    \begin{equation*}
        \mm \j = 0 - \laplacian \A \implies \laplacian \A = - \mm \j.
    \end{equation*}
	Note the similarity to the Poisson equation for electric potential.
	
    \item Formally, we would solve this equation using a vector Green's function, similarly to how we solved the electrostatic Poisson equation with a Green's function. However, given the close similarity between the magnetostatic and electrostatic Poisson equations, we can simply guess the solution for $ \A $ with reference to the solution for $ \phi $. Without derivation, the result comes out to be
	\begin{equation*}
		\A(\r) = \frac{\mm}{4\pi} \iiint_{V} \frac{\j(\t{\r})}{\abs{\r - \t{\r}}} \dtr,
	\end{equation*}
	where $ V $ is the region of space with non-zero current density $ \j $. This is the general expression for magnetic vector potential $ \A $. 
	
\end{itemize}


\subsection{Magnetic Vector Potential of a Long, Straight Inductor}
\textit{State the magnetic vector potential inside and outside a long, straight inductor. Give an overview of the derivation process, and explain how it is possible to simplify the magnetic field outside the inductor using gauge transformations.}

\begin{itemize}
    \item The magnetic vector potential inside and outside a long straight inductor, which we can assume contains a homogeneous magnetic field $ \B_{0} $, is
    \begin{equation*}
        \A_{\text{in}} = \frac{1}{2} \B_{0} \cross \r \qquad \text{and} \qquad \A_{\text{out}} = \frac{a^{2}}{2}\B_{0} \cross \frac{\r}{r^{2}}.
    \end{equation*}
     Note that although although the magnetic field in the inductor is homgeneous, $ \A_{\text{in}} $ has a much more complicated dependence on position. Likewise, $ A_{\text{out}} $ is more more complex than the trivial $ \B_{\text{out}} = 0 $. Finally, we note that $ \A $ changes continuously across the inductor boundary at $ r = a $, which we can see by comparing the expressions for $ \A $ inside and outside the inductor.

     \item With an approriate gauge transformation, it is possible to simplify the vector potential outside the inductor to
     \begin{equation*}
         \A_{\text{out}}' = \pi a \B_{0} \delta(\phi - \pi) \uvec{e}_{\phi},
     \end{equation*}
     where $ a $ is the inductor's radius and $ \phi $ is the polar angle in the inductor's circular cross section. This is discussed more thoroughly below.
    
\end{itemize}

\subsubsection{Discussion: Gauges and the Magnetic Potential of an Inductor}
\begin{itemize}
	
	% In the following discussion of gauge transforms, keep in mind that $ \A $ is the vector potential corresponding to the magnetic field $ \B $, which is the quantity actually measured in experiment.
	
	\item For more convenient calculations, we are free to define a new vector potential using the transform
	\begin{equation*}
		\A' = \A + \grad \zeta (\r).
	\end{equation*}
	We can safely add the term $ \grad \zeta (\r) $ because the curl of the gradient of a scalar field is always zero, and won't affect the end result for $ \B $, which is the quantity actually measured in experiment.

    In other words, both $ \A $ and $ \A' $ correspond to the same magnetic field $ \B $, i.e.
	\begin{equation*}
		\B = \curl \A = \curl \A'.
	\end{equation*}
	
	\item Without derivation, for the specific case of a long, straight inductor, we can simplify the expression for $ \A_{\text{out}} $ by choosing
	\begin{equation*}
		\zeta (\r)  = -\frac{B_{0}a^{2}}{2} \arctan \frac{y}{x}.
	\end{equation*}
	For this choice of $ \zeta $, the vector potential outside the inductor simplifies to
	\begin{align*}
		\A_{\text{out}}' &= \A_{\text{out}} + \grad \zeta (\r) = \frac{a^{2}}{2}\B_{0}\curl \frac{\r}{r^{2}} - \grad \left(\frac{\B_{0}a^{2}}{2} \arctan \frac{y}{x}\right)\\
        & = \frac{\B_{0}a^{2}}{2} \frac{2\pi}{a} \delta (\phi - \pi) \uvec{e}_{\phi} = \pi a \B_{0} \delta(\phi - \pi) \uvec{e}_{\phi}.
	\end{align*}
	In this case, $ \A_{\text{out}}' $ is zero everywhere except along the curve $ \phi = \pi $, corresponding to the negative portion of the $ x $ axis.
	
	\item To summarize, gauge transforms allow us to simplify the expression for vector potential $ \A $ without changing the value of the magnetic field $ \B $, which is the physically relevant quantity measured in experiment.
\end{itemize}


\subsubsection{Discussion: Magnetic Vector Potential of an Inductor}
\begin{itemize}
    \item We consider a long, straight inductor, which has the simple magnetic field: inside the inductor, the magnetic field is approximately homogeneous and obeys $ \B(\r) = \B_{0} $. Outside the inductor, the magnetic field is zero. We choose our coordinate system so that $ \B_{0} = (0, 0, B_{0}) $. 
	
	\item Without derivation, the magnetic vector potential inside the inductor is
	\begin{equation*}
		\A_{\text{in}} = \frac{1}{2}\B_{0} \cross \r.
	\end{equation*}
    Deriving the result is simply a matter of vector calculus acrobatics, which we delegate to an appropriate mathematics course. We can confirm, however, that this expression for $ \A_{\text{in}} $ satisfies $ \B = \curl \A $:
    \begin{align*}
        \B &= \curl \left[ \frac{1}{2} \B_{0} \cross \r \right] = \frac{1}{2}\Big[ (\div \r) + \r \cdot \grad \Big] \B_{0} - \frac{1}{2} \Big[ (\div \B_{0}) + \B_{0} \cdot \grad \Big] \r\\
        & = \frac{1}{2} \big( 3\B_{0} + 0 \big) - \frac{1}{2} \big( 0 + \B_{0} \big) = \B_{0},
    \end{align*}
    where the various derivatives of $ \B_{0} $ vanish because $ \B_{0} $ is homogeneous.
    
	
	\item Outside the inductor, we know $ \B = 0 $. What about $ \A $? We assume inductor has radius $ a $ and consider a loop just hugging the outside of the inductor and bounding the planar surface $ S $, which aligns with the inductor's circular cross section.
	
	The magnetic flux through the surface $ S $ is then
	\begin{equation*}
		\Phi_{\text{M}} = \iint_{S} \B \cdot \diff \vec{S} = \B_{0} (\pi a^{2}).
	\end{equation*}
    Note that only the porition of $ S $ inside the inductor (for $ r < a $) contributes non-zero magnetic flux, since $ \B = 0 $ outside the inductor.
	
	\item Next, we recall the general definition of flux in terms of vector potential, which reads
	\begin{equation*}
		\Phi_{\text{M}} = \iint_{S} \B \cdot \diff \vec{S} = \oint_{\partial S} \A \cdot \diff \r.
	\end{equation*}
    Comparing this general definition to our just-derived result $ \Phi_{\text{M}} = \pi a^{2}\B_{0} $ gives
	\begin{equation*}
		\oint_{\partial S} \A \cdot \diff \r = \Phi_{\text{M}} = \B_{0} (\pi a^{2}) \neq 0  \implies \A \neq 0.
	\end{equation*}
    In other words, the vector potential $ \A $ is non-zero outside the inductor, even though $ \B $ is zero in the same region. Without derivation, the vector potential outside the inductor turns out to be
	\begin{equation*}
		\A = C \B_{0} \cross \frac{\r}{r^{2}},
	\end{equation*}
	where $ C $ is a constant. Using this expression for $ \A $, the magnetic flux $ \Phi_{\text{M}} $ through the loop $ \partial S $ comes out to
	\begin{equation*}
		\Phi_{\text{M}} = \oint_{\partial S} \A \cdot \diff \r  = \oint_{\partial S} C\left(\B_{0} \cross \frac{\r}{r^{2}}\right) \diff \r = \cdots = 2\pi C B_{0}
	\end{equation*}
	The result of the integral is quoted without derivation. Combining the above result with the earlier equality $ \Phi_{M} = B_{0}\pi a^{2} $ gives
	\begin{equation*}
		B_{0}\pi a^{2} = \Phi_{M} = 2\pi C B_{0} \implies C = \frac{a^{2}}{2}.
	\end{equation*}
	The magnetic vector potential outside the inductor is thus
	\begin{equation*}
		\A = \frac{a^{2}}{2} \B_{0} \cross \frac{\r}{r^{2}}.
	\end{equation*}

\end{itemize}



\subsection{Magnetic Energy in an External Magnetic Field}
\textit{State and derive the expression for the magnetic energy of a charge distribution in an external magnetic field.}

\begin{itemize}
    \item The magnetostatic potential energy $ W_{\text{M}} $ of a current distribution $ \j $ placed in an external magnetic field with vector potential $ \A $ is
    \begin{equation*}
        W_{\text{M}} = - \iiint_{V} \j(\r)\cdot \A(\r) \dr.
    \end{equation*}
    Keep in mind that $ \A(\r) $ is the potential of the external magnetic field, and is completely unrelated to the current distribution $ \j $.

    \item In terms of the external current distribution $ \widetilde{\j} $ generating the external potential $ \A $, the magnetostatic potential energy of the current distribution $ \j $ is
    \begin{equation*}
		W_{\text{M}} = - \frac{\mm}{4\pi} \iiint_{V} \iiint_{\tilde{V}} \frac{\j(\r) \widetilde{\j}(\tilde{\r}) \dr \dtr}{\abs{\r - \tilde{\r}}},
    \end{equation*}
    which follows from writing $ \A $ using the general solution of the magnetic Poisson equation
	\begin{equation*}
		\A(\r) = \frac{\mm}{4\pi}\iiint_{\tilde{V}}\frac{\widetilde{\j}(\r)}{\abs{\r - \t{\r}}} \dtr.
	\end{equation*}
    
\end{itemize}


\textbf{Derivation: Magnetic Field Energy in an External Field}
\begin{itemize}
    \item We begin by considering a current loop $ C $ carrying current $ I $ in an external magnetic field $ \B $, which is unrelated to the current $ I $. We will work in the quasistatic regime where $ \dv{I}{t} = 0 $. Finally, for use in the next bullet, recall that the magnetic force on a current carrying-conductor is.
	\begin{equation*}
		\vec{F} = \oint_{C} I \diff \vec{l} \cross \B = I \oint_{C} \uvec{t} \cross \B \diff l,
	\end{equation*}
	where $ \uvec{t} $ is the tangent to the conducting loop pointing in the direction of current.
	
	\item We then consider the work $ W $ associated with moving the loop by the infinitesimal displacement $ \diff \r $ in the external magnetic field. The plan is then to relate the work to the desired magnetic energy $ W_{\text{M}} $ using the work-energy theorem. 

    First, the work $ W $ associated with the displacement $ \diff \r $ of the current loop is
	\begin{align*}
		\diff W &= - \vec{F} \cdot \diff \r =  -I \oint_{C} \uvec{t} \cross \B \diff l \cdot \diff \r\\
		& = - I \oint_{C}\diff \r \cdot (\uvec{t} \cross \B) \diff l.
	\end{align*}
    The last result is a scalar triple product, which we rearrange to get
	\begin{equation*}
		\diff W = - I \oint_{C} (\diff \r \cross \uvec{t})\cdot \B \diff l.
	\end{equation*}
	
    \item Next, we recognize that quantity $ (\diff \r \cross \uvec{t}) \diff l $ is the surface element $ \diff \vec{S} $  of the loop during the displacement $ \diff \r $. In terms of $ \diff \vec{S} $, the work reads
	\begin{equation*}
		W = - I \iint_{S} \B \cdot \diff \vec{S} \equiv - I \Phi_{\text{M}}.
	\end{equation*}
    We stress that the magnetic flux $ \Phi_{\text{M}} $ in the above expression is the flux arising from the external field $ \B $ only, and is unrelated to the current $ I $. We then use $ \B = \curl \A $ to write the work in terms of the magnetic potential, which produces
	\begin{equation*}
		W = - I \iint_{S}\B \cdot \diff \vec{S} = - I \iint_{S} (\curl \A) \cdot \diff \vec{S}.
	\end{equation*}
	The last term is the surface integral of a curl quantity, which we can rewrite as an integral over the surface boundary $ \partial S $ using Stokes' theorem. But which surface boundary---this is important! Assume the initial conductor before the displacement by $ \diff \r $ is described by the curve $ C_{1} $, and that after the displacement the conductor is described by the curve $ C_{2} $. In this case, the appropriate surface is the surface between the curves $ C_{1} $ and $ C_{2} $, and the expression for work becomes
	\begin{equation*}
        W = - I \iint_{S} (\curl \A) \cdot \diff \vec{S} = - I \oint_{C_{2}} \A \cdot \diff \r  + I \oint_{C_{1}} \A \cdot \diff \r,
	\end{equation*}
	which is the difference of the line integrals over the two curves.
	
	\item Next, we generalize our expression from a current loop $ C $ to a generalized current distribution described by the current density $ \j $. 

    First, an intermediate step: we use the identity $ I \diff \r = \j \dr $ to write
	\begin{equation*}
		I \oint_{C} \A \cdot \diff \r = \iiint_{V} \j(\r)\A(\r) \cdot \dr,
	\end{equation*}
	where $ V $ is the region of space containing the current density $ \j $. Using this relantionship, the work energy theorem for magnetic energy reads
	\begin{equation*}
		W \equiv W_{\text{M}}^{(1)} - W_{\text{M}}^{(2)} = - \iiint_{V_{2}} \j_{2}(\r) \A(\r) \cdot \dr + \iiint_{V_{1}} \j_{1}(\r) \A(\r) \cdot \dr
	\end{equation*}
    where the $ (2) $ terms correspond to quantities after the displacement $ \diff \r $ and the $ (1) $ terms correspond to quantities before the displacement $ \diff \r $. The magnetostatic energy $ W_{\text{M}} $ of a charge distribution $ \j $ in an \textit{external} magnetic field $ \B $ is thus
	\begin{equation*}
		W_{\text{M}} = - \iiint_{V}\j(\r) \A(\r) \dr,
	\end{equation*}
	and the corresponding magnetic energy density $ w_{\text{M}} $ is
	\begin{equation*}
		w_{\text{M}}(\r) = - \j(\r) \A (\r).
	\end{equation*}
	
\end{itemize}
    
\subsection{Total Magnetostatic Field Energy} \label{ss:total-M-energy}
\textit{Derive the expression for the total magnetic field energy associated with a magnetic field. Give the result both in terms both in terms of magnetic field and magnetic vector potential. Discuss the differences between finding the total magnetic field energy of a magnetic field and finding the total electric field energy of an electric field.}

\begin{itemize}
    \item The difference between finding field energy in electrostatics and magnetostatics is:
    \begin{quote}
        In electrostatics, the field source is charge, which is a static quantity---once it is assembled, it doesn't require any additional energy to maintain. Meanwhile, in magnetostatics, the field source is electric current, and establishing and maintaining an electric current requires energy input. As a result, a complete expression for magnetic field energy should also consider the energy needed to maintain the magnetic field source.
    \end{quote}

    \item The total energy associated with a magnetic field $ \B $ with vector potential $ \A $ generated by a current distribution $ \j $ is
    \begin{equation*}
        W_{\text{M}} = \frac{1}{2} \iiint_{V} \j \cdot \A(\r) \dr = \frac{1}{2 \mm} \iiint_{V} B^{2}(\r) \dr.
    \end{equation*}
    This expression accounts for both the energy in the field itself and the energy required to maintain the field source. The associated magnetic energy density is
    \begin{equation*}
        w_{\text{M}} = \frac{1}{2 \mm} B^{2}.
    \end{equation*}

    \item The magnetic energy associated with only the magnetic field (ignoring the energy required to maintain the field) is
    \begin{equation*}
        W_{\text{M}_{0}} = -\frac{1}{2} \iiint_{V} \j \cdot \A(\r) \dr,
    \end{equation*}
    while the energy required to maintain the current distribution $ \j $ creating the field is
    \begin{equation*}
        W_{\j} = \iiint_{V} \j \cdot \A(\r) \dr.
    \end{equation*}
    As expected, the sum of these two energies comes out to total magnetic field energy:
    \begin{equation*}
        W_{\text{M}_{0}} + W_{\j} = -\frac{1}{2} \iiint_{V} \j \cdot \A(\r) \dr + \iiint_{V} \j \cdot \A(\r) \dr = \frac{1}{2} \iiint_{V} \j \cdot \A(\r) \dr = W_{\text{M}}.
    \end{equation*}
    
\end{itemize}


\subsubsection{Derivation: Total Magnetic Field Energy in Terms of Potential}
\begin{itemize}

    \item Following the same procedure as for total electric field energy, we introduce a parameter $ \alpha \in [0, 1] $, which controls ``turning on'' the total magnetic field. As $ \alpha $ increases from $ 0 $ to $ 1 $, the distribution's current density correspondingly increases from $ 0 $ to $ \j $.

    We then consider the change in energy $ \diff W_{\text{M}} $ from adding additional current $ \diff \j = \j \diff \alpha $ to the current density we have already accumulated. The existing current creates an external magnetic field like with associated magnetostatic potential energy is $ W_{\text{M}_{0}} = - \iiint_{V} \tilde{\j} \tilde{\A}(\r) \dr $. The corresponding differential $ \diff W_{\text{M}_{0}} $ is
    \begin{equation*}
        \diff W_{\text{M}_{0}} = - \iiint_{V} \diff \tilde{\j} \tilde{\A}(\r) \dr = - \iiint_{V} \j \diff \alpha \big[ \alpha \A(\r) \big] \dr,
    \end{equation*}
    where the transition between the two integrals relies on the linearity of the vector Poisson equation, i.e. $ \laplacian[\alpha \A] = - \mm \alpha \j $.

    \textit{Note}: So far we are neglecting the energy need to maintain the magnetic field, which is the reason for using the subscript $ 0 $ in $ W_{\text{M}_{0}} $.

    \item Next, integrate over $ \alpha $ to ``turn on'' the current distribution
    \begin{equation*}
        W_{\text{M}_{0}}  = - \int_{0}^{1} \alpha \diff \alpha \iiint_{V} \j(\r)\A(\r)\dr = - \frac{1}{2} \iiint_{V} \j(\r) \cdot \A(\r) \dr.
    \end{equation*}
    This expression is the field energy assuming the current distribution $ \j $ already exists. Note that $ \A $ in the above expression is the magnetic vector potential formed by $ \j $.
    
    \item We now consider the energy needed to maintain the field. Maintaining the current distribution generating the field requires a power input
    \begin{align*}
        P &= - UI = -I \oint_{C} \E \cdot \diff \r = - I \iint_{S} \curl \E \cdot \diff \vec{S} = - I \iint_{S} \left( - \pdv{\B}{t} \right) \cdot \diff \vec{S}\\
        &= I \pdv{t} \iint_{S} \B \cdot \diff \vec{S},
    \end{align*}
    where the line integral runs over the curve(s) carrying the current. 

    Combining the above result with the definition of power, i.e. $ P = \dv{W}{t} $, gives 
    \begin{equation*}
        P \equiv \dv{W}{t} = I \pdv{t} \iint_{S} \B \cdot \diff \vec{S} \implies W_{\j} = I \iint_{S} \B \cdot \diff \vec{S},
    \end{equation*}
    where we have used $ W_{\j} $ to denote the energy required to maintain the magnetic field.
    

    \item By putting the pieces from the last two equations together and using $ \B = \curl \A $, the energy $ W_{\j} $ required to maintain the magnetic field comes out to
    \begin{equation*}
        W_{\j} = I \iint_{S} \B \cdot \diff \vec{S} = I \iint_{S} \curl \A \cdot \diff \vec{S} = \oint_{C} I \A \cdot \diff \r = \iiint_{V} \j \cdot \A \dr.
    \end{equation*}
    
    \item The total energy associated with the magnetic field of a current distribution is then
    \begin{equation*}
        W_{\text{M}} = W_{\text{M}_{0}} + W_{\vec{j}} = \left( -\frac{1}{2} + 1 \right) \iiint_{V} \j \cdot \A \dr = + \frac{1}{2} \iiint_{V} \j \cdot \A \dr.
    \end{equation*}
    This is the total energy associated with the magnetic field generated by the current distribution $ \j $, and includes the energy required to create and sustain the current distribution.
    
\end{itemize}

\subsubsection{Derivation: Magnetic Field Energy in Terms of Magnetic Field}
\begin{itemize}
    \item We begin by substituting Ampere's law $ \curl \B = \mm \j $ into the just-derived expression for total magnetic field energy $ W_{\text{M}} $ to get
    \begin{equation*}
        W_{\text{M}} = \frac{1}{2} \iiint_{V} \j(\r) \A(\r) \dr = \frac{1}{2\mm}\iiint_{V}(\curl \B) \A \dr.
    \end{equation*}
    Next, we reverse-engineer the vector identity $ \div (\B \cross \A) = - \B (\curl \A) + \A (\curl \B) $ to get
    \begin{equation*}
        W_{\text{M}} = \frac{1}{2\mm} \iiint_{V} \B(\curl \A)\dr + \frac{1}{2\mm} \iiint_{V} \div (\B \cross \A) \dr.
    \end{equation*}
    \item Using $ \B = \curl \A $, the first integrand simplifies to $ B^{2} $, while the second can be rewritten using the divergence theorem. The result is
    \begin{equation*}
        W_{\text{M}} = \frac{1}{2\mm} \iiint_{V} B^{2} \dr + \oiint_{\partial V} (\B \cross \A) \diff \S.
    \end{equation*}
    Using an similar argument as in the analogous treatment of electric field energy, the second term vanishes for large $ \r $. Here's why: the surface term $ \diff \S $ grows as $ \sim r^{2} $, the field term $ \B $ falls as $ \sim 1/r $, and the potential term $ \A $ falls as $ \sim 1/r^{2} $. The result is that the entire integrand falls as $ 1/r $, which vanishes for large $ r $, i.e. for a large enough integration volume $ V $ containing the charge distribution. In this limit, the magnetic field energy reduces to
    \begin{equation*}
        W_{\text{M}} = \frac{1}{2\mm} \iiint_{V}B^{2}(\r)\dr.
    \end{equation*}
    
\end{itemize}

\subsection{Magnetic Force}
\textit{State and derive the magnetic force on current distribution in an external magnetic field in terms of both magnetic field and the magnetostatic stress tensor. Use the result to calculate the magnetic force between two straight, parallel conducting wires carrying an electric current in (a) equal directions and b.) opposite directions.}

\begin{itemize}
    \item In terms of magnetic field, the magnetic force on a current distribution in an external magnetic field is
    \begin{equation*}
        \vec{F}_{\text{M}} = \frac{1}{\mm} \oiint_{\partial V} \left[ \B \otimes \B - \frac{1}{2}B^{2} \mat{I} \right] \diff \vec{S},
    \end{equation*}
    where $ \mat{I} $ is the identity matrix and $ \partial V $ is a surface enclosing the charge distribution.

    \item In terms of magnetostatic stress tensor, the magnetic force on a current distribution in an external magnetic field is
    \begin{equation*}
        F_{\text{M}_{i}} = \oiint_{\partial V} \TT_{ik} \hat{n}_{k}\diff S = \iiint_{V} \pdv{T_{ik}}{x_{k}} \dr,
    \end{equation*}
    where the components of the magnetostatic stress tensor are defined by
    \begin{equation*}
        \TT_{ik} = \frac{1}{\mm} \left[ B_{i}B_{k} - \frac{1}{2}B^{2} \delta_{ik} \right].
    \end{equation*}
    
\end{itemize}

\subsubsection{Derivation: Magnetic Force as a Function of Magnetic Field}

\begin{itemize}
	\item Consider a current distribution $ \j(\r) $ consisting of an arbitrary system of closed current loops. We then place the current distribution in an external magnetic field $ \B $ and consider the total magnetic field, i.e. the sum of the external magnetic field and the field generated by the current distribution. 
	
	We begin with the general expression for magnetic force on a current distribution:
	\begin{equation*}
		\vec{F}_{\text{M}} = \iiint_{V}\j \cross \B_{\text{ext}} \dr,
	\end{equation*}
	where $ V $ is the region of space containing the current distribution $ \j $ and $ \B_{\text{ext}} $ is the external magnetic field. 

    \item Like in the analogous treatment of electric force, we can safely replace $ \B_{\text{ext}} $ with the total magnetic field $ \B \equiv \B_{\text{ext}} + \B_{\j} $, which includes both the external field and the ``internal'' field $ \B_{\j} $ generated by the current distribution itself. This replacement is valid because the contribution of the internal field to the magnetic force integrates to zero. This should make intuitive sense---the current distribution shouldn't be able to generate a force on itself. 

    By writing $ \B_{\text{ext}} \to \B $, the force on the current distribution becomes
    \begin{equation*}
		\vec{F}_{\text{M}} = \iiint_{V}\j \cross \B \dr,
    \end{equation*}
    where $ \B = \B_{\text{ext}} + \B_{\j} $ is the total magnetic field.
    
    \item Next we use Ampere's law $ \curl \B_{\j} = \mm \j $ to write the magnetic force in the form
    \begin{equation*}
        \vec{F}_{\text{M}} = \frac{1}{\mm}\iiint (\curl \B_{\j}) \cross \B \dr \to \frac{1}{\mm}\iiint (\curl \B) \cross \B \dr.
    \end{equation*}
    The justification for replacing $ \B_{\j} $ with $ \B $ is the same as the argument used in the analogous treatment of electrostatic force, included in the ``\textit{Technicality}'' note of \hyperref[ss:e-force]{\underline{Subsubsection \ref{ss:e-force}}}, and I am leaving it out here.
    
    \item Next, we quote the general vector calculus identity
	\begin{equation*}
		\B \cross (\curl \B) = \frac{1}{2}\grad B^{2} - \div (\B \otimes \B) + \B(\div \B).
	\end{equation*}
    We simplify this using $ \div \B = 0 $ (absence of magnetic monopoles) and rearrange using the cross product identity $ \vec{a} \cross \vec{b} = - \vec{b} \cross \vec{a} $ to get
    \begin{equation*}
        (\curl \B) \cross \B = \div (\B \otimes \B) - \frac{1}{2} \grad B^{2},
    \end{equation*}
    which we then substitute into the equation for magnetic force to get
	\begin{equation*}
		\vec{F}_{\text{M}} = \frac{1}{\mm}\iiint_{V} \div (\B \otimes \B)\dr - \frac{1}{2\mm}\iiint_{V} \grad B^{2}\dr.
	\end{equation*}
	
	\item Finally, we write $ \vec{F}_{\text{M}} $ in terms of a surface integral using the divergence theorem:
	\begin{equation*}
		\vec{F}_{\text{M}} = \frac{1}{\mm}\oiint_{\partial V}\left(\B \otimes \B - \frac{1}{2}B^{2}\mat{I}\right)\diff \vec{S},
	\end{equation*}
    which is the equation quoted at the begining of this subsection. As before, $ \mat{I} $ is the identity matrix and $ \partial V $ is the surface of the region containing the original current distribution $ \j $. Note that the quantity $ \B $, which is the total field, now implicitly contains the current distribution $ \j $.
\end{itemize}


\subsubsection{Magnetostatic Stress Tensor}
\begin{itemize}
	\item We write the magnetostatic force in terms of the magnetostatic stress tensor, using Einstein summation notation, as
	\begin{equation*}
		F_{\text{M}_{i}} = \oiint_{\partial V} \TT_{ik} \hat{n}_{k}\diff S,
	\end{equation*}
	where $ \hat{n}_{k} $ is the normal to the surface $ \partial V $ and $ \mat{T} $ is the $ (3 \cross 3) $ magnetostatic stress tensor with components given by
	\begin{equation*}
		T_{ik} = \frac{1}{\mm}\left(B_{i}B_{k} - \frac{1}{2}B^{2}\delta_{ik}\right).
	\end{equation*}
	
	\item We can also write the force as a volume integral in the form
	\begin{equation*}
		F_{\text{M}_{i}} = \iiint_{V}\pdv{\TT_{ik}}{x_{k}}\dr \equiv \iiint_{V}f_{\text{M}_{i}} \dr,
	\end{equation*} 
	where we have introduced the volume force density $ f_{\text{M}} $, motivated by the fact that $ \pdv{\TT_{ik}}{x_{k}} $ has units $ \si{\newton \, m^{-3}} $.
	
\end{itemize}
    

\subsection{Multipole Expansion of the Magnetic Potential}
\textit{State and derive the multipole expansion of the magnetic vector potential to the dipole term. Use the result to define magnetic dipole moment, and write the multipole expansion in terms of magnetic moment.}

\begin{itemize}
    \item Up to the dipole term, the magnetic vector potential of current distribution $ \j $ is
    \begin{equation*}
        \A(\r) = \frac{\mm}{4 \pi r^{3}} \iiint_{V} (\r \cdot \s)\j(\s)\ds.
    \end{equation*}
    The magnetic monopole term is zero, which corresponds to the fact that current loops are closed in magnetostatics.

    \item The magnetic dipolemoment of a current distribution $ \j $ is
    \begin{equation*}
        \m = \frac{1}{2}\iiint_{V} \s \cross \j(\s) \ds.
    \end{equation*}
    In terms of magnetic dipole moment, the dipole magnetic potential term reads
    \begin{equation*}
        \A(\r) = \frac{\mm}{4 \pi} \frac{\m \cross \r}{r^{3}} = \frac{\mm}{4\pi} \curl \frac{\m}{\abs{\r}}.
    \end{equation*}
    
\end{itemize}

\textbf{Derivation: Dipole Expansion of Magnetic Potential}
\begin{itemize}
	\item We begin with the general expression for $ \A $, from the solution to the magnetic Poisson equation, which reads
	\begin{equation*}
		\A(\r) = \frac{\mm}{4\pi}\iiint_{V}\frac{\j(\s)}{\abs{\r - \s}}\ds,
	\end{equation*}
    where $ V $ is the region of space containing the current density $ \j $. 

    We then consider the behavior of $ \A(\r) $ far from the source of $ \A $, and expand $ \A $ in the regime of $ \abs{\r} \gg \abs{\s} $ using the first-order expansion
	\begin{equation*}
		\frac{1}{\abs{\r - \s}} \approx \frac{1}{r} - (\s \cdot \grad) \frac{1}{r} + \cdots = \frac{1}{r} + \frac{\s \cdot \r}{r^{3}} + \cdots.
	\end{equation*}
    We then substitute this expansion into the expresion for magnetic potential to get
	\begin{equation*}
		\A(\r) \approx \frac{\mm}{4\pi r}\iiint_{V} \j(\s) \ds + \frac{\mm}{4\pi r^{3}} \iiint_{V}(\r \cdot \s)\j(\s)\ds.
	\end{equation*}
	The first term in the above expansion is called the monopole term---note that the monopole is a vector quantity (as opposed to the electric monopole, which corresponds to a point charge, which is a scalar quantity).
	
	The second term---the dipole term---is a rank-two tensor (while the electric field dipole is a vector). In general, the quantities arising in the magnetic multipole expansion have one more index than the analogous terms in the electric multipole expansion because $ \A $ is a vector field while electric potential $ \phi $ is a scalar field.

    \item The monopole term is zero, which follows from the divergence theorem and and the fact that in magnetostatics all current loops contributing the current density $ \j $ are closed, meaning $ \div \j = 0 $. The derivation reads
	\begin{equation*}
        \iiint_{V} \j(\s) \ds = \oiint_{\partial V} \div \j(\s) \diff \vec{S} = \oiint_{\partial V} 0 \cdot \diff \vec{S} = 0.
	\end{equation*}
    Note, however, that $ \div \j = 0 $ only applies in magnetostatics.

    \item Since the monopole term is zero, the multiple expression for $ \A(\r) $ simplifies to
	\begin{equation*}
		 \A(\r) = \frac{\mm}{4\pi r^{3}} \iiint_{V}(\r \cdot \s)\j(\s)\ds.
	\end{equation*}
    Without derivation, we state that we can use the divergence theorem and some vector calculus acrobatics to write $ \A(\r) $ in the form
	\begin{equation*}
		\A(\r) = \frac{\mm}{4\pi}\frac{\m \cross \r}{r^{3}},
	\end{equation*}
	where we have defined the magnetic dipole moment
	\begin{equation*}
		\m = \frac{1}{2}\iiint_{V} \s \cross \j(\s) \ds.
	\end{equation*}
	As a side note, the dipole term can also be written in the form
	\begin{equation*}
		\A(\r) =  \frac{\mm}{4\pi} \curl \frac{\m}{\abs{\r}}
	\end{equation*}
\end{itemize}


\subsection{Magnetic Field and Potential of a Magnetic Dipole}
\textit{State and derive the magnetic field of a magnetic dipole. Explain the concept of Ampere equivalence.}

\begin{itemize}
    \item The magnetic field of a magnetic dipole with magnetic dipole moment $ \m $ is
    \begin{equation*}
        \B = \frac{\mm}{4\pi} \frac{3 \r (\r \cdot \m) - \m r^{2}}{r^{5}}.
    \end{equation*}
    \begin{quote}
        \textit{Derivation}: We start with the just-derived expression for magnetic vector potential, which reads
	\begin{equation*}
		\A(\r) = \frac{\mm}{4\pi}\frac{\m \cross \r}{r^{3}}.
	\end{equation*}
	Using $ \A $, we find the magnetic field $ \B \equiv \curl \A $ via
	\begin{equation*}
		\B = \curl \A = \curl \left(\frac{\mm}{4\pi}\frac{\m \cross \r}{r^{3}}\right) = \frac{\mm}{4\pi} \frac{3\r(\r \cdot \m) - \m r^{2}}{r^{5}}.
	\end{equation*}
    \end{quote}
    
	\item Ampere equivalence refers to the following concept:
    \begin{quote}
        In an external magnetic field, a circular current-carrying loop is equivalent to a magnetic dipole.
    \end{quote}
\end{itemize}

\textbf{Derivation: Ampere Equivalence for a Circular Current Loop}
\begin{itemize}
	\item We start with the magnetic dipole moment of a circular current loop $ C $ of radius $ a $ and carrying current $ I $. The circular loop's magnetic dipole moment is
	\begin{equation*}
		\m = \frac{1}{2}\iiint_{V}\r \cross \j(\r)\dr =  \frac{1}{2}\oint_{C} \r \cross (I \diff \vec{l}),
	\end{equation*}
    where the last integral transitions from the general expression for $ \m $ in terms of current density $ \j $ to simpler expresion in terms of current $ I $ via $ \j \dr = I \diff \vec{l} $.

    \item We then transition to polar coordinates, in which $ \vec{r} \cross \diff \vec{l} = a \uvec{e}_{r} \cross \uvec{e}_{\phi} $, to get
	\begin{equation*}
		\m = \frac{1}{2}I\oint_{C} (a \uvec{e}_{r}) \cross  (\uvec{e}_{\phi})\diff l = \frac{aI}{2} \oint_{C} \uvec{e}_{z} \diff l = \pi a^{2}I \uvec{e}_{z} = IS\uvec{e}_{z},
	\end{equation*}
    where $ S = \pi a^{2} $ is the loop's area. The result $ \m = I S \uvec{e}_{z} $ is the known expression for a circular loop's magnetic dipole moment.
	
\end{itemize}
    
\subsection{Multipole Expansion of Magnetostatic Energy, Force and Torque}
\textit{State and derive the multipole expansion of magnetic energy up to the dipole term. Use the result to derive the force and torque on a magnetic dipole in an external electric field.}
    
\begin{itemize}
    \item The magnetic potential energy $ W_{\text{M}} $ of a magnetic dipole arising from a current distribution $ \j $ localized around the position $ \r_{0} $ in an external magnetic field $ \B $ with vector potential $ \A $
	\begin{align*}
		W_{\text{M}} &=  -\frac{1}{2} \big[\grad_{0}\cross \A(\r_{0})\big] \cdot \iiint_{V}\big[(\r - \r_{0}) \cross \j\big] \dr\\
		&= - \B(\r_{0})  \cdot \m,
	\end{align*}
    where $ \m $ is the current distribution's magnetic dipole moment.

    \item The magnetic force on a magnetic dipole in an external magnetic field is
	\begin{equation*}
		\vec{F}_{\text{M}} = (\m \cdot \grad)\B(\r).
	\end{equation*}
	In other words, the magnetic force on a magnetic dipole in an external magnetic field is the directional derivative of $ \B $ in the direction of $ \m $.

	\item The magnetic torque on a magnetic dipole in an external magnetic field is
	\begin{equation*}
		\vec{M}_{\text{M}} = \m \cross \B.
	\end{equation*}

\end{itemize}
\subsubsection{Derivation: Multipole Expansion of Magnetic Energy}
\begin{itemize}
	\item We consider an arbitrary system of current carrying loops with current density $ \j $ localized in space around the position $ \r_{0} $ and exposed to an \textit{external} magnetic field $ \B $ with vector potential $ \A $. The current distribution's magnetic field energy in the external magnetic field is
	\begin{equation*}
		W_{\text{M}} = - \iiint_{V} \j(\r) \cdot \A(\r) \dr,
	\end{equation*}
	where $ V $ is the region containing the localized current distribution $ \j $. 
	
	\item We then expand the magnetic potential about $ \r_{0} $ to get
	\begin{equation*}
		\A(\r) \approx \A(\r_{0}) + \big[(\r - \r_{0}) \cdot \grad_{0}\big] \A(\r_{0}) + \cdots,
	\end{equation*}
	where the gradient acts on $ \r_{0} $. We then substitute the expansion into $ W_{\text{M}} $ to get
	\begin{align*}
		W_{\text{M}} &= - \iiint_{V}\j(\r) (\r)\cdot \A(\r_{0}) \dr - \iiint_{V} \j(\r) \big[(\r - \r_{0}) \cdot \grad_{0}\big] \A(\r) \dr\\
		& = - \A(\r_{0})\iiint_{V} \j(\r)\dr - \iiint_{V} \j(\r)\big[(\r - \r_{0})\cdot \grad_{0}\big] \A(\r_{0})\dr.
	\end{align*}
	The first term is the monopole term and integrates to zero. The second term is more complicated---we will handle it by components in the next bullet point.

    \item First, we note that
	\begin{equation*}
		\grad_{0} \A(\r_{0}) = \pdv{\A(\r_{0})_{j}}{\r_{0_{i}}}.
	\end{equation*}
	This term depends only on $ \r_{0} $, and thus can be moved outside the integral over $ \r $ in the energy expression, which gives
    \begin{equation*}
        W_{\text{M}} = - \pdv{\A(\r_{0})_{j}}{\r_{0_{i}}} \iiint_{V} \j(\r)_{j} (\r - \r_{0})_{i} \dr.
    \end{equation*}
    
    \item Next, we apply tensor symmetrization (justified in the accompanying theory notes and left out here for conciseness), which produces
	\begin{equation*}
		\iiint_{V}\j(\r)_{j} (\r - \r_{0})_{i} \dr = - \iiint_{V} \j(\r)_{i}(\r - \r_{0})_{j}\dr.
	\end{equation*}
    Basically, the indexes $ i $ and $ j $ of the components of the vectors $ \j $ and $ (\r - \r_{0}) $ are switched, with the addition of a minus sign. We use this symmetry to write
	\begin{equation*}
		\iiint_{V}\j(\r)_{j} (\r - \r_{0})_{i} \dr = \frac{1}{2}\iiint_{V}  \big[\j(\r)_{j} (\r - \r_{0})_{i} - \j(\r)_{i}(\r - \r_{0})_{j}\big]\dr.
	\end{equation*}
	
	\item Putting the pieces together, the magnetic energy is
	\begin{equation*}
		W_{\text{M}} = -\frac{1}{2}\pdv{\A(\r_{0})_{j}}{\r_{0_{i}}} \iiint_{V}  \big[\j(\r)_{j} (\r - \r_{0})_{i} - \j(\r)_{i}(\r - \r_{0})_{j}\big]\dr.
	\end{equation*}
	Back in vector form, the magnetic energy reads
	\begin{equation*}
		W_{\text{M}} = - \frac{1}{2}\iiint_{V} \Big\{\j(\r) \big[ (\r - \r_{0})\cdot \grad_{0} \big ] \A(\r_{0}) - (\r - \r_{0})(\j\cdot \grad_{0})\A(\r_{0}) \Big\} \dr.
	\end{equation*}
	
    \item Next, to make a subsequent step more clear, we define the following vectors:
	\begin{equation*}
		\vec{a} \equiv (\r - \r_{0}) \qquad \vec{b} \equiv \j(\r) \qquad \vec{c} \equiv \grad_{0} \qquad \vec{d} \equiv \A(\r_{0}).
	\end{equation*}
	In this notation, the magnetic energy is
	\begin{align*}
		W_{\text{M}} &= -\frac{1}{2}\iiint_{V}\Big[\vec{b}   (\vec{a} \cdot \vec{c}) \vec{d} - \vec{a}(\vec{b}\cdot \vec{c})\vec{d} \Big] \dr\\
		& = -\frac{1}{2}\iiint_{V}\Big[(\vec{a} \cdot \vec{c}) (\vec{b}\cdot\vec{d}) - (\vec{a} \cdot \vec{d}) (\vec{b}\cdot \vec{c}) \Big] \dr.
	\end{align*}
	We then use the vector identity
	\begin{equation*}
		(\vec{a}\cross \vec{b}) \cdot (\vec{c} \cross \vec{d}) = 	(\vec{a}\cdot \vec{c}) (\vec{b} \cdot \vec{d}) - 	(\vec{a}\cdot \vec{d}) (\vec{b} \cdot \vec{c})
	\end{equation*}
	to rewrite the magnetic field energy in the form
	\begin{align*}
		W_{\text{M}} &= - \frac{1}{2}\iiint_{V} \Big\{\big[(\r - \r_{0}) \cross \j\big] \cdot \big[\grad_{0}\cross \A(\r_{0})\big] \Big\}\dr\\
		& = -\frac{1}{2} \big[\grad_{0}\cross \A(\r_{0})\big] \cdot \iiint_{V}\big[(\r - \r_{0}) \cross \j\big] \dr,
	\end{align*}
	where we have moved the $ \r_{0} $-dependent term $ \grad_{0} \cross \A $ out of the integral over $ \r $. 
	
	\item Recognizing $ \B = \curl \A $ and the magnetic dipole in the integrand, the result is
	\begin{align*}
		W_{\text{M}} &=  -\frac{1}{2} \big[\grad_{0}\cross \A(\r_{0})\big] \cdot \iiint_{V}\big[(\r - \r_{0}) \cross \j\big] \dr\\
		&= - \B(\r_{0})  \cdot \m,
	\end{align*}
	which is the expression quoted at the beginning of the subsection.
	
\end{itemize}

\subsubsection{Derivation: Force on a Magnetic Dipole in an External Magnetic Field}
\begin{itemize}
	\item We begin with the general relationship between force and energy, which reads
	\begin{equation*}
		\diff W_{\text{M}} = - \vec{F}_{\text{M}} \cdot \diff \r,
	\end{equation*}
	and take $ \diff \r $ to be a small displacement of a current distribution $ \j $ in an external magnetic field. 

    \item Next, we combine the energy of a magnetic dipole in an external magnetic field, i.e. 
    $ W_{\text{M}} = - \B(\r) \cdot \m $, with the relationship $ \vec{F}_{\text{M}} = - \grad W_{\text{M}}  $, which produces
    \begin{equation*}
        \vec{F}_{\text{M}} = - \grad W_{\text{M}} = \grad [ \m \cdot \B(\r) ].
    \end{equation*}
    Combining $ \vec{F}_{\text{M}} = \grad [\m \cdot \B(\r)] $ with $ \diff W_{\text{M}} = - \vec{F}_{\text{M}} \cdot \diff \r $ gives
	\begin{equation*}
		\diff W_{\text{M}} = - \grad\big[\m \cdot \B(\r)\big] \cdot \diff \r.
	\end{equation*}
	The gradient evaluates to 
	\begin{equation*}
		\grad\big[\m \cdot \B(\r)\big] = \m \cross (\curl \B) + (\m \cdot \grad)\B,
	\end{equation*}
    which results in
    \begin{equation*}
        \diff W_{\text{M}} = - \big[ \m \cross (\curl \B) + (\m \cdot \grad)\B \big] \cdot \diff \r.
    \end{equation*}
    
	\item We know from Ampere's law that $ \curl \B = \mm \j $. But careful here! So far, when evaluating energy and force, we have considered only the external magnetic field, and not the contribution of the current distribution $ \j $ to the total magnetic field. However, the magnetic field in the expression $ \m \cdot \B $ and thus $ \curl \B $ refers only to the external field.
	
	To make this distinction clear, we write Ampere's law for our problem as 
	\begin{equation*}
		\curl \B = \mm \t{\j},
	\end{equation*}
	where $ \t{\j} $ is the current distribution generating the external field $ \B $---note that the $ \t{\j} $ generating the external field is unrelated to the current $ \j $ for which we are calculating magnetic force.
	
	\item We now make an approximation---we assume the current distribution $ \t{\j} $ generating the external field is far from the current distribution $ \j $ for which we are calculating magnetic force. This assumption implies $ \big |\t{\j}\big | \ll \abs{\j} $ near the localized region of space containing $ \j $. Since $ \j $ sets the scale for our problem, and $ \t{\j} $ is negligible compared to $ \j $, we can make the approximation
	\begin{equation*}
		\tilde{\j} \approx 0 \implies \curl \B = \mm \t{\j} \approx 0.
	\end{equation*}

	\item Using $ \curl \B \approx 0 $ leads to
	\begin{equation*}
		\grad\big[\m \cdot \B(\r)\big] = \m \cross (\curl \B) + (\m \cdot \grad)\B \approx (\m \cdot \grad)\B,
	\end{equation*}
	and thus
	\begin{equation*}
		\diff W_{\text{M}} = - \grad\big[\m \cdot \B(\r)\big]\cdot  \diff \r \approx - \big[ (\m \cdot \grad)\B\big]\cdot \diff \r.
	\end{equation*}
	Comparing this to the general force-energy relation $ \diff W_{\text{M}} = - \vec{F}_{\text{M}}\diff \r $ produces
	\begin{equation*}
		\vec{F}_{\text{M}} = (\m \cdot \grad)\B(\r),
	\end{equation*}
    which is the magnetic force on a magnetic dipole in an external magnetic field.
    
\end{itemize}

\subsubsection{Derivation: Torque on a Magnetic Dipole in an External Magnetic Field}
\begin{itemize}
	\item We begin with the general relationship between torque and energy, which reads
	\begin{equation*}
		\diff W = - \vec{M} \cdot \diff \vec{\phi},
	\end{equation*}
    where $ \M $ is torque, $ \diff \vec{\phi} $ is an infinitesimal rotation in space, and $ \diff W $ is the corresponding change in potential energy.
	
	\item A small rotation of a magnetic dipole $ \m $ in an external magnetic field reads
	\begin{equation*}
		\diff \vec{m} = \diff \vec{\phi} \cross \m,
	\end{equation*}
	and the corresponding change in the dipole's energy because of this rotation, using the potential energy of an magnetic dipole, i.e. $ W_{\text{M}} = - \m \cdot \B $, is
	\begin{equation*}
		\diff W_{\text{M}} = - \diff \m \cdot \diff \B = - (\diff \vec{\phi} \cross \m)\cdot \B.
	\end{equation*}
	This is a scalar triple product, which we can rearrange to get
	\begin{equation*}
		\diff W_{\text{M}} = - \diff \vec{\phi} \cdot (\m \cross \B).
	\end{equation*}
	

    \item Comparing $ \diff W_{\text{M}} = - \diff \vec{\phi} \cdot (\m \cross \B) $ to the general relation $ \diff W = - \vec{M} \cdot \diff \vec{\phi} $ produces the desired expression for magnetic torque:
	\begin{equation*}
		\vec{M}_{\text{M}} = \m \cross \B.
	\end{equation*}

\end{itemize}

    
\newpage
\section{Quasistatic Electromagnetic Fields}

\subsection{Maxwell Equations in the Quasistatic Regime}
\textit{State the Maxwell equations in the regime of quasistatic electromagnetic fields, and derive the equations that are different from their static analogs. Show that in the quasistatic regime, the Maxwell equations correspond to closed current loops.}

\begin{itemize}
    \item In the quasistatic regime of electromagnetism, the Maxwell equations read
	\begin{align*}
		&\div \E = \frac{\rho}{\ee} \qquad \curl \E = - \pdv{\B}{t}\\
		& \div \B = 0 \ \, \qquad \curl \B = \mm \j.
	\end{align*}
    The above Maxwell equations completely describe quasistatic electromagnetism in which $ \div \j = 0 $. They do not, however, describe all of electromagnetism---the equation $ \curl \B = \mm \j  $ must be generalized to allow for $ \div \j \neq 0 $.
    
    \item The equations $ \div \E = \frac{\rho}{\ee} $ and $ \div \B  = 0 $, which link the $ \E $ and $ \B $ fields to their sources, are the same as their static analogs.

    The equation for $ \curl \E $ is different from the static version and is derived below.

\end{itemize}

\textbf{Derivation: The Electric Field Curl Equation}
\begin{itemize}
    \item To derive the equation $ \curl \E = - \pdv{\B}{t} $, we begin by recalling Lenz's law:
    \begin{quote}
        The change in magnetic flux through a current loop induces an electric current that opposes the magnetic flux inducing the current. 
    \end{quote}
	Note that the law in this form is qualitative. We will instead use Maxwell's formulation of Lenz's law, which states that the electric circulation $ \Gamma_{\mathrm{E}} $ in a current loop and the magnetic flux $ \Phi_{\text{M}} $ through the loop are observed to obey the relationship
	\begin{equation*}
		\Gamma_{\mathrm{E}} = -\dv{t} \Phi_{\text{M}}.
	\end{equation*}
    We stress that this result comes from experiment. We have not derived it from fundamental principles; it is simply the observed physical behavior.
	
	\item In integral form, the quantitative formulation of Lenz's law reads
	\begin{equation*}
		\oint_{\partial S}\E \cdot \diff \r = - \dv{t} \iint_{S} \B \cdot \diff \vec{S}.
	\end{equation*}
	We then apply Stokes' theorem to transform this equation to the differential form
	\begin{equation*}
		\iint \curl \E \cdot \diff \vec{S} = - \iint \pdv{\B}{t} \cdot \diff \vec{S},
	\end{equation*}
    where we have assumed the shape of the current loop is constant through time. Comparing the integrands in the above equation gives the desired result
	\begin{equation*}
		\curl \E = - \pdv{\B}{t}.
	\end{equation*}
	
\end{itemize}

\subsection{Canonical Momentum}
\textit{Define canonical momentum in the context of electromagnetism, and explain the motivation for its definition.}

\begin{itemize}
    \item In electromagnetism, the canonical momentum of a particle of charge $ q $ in an external magnetic potential $ \A $ is defined as
    \begin{equation*}
        \vec{p}_{\text{canonical}} = m \dot{\vec{r}} + q\A,
    \end{equation*}
    which rests on the fact that $ q\A $ has units of momentum. The quantity $ m \dot{\vec{r}} $ is called kinetic momentum.

    \item Canonical momentum should really be discussed in the context of Hamiltonian mechanics---see e.g. \hyperref[ss:hamilton]{\underline{Subsection \ref{ss:hamilton}}}.

\end{itemize}

\textbf{Motivation: Definition of Canonical Momentum}
\begin{itemize}
	\item We begin by combining the Maxwell equation for $ \curl \E $ with $ \B = \curl \A $ to get
	\begin{equation*}
		\curl \E = - \pdv{\B}{t} = - \curl \pdv{\A}{t},
	\end{equation*}
	which leads to to relationship
	\begin{equation*}
		\E = - \pdv{\A}{t}.
	\end{equation*}
	\item We then combine the above expression for $ \E $ with Newton's law and the electrostatic force to get
	\begin{equation*}
		\dv{\mat{p}}{t} = \vec{F} = q \E = - \pdv{(q \A)}{t}.
	\end{equation*} 
    In other words, we've derived the relationship $ \vec{p} = - q \A $---which implies that $ q\A $ behaves as a momentum.
	
    % \item The canonical momentum $ \vec{p} = q \A $ leads to an interesting interpretation. Namely, we can interpret magnetic induction as the impulse of the momentum $ \vec{p} = q \A $. 
\end{itemize}

    
\subsection{Electric Field in Terms of the Electromagnetic Potentials}
\textit{State and derive the complete relationship between electric field strength and the electric and magnetic potentials. Use the result to calculate the curl of the electric field and interpret the result.}

\begin{itemize}
    \item The electric field $ \E $ and magnetic field $ \B $ are written in terms of the electrostatic potential $ \phi $ and magnetic vector potential $ \A $ as
    \begin{equation*}
        \E = - \grad \phi - \pdv{\A}{t} \qquad \text{and} \qquad \B = \curl \A.
    \end{equation*}

    \item The expression for magnetic field in terms of $ \A $ was introduced in \hyperref[ss:magnetic-potential]{\underline{Subsection \ref{ss:magnetic-potential}}}.

    Recall that the definition of magnetic potential via $ \B = \curl \A $ rested on the Maxwell equation $ \div \B = 0 $, which is satisfied if $ \B $ is written as the curl of a vector field, since the divergence of a curl of a vector field is always zero. Since $ \div \B = 0 $ is always valid, the expression $ \B = \curl \A $ thus holds in dynamic as well as quasistatic situations, and we don't need to generalize if further.

    \item The electrostatic expression for electric field $ \E = - \grad \phi $ was possible because of the static Maxwell equation $ \curl \E = 0 $. However, $ \curl \E = 0 $ does not hold in quasistatic or dynamic situations, which is why we must generalize the expression for $ \E $.

\end{itemize}

\textbf{Derivation: Electric Field in Terms of Electromagnetic Potential}
\begin{itemize}

    \item We begin with the Maxwell equation for $ \curl \E $, which reads
    \begin{equation*}
		\curl \E = - \pdv{\B}{t} = - \curl \pdv{\A}{t},
	\end{equation*}
    where we have used $ \B = \curl \A $ in the last equality.

    \item We then rewrite the above equation for $ \curl \E $ in the form
	\begin{equation*}
		\curl \left(\E + \pdv{\A}{t}\right) = 0.
	\end{equation*}
	Recall that the curl of the gradient of a scalar field is always zero. This means that the above expression in parentheses is determined only up to the gradient of a scalar field. This implies
	\begin{equation*}
        \E + \pdv{\A}{t} = - \grad \phi \qquad \text{or} \qquad \E = - \grad \phi - \pdv{\A}{t}.
	\end{equation*} 
	In the quasistatic regime, the complete expression for electric field is thus
	\begin{equation*}
		\E = - \grad \phi - \pdv{\A}{t}.
	\end{equation*}
	
\end{itemize}
    

\subsection{Ohm's Law and Electric Fields in Conductors}
\textit{State Ohm's law in terms of electric field and current density. Discuss the basic properties of conductors, and use Ohm's law to explain the behavior of the electric field within a conductor and the electric potential on the conductor's surface.}

\vspace{2mm}
\textit{State and derive the expression for dissipation of electric energy in an Ohmic conductor.}

\begin{itemize}

	\item For the purposes of this course, conductors are materials permitting the motion of charged particles through the material. The charged particles are typically electrons, holes, or ions.

    \item The electric field $ \E $ and current density $ \j $ in a conductor are related by
    \begin{equation*}
        \j = \sigma_{\text{E}} \E,
    \end{equation*}
    where $ \sigma_{\text{E}} $ is the conductor's electrical conductivity. The above equation is the general form of Ohm's law.
    
    \item When conductors are placed in an external electric field:
    \begin{itemize}
        \item Charge in the conductor moves from the conductor's body to its surface.

        \item The induced charge on the conductor's surface exactly cancels the external electric field within the conductor, so that the net electric field inside the conductor is zero.

        \item The conductor's surface is an equipotential surface, and the electric field at the surface is normal to the surface.
    \end{itemize}
    
    \item The electric energy dissipated in an Ohmic conductor because of charge carrier motion corresponding to a current density $ \j $ under the influence of an electric field $ \E $ is encoded by
    \begin{equation*}
        P = \iiint_{V} \j \cdot \E \dr,
    \end{equation*}
    where $ P $ is the dissipated power. Note that for Ohmic materials we have $ \j \propto \E $ and thus $ P \propto E^{2} $. 
\end{itemize}

\subsubsection{Further Discussion: Conductors}
\begin{itemize}
	
    \item Conductors are electrically neutral. The charges in the material my move around, but the conductor as a whole is neutral.
	
	\item When a conductor is placed in an external electric field, positive charges move in the direction of the external field and negative charges opposite the external field. 
	
	The effect is that the free charges in the conductor rearrange in a configuration that cancels out the external field. The total electric field in the conductor at equilibrium, after the charge redistribution, is thus zero--if it weren't zero, the charges would keep moving! At equilibrium, we have $ \j = 0 $ and $ \E = 0 $ within the conductor.
	
	\item Continuing with the example of an conductor in an external electric field at equilibrium, we recall that $ \E $ and $ \rho $ are related by Gauss's law via
	\begin{equation*}
		\div \E = \frac{\rho}{\ee}.
	\end{equation*}
    Because $ \E = 0 $ at equilibrium, it follows from Gauss's law that $ \rho $ is also zero at equilibrium. In other words, there is zero volume charge density in a conductor at equilibrium. If the charges are not within the conductor, they must then be at the conductor's surface.

    We let $ \uvec{n} $ and $ \sigma $ denote the normal to the surface and the surface charge density, respectively. Near the conductor's surface we must have 
	\begin{equation*}
		\E \cdot \uvec{n} = \frac{\sigma}{\ee},
	\end{equation*}
	which means $ \E $ is perpendicular to the conductor's surface.
	
	\item \textit{Technicality}: Assume we increase the external electric field. As we increase the external field, the conductor must naturally redistribute more and more charge to its surface to cancel out the external field. 
	
	In principle, at some point, there might not be enough charge left in the material to cancel out the external field. In this course, however, we will assume conductors always have enough available charge pairs (e.g. electrons and holes) to cancel out an external magnetic field. 
	
	\item For the conductor to be in equilibrium, the electric field must always be normal to the surface (i.e. $ \E \cross \uvec{n} = 0 $). If the electric field at the surface had a tangential component, charges would move along the surface in the tangent direction, and the conductor wouldn't be in equilibrium. 
	
	Since $ \E $ is perpendicular to the conductor's surface at equilibrium, the surface is an equipotential surface and obeys
	\begin{equation*}
		\phi_{2} - \phi_{1} = \int \E \cdot \diff \vec{l} = \int \E \cdot \uvec{t} \diff l = \int 0 \diff l = 0 \implies \phi \text{ constant}.
	\end{equation*}
	where $ \uvec{t} $ is the tangent to the surface---$ \E \cdot \uvec{t} = 0$ since $ \E $ is normal to the surface. 

\end{itemize}

\subsubsection{Derivation: Energy Dissipation in Ohmic Conductors} \label{sss:ohmic-dissipation}
\begin{itemize}
    \item In general a charge distribution experiences two forces---the electric and magnetic forces. The magnetic force $ \vec{F}_{\text{M}} = q \vec{v} \cross \B  $ is perpendicular to velocity and cannot do work, so magnetic force does not contribute to dissipative forces. 
	
    \item More formally, the dissipative power associated with a charge carrier in terms of electromagnetic force density $ \vec{f}_{\text{em}} = \rho \E + \j \cross \B $ is
	\begin{equation*}
		P = \iiint_{V} \vec{v} \cdot \vec{f}_{\text{em}} \dr = \iiint_{V} \frac{\j}{\rho} \left(\rho \E + \j \cross \B\right)\dr,
	\end{equation*}
	where we have substituted in $ \vec{v} = \j/\rho $. The second integrand contains the term $ \j \cdot (\j \cross \B) = 0 $, which is why the magnetic force cannot dissipate power. The dissipated power thus reduces to the desired expression
	\begin{equation*}
		P = \iiint_{V} \j \cdot \E \dr.
	\end{equation*}
		
\end{itemize}

\subsection{The Relaxation Time Constant of a Conductor}
\textit{Discuss the concept of a conductor's relaxation time. What does this time represent, how is it related to electrical conductivity, and what is a typical order of magnitude?}

\begin{itemize}
	\item A conductor's relaxation encodes how quickly a conductor can react to an external electric field, i.e. how quickly charges redistribute in response to the external field to create equilibrium within the conductor.
	
	\item We begin with the continuity equation, which reads
	\begin{equation*}
		\div \j + \pdv{\rho}{t} = 0.
	\end{equation*}
	Note that this is a generalization of the earlier expression $ \div \j = 0 $ to electrodynamic situations involving time-varying charge distributions. 
	
	\item We then substitute Ohm's law $ \j = \sigma_{\mathrm{E}}\E $ into the continuity equation to get
	\begin{equation*}
		\div (\sigma_{\mathrm{E}} \E) + \pdv{\rho}{t} = 0.
	\end{equation*}
	Next, we write $ \E $ in terms of $ \rho $ using Gauss's law $ \div \E = \frac{\rho}{\ee} $, which produces
	\begin{equation*}
		\pdv{\rho}{t} + \frac{\sigma_{\mathrm{E}}}{\ee} \rho = 0.
	\end{equation*}
	The general solution for $ \rho $ is exponential:
	\begin{equation*}
        \rho(\r, t) = \rho(\r, 0)e^{-\frac{t}{\tau}}, \quad \text{where } \tau = \frac{\ee}{\sigma_{\text{E}}}.
	\end{equation*}
	The larger the conductivity $ \sigma_{\mathrm{E}} $, the smaller the characteristic response time $ \tau $, and the sooner the conductor reaches equilibrium in an external electric field. 
	
	\item As an example, for iron, we have $ \tau \approx \SI{8.85e-19}{\second} $. In other words, the time constant is incredibly small---on a macroscopic scale, the conductor reaches equilibrium essentially instantly.
	
\end{itemize}

\subsection{Drude Model of Electrical Conduction}
\textit{Derive Ohm's law from the Drude model of electrical conduction. Discuss the Drude model's prediction for a material's electrical conductivity. }

\begin{itemize}
	\item The Drude model is a simple model of electrical conduction in terms of Newtonian motion of microscopic charge carriers, and can be used to derive Ohm's law from fundamental principles.
	
	We begin with Newton's law for a charge carrier moving through a conductor in an electric field. We consider two forces on the charge carrier:
	\begin{enumerate}
		\item a dissipative velocity-dependent force $ - m \gamma \vec{v}(t) $, where $ \gamma $ is a damping constant,
		
		\item and the accelerating electric force $ q \E(t) $.
	\end{enumerate}
	In terms of these two forces, Newton's law for the particle in the conductor reads
	\begin{equation*}
		m \dv{v}{t} = - m \gamma \vec{v}(t) + q \E(t).
	\end{equation*}
	
	\item When $ \E = 0 $, the solution is $ \vec{v}(t) = \vec{v}_{0} e^{- \gamma t}$. In this case, the particle's velocity exponentially decays with time, which means particles in the conductor rapidly stop moving. This makes sense---electric current is not observed to flow in the absence of an external electric field.
	
	\item When $ \E \neq 0 $, we use the ansatz (which is quoted, not derived or explained further)
	\begin{equation*}
		\vec{v}(t) = \frac{q}{m}\int_{-\infty}^{t}e^{-\gamma(t - \t{t})}\E(\t{t}) \diff \t{t},
	\end{equation*}
    where $ \tilde{t} $ is a placeholder variable for time integration.
	
    \item Next, we write current density to charge carrier velocity $ \vec{v} $ via
	\begin{equation*}
		\j = \rho \vec{v} = n q \vec{v},
	\end{equation*}
	where $ q $ is the charge of a single charge carrier, $ n $ is the number density of charge carriers in the conductor, and $ \vec{v} $ is the charge carriers' drift velocity. Substituting the ansatz for charge carrier velocity $ \vec{v} $ into the equation for $ \j $ gives
	\begin{equation*}
		\j = \frac{nq^{2}}{m}\int_{-\infty}^{t}e^{-\gamma(t - \t{t})}\E(\t{t}) \diff \t{t}.
	\end{equation*}
	
	\item For a constant field, $ \E(t) = \E_{0} $, we can easily solve the integral to get Ohm's law
	\begin{equation*}
		\j = \frac{nq^{2}}{m\gamma}\E_{0}.
	\end{equation*}
	The Drude model thus gives a prediction for conductivity
	\begin{equation*}
		\sigma_{\mathrm{E}} = \frac{nq^{2}}{m \gamma}
	\end{equation*}
    Note that the Drude model gives an expression for conductivity in terms of fundamental quantities, rather than taking conductivity itself to be fundamental.

\end{itemize}
	



\subsection{Capacitance}
\textit{What is capacitance? State and derive the expression for the capacitance of an arbitrary configuration of $ N $ conductors. How is the generalized form of capacitance related to a capacitive system's total electric field energy?}

\begin{itemize}
    \item Capacitance describes how much charge can accumulate on a capacitor at a given potential difference. A conductor's capacitance depends only on its geometry (and also permittivity, which we discuss in the chapter on electromagnetism in matter).

	\item For a coupled system of $ N $ conductors, the system's total capacitance is encoded by a $ N \cross N $ rank-two tensor with elements $ C_{ik} $ given by
	\begin{equation*}
		\frac{1}{C_{ik}} = \frac{1}{4\pi \ee q_{i} q_{k}} \iint_{S} \iint_{S} \frac{\sigma_{i}\sigma_{k}\diff S_{i}\diff S_{k}}{\abs{\r_{i} - \r_{k}}}.
	\end{equation*}

	\item In terms of capacitance, the system's total electric field energy is
	\begin{equation*}
		W_{\text{E}} = \frac{1}{2}\sum_{i, k}(C_{ik})^{-1}q_{i}q_{k},
	\end{equation*}
    where $ q_{i} $ denotes the electric charge on the $ i $-th capacitive element.

\end{itemize}

\textbf{Derivation: Generalized Form of Capacitance}
\begin{itemize}
	
	\item We begin by considering $ N $ conductors indexed by $ i = 1, \ldots, N $. For example, for a parallel-plate capacitor with two plates, we would have $ N = 2 $. In a more general situation, however, we can have many ``conductors'' contributing to a cumulative capacitance. 
	
    \item The surface of any conductor is an equipotential surface, which we write for our system of $ i = 1, \ldots, N $ conductors in the form
	\begin{equation*}
        \phi(\r) \big |_{\partial V_{i})}  = k_{i},
	\end{equation*}
	where the $ k_{i} $ are constants encoding the constant potential on the $ i $-th conductor's surface.
	
	\item Next, we consider the total electric energy associated with the collection of conductors, which is
	\begin{equation*}
		W_{\text{E}} = \frac{1}{2}\iiint_{V}\rho(\r)\phi(\r)\dr,
	\end{equation*}
	where $ \rho(\r) $ is the volume charge density in the conductor collection. However, since all charges in the conductors occur at the surface, we can change the expression for $ W_{\text{E}} $ to a surface integral of the form
	\begin{equation*}
		W_{\text{E}} = \frac{\phi}{2} \iint_{S} \sigma(\r) \diff \vec{S},
	\end{equation*}
	where $ \phi $ is moved out of the integral because it is constant along the surface. 

    \item In terms of the individual contributions of each conductor, the energy becomes
	\begin{equation*}
		W_{\text{E}} = \frac{1}{2}\sum_{i} \phi_{i} \iint_{S} \sigma_{i}(\r) \diff \vec{S}_{i} = \frac{1}{2}\sum_{i} \phi_{i} q_{i}
	\end{equation*}
	where $ q_{i} $ is the charge on the $ i $th conductor's surface. The total electric field energy in the conductor collection is thus
	\begin{equation*}
		W_{\text{E}} = \frac{1}{2}\sum_{i} \phi_{i}q_{i}.
	\end{equation*}
	
	\item We will now find the same total electric field energy with a different approach. We will then equate the two expressions for $ W_{\text{E}} $ to get a generalized expression for capacitance. As before, we begin with the definition
	\begin{equation*}
		W_{\text{E}} = \frac{1}{2}\iiint_{V} \rho(\r) \phi(\r) \dr.
	\end{equation*}
	Our goal is to write the integrand purely in terms of charge $ q $. We begin by writing $ \phi $ in terms of $ \rho $ using the Poisson equation, which gives
	\begin{equation*}
		\phi(\r) = \frac{1}{4\pi \ee}\iiint_{V}\frac{\rho(\t{\r})}{\abs{\r - \t{\r}}} \dtr.
	\end{equation*}
    Using this expression for $ \phi $, the electric field energy is thus
	\begin{equation*}
		W_{\text{E}} = \frac{1}{8\pi \ee} \iiint_{V}\left(\iiint_{V} \frac{\rho(\r)\rho(\t{\r})}{\abs{\r - \t{\r}}}\dtr \right) \dr.
	\end{equation*}
	
    \item Because the conductor charge is on the surface only, we re-write the integral in terms surface charge density and surface elements, which gives
	\begin{equation*}
		W_{\text{E}} = \frac{1}{8\pi \ee}\sum_{i, k} \iint_{S} \iint_{S} \frac{\sigma_{i}\sigma_{k}\diff S_{i}\diff S_{k}}{\abs{\r_{i} - \r_{k}}},
	\end{equation*}
	where $ \diff S_{i} $ and $ \diff S_{k} $ are the surface elements of the $ i $th and $ k $th conductors, and $ \r_{i} $ and $ \r_{k} $ are the position vectors on the $ i $th and $ k $th surfaces. 
	
	Finally, we multiply above and below by $ q_{i}q_{k} $ to get
	\begin{equation*}
		W_{\text{E}} = \frac{1}{2} \sum_{i, k} \frac{1}{4\pi \ee q_{i}q_{k}}q_{i}q_{k} \iint_{S} \iint_{S} \frac{\sigma_{i}\sigma_{k}\diff S_{i}\diff S_{k}}{\abs{\r_{i} - \r_{k}}}.
	\end{equation*}
	
	\item Next, we equate the above expression for $ W_{\text{E}} $ to the earlier $ W_{\text{E}} $, which results in the general expression for capacitance
	\begin{equation*}
		\frac{1}{C_{ik}} = \frac{1}{4\pi \ee q_{i} q_{k}} \iint_{S} \iint_{S} \frac{\sigma_{i}\sigma_{k}\diff S_{i}\diff S_{k}}{\abs{\r_{i} - \r_{k}}}.
	\end{equation*}
	In terms of capacitance, the system's total electric field energy is
	\begin{equation*}
		W_{\text{E}} = \frac{1}{2}\sum_{i}\phi_{i}q_{i} = \frac{1}{2}\sum_{i, k}(C_{ik})^{-1}q_{i}q_{k}.
	\end{equation*}

	\item Finally, we note that the expression for capacitance $ C_{ik} $ is consistent with capacitance depending only on geometry (and not charge): the $ q_{i}q_{k} $ in the denominator cancel with the $ \sigma_{i} \sigma_{k} $ after the charge densities are integrated over the surfaces $ \diff S_{i} $ and $ \diff S_{k} $. This exact cancellation works out since the expression $ \abs{\r_{i} - \r_{k}} $ ``normalizes'' the geometric properties of the potentially varying surface charge densities. 

	% Finally, note that the charge on a given conductor depends on the capacitance and potential difference of all of the other conductors in the collection---this is a consequence of the principle of superposition. 
	
\end{itemize}

\subsection{Inductance}

\textit{What is inductance? State and derive the expression for the inductance of an arbitrary configuration of $ N $ conductors. How is the generalized form of inductance related to a inductive system's total magnetic field energy and the magnetic flux through the inductive system?}

\begin{itemize}

	\item Inductance relates the magnetic flux through a conducting loop to the electric current through the conductor. Like capacitance, inductance depends only on geometric properties.

    \item For a coupled system of $ N $ inductors, the system's total inductance is encoded by a $ N \cross N $ rank-two tensor with elements $ L_{ik} $ given by
	\begin{equation*}
		L_{ik} = \frac{\mm}{4\pi}\oint_{C_{i}}\oint_{C_{k}}\frac{\diff \vec{l}_{i}\diff \vec{l}_{k}}{\abs{\r(l_{1}) - \r(l_{k})}}.
	\end{equation*}
    The tensor $ L_{ik} $'s diagonal terms $ L_{ii} $ are called self-inductance, while the off-diagonal terms correspond to mutual inductance between different inductors.

    \item In terms of inductance, the system's total magnetic field energy is
	\begin{equation*}
        W_{\text{M}} = \frac{1}{2}\sum_{i}I_{i}\Phi_{\text{M}_{i}} = \frac{1}{2} \sum_{i, k}I_{i}I_{k} L_{ik},
	\end{equation*}
    where $ I_{i} $ denotes the current through the $ i $-th inductor.
	
    \item In general, the magnetic flux through the $ i $th loop is
    \begin{equation*}
        \Phi_{\text{M}_{i}} = \sum_{i} L_{ik}I_{k}.
    \end{equation*}
	
\end{itemize}

\textbf{Derivation: Generalized Form of Inductance}
\begin{itemize}

	\item We begin with a generalized system of inductors, modeled by $ N $ current-carrying loops carrying currents $ I_{1}, I_{2}, \ldots, I_{N} $.
	
	\item We first find the system's total magnetic field energy. We begin with the definition
	\begin{equation*}
		W_{\text{M}} = \frac{1}{2}\iiint_{V}\j \cdot \A \dr,
	\end{equation*}
	where we integrate over the volume with non-zero current density $ \j $. We then rewrite the current in the form $ \j \dr = I \diff \vec{l} $ and consider all $ N $ loops together to get
	\begin{equation*}
		W_{\text{M}} = \frac{1}{2}\sum_{i}I_{i}\oint_{C_{i}} \A \cdot \diff \vec{l}_{i},
	\end{equation*}
	where the current $ I_{i} $ in the $ i $-th conductor is constant along the conductor's curve $ C_{i} $, and thus moves out of the integral. Finally, we write the integrand in terms of magnetic flux, for which we use Stokes's theorem:
	\begin{align*}
        W_{\text{M}} &= \frac{1}{2}\sum_{i}I_{i}\oint_{C_{i}} \A \cdot \diff \vec{l}_{i} = \frac{1}{2}\sum_{i}I_{i}\iint_{S_{i}} \curl \A \cdot \diff \vec{S}_{i} \\
        &= \frac{1}{2}\sum_{i}I_{i}\iint_{S_{i}} \B \cdot \diff \vec{S}_{i} = \frac{1}{2}\sum_{i}I_{i} \Phi_{\text{M}_{i}}.
	\end{align*}
	
	\item We will now find the same magnetic field energy with a second approach. We will then equate the two results for $ W_{\text{M}} $ to get a generalized expression for inductance. As before, we begin with the general definition
	\begin{equation*}
		W_{\text{M}} = \frac{1}{2}\iiint_{V} \j \cdot \A \dr.
	\end{equation*}
	We then express $ \A $ in terms of $ \j $ via
	\begin{equation*}
		\A(\r) = \frac{\mm}{4\pi}\iiint_{V}\frac{\j(\t{\r})}{\abs{\r - \t{\r}}}\dtr,
	\end{equation*}
    and substitute this expression for $ \A(\r) $ into $ W_{\text{M}} $ to get
	\begin{equation*}
		W_{\text{M}} = \frac{\mm}{8\pi}\iiint_{V} \left(\iiint_{V}\frac{\j(\r)\j(\t{\r})}{\abs{\r - \t{\r}}}\dtr\right)\dr.
	\end{equation*}
	Finally, we convert $ \j \dr $ to current, $ I \diff \vec{l} $, and sum over all loop pairs to get
	\begin{equation*}
		W_{\text{M}} = \frac{\mm}{8\pi}\sum_{i, k}I_{i}I_{k}\oint_{C_{i}}\oint_{C_{k}}\frac{\diff \vec{l}_{i}\diff \vec{l}_{k}}{\abs{\r(l_{1}) - \r(l_{k})}},
	\end{equation*}
	where $ C_{i} $ and $ C_{k} $ are the space curves associated with the $ i $th and $ k $th conducting loops.
	
	\item We then equate the two expressions for $ W_{\text{M}} $ to get
	\begin{equation*}
		\frac{1}{2}\sum_{i}I_{i}\Phi_{\text{M}_{i}} = \frac{\mm}{8\pi}\sum_{i, k}I_{i}I_{k}\oint_{C_{i}}\oint_{C_{k}}\frac{\diff \vec{l}_{i}\diff \vec{l}_{k}}{\abs{\r(l_{1}) - \r(l_{k})}}.
	\end{equation*}
    This relationship motivates the definition of inductance $ L_{ik} $ as
	\begin{equation*}
		L_{ik} = \frac{\mm}{4\pi}\oint_{C_{i}}\oint_{C_{k}}\frac{\diff \vec{l}_{i}\diff \vec{l}_{k}}{\abs{\r(l_{1}) - \r(l_{k})}},
	\end{equation*}
    in terms of which the relationship between the two forms of $ W_{\text{M}} $ simplifies to
	\begin{equation*}
        W_{\text{M}} = \frac{1}{2}\sum_{i}I_{i}\Phi_{\text{M}_{i}} = \frac{1}{2} \sum_{i, k}I_{i}I_{k} L_{ik}.
	\end{equation*}

\end{itemize}

    
\subsection{The Skin Effect}
\textit{What is the skin effect? State the equations needed to analyze the skin effect, give their solutions in the case of a cylindrical conductor, and discuss the results. Discuss the limit cases of complex impedance for small and large frequencies.}

\begin{itemize}
	\item Qualitatively, the skin effect is summarized as follows:
    \begin{quote}
        At high frequencies, alternative current tends to run mostly along a conductor's surface. More formally, the current density is largest at a conductor's surface for high frequency alternating currents.
    \end{quote}

    \item The analysis of the skin effect in conductors rests on the diffusion equations
    \begin{equation*}
		\laplacian \E = \mm \sigma_{\text{E}}\pdv{\E}{t} \eqtext{and} \laplacian \B = \mm \sigma_{\text{E}} \pdv{\B}{t}.
    \end{equation*}
    
	\item In one dimension, the position-dependent components of the solutions for $ \E $ and $ \B $ are
	\begin{equation*}
		\begin{Bmatrix}
            \E(z)\\
            \B(z)
		\end{Bmatrix}
        \sim \exp\left(- \frac{\sqrt{2}}{2}\sqrt{\omega \mm \sigma_{\text{E}}}\cdot z\right)\exp\left(i \frac{\sqrt{2}}{2}\sqrt{\omega \mm \sigma_{\text{E}}}\cdot z\right),
	\end{equation*}
    where $ \omega $ is the alternating current frequency and $ \sigma_{\text{E}} $ is the conductor's conductivity. 

	Qualitatively: the $ \sim e^{iz} $ term describes oscillation, while the $ \sim e^{-z} $ term describes exponential decay, with characteristic decay length $ z_{0} = \sqrt{\frac{2}{\omega \mm \sigma}} $. Note that $ z_{0} $ decreases with increasing $ \omega $, so the electric field (and thus $ \j $ via Ohm's law) decays more rapidly with position at increasing frequencies. 


    \item In the specific case of a cylindrical, wire-like conductor, with longitudinal axis in the $ z $ direction, the electric and magnetic fields obey
    \begin{equation*}
        \E(\r, t) = (0, 0, E_{z}) \qquad \text{and} \qquad \B(\r, t) = (0, B_{\phi}, 0),
    \end{equation*}
    where the components $ E_{z} $ and $ B_{\phi} $ are
	\begin{equation*}
        E_{z}(\r, t) = A J_{0}(kr)e^{-i \omega t} \eqtext{and} B_{\phi}(\r, t) = i A \frac{k}{\omega} J_{1}(kr)e^{-i \omega t},
	\end{equation*}
	where $ A $ is a constant, $ k = \frac{\sqrt{2}}{2}(1- i)\sqrt{\omega \mm \sigma_{\text{E}}} $,	and $ J_{0} $ and $ J_{1} $ are the Bessel functions.
    
    Finally, assuming the wire has radius $ a $, the electric current through the wire is
    \begin{equation*}
        I(t) = \frac{2\pi a}{\mm}B_{\phi}(a, t).
    \end{equation*}
    
    
\end{itemize}

\subsubsection{Derivation: Skin Effect Diffusion Equations}
\begin{itemize} 

    \item We begin by consider a conductor, in which volume charge density is $ \rho = 0 $ in a conductor. The Maxwell equations for electric field are
	\begin{equation*}
		\div \E = \frac{\rho}{\ee} = 0 \eqtext{and} \curl \E = - \pdv{\B}{t}.
	\end{equation*}
    Using Ohm's law $ \j = \sigma \E $, the Maxwell equations for magnetic field are
	\begin{equation*}
		\div \B = 0 \eqtext{and} \curl \B = \mu_{0}\j = \mm \sigma_{\text{E}} \E.
	\end{equation*}

    \item We then take the curl of $ \curl \E $ and $ \curl \B $  to get
	\begin{align*}
        & \curl \big[ \curl \E \big] = - \pdv{t}\big[ \curl \B\big] = - \mm \sigma_{\text{E}} \pdv{\E}{t}\\
        & \curl [\curl \B] = \mm \sigma_{\text{E}}\curl \E = - \mm \sigma_{\text{E}}\pdv{\B}{t}.
	\end{align*}
	
	\item Finally, we use the general vector calculus identity
	\begin{equation*}
		\curl (\curl \vec{F}) = \grad (\div \vec{F}) - \laplacian \vec{F},
	\end{equation*}
    where $ \vec{F} $ is a vector field. We apply this identity to both double curl equations, together with $ \div \E = \div \B = 0 $, to replace the double curl with the Laplacian:
	\begin{equation*}
		\laplacian \E = \mm \sigma_{\text{E}}\pdv{\E}{t} \eqtext{and} \laplacian \B = \mm \sigma_{\text{E}} \pdv{\B}{t}.
	\end{equation*}
	These are diffusion equations for $ \E $ and $ \B $---their symmetry is a consequence of $ \rho = 0 $ for a conductor and $ \j = \sigma_{\text{E}} \E $ from Ohm's law. 
	
\end{itemize}

\textbf{Solution}
\begin{itemize}
	\item We solve the diffusion equations with separation of variables and the ansatzes
	\begin{equation*}
		\E(\r, t) = \E(\r)e^{-i\omega t} \eqtext{and} \B(\r, t) = \B(\r)e^{-i\omega t}.
	\end{equation*}
	These ansatzes produce
	\begin{equation*}
		\laplacian \E = k^{2} \E \eqtext{and} \laplacian \B = k^{2} \B,
	\end{equation*}
	where $ k^{2} = -i \omega \mm \sigma_{\text{E}} $. The value of $ k $, using the identity $ \sqrt{i} = \frac{1-i}{\sqrt{2}} $, is
	\begin{equation*}
		k = \frac{\sqrt{2}}{2}(1 - i) \sqrt{\omega \mm \sigma_{\text{E}}}.
	\end{equation*}
	In terms of $ k $, in one dimension, the position-dependent solutions are
	\begin{equation*}
		\begin{Bmatrix}
            \E(z)\\
            \B(z)
		\end{Bmatrix}
        \sim e^{-kz} = \exp\left(- \frac{\sqrt{2}}{2}\sqrt{\omega \mm \sigma_{\text{E}}}\cdot z\right)\exp\left(i \frac{\sqrt{2}}{2}\sqrt{\omega \mm \sigma_{\text{E}}}\cdot z\right).
	\end{equation*}
	
\end{itemize}

\subsubsection{Derivation: The Skin Effect In a Cylindrical Conductor}
\begin{itemize}
	\item We now consider a long, straight wire, which we model as a cylinder of radius $ a $ with the longitudinal axis in the $ z $ direction. We use a cylindrical basis and the cylindrical coordinates $ (r, \phi, z) $. 
	
	\item The current runs in the $ z $ direction, so $ \j = (0, 0, j_{z}) $, which corresponds to an electric field $ \E = (0, 0, E_{z}) $. Because of rotational symmetry about the angle $ \phi $, the electric field $ E_{z} $ depends only on $ r $:
	\begin{equation*}
		E_{z}(\r, t) = E_{z}(r)e^{-i\omega t}.
	\end{equation*}
	The magnetic field points tangent to the cylinder's circular cross section, i.e. in the $ \uvec{e}_{\phi} $ direction in cylindrical coordinates. The expression for magnetic field is thus
	\begin{equation*}
		B_{\phi}(\r, t) = B_{\phi}(r)e^{-i\omega t}.
	\end{equation*}
	
    \item Next, recall the general diffusion equations for $ \E $ and $ \B $ are
	\begin{equation*}
		\laplacian \E = \mm \sigma_{\text{E}}\pdv{\E}{t} \eqtext{and} \laplacian \B = \mm \sigma_{\text{E}} \pdv{\B}{t}.
	\end{equation*}
	We evaluate the Laplacian operators in cylindrical coordinates to get
	\begin{align*}
		& \frac{1}{r}\pdv{r}\left(r\pdv{E_{z}}{r}\right)  = -i \mm \omega \sigma_{\text{E}}E_{z} \\
		& \frac{1}{r}\pdv{r}\left(r\pdv{B_{\phi}}{r}\right) - \frac{B_{\phi}}{r^{2}} = -i \mm \omega \sigma_{\text{E}}B_{\phi}.
	\end{align*}
	As before, we define $ k^{2} = -i\mm \omega \sigma_{\text{E}} $. 

    \item Next, we recall the Maxwell equation $ \curl \E = - \pdv{\B}{t} $. In our cylindrical basis and coordinate system, this equation implies $ \E $ and $ \B $ are related according to
	\begin{equation*}
		i \omega B_{\phi} = (\curl \E)_{\phi} = -\pdv{E_{z}}{r}.
	\end{equation*}
	If we solve for $ E_{z}(r) $, the solution turns out to be
	\begin{equation*}
		E_{z}(r) = A J_{0}(kr) \eqtext{and} B_{\phi} = i A \frac{k}{\omega} J_{1}(kr),
	\end{equation*}
	where $ A $ is an unknown constant, $ k = \frac{\sqrt{2}}{2}(1- i)\sqrt{\omega \mm \sigma_{\text{E}}} $,	and $ J_{0} $ and $ J_{1} $ are the Bessel functions.
	
	\item We will now find the electric current in the conductor. We start with current density, which we find with Ohm's law according to
	\begin{equation*}
		\j = \sigma_{\text{E}}\E = \sigma_{\text{E}}AJ_{0}(kr)\uvec{e}_{z}.
	\end{equation*}
	The total current through the conductor is thus
	\begin{equation*}
		I = \iint_{S} \j \cdot \diff \vec{S} = \iint_{S} (\j \cdot \uvec{n})\diff S = \sigma_{\text{E}} \int_{0}^{a}E_{z}(r)(2\pi r\diff r).
	\end{equation*}
	
	\item Next, without derivattion, we can write the electric field component $ E_{z} $ in cylindrical coordinates in the form
	\begin{equation*}
		E_{z}(r) \cdot r = \frac{1}{-i\mm \sigma_{\text{E}}\omega} \pdv{r}\left(r \pdv{E_{z}(r)}{r}\right).
	\end{equation*}
	We then subsitute this expression for $ E_{z} $ in the expression for $ I $, which eliminiates the integral and leaves us with the desired expression
	\begin{equation*}
		I = \frac{2\pi a}{-i\omega \mm}\pdv{E_{z}}{r}\bigg |_{0}^{a} = \frac{2\pi a}{\mm}B_{\phi}(a),
	\end{equation*}
	where the last equality follows from $ \curl \E = - \pdv{\B}{t} $ and $ \pdv{E_{z}}{r} = -i \omega B_{\phi}$.
	
\end{itemize}
    
\newpage
\section{Maxwell Equations}

\subsection{Maxwell Equations in Free Space and Charge Conservation}
\textit{State and derive the Maxwell equations in free space. Discuss the mathematical foundation of the Maxwell equations with reference to the Helmholtz decomposition theorem.}

\vspace{2mm}
\textit{Discuss the relationship between displacement current and conservation of charge.}

\begin{itemize}
    \item The Maxwell equations in free space read 
        \begin{align*}
            & \div \E = \frac{\rho}{\ee} && \div \B = 0\\
            & \curl \E = - \pdv{\B}{t} && \curl \B = \mm \j + \mm \ee \pdv{\E}{t}.
        \end{align*}

    \item Conservation of charge is encoded by the continuity equation, which reads
    \begin{equation*}
        \div \j + \pdv{\rho}{t} = 0.
    \end{equation*}

    \item Displacement current is needed in the $ \curl \B $ equation to preserve conservation of charge.

    \begin{quote}
        \textit{Derivation}: We take the divergence of the $ \curl \B $ equation and apply the vector calculus identity $ \div [\curl \B ] = 0 $ and Gauss's law $ \div \E = \frac{\rho}{\ee} $ to get
        \begin{align*}
            0 & \equiv \div [ \curl \B ] = \div \left[ \mm \j + \mm \ee \pdv{\E}{t} \right] = \mm \div \j + \ee \mm \pdv{t} (\div \E) \\
            & = \mm \div \j + \mm \pdv{\E}{t} = 0 \implies \div \j + \pdv{\rho}{t} = 0.
        \end{align*}
    \end{quote}
    
\end{itemize}

\subsubsection{The Maxwell Equations the Helmholtz Decomposition}

\begin{itemize}
	\item The Maxwell equations relate the fields $ \E(\r, t) $ and $ \B(\r, t) $ to their sources, which are charges for electric field and currents for magnetic field. More generally, these sources are analyzed in terms of charge density $ \rho(\r, t) $ and current density $ \j(\r, t) $.
	
	\item Note that the fields $ \E $ and $ \B $ are expressed in terms of their divergence and curl. The mathematical basis for this formulation is the Helmholtz decomposition theorem, which states that a vector field is uniquely determined by its divergence and curl, also discussed in \hyperref[ss:magnetic-poisson]{\underline{Subsection \ref{ss:magnetic-poisson}}} in the context of gauge transformations of the magnetic vector potential.
        
	\item Interesting interpretation: the Helmholtz theorem (i.e. mathematical theory) tells us that $ \E $ and $ \B $ are fully determined by the expressions
	\[
	\begin{array}{ll}
		\div \E = \text{something} & \quad \div \B = \text{something} \\
		\curl \E = \text{something} & \quad \curl \B = \text{something}.
	\end{array}
	\]
	It is then up to physics to determine what these ``something'' terms are.
	
\end{itemize}

\subsubsection{Derivation: Charge Conservation and the Continuity Equation}
\begin{itemize}
	\item We begin by considering a region of space $ V $ containing a charge density $ \rho(\r, t) $. In general, the region can exchange charge with its surroundings, and we encode the flow of charge in and out with current density $ \j $.
	
	\item The total charge $ q $ in the volume $ V $ is
	\begin{equation*}
		q(t) = \iiint_{V}\rho(\r, t)\dr,
	\end{equation*}
	while the change in charge over time is
	\begin{equation*}
		\dv{q}{t} = - \iint_{\partial V} \j(\r, t)\uvec{n}\diff S,
	\end{equation*}
	where $ \uvec{n} $ is the normal to the surface $ \partial V $---we assume $ \uvec{n} $ points out of the region $ V $. The charge's time derivative is then
	\begin{equation*}
		\dv{q}{t} = \pdv{t}\iiint_{V} \rho(\r, t)\dr = \iiint_{V} \pdv{t}\rho(\r, t)\dr \equiv I = - \iint_{\partial V} \j(\r, t)\diff \vec{S}.
	\end{equation*}

	\item We then rewrite the surface integral with the divergence theorem, which produces
	\begin{equation*}
		\pdv{\rho}{t} = - \div \j.
	\end{equation*}
	This important result is the continuity equation. It generalizes the electrostatic relationship $ \div \j = 0 $ to allow for conservation of charge in dynamic situations.
	
    The time dependence in the continuity equations allows for the possibility that, as long as $ \pdv{\rho}{t} $ is nonzero, we have $ \div \j \neq 0 $, meaning that current loops are not closed---they could simply end, and charge could accumulate at their ends. 
\end{itemize}


\subsubsection{Derivatino: Displacement Current}
\begin{itemize}
	\item First, we recall the (incomplete) quasistatic Maxwell equations, which read 
	\begin{align*}
		&\div \E = \frac{\rho}{\ee} && \div \B = 0\\
		& \curl \E = - \pdv{\B}{t} && \curl \B = \mm \j.
	\end{align*}
	The last equation, $ \curl \B = \mm \j $, is based on the assumption that all current loops are closed. This holds in the quasi-static regime, but violates the continuity equation for more general dynamic situations. To explicitly show this violation, we take the equation's divergence, which produces
	\begin{equation*}
        \mm \div \j = \div (\curl \B) = 0 \implies \div \j = 0 \stackrel{!}{\neq} - \pdv{\rho}{t},
	\end{equation*}
    where we have used the fact that the divergence of a curl is always zero.

    \item To satisfy the continuiuty equation, we generalize the $ \curl \B $ equation to include a displacement current $ \ee \mm \pdv{\E}{t} $, which produces
	\begin{equation*}
		\curl \B = \mm \j + \ee \mm \pdv{\E}{t}.
	\end{equation*}
	The displacement current is created by time-dependent electric fields. 

    Now that we have added the displacement current term, taking the divergence of the last Maxwell equation leads to
	\begin{equation*}
		\mm \div \j + \ee \mm \pdv{t}(\div \E) = 0 \implies \div \j = - \pdv{\rho}{t},
	\end{equation*}
    in agreement with the continuity equation.
	
\end{itemize}

\subsection{The Poynting Theorem}
\textit{Define and state the physical meaning of the Poynting vector and electromagnetic energy density.}

\vspace{2mm}
\textit{State, derive, and interpret the Poynting theorem for conservation of electromagnetic energy in free space.}

\begin{itemize}
    \item The Poynting vector, which I will denote with an upright $ \S $, is defined as
    \begin{equation*}
        \S = \frac{1}{\mm} \E \cross \B.
    \end{equation*}
    Physically, the Poynting vector has units of energy per unit time per unit area (i.e. intensity), and encodes the power transmitted by a electromagnetic waves per unit cross-sectional area.
    
    \item Electromagnetic energy density, denoted by $ w $, is defined as
    \begin{equation*}
        w = \frac{\ee}{2}\E^{2} + \frac{1}{2\mm} \B^{2},
    \end{equation*}
    and represents the energy per unit volume in an electromagnetic field.
    
    \item In free space, the Poynting theorem reads
    \begin{equation*}
        \pdv{w}{t} + \div \S + \j \cdot \E = 0.
    \end{equation*}
    In integral form, for a region of space $ V $, the Poynting theorem reads
    \begin{equation*}
        \pdv{W_{\text{em}}}{t} + \oiint_{\partial V} \S \cdot \diff \vec{S} + \iiint_{V} \j \cdot \E \dr = 0,
    \end{equation*}
    where $ W_{\text{em}} $ is the total electromagnetic energy in the region $ V $.

	Interpretation: in a given volume $ V $, electromagnetic energy changes either because of energy flow through the surface, encoded by $ \S $, or because of dissipative losses inside the volume, encoded by $ \j \cdot \E $. 
    
\end{itemize}


\textbf{Derivation: The Poynting Theorem}
\begin{itemize}
	\item First, we cross-multiply the third and fourth Maxwell equations, which produces
	\begin{align*}
		& \B \cdot (\curl \E) = - \B \cdot \pdv{\B}{t}\\
		& \E \cdot (\curl \B) = \mm \j \cdot \E + \mm \ee \E \cdot \pdv{\E}{t}.
	\end{align*}
	We then subtract the two equations to get
	\begin{equation*}
		\mm \ee \E \pdv{\E}{t} + \B \pdv{\B}{t} = \E\cdot(\curl \B) - \B\cdot(\curl \E) - \mm \j \E.
	\end{equation*}
	Next, we divide by $ \mm $ and rewrite the right hand side with a reverse-engineered ``divergence product rule'' to get
	\begin{equation*}
		\ee \E \pdv{\E}{t} + \frac{1}{\mm}\B \pdv{\B}{t} = - \frac{1}{\mm}\div(\E\cross\B) - \j \E.
	\end{equation*}

	We now turn our attention to the left hand side, and reverse-engineer the time derivative to get
	\begin{equation*}
		\pdv{t}\left(\frac{\ee}{2} \E^{2} + \frac{1}{2\mm}\B^{2}\right) = - \frac{1}{\mm}\div(\E\cross\B) - \j \E,
	\end{equation*}
	where the left hand side is exactly the electromagnetic field energy density $ w $, i.e.
	\begin{equation*}
		w = \frac{\ee}{2} \E^{2} + \frac{1}{2\mm}\B^{2}.
	\end{equation*}
	
	\item In terms of energy density $ w $, the energy continuiuty equation becomes
	\begin{equation*}
		\pdv{w}{t} + \div \S - \j \cdot \E,
	\end{equation*}
    where we have introduced the \Poy vector
	\begin{equation*}
		\S \equiv \frac{1}{\mm}\E \cross \B.
	\end{equation*}
	The \Poy vector corresponds to energy current density, i.e. power per unit cross-sectional area or simply intensity. Finally, we note that the term $ \j \cdot \E $ corresponds to Ohmic (dissipative) energy losses, which we showed in \hyperref[sss:ohmic-dissipation]{\underline{Subsubsection \ref{sss:ohmic-dissipation}}} when discussing the energy dissipation in the chapter on quasistatic electromagnetic fields.
	
	\item Putting the pieces together, conservation of electromagnetic energy is encoded in the equation
	\begin{equation*}
        \pdv{w}{t} + \div \S + \j \cdot \E = 0,
	\end{equation*}
    which is the differential form of the \Poy theorem. In integral form for a region of space $ V $, using the divergence theorem, the \Poy theorem reads
	\begin{equation*}
		\pdv{t}\iiint_{V}w \dr \equiv \pdv{W_{\text{em}}}{t} = - \oiint_{\partial V} \S \cdot \uvec{n} \diff \vec{S} - \iiint_{V} \j \cdot \E \dr,
	\end{equation*}
    where we have defined the total electric field energy
    \begin{equation*}
        W_{\text{em}} = \iiint_{V} w \dr.
    \end{equation*}
    
\end{itemize}

    
\subsection{Conservation of Electromagnetic Momentum}
\textit{State, derive and interpret the Cauchy continuity equation encoding conservation of electromagnetic momentum in both differential form.}

\begin{itemize}
    \item In differential form, in terms of vector and tensor components, the Cauchy continuity equation for conservation of electromagnetic momentum reads
    \begin{equation*}
        \pdv{p_{\text{em}_{i}}}{t} - \pdv{\TT_{ik}}{x_{k}} + f_{\text{em}_{i}} = 0,
    \end{equation*}
    where the three terms are:
    \begin{itemize}
        \item electromagnetic momentum $ \vec{p}_{\text{em}} = \ee (\E \cross \B) $,

        \item the electromagnetic stress tensor 
        \begin{equation*}
            \TT_{ik} = \ee \big(E_{i}E_{k} - \frac{1}{2}E^{2}\delta_{ik}\big) + \frac{1}{\mm}\left(B_{i}B_{k} + \frac{1}{2}B^{2}\delta_{ik}\right),
        \end{equation*}
        
        \item and the Lorentz force density $ \vec{f}_{\text{em}} = \rho \E + \j \cross \B $.
    \end{itemize}
    
    \item In integral form for a region of space $ V $ the Cauchy equation reads
    \begin{equation*}
        \pdv{t} \iiint_{V} p_{\text{em}_{i}} \dr = \oiint_{\partial V} \TT_{ik}\hat{n}_{k}\diff \vec{S} - \iiint_{V}f_{\text{em}_{i}}\dr.
    \end{equation*}
	Interpretation: electromagnetic momentum in a region of space $ V $ can change because of of momentum flux through the surface (encoded by $ \TT_{\text{ik}} $) and because of the Lorentz force acting on the entire volume (encoded by $ \vec{f}_{\text{em}} $). 
    
    
    
\end{itemize}

\subsubsection{Conservation of Momentum and Maxwell's Equations}
\begin{itemize}
	\item We begin by considering the time derivative
	\begin{equation*}
		\pdv{t}\ee (\E \cross \B).
	\end{equation*}
	Evaluating the derivative with the product rule produces
	\begin{equation*}
		\pdv{t}\ee (\E \cross \B) = \ee \left[\pdv{\E}{t}\cross \B + \E \cross \pdv{B}{t}\right].
	\end{equation*}
	We then substitute in the third and fourth Maxwell equation to replace the time derivatives, which leads to
	\begin{equation*}
		= \ee \left[\frac{1}{\ee \mm}(\curl \B)\cross \B - \frac{1}{ee}\j \cross \B - \E \cross (\curl \E)\right].
	\end{equation*}
	
    \item Next, some vector calculus acrobatics (quoted and not proven) leads to the expression
	\begin{equation*}
		\pdv{t}\big[\ee (\E \cross \B)\big] = \grad \left[\ee \E \otimes \E - \frac{1}{2}\ee \E^{2} + \frac{1}{\mm} \B \otimes \B - \frac{1}{2\mm}\B^{2}\right] - \big[\rho \E - \j \cross \B\big].
	\end{equation*}
	The large brackets contain the electromagnetic stress tensor. Meanwhile, the very last term is the electromagnetic (or Lorentz) force density. Note that in general (e.g. from basic mechanics) the time derivative of momentum is force. This implies the quantity $ \ee (\E \cross \B) $ on the left hand side of the equation is some form of momentum.
	
	\item Collected in one place, these terms are:
    \begin{itemize}
        \item electromagnetic momentum $ \vec{p}_{\text{em}} = \ee (\E \cross \B) $,

        \item the electromagnetic stress tensor 
        \begin{equation*}
            \TT_{ik} = \ee \big(E_{i}E_{k} - \frac{1}{2}E^{2}\delta_{ik}\big) + \frac{1}{\mm}\left(B_{i}B_{k} + \frac{1}{2}B^{2}\delta_{ik}\right),
        \end{equation*}
        
        \item and the Lorentz force density $ \vec{f}_{\text{em}} = \rho \E + \j \cross \B $.
    \end{itemize}
	In this notation, the complicated momentum relationship between $ \E $ and $ \B $ becomes
	\begin{equation*}
		\pdv{p_{\text{em}_{i}}}{t} - \pdv{\TT_{ik}}{x_{k}} + f_{\text{em}_{i}} = 0,
	\end{equation*}
	which is the Cauchy continuity equation for electromagnetic momentum.
	
	\item In integral form for a volume $ V $ (using the divergence theorem), the momentum continuity equation reads
	\begin{equation*}
		\pdv{t}\iiint_{V}p_{\text{em}_{i}}\dr = \oiint_{\partial V}\TT_{ik}\hat{n}_{k} \diff \vec{S} - \iiint_{V} f_{\text{em}_{i}}\dr.
	\end{equation*}
	
%	\item Note: we have just ascribed momentum to a mass-less field, as opposed to a body with mass as in introductory mechanics. Cool!
	
\end{itemize}


\newpage
\section{Electromagnetic Fields in Matter}

\subsection{Bound Charge, Polarization and Electric Susceptibility}
\textit{Define and discuss free and bound charge, polarization and electric susceptibility. State the definition of the $ \D $ field, discuss some of its important properties, and explain its role in the study of electric fields in matter.}

\subsubsection{Bound Charge}
\begin{itemize}
	\item Bound charge is immobile charge built into the crystal structure of matter, and arises from matter's constituent electrons and proteons. Free charge, as the name suggests, can move freely through space.

    Total charge is the sum of bound and free charge, and reads
    \begin{equation*}
        \rho = \rho_{\text{f}} + \rho_{b}.
    \end{equation*}
	The $ \rho $ in Maxwell's equations is total charge density.
	
	\item We define bound charge charge in a material containing point charges $ q_{i} $ via the sum
	\begin{equation*}
		\rho_{\text{b}} = \ev{\sum_{i}q_{i}\delta^{3}(\r - \r_{i})},
	\end{equation*}
    where sum runs over all charges $ q_{i} $ bound in the material's crystal lattice and the brackets denote an average calculated over ``hydrodynamic volume'', which allows for microscopic variation around the discrete charge positions in material and allows us to treat bound charge as a continuous quantity. 
	
	\item In terms of free and bound charge, the first Maxwell equation reads
	\begin{equation*}
		\div \E (\r, t) = \frac{\rho(\r, t)}{\ee}  = \frac{\rho_{\text{f}}(\r, t)}{\ee} + \frac{\rho_{\text{b}}(\r, t)}{\ee}.
	\end{equation*}
\end{itemize}

\subsubsection{Electric Polarization and the Displacement Field}
\begin{itemize}
	\item Bound charge and electric polarization in material are related according to
	\begin{equation*}
		\rho_{\text{b}} = - \div \P.
	\end{equation*}

    \item The $ \D $ field, sometimes called the electric displacement field, is defined as
    \begin{equation*}
        \D = \ee \E + \P.
    \end{equation*}
\end{itemize}

\textbf{Motivation: Definition of the $ \D $ Field}
\begin{itemize}
        \item We first substitute the identity $ \rho_{\text{b}} = - \div \P $ into the first Mawell equation, i.e.
    \begin{equation*}
		\div \E (\r, t) = \frac{\rho_{\text{f}}(\r, t)}{\ee} + \frac{\rho_{\text{b}}(\r, t)}{\ee}.
    \end{equation*}
    
    \item Next, we group the divergence terms together, which produces the relationship
	\begin{equation*}
		\div (\ee \E + \P) = \rho_{\text{f}},
	\end{equation*}
    and motivates the definition of the $ \D $ field as
	\begin{equation*}
		\D = \ee \E + \P \implies \div \D = \rho_{\text{f}}.
	\end{equation*}

\end{itemize}
	
\textbf{Example: Homogeneous Polarization}
\begin{itemize}
    \item In general, in a material with homogeneous polarization, bound charge occurs only along the boundaries.

	\item As an example, we consider homogeneous polarization $ \P = (0, 0, P_{0}) $ in a long and wide rectangular material whose height, which is much smaller than both the length and width, aligns with the $ z $ axis and varies from $ z = 0 $ to $ z = h $. The bound charge in the material is
	\begin{equation*}
		\rho_{\text{b}} = - \div \P = - \pdv{P_{z}}{z},
	\end{equation*}
    where we have used the material's long and wide geometry to approximate the divergence operator with $ \div \to \pdv{x} $. Since $ \P $ is homogeneous, the derivative is nonzero only at the boundary between material and free space, which leads to
	\begin{equation*}
		\rho_{\text{b}} = - \div \P = - \pdv{P_{z}}{z} = - P_{z}\delta(x) + P_{z}\delta(z - h).
	\end{equation*}
	In other words, the bound charge is confined to the planes at $ z = 0 $ and $ z = h $, which are the material's surface. 

\end{itemize}

\subsubsection{The Constitive Relation for Electric Field in Matter} \label{sss:const-rel-electric}
\begin{itemize}
    \item The constitutive relation is the relationship between polarization and the $ \D $ field in matter. In general, it is a non-linear relationship with the general form
	\begin{equation*}
		\P = \P(\D).
	\end{equation*}


    \item In isotropic, homogeneous matter, the constitutive relation is well-approximated by
	\begin{equation*}
		\P(\D) \approx \chi_{\text{E}} \D + \mathcal{O}(\D^{2}),
	\end{equation*}
    which is just an expansion of polarization to first order in $ \D $. The constant $ \chi_{\text{E}} $ is called electric susceptibility and is the proportionality constant between polarization and displacement field.

    \item Electric susceptibility is related to the dielectric constant $ \epsilon $ according to
	\begin{equation*}
        \chi_{\text{E}} = 1 - \frac{1}{\epsilon}.
	\end{equation*}
	
	\item In terms of $ \chi_{\text{E}} $ and $ \epsilon $, the linear constitutive relation reads
	\begin{equation*}
        \D = \ee \E + \P \approx \ee \E + \chi_{\text{E}} \D = \ee \E + \left( 1 - \frac{1}{\epsilon} \right) \D.
	\end{equation*}
	Thus, with a little algebra, the $ \E $ and $ \D $ fields are related to first order by
	\begin{equation*}
        \D = \epsilon \ee \E.
	\end{equation*}
    In terms of $ \D = \epsilon \ee \E $, the relationship between polarization and the $ \E $ field reads
	\begin{equation*}
        \P = \D - \ee \E \approx \epsilon \ee \E - \ee \E = \ee (\epsilon - 1)\E.
	\end{equation*}
    \textit{Important:} We stress that these relationships between $ \E $, $ \D $ and $ \P $ are not general. They hold only for an isotropic, homogeneous material in the regime of a first-order approximation of the constitutive relation $ \P = \P(\D) $.
	
\end{itemize}

    
\subsection{TODO: Electric Polarization and Electric Dipole Density}
\textit{Show that electric polarization equals the volume density of electric dipoles in matter.}
    
\subsection{Bound Current Density, Magnetization and Magnetic Susceptibility}
\textit{Define bound current density, magnetization and magnetic susceptibility. State the definition of the $ \H $ field, discuss some of its important properties, and explain its role in the study of magnetic fields in matter.}

\subsubsection{Bound Currents}
\begin{itemize}

    \item Bound current density, analogously to bound charge, is highly localized current confined to a material's crystal structure. Bound currents arise from microscopic magnetic dipoles in a material's atomic structure. 

    Free current density, as the name suggests, is free to move through space. Total current density is the sum of bound and free current density, and reads
    \begin{equation*}
        \j = \j_{\text{f}} + \j_{\text{b}}.
    \end{equation*}
    As for charge, the current density in Maxwell's equations is total current density.


	\item As for bound charge, bound current density $ \j_{\text{b}} $ in a material is defined as
	\begin{equation*}
		\j_{\text{b}}(\r, t) = \ev{\sum_{i}\j_{i}\delta^{3}(\r - \r_{i})},
	\end{equation*}
    where the sum runs over the positions $ \r_{i} $ of atoms or molecules in the material and the brackets denote a hydrodynamic average, which allows us to treat the bound current density as a continuous quantity (even though the microscopic currents are technically discrete on a quantum level).

\end{itemize}

\subsubsection{Magnetization and Magnetic Field Strength}
\begin{itemize}
    \item Magnetization, denoted by $ \M $, is defined by the implicit relationship
	\begin{equation*}
		\j_{\text{b}} = \curl \M + \pdv{\P}{t}.
	\end{equation*}
	Note that polarization and magnetization are coupled---which means that bound currents depend on the time derivative of $ \P $ as well as on magnetization. 
	
	\item \textit{Note}: The time derivative of $ \P $ is needed in the above relationship between bound current and magnetization to satisfy the continuity equation
    \begin{equation*}
        \div \j_{\text{b}} + \pdv{\rho_{\text{b}}}{t} = 0.
    \end{equation*}
    \begin{quote}
        \textit{Derivation}: We substitute the expression for $ \j_{\text{b}} $ in terms of $ \M $ and $ \P $ into the continuiuty equation to get
        \begin{equation*}
            \div \j_{\text{b}} + \pdv{\rho_{\text{b}}}{t} = \div (\curl \M) + \div \pdv{\P}{t} - \pdv{\div \P}{t} = 0.
        \end{equation*}
        Since the divergence of a curl quantity is zero, we have $ \div (\curl \M) = 0 $ and thus
        \begin{equation*}
            \div \j_{\text{b}} + \pdv{\rho_{\text{b}}}{t} = 0 + \div \pdv{\P}{t} - \pdv{\div \P}{t} = 0,
        \end{equation*}
        which satisfies the continuity equation $ \div \j_{\text{b}} + \pdv{\rho_{\text{b}}}{t} = 0 $. 
    \end{quote}
    
    \item The $ \H $ field, sometimes called magnetic field strength, is defined as
    \begin{equation*}
        \H = \frac{\B}{\mm} - \M \implies \B = \mm (\H + \M).
    \end{equation*}
    
\end{itemize}

\textbf{Motivation: Definition of the $ \H $ Field}
\begin{itemize}
    
    \item We begin by writing the third Maxwell equation in terms of $ \j_{\text{b}} $ and $ \j_{\text{b}} $ to get
    \begin{equation*}
        \curl \B = \mm \j_{\text{f}} + \mm \j_{\text{b}} + \mm \ee \pdv{\E}{t}.
    \end{equation*}
    We then divide through by $ \mm $ and substitute in $ \j_{\text{b}} $ in terms of $ \M $ and $ \P $ to get
    \begin{equation*}
        \frac{1}{\mm} \curl \B = \j_{\text{f}} + \curl \M + \pdv{\P}{t} + \ee \pdv{\E}{t}.
    \end{equation*}
    We then group derivative quantities together and substitute in $ \D = \ee \E + \P $ to get
    \begin{equation*}
        \curl \left( \frac{\B}{\mm} - \M \right) = \j_{\text{f}} + \pdv{t}\big[ \ee \E + \P \big] = \j_{\text{f}} + \pdv{\D}{t}.
    \end{equation*}
    
    \item The result reads
	\begin{equation*}
		\curl \left(\frac{\B}{\mm} - \M\right) = \j_{\text{f}} + \pdv{\D}{t},
	\end{equation*}
    which motivates the definition of magnetic field strength $ \H $ as
	\begin{equation*}
		\H = \frac{\B}{\mm} - \M.
	\end{equation*}
	Note that $ \H $ arises only from free currents and free charge, analogously to how the $ \D $ field arises from only free charge. In terms of $ \H $, the Maxwell equation reads
	\begin{equation*}
		\curl \H = \j_{\text{f}} + \pdv{\D}{t}.
	\end{equation*}
\end{itemize}

\textbf{Example: Homogeneous Magnetization}
\begin{itemize}
	\item  In general, the bound currents in a material with homogeneous magnetization occur only on the material's lateral surface enclosing the axis of magnetization. 

	\item As an example, consider a cuboid of base $ a \cross b $ and height $ h $ with homogeneous magnetization $ \M = (0, 0, M_{0}) $ in the $ z $ direction. 

    The bound current corresponding to this homogeneous magnetization is
	\begin{equation*}
		\j_{\text{b}} = \curl \M + \pdv{\P}{t} = \curl \M + 0 = \left(\pdv{M_{z}}{y}, - \pdv{M_{z}}{x}, 0\right).
	\end{equation*}
	The derivatives are nonzero only at the borders, which produces
	\begin{equation*}
		\j_{\text{b}} = \cdots = \big[M_{0}\delta(y) - M_{0}\delta(y-b), -M_{0}\delta(x) + M_{0}\delta(x - a), 0\big].
	\end{equation*}
	The surface currents shadow the magnetic field in the material, analogously to how surface charge shadows electric field in conductors.

\end{itemize}

\subsubsection{The Constitive Relation for Magnetic Field in Matter} \label{sss:const-rel-magnetic}
\begin{itemize}

    \item The constitutive relation is the relationship between magnetization and the $ \H $ field in matter. In general, it is a non-linear relationship with the general form
	\begin{equation*}
		\M = \M(\H).
	\end{equation*}


    \item In isotropic, homogeneous matter, the constitutive relation is well-approximated by
	\begin{equation*}
		\M(\H) \approx \chi_{\text{M}} \H + \mathcal{O}(\H^{2}),
	\end{equation*}
    which is just an expansion of magnetization to first order in $ \H $. The constant $ \chi_{\text{M}} $ is called magnetic susceptibility and is the proportionality constant between magnetization and the $ \H $ field.

    \item Magnetic susceptibility is related to the relative permeability $ \mu $ according to
	\begin{equation*}
        \chi_{\text{M}} = \mu - 1.
	\end{equation*}
    For most materials $ \chi_{\text{M}} $ is close to zero and $ \mu $ is close to one. Magnetic permeability $ \mu $ is the magnetic analog of electric permittivity $ \epsilon $, i.e. the dielectric constant.
	
	\item In terms of $ \chi_{\text{M}} $ and $ \mu $, the linear constitutive relation reads
	\begin{equation*}
        \H = \frac{\B}{\mm} - \M \approx \frac{\B}{\mm} - \chi_{\text{M}} \H = \frac{\B}{\mm} - (\m - 1)\H
	\end{equation*}
    Thus, to first, order, the $ \B $ and $ \H $ fields are related by
	\begin{equation*}
        \H = \frac{\B}{\mu \mm} \implies \B = \mu \mm \H.
	\end{equation*}
    In terms of $ \B = \mu \mm \H $, the relationship between magnetization and the $ \B $ field reads
	\begin{equation*}
        \M = \left( 1 - \frac{1}{\mu} \right)\frac{\B}{\mm}.
	\end{equation*}
    \textit{Important:} as for electric field, these relationships between $ \B $, $ \H $ and $ \M $ are not general. They hold only for an isotropic, homogeneous material in the regime of a first-order approximation to the constitutive relation $ \M = \M(\H) $.

\end{itemize}

    
\subsection{TODO: Magnetization and Magnetic Dipole Density}
\textit{Show that magnetization equals the volume density of the magnetic dipole moment in matter, and interpret the result in terms of Ampere equivalence.}
    
\subsection{Maxwell Equations in Matter and the Constitutive Relations}
\textit{State Maxwell's equations in matter. State and discuss the constitutive relations for the electric and magnetic fields in the linear regime.}

\begin{itemize}
	\item In matter, the Maxwell equations are written
	\begin{align*}
		& \div \D = \rho_{\text{f}} && \div \B = 0\\
		& \curl \E = - \pdv{\B}{t} &&  \curl \H = \j_{\text{f}} + \pdv{\D}{t}.
	\end{align*}
	
    \item To complete the description of electomagnetism in matter, we must also consider the constitutive relationships
    \begin{equation*}
        \P = \P(\D) \qquad \text{and} \qquad \M = \M(\H).
    \end{equation*}
    
    \item In isotropic, homogeneous materials, we can expand the constitutive relations to first order in $ \D $ and $ \H $ to produce the relationships
    \begin{equation*}
        \D = \e \ee \E \qquad \text{and} \qquad \B = \mu \mm \H
    \end{equation*}
    between the field flux densitities ($ \D $ and $ \B $) and their strengths ($ \E $ and $ \H $).
	In anisotropic materials, $ \e $ and $ \mu $  generalize to rank-two tensors.

    \vspace{2mm}
    \textit{Note}: The constitutive relationships are discussed in Subsubsections \ref{sss:const-rel-electric} and \ref{sss:const-rel-magnetic}.
	
\end{itemize}

\textbf{Derivation: First Maxwell Equation in Matter}
\begin{itemize}
    \item We first substitute the identity $ \rho_{\text{b}} = - \div \P $ into the first Mawell equation, i.e.
    \begin{equation*}
        \div \E (\r, t) = \frac{\rho_{\text{f}}(\r, t)}{\ee} + \frac{\rho_{\text{b}}(\r, t)}{\ee}.
    \end{equation*}
    We then group the divergence terms together, which produces the relationship
    \begin{equation*}
        \div (\ee \E + \P) = \rho_{\text{f}}.
    \end{equation*}
    In terms of the $ \D $ field, i.e. $ \D = \ee \E + \P $, this becomes
    \begin{equation*}
        \div \D = \rho_{\text{f}}.
    \end{equation*}
\end{itemize}

\textbf{Derivation: Third Magnetic Equation in Matter}
\begin{itemize}
    \item We begin by writing the third Maxwell equation in terms of $ \j_{\text{b}} $ and $ \j_{\text{b}} $ to get
    \begin{equation*}
        \curl \B = \mm \j_{\text{f}} + \mm \j_{\text{b}} + \mm \ee \pdv{\E}{t}.
    \end{equation*}
    We then divide through by $ \mm $ and substitute in $ \j_{\text{b}} = \curl \M + \pdv{\P}{t} $ to get
    \begin{equation*}
        \frac{1}{\mm} \curl \B = \j_{\text{f}} + \curl \M + \pdv{\P}{t} + \ee \pdv{\E}{t}.
    \end{equation*}
    We then group derivative quantities together and substitute in $ \D = \ee \E + \P $ to get
    \begin{equation*}
        \curl \left( \frac{\B}{\mm} - \M \right) = \j_{\text{f}} + \pdv{t}\big[ \ee \E + \P \big] = \j_{\text{f}} + \pdv{\D}{t}.
    \end{equation*}
    In terms of the $ \H $ field, i.e $ \H = \frac{\B}{\mm} - \M $, the above equation becomes
    \begin{equation*}
        \curl \H = \j_{\text{f}} + \pdv{\D}{t}.
    \end{equation*}
\end{itemize}

    
\subsection{Electromagnetic Energy in Matter}
\textit{How does the Poynting theorem for conservation of electromagnetic energy generalize to matter? How do electric and magnetic field energies change in matter compared to free space?}

\subsubsection{Conservation of Energy}
\begin{itemize}
    \item The Poynting theorem in matter is the same as in free space, i.e.
	\begin{equation*}
		\pdv{w}{t} + \div \S  + \j \cdot \E = 0,
	\end{equation*}
    with the following two changes:
    \begin{enumerate}

        \item The Poynting vector is defined as $ \S = \E \cross \H $.

        \item The definition of electromagnetic energy density $ w $ is generalized to
        \begin{equation*}
            w = \int_{0}^{\D} \E(\D') \diff \D' + \int_{0}^{\B}\H(\B')\diff \B',
        \end{equation*}
        For a linear constitutive relation i.e. $ \E \propto \D $ and $ \H \propto \B $, then $ w $ simplifies to an expression involving quadratic forms, just like for $ w $ in free space.

    \end{enumerate}


    \item The change in electric field energy in a region of space $ V $ permeated with an electric field $ \E_{0} $ that is then filled with an electrically active material is
	\begin{align*}
		W - W_{0} = -\frac{1}{2}\iiint_{V}\P \cdot \E_{0}\dr,
	\end{align*}
    where $ \P $ is the electric polarization induced in the material and $ W_{0} $ and $ W $ are the electric field energies before and after filling the region with the polarized material, respectively. This result holds only for a linear constitutive relation.
	
    \item The change in magnetic field energy in a region of space $ V $ permeated with a magnetic field $ \B_{0} $ that is then filled with a magnetically active material is
	\begin{align*}
		W - W_{0} = \frac{1}{2}\iiint_{V}\M \cdot \B_{0}\dr.
	\end{align*}
    where $ \M $ is the magnetization induced in the material and $ W_{0} $ and $ W $ are the magnetic field energies before and after filling the region with the magnetized material, respectively. As before, this result holds only for a linear constitutive relation.

\end{itemize}

\subsubsection{Derivation: Change in Electric Field Energy in Matter}
\begin{itemize}
	
	\item Using $ W $ to denote electric field energy and the subscript zero to denote vacuum, we integrate the generalized expression for electric energy density $ w $ in matter to get
	\begin{equation*}
        W - W_{0} = \iiint_{V}(w - w_{0})\dr = \iiint_{V} \left[\int \E(\D) \diff \D - \int \E_{0}(\D_{0}) \diff \D_{0} \right] \dr.
	\end{equation*}
	We then assume a linear constitutive relation, which implies
	\begin{equation*}
		\E \propto \D \implies \int \E(\D) \diff \D \propto \frac{\D^{2}}{2}.
	\end{equation*}

    \item Next, we substitute the relation $ \int \E(\D) \diff \D \propto \frac{\D^{2}}{2} $ into the energy difference, move the factor $ 1/2 $ otuside the integral, and rewrite $ \D $ in terms of $ \E $ to get
	\begin{equation*}
		W - W_{0} = \frac{1}{2}\iiint_{V} \E \cdot \D \dr - \frac{1}{2}\iiint_{V}\E_{0}\cdot \D_{0}\dr.
	\end{equation*}
	Next, a trick: using reverse-engineered multiplication, we rewrite the above expression in the algebraically equivalent form
	\begin{equation*}
		W - W_{0} = \frac{1}{2}\iiint_{V}(\E \cdot \D_{0} - \E_{0}\cdot\D)\dr + \frac{1}{2}\iiint_{V}(\E + \E_{0})\cdot (\D - \D_{0})\dr.
	\end{equation*}
    It turns out the second term is zero, which we prove in the following bullet points.

    \item To show the aforementioned second term is zero, we first consider the sum $ \E + \E_{0} $, which we rewrite in terms of electric potential as $ \E + \E_{0} = - \grad \phi $. Using this expression for $ \E $ and some vector calculus acrobatics, the second term becomes
	\begin{align*}
        \mathcal{I} & \equiv \frac{1}{2}\iiint_{V}(\E + \E_{0})\cdot (\D - \D_{0})\dr = \frac{1}{2}\iiint_{V} (- \grad \phi) \cdot (\D - \D_{0}) \dr \\
        & = -\frac{1}{2}\iiint_{V}\Big\{ \div \big[ \phi \cdot (\D - \D_{0}) \big] - \div (\D - \D_{0})\phi \Big\}\dr.
	\end{align*}
    Next, recall that the $ \D $ field arises from free charge via $ \div \D = \rho_{\text{f}} $. Since placing a dielectric material into a region of space changes only bound charge and not free charge, the free charge in the region of space is the same before and after filling it with dielectric material, i.e. $ \rho_{\text{f}} = \rho_{\text{f}_{0}}$, which implies $ \div (\D - \D_{0}) = 0 $ and thus
    \begin{equation*}
        \mathcal{I} = -\frac{1}{2}\iiint_{V}\div (\phi \cdot (\D - \D_{0}))\dr.
    \end{equation*}
	The remaining term is a divergence integrated over the region of space $ V $ containing the material, which we rewrite with the divergence theorem to get
	\begin{equation*}
        \mathcal{I} = -\frac{1}{2}\oiint_{\partial V} \phi \cdot (\D - \D_{0})\diff S.
	\end{equation*}
    But along the region's surface $ \D = \D_{0} $, so we are integrating $ \phi \cdot 0 $ over the surface, which is zero. The result is the quoted expression $ \mathcal{I} = 0 $.
	
    \item The difference in electric field energies, using the just-derived result $ \mathcal{I} = 0 $ and the linear constitutive identity $ \D = \epsilon \ee \E $  and $ \D_{0} = \ee \E_{0} $ to relate $ \E $ and $ \D $, is
	\begin{align*}
		W - W_{0}  &= \frac{1}{2}\iiint_{V}(\E \cdot \D_{0} - \E_{0}\cdot\D)\dr + 0\\
		& = - \frac{1}{2}\iiint_{V} \ee(\epsilon - 1)\E_{0}\E \dr\\
		& = -\frac{1}{2}\iiint_{V}\P \cdot \E_{0}\dr,
	\end{align*}
	where $ \P $ is the electric polarization in the dielectric material and $ \E_{0} $ is the electric field that would be present in the region of space in the absence of the material.
	
\end{itemize}

\subsubsection{Magnetic Field Energy in Matter}
\begin{itemize}
	\item The derivation is analogous to the above derivation for electric field energy. We assume a linear constitutive relation for the magnetic field, and use $ W $ to denote magnetic field energy, and write the energy difference $ W - W_{0} $ using an integral of energy density difference $ w - w_{0} $. This reads
	\begin{align*}
        W - W_{0} &= \frac{1}{2}\iiint_{V}(w - w_{0}) \dr = \frac{1}{2} \iiint_{V} \H \cdot \B  \dr - \frac{1}{2}\iiint_{V}\H_{0}\cdot\B_{0}\dr\\
		& = \frac{1}{2}\iiint_{V}\big(\B \H_{0} - \B_{0}\H\big)\dr + \frac{1}{2}\iiint_{V}(\B + \B_{0})\cdot(\H - \H_{0}) \dr.
	\end{align*}
    As for electric energy, the second integral comes out to zero, as derived below.

    \item To show the second integral is zero, we use $ \B + \B_{0} = \curl \A $ and some vector calculus acrobatics to rewrite the second integral in the form
    \begin{align*}
        \mathcal{I} & \equiv \frac{1}{2}\iiint_{V}(\B + \B_{0})\cdot(\H - \H_{0}) \dr = \frac{1}{2}\iiint_{V}\curl \A \cdot (\H - \H_{0}) \dr\\
        & = \frac{1}{2} \iiint_{V} \Big\{ \div \big[ \A \cross (\H - \H_{0}) \big] + \A \cdot \big[ \curl (\H - \H_{0}) \big] \Big \} \dr
    \end{align*}
    If we assume $ \pdv{\D}{t} = 0 $ and that free currents in the region of space do not change after adding the magnetically active material (just like we assumed free charge didn't change when adding a dielectric material), we have $ \curl (\H - \H_{0}) = \j_{\text{f}} - \j_{\text{f}_{0}} = 0 $. 

    The above integral $ \mathcal{I} $ then reduces to
    \begin{equation*}
        \mathcal{I} = \frac{1}{2} \iiint_{V} \Big\{ \div \big[ \A \cross (\H - \H_{0}) \big] \Big \} \dr = \frac{1}{2} \oiint_{\partial V} \big[ \A \cross (\H - \H_{0}) \big] \dr,
    \end{equation*}
    where we have written the second equality with the divergence theorem. Since $ \H = \H_{0} $ on the region's surface $ \partial V $, we have $ \mathcal{I} = 0 $.

    \item The difference in magnetic field energies, using the just-derived result $ \mathcal{I} = 0 $ and the linear constitutive identity $ \B = \mu \mm \H $  and $ \H_{0} = \mu_{0} \H_{0} $ to relate $ \H $ and $ \B $, is
	\begin{align*}
		W - W_{0} &= \frac{1}{2}\iiint_{V}\big(\B \H_{0} - \B_{0}\H\big)\dr \\
        & = \frac{1}{2} \iiint_{V}\mm (\mu - 1)\H_{0}\H\dr\\
		& = \frac{1}{2}\iiint_{V}\M \cdot \B_{0}\dr.
	\end{align*}
	Lesson: the change in magnetic field energy that results from placing a magnetically active material in empty space permeated by the magnetic field $ \B_{0} $ is proportional to the original field $ \B_{0} $ and the magnetization $ \M $ induced in the magnetically active material. Keep in mind that this result holds only for a linear constitutive relation.
\end{itemize}

\subsection{Electrostatic Force on a Nonhomogeneous Material}
\textit{Derive an expression for the electrostatic force on a nonhomogeneous material, and discuss the limiting cases of a dielectric material in the quasistatic regime.}
\begin{itemize}
	\item In a nonhomogeneous material, $ \e = \e(\r) $ and $ \mu = \mu(\r) $ vary throughout the material. If we move the material within the field, the value of $ \e $ and $ \mu $ change relative to the field $ \E $ and $ \B $, so the electromagnetic force on the material changes. 

    \item Under certain assumptions, listed below, the electrostatic force on a nonhomogeneous material with dielectric constant $ \epsilon = \epsilon(\r) $ is approximately given by
	\begin{equation*}
		\vec{F}_{\text{E}} = - \frac{1}{2}\iiint_{V}(\grad \e(\r))\ee E^{2}\dr.
	\end{equation*}
	In other words, $ \vec{F} $ at a given point depends on the gradient of $ \e $. 

    The above expression holds under the following two assumptions:
    \begin{enumerate}
        \item We assume that $ \div \D \approx 0 $, which corresponds to working in the absence of appreciable free charge (recall $ \div \D \equiv \rho_{\text{f}} $). The approximation $ \div \D \approx 0 $ often gives satisfactory results, but is of course not valid in general.
        
        \item We also assume $ \curl \E = 0 $, which corresponds to working in a system without induction, i.e. in the absence of a time-varying magnetic field (recall the relationship $ \curl \E = - \pdv{\B}{t} $).
    \end{enumerate}

\end{itemize}


\textbf{Derivation: Electrostatic Force on an Nonhomogeneous Dielectric}
\begin{itemize}

    \item Assuming we can assign the field an electrostatic stress tensor, the electric force on the material can be written in terms of the tensor as
	\begin{equation*}
		F_{\text{E}_{i}} = \iiint_{V}\pdv{\TT_{ik}}{x_{k}} \dr.
	\end{equation*}
	 Using the linear constitutive relation $ \D = \e \ee \E $, the tensor's components are
	\begin{align*}
        \pdv{\TT_{ik}}{x_{k}} &= \pdv{x_{k}} \left[ E_{i}D_{k} - \frac{1}{2}(\E \cdot \D) \delta_{ik}\right]\\
        &= E_{i}\pdv{D_{k}}{x_{k}} + E_{k}\pdv{D_{i}}{x_{k}} - \frac{1}{2}\pdv{\e}{x_{i}}\ee E^{2} - \frac{1}{2}\e \ee \pdv{E^{2}}{x_{i}},
	\end{align*}
    where we note that $ \e = \e(\r) $ is dependent on position.
	
    \item Next, we rewrite the term $ \pdv{E^{2}}{x_{i}} $ using the vector calculus identity
	\begin{equation*}
		\frac{1}{2}\grad E^{2} = \E \cross (\curl \E) + (\E \cdot \grad)\E,
	\end{equation*}
	which leads to a vector expression for electric force:
	\begin{align*}
		\vec{F}_{\text{E}} = &\iiint_{V}\left[\E (\div \D) + (\E \cdot \grad )\D - \D \cross (\curl \E) - (\E \cdot \grad )\D\right] \dr\\
		& - \frac{1}{2}\iiint_{V}(\grad \e(\r))\ee E^{2}\dr.
	\end{align*}
	
	\item Next, we consider a simplified case of the above result. We make two assumptions:
    \begin{enumerate}
        \item We assume the $ \div \D \approx 0 $, which corresponds to working in the absence of appreciable free charge (recall $ \div \D \equiv \rho_{\text{f}} $). The approximation $ \div \D \approx 0 $ often gives satisfactory results, but is of course not valid in general.
        
        \item We also assume $ \curl \E = 0 $, which corresponds to working in a system without induction, i.e. in the absence of a time-varying magnetic field (recall the relationship $ \curl \E = - \pdv{\B}{t} $).
    \end{enumerate}
    Under these assumptions, the first integral vanishes and the force on the nonhomogeneous dielectric simplifies to the desired expression
	\begin{equation*}
		\vec{F}_{\text{E}} = - \frac{1}{2}\iiint_{V}(\grad \e(\r))\ee E^{2}\dr,
	\end{equation*}
    as quoted at the begining of the subsection.
	
\end{itemize}


\subsection{Boundary Conditions for the Maxwell Equations}
\textit{State and derive the boundary conditions for the Maxwell equations along the boundary between two materials with different electromagnetic properties.}
    
\begin{itemize}

    \item Consider a boundary between two materials with different electromagnetic properties, i.e. different values of $ \epsilon $ and $ \mu $, where we assign each material a surface normal vector $ \uvec{n} $ pointing \textit{into} the material. 


    \item Along the boundary between the two regions:
    \begin{itemize}

        \item The $ \B $ field obeys the condition
        \begin{equation*}
            \B_{1} \cdot \uvec{n}_{1} + \B_{2} \cdot \uvec{n}_{2} = 0 \implies B_{1}^{\perp} - B_{2}^{\perp} = 0,
        \end{equation*}
        where $ B^{\perp} $ denotes the magnetic field component perpendicular to the surface.
        
        \item The $ \D $ field obeys the condition
            \begin{equation*}
                \D_{1} \cdot \uvec{n}_{1} + \D_{2} \cdot \uvec{n}_{2} = \sigma_{\text{f}} \implies D_{1}^{\perp} - D_{2}^{\perp} = \sigma_{\text{f}}.
	\end{equation*}
        where $ D^{\perp} $ denotes the magnetic field component perpendicular to the surface and $ \sigma_{\text{f}} $ is the free surface charge density along the boundary.

        \item The $ \E $ field obeys the condition
        \begin{equation*}
            E_{1}^{\parallel} - E_{2}^{\parallel} = 0,
        \end{equation*}
        where $ E^{\parallel} $ denotes the component of the electric field parallel to the boundary.

        Alternatively, in terms of the normal to the surface, the boundary condition is
        \begin{equation*}
            (\uvec{n}_{1} \cross \E_{1}) - (\uvec{n}_{2}\cross \E_{2}) = 0.
        \end{equation*}
        This expression is useful, since the normal to the material interface is always well-defined, while the tangent is sometimes ambiguous.
    
        \item The $ \H $ field obeys the condition
        \begin{equation*}
            H_{1}^{\parallel} - H_{1}^{\parallel} = k,
        \end{equation*}
        where $ k $ is the magnitude of the surface current density. Alternatively, in terms of the normal vectors to the materials, the boundary condition reads
        \begin{equation*}
            (\uvec{n}_{1} \cross \H_{1}) - (\uvec{n}_{2}\cross \H_{2}) = \vec{k}, \qquad \uvec{k} = k \uvec{e},
        \end{equation*}
        where $ \uvec{e} $ is normal to the vectors $ \uvec{n}_{1} $ and $ \uvec{n}_{2} $.
    \end{itemize} 

\end{itemize}



\subsubsection{Derivation: Boundary Condition for the \textit{B} Field}
\begin{itemize}
	\item Let the magnetic field in the first and second materials be $ \B_{1} $ and $ \B_{2} $, respectively. We begin with the Maxwell equation
	\begin{equation*}
        \div \B = 0 \implies \iiint_{V} \div \B \dr = 0,
	\end{equation*}
	which must hold for any region of space $ V $. 
	
	
    \item We then consider a cylinder of infinitesimal height $ \diff \vec{l} $ just enclosing the boundary between the two materials (a so-called Gaussian pillbox). We integrate over the cylindrical volume to get
	\begin{align*}
        0 & \equiv \iiint_{V} \div \B \dr = \oiint_{\partial V} \B \cdot \diff \vec{S}  \\
        & = \iint_{S_{1}}\B_{1}\cdot \uvec{n}_{1}\diff S + \iint_{S_{2}}\B_{2}\cdot \uvec{n}_{2}\diff S + \iint_{S_{\text{lat}}}\B_{3}\cdot \uvec{n}\diff S,
	\end{align*}
	where $ S_{\text{lat}} $ is the cylinder's lateral surface area. 

    \item We send the cylinder height $ \diff \vec{l} $ to zero, since we're interested only in the boundary between the two materials. The third integral over the lateral surface area vanishes in the limit $ \diff \vec{l} \to 0 $, leaving
	\begin{equation*}
		\iint_{S_{1}}\B_{1}\cdot \uvec{n}_{1}\diff S + \iint_{S_{2}}\B_{2}\cdot \uvec{n}_{2}\diff S = 0.
	\end{equation*}
    At the boundary, the $ \B $ field thus obeys the relationship
	\begin{equation*}
        \B_{1} \cdot \uvec{n}_{1} + \B_{2} \cdot \uvec{n}_{2} = 0 \implies B_{1}^{\perp} - B_{2}^{\perp} = 0,
	\end{equation*}
    where $ B^{\perp} $ denotes the magnetic field component perpendicular to the surface. Note the change in sign of the component expression relative to the vector expression, which occurs because the normal vector $ \uvec{n}_{1} $ points in the opposite direction as $ \uvec{n}_{2} $.
	
	
\end{itemize}

\subsubsection{Derivation: Boundary Condition for the \textit{D} Field}
\begin{itemize}
	\item We consider the same interface between two materials as before and let $ \D_{1} $ and $ \D_{2} $ denote the $ \D $ fields in the first and second material, respectively. We now base our analysis on the Maxwell equation
	\begin{equation*}
		\div \D = \rho_{\text{f}}.
	\end{equation*}
	
	\item As before, we consider a cylinder with infinitesimal height $ \diff \uvec{l} $ enclosing the boundary between the two materials. In integral form, the Maxwell equation reads
	\begin{equation*}
		\iiint_{V} \div \D \dr = \iiint_{V}\rho_{\text{f}}\dr.
	\end{equation*}
	We use the divergence theorem to write the left-hand integral as an integral over the cylinder surface, which gives
    \begin{equation*}
        \iiint_{V} \rho_{\text{f}} \dr = \oiint_{\partial V} \D \cdot \diff \vec{S} = \iint_{S_{1}}\D_{1} \cdot \uvec{n}_{1}\diff S +  \iint_{S_{2}}\D_{2} \cdot \uvec{n}_{2}\diff S +  \iint_{S_{\text{lat}}}\D_{3} \cdot \uvec{n}\diff S,
    \end{equation*}
	where $ S_{\text{lat}} $ again denotes the cylinder's lateral surface area. 
    
    \item We then send the cylinder height to zero, in which case the integral $ \iiint_{V}\rho_{\text{f}}\dr $ becomes a surface integral over the cylinder's cross-sectional area and the volume charge density $ \rho_{\text{f}} $ approaches the surface charge density $ \sigma_{\text{f}} $, leaving us with
	\begin{equation*}
		 \iint_{S_{1}}\D_{1} \cdot \uvec{n}_{1}\diff S +  \iint_{S_{2}}\D_{2} \cdot \uvec{n}_{2}\diff S = \iint_{S}\sigma_{\text{f}} \diff S.
	\end{equation*}
    Along the boundary between the materials, the $ \D $ field must thus obey
	\begin{equation*}
        \D_{1} \cdot \uvec{n}_{1} + \D_{2} \cdot \uvec{n}_{2} = \sigma_{\text{f}} \implies D_{1}^{\perp} - D_{2}^{\perp} = \sigma_{\text{f}}.
	\end{equation*}
	
\end{itemize}

\subsubsection{Derivation: Boundary Condition for the \textit{E} Field}
\begin{itemize}
	\item We now consider the boundary between two materials with electric fields $ \E_{1} $ and $ \E_{2} $, respectively. This time around, we use the Maxwell equation
	\begin{equation*}
		\curl \E = - \pdv{\B}{t}.
	\end{equation*}
	We then integrate this equation over the boundary and apply Stoke's theorem to get
	\begin{equation*}
        - \pdv{t}\iint_{S}\B \cdot \diff \vec{S} = \iint_{S} \curl \E \cdot \diff \vec{S} = \oint_{\partial S} \E \cdot \diff \vec{s},
	\end{equation*}
    where $ S $ is a rectangular surface of length $ l $ and height $ h $ whose boundary $ \partial S $ encloses the interface between the two materials. The loop's length runs parallel to the interface while the height is perpendicular to the interface, so that decreasing the height $ \diff h $ causes the loop to approach the interface between the materials. The tangents to the loop (along its lengths) are $ \uvec{t}_{1} = - \uvec{t}_{2} $ in the two regions, respectively. 
    
    \item Next, we split the closed line integral over the rectangular surface's boundary $ \partial S $ into integrals over the loop's length and height to get
    \begin{equation*}
        - \pdv{t}\iint_{S}\B \cdot \diff \vec{S} = \int_{l_{1}}\E \cdot \diff \vec{l} + \int_{l_{2}}\E \cdot \diff \vec{l} + \int_{h_{1}}\E \cdot \diff \vec{h} + \int_{h_{2}}\E \cdot \diff \vec{h}.
    \end{equation*}
    
    \item We then send the surface's height to zero, i.e. $ \diff h \to 0 $, in which case the integrals over height vanish, leaving us with
    \begin{align*}
        - \pdv{t}\iint_{S}\B \cdot \diff \vec{S} &= \int_{l_{1}}\E \cdot \diff \vec{l} + \int_{l_{2}}\E \cdot \diff \vec{l} + 0 + 0\\
        & = \int_{l} (\E_{1}\cdot \uvec{t}_{1} + \E_{2}\cdot \uvec{t}_{2}) \diff l,
    \end{align*}
    where we have written the line elements in terms of the tangent vectors $ \uvec{t}_{1} $ and $ \uvec{t}_{2} $. However, we can write the surface element $ \diff S $ in the form $ \diff S = \diff l \diff h $, which shows that $ \diff S $ also vanishes\footnote{The fact that the surface integral vanishes should make intuitive sense; as $ \diff h \to 0 $ the rectangle approaches a line and there is no surface left to integrate over!} in the limit $ \diff h \to 0 $, leaving us with
    \begin{equation*}
        \int_{l} (\E_{1}\cdot \uvec{t}_{1} + \E_{2}\cdot \uvec{t}_{2}) \diff l.
    \end{equation*}
    It follows that at the boundary between the two materials, the $ \E $ field must obey
    \begin{equation*}
        E_{1}^{\parallel} - E_{2}^{\parallel} = 0,
    \end{equation*}
    where $ E^{\parallel} $ denotes the component of the electric field parallel to the boundary between the two materials. Alternatively, in terms of the normal to the surface, we can write the above condition as
	\begin{equation*}
		(\uvec{n}_{1} \cross \E_{1}) - (\uvec{n}_{2}\cross \E_{2}) = 0.
	\end{equation*}
	This expression is useful, since the normal to the material interface is always well-defined, while the tangent is more ambiguous.

\end{itemize}

\subsubsection{Derivation: Boundary Condition for the \textit{H} Field}
\begin{itemize}
	\item We now consider an interface between two materials with $ \H $ fields $ \H_{1} $ and $ \H_{2} $, and base our analysis on the Maxwell equation
	\begin{equation*}
		\curl \H = \j + \pdv{D}{t}.
	\end{equation*}
	In an analogous procedure as for $ \E $, we introduce a small planar surface $ S $ in the interface between the two materials with height $ h $ perpendicular to the interface and length $ l $, defined by the tangent vectors $ \uvec{t}_{1} $ and $ \uvec{t}_{2} $, parallel to the interface.
	
	\item Using Stokes' theorem, the integral of the above Maxwell equation is
    \begin{equation*}
        \iint_{S} \j \cdot \diff \vec{S} + \pdv{t} \iint_{S} \D \cdot \diff \vec{S} = \iint_{S} \curl \H \cdot \diff \vec{S} = \oint_{\partial S} \H \cdot \diff \vec{s}.
	\end{equation*}

    \item We then perform a geometrically identical procedure as in the analysis of the $ \E $ field boundary condition, in which we send the loop height to zero, i.e. $ \diff h \to 0 $ and thus $ \diff S = \diff l \diff h \to 0 $. In this case, the $ \D $-dependent term vanishes, while the line integral over $ \partial S $ keeps only the portions over the loop's length, leaving us with
	\begin{equation*}
        \iint_{S} \j \cdot \diff \vec{S} = \iint_{S} \j \cdot \uvec{e} \diff S = \int (\H \cdot \uvec{t}_{1} + \H_{2}\cdot \uvec{t}_{2}) \diff l.
	\end{equation*}
    
    \item Note that we have kept the surface integral containing $ \j $. The integrand contains $ \j \cdot \uvec{e} $, i.e. the component of $ \j $ parallel to the normal of the Stokes' theorem-derived integration surface $ \uvec{S} $. Keep in mind that $ \uvec{e} $ is normal to the vectors $ \uvec{n}_{1} $ and $ \uvec{n}_{2} $, which are the normal vectors to the interface between the two materials.

    \item As $ \diff h \to 0 $, the current density $ \j $ aligns with $ \uvec{n}_{S} $, and represents a ``surface current density'' with units $ \si{\ampere \, \meter^{-1}} $, which we will denote $ \vec{k} $; the direction of $ \vec{k} $ is determined by the right hand rule in the direction of the tangent vectors $ \uvec{t}_{1} $ and $ \uvec{t}_{1} $ round the loop. In terms of $ \vec{k} $ in the limit $ \diff h \to 0 $, the integral of $ \j $ over the surface becomes
    \begin{equation*}
        \iint_{S} \j \cdot \uvec{e} \diff S \to \int \vec{k} \cdot \diff \vec{l},
    \end{equation*}
    which produces the relationship
    \begin{equation*}
        \int \vec{k} \cdot \diff \vec{l} = \int (\H \cdot \uvec{t}_{1} + \H_{2}\cdot \uvec{t}_{2}) \diff l.
    \end{equation*}
    Thus, at the interface between the two materials, the $ \H $ field obeys the condition 
	\begin{equation*}
        H_{1}^{\parallel} - H_{1}^{\parallel} = k,
	\end{equation*}
    where $ k $ is the magnitude of the surface current density. Alternatively, in terms of the normal vectors to the materials, the boundary condition reads
	\begin{equation*}
        (\uvec{n}_{1} \cross \H_{1}) - (\uvec{n}_{2}\cross \H_{2}) = \vec{k}, \qquad \uvec{k} = k \uvec{e}.
	\end{equation*}
\end{itemize}


\newpage
\section{Frequency Dependence of the Dielectric Constant}

\subsection{Relationship Between the Dielectric Function and Polarization}
\textit{State and derive the relationship between electric field, polarization, and the dielectric function in the frequency domain. Discuss the physical significance of the dielectric function's real and imaginary components in the frequency domain.}

\begin{itemize}
        \item The real component $ \Re(\epsilon) $ is associated with reflection and refraction and encodes the classical index of refraction, as in the context of the law of refraction. Meanwhile, the imaginary component $ \Im(\epsilon) $ encodes the absorption of electromagnetic waves in the material---a large imaginary component corresponds to high absorption. 

     \item The polarization and electric field in a material whose bound charge arises from discrete particles with mass $ m $, charge $ q $ and volume density $ n $ is
        \begin{equation*}
            \P(\omega) = \frac{q^{2}n}{m} \frac{\E(\omega)}{\omega_{0}^{2} - \omega^{2} - i \gamma \omega},
        \end{equation*}
        where $ \omega $ is the electric field frequency, the constants $ \omega_{0} $ and $ \gamma $ depend, respectively, on the nature of the potential binding the bound charges and the dissipative forces acting on the charges should they move from their equilbrium positions.

        \item In the same material, the dielectric constant $ \epsilon $ depends on frequency as
        \begin{equation*}
            \ee \big[ \epsilon(\omega) -1 \big] = \frac{q^{2}n}{m} \frac{1}{\omega_{0}^{2} - \omega^{2} - i\gamma\omega}. 
        \end{equation*}
        Both relationships hold only for a linear constitutive relation $ \D = \epsilon \epsilon_{0} \E $.
    
\end{itemize}

\textbf{Derivation: Frequency Dependence of the Dielectric Constant}
\begin{itemize}

    \item We begin by assuming a linear constitutive relation between $ \E $ and $ \D $. In both the time and frequency domains, the $ \E $ and $ \D $ fields are related by
    \begin{equation*}
        \D(\r, t) = \epsilon(t)\epsilon_{0} \E(\r, t) \qquad \text{and} \qquad \D(\r, \omega) = \epsilon(\omega)\epsilon_{0}\E(\r, \omega).
    \end{equation*}
    Note that $ \epsilon(\omega) $ is a complex number under the Fourier transform.

    \item We then assume a basic classical equation of motion for a bound charge $ q $ in a crystal lattice exposed to a time-dependent electric field $ \E(t) $. This reads:
    \begin{equation*}
        m \dv[2]{\r}{t} = - m \omega_{0}^{2}\r - m\gamma \dot{\r} + q\E(t).
    \end{equation*}
    The term $ - m \omega_{0}^{2} \r $ assumes the charge is bound within the dielectric's crystal lattice by a harmonic potential with harmonic frequency $ \omega_{0} $, the second term encodes a velocity-dependent dissipative force, parameterized by $ \gamma $, if the particle moves from its lattice position. The final term is the electric force on a charge.

    \item We then take the Fourier transform of the equation of motion. The time derivatives in Fourier space simplify to multiplication by $ i\omega $, and the result is
    \begin{align*}
        -m\omega^{2}\r(\omega) &= - m\omega_{0}^{2}\r(\omega) + i\omega\gamma\r(\omega) + q\E(\omega),\\
        \r(\omega) &= \frac{q}{m} \frac{\E(\omega)}{(\omega_{0}^{2} - \omega^{2}) - i \gamma\omega}.
    \end{align*}

    \item We now want to derive an expression for $ \epsilon $, and begin by introducing polarization
    \begin{equation*}
        \P(\omega) = nq\r(\omega),
    \end{equation*}
    where $ n $ is the volume charge density of electric dipoles contributing to the polarization. We then substitute the above expression for $ \r(\omega) $ into the expression for polarization, which gives
    \begin{equation*}
        \P(\omega) = \frac{q^{2}n}{m} \frac{\E(\omega)}{\omega_{0}^{2} - \omega^{2} - i \gamma \omega}.
    \end{equation*}
    Finally, we use $ \P(\omega) = \ee \big[ \epsilon(\omega)-1 \big]\E(\omega) $ to get the desired relationship for $ \epsilon $, i.e.
    \begin{equation*}
        \ee \big[ \epsilon(\omega) -1 \big] = \frac{q^{2}n}{m} \frac{1}{\omega_{0}^{2} - \omega^{2} - i\gamma\omega}. 
    \end{equation*}
    
    \item We often work with the above expression for $ \epsilon $ in one of three regimes, called
    \begin{itemize}
        \item Debye relaxation,

        \item Lorentz relaxation,

        \item and plasma relaxation.
    \end{itemize}
    Since electromagnetic frequencies vary over many orders of magnitude, it makes sense that we might have different dissipative and binding mechanism at different frequencies. We note also that in real materials, we often have multiple sources of harmonic-like potentials, which results in multiple eigenfrequencies $ \omega_{0_{i}} $ and multiple modes of dissipation (multiple $ \gamma_{i} $).

\end{itemize}


\subsection{Models of Dielectric Relaxations}
\textit{Discuss the Debye, Lorentz and plasma models of dielectric relaxation and derive and sketch the associated expressions for the real and imaginary components of the dielectric function as a function of electric field frequency. Provide an example of a real-world material's dielectric function.}

\begin{itemize}
    \item Each of the following dielectric relaxation models is based on the relationships
    \begin{equation*}
        \P(\omega) = \frac{q^{2}n}{m} \frac{\E(\omega)}{\omega_{0}^{2} - \omega^{2} - i \gamma \omega} \qquad \text{and} \qquad \ee \big[ \epsilon(\omega) -1 \big] = \frac{q^{2}n}{m} \frac{1}{\omega_{0}^{2} - \omega^{2} - i\gamma\omega},
    \end{equation*}
    which are derived in the section above.

\end{itemize}

\subsubsection{Debye Relaxation}
\begin{itemize}
    \item Debye relaxation is relevant at low frequencies, e.g. in the range $ \omega \sim $ \SIrange{e7}{e9}{\hertz}, and is commonly used to model the relaxation of electric dipoles in matter.

    At low frequencies we can neglect the term proportional to $ \omega^{2} $, in which case the expression for polarization simplifies to
    \begin{equation*}
        \P(\omega) = \frac{q^{2}n}{m\omega_{0}^{2}} \frac{\E(\omega)}{1 - i\omega\tau},
    \end{equation*}
    where the Debye relaxation time $ \tau $ is defined as $ \tau = \frac{\gamma}{\omega_{0}} $.
    
    \item Using this expression for polarization, the dielectric constant in the Debye regime is
    \begin{equation*}
        \epsilon(\omega) = 1 + \frac{(\epsilon(0) -1)(1 + i\omega\tau)}{1 + \omega^{2}\tau^{2}},
    \end{equation*}
    where $ \epsilon(0) $ is the dielectric constant at $ \omega = 0 $ and we have assumed the relationship
    \begin{equation*}
        \ee \big[ \epsilon(0) -1 \big] = \frac{q^{2}n}{m\omega_{0}^{2}}.
    \end{equation*}
    The real component of $ \epsilon $ decays to 0 at large frequencies in a sort of half-sigmoid curve. The imaginary component grows to a maximum and then falls to zero at large frequencies.
\end{itemize}

\subsubsection{Lorentz Relaxation}
\begin{itemize}
    \item The Lorentz regime is not an approximation at all---it accounts for all frequency terms in the general expression for $ \epsilon(\omega) $. Lorentz relaxation applies to frequencies in the range \SIrange{e12}{e16}{\hertz}, which includes the range of visible light, and is used, for example, to model the oscillation of molecules. 

    \item In the regime of Lorentz relaxation, polarization retains the general expression
    \begin{equation*}
        \P(\omega) = \frac{q^{2}n}{m} \frac{\E(\omega)}{(\omega_{0}^{2} - \omega^{2}) - i \gamma \omega},
    \end{equation*}
    while the corresponding dielectric function is given by the relationship
    \begin{equation*}
        \ee \big[ \epsilon(\omega) - 1 \big] = \frac{q^{2}n}{m}\frac{1}{\omega_{0}^{2} - \omega^{2} - i \gamma\omega}.
    \end{equation*}
    In terms of the dielectric constant at $ \omega = 0 $, the dielectric function reads
    \begin{equation*}
         \epsilon(\omega) = 1 + \frac{ \big[ \epsilon(0) - 1 \big]\omega_{0}^{2}}{\omega_{0}^{2} - \omega^{2} - i \gamma\omega}.
    \end{equation*}
    The real component has two extrema and a node at the resonance frequency; in general form, it resembles the first excited quantum state of a finite potential well, although the functional dependence is of course different.

    The imaginary component has a single extrema at the resonance frequency, and decays to zero at $ \omega \to \pm \infty $. 

\end{itemize}

\subsubsection{Plasma Relaxation}
\begin{itemize}
    \item Plasma relaxation applies to very high frequencies, where we can neglect all $ \omega $-dependent terms in the equation of motion except the accelerating electric force.

    \item In the plasma regime, the polarization reads
    \begin{equation*}
        \P(\omega) = - \frac{q^{2}n}{m} \frac{\E(\omega)}{\omega^{2}},
    \end{equation*}
    and the corresponding dielectric constant is
    \begin{equation*}
        \epsilon(\omega) = 1 - \frac{\omega_{\text{p}}^{2}}{\omega^{2}} \qquad \text{where} \qquad \omega_{\text{p}} = \frac{nq^{2}}{m\epsilon_{0}}. 
    \end{equation*}
    
    \item The plasma model corresponds to electrons behaving as free particles in material, and begins to apply at frequencies approximately greater than \SI{e16}{\hertz}.
\end{itemize}
    
\subsubsection{Example: Dielectric Function in Water}
\begin{itemize}
    \item As mentioned in the introduction to this section, real materials have multiple dissipative and binding mechanics, and thus multiple eigenfrequencies $ \omega_{0_{i}} $ that contribute differently at different electric field frequencies. The resulting dielectric function is quite complex, as seen in the following example of water. 

    \item We model water with a single Debye relaxation term for low electric field frequencies and fully eleven Lorentz relaxation terms to encode various binding mechanisms at larger frequencies. The result is the phenomenolgical relationship
    \begin{equation*}
        \epsilon(i\omega) = 1 + \frac{d}{1 + \omega\tau} + \sum_{i = 1}^{11} \frac{f_{i}}{\omega_{0_{i}}^{2} + g_{i}\omega + \omega^{2}},
    \end{equation*}
    where $ d $, $ \tau $, $ \omega_{0_{i}} $, $ g_{i} $, and $ f_{i} $ are phenomenologically-determined constants found with measurements and fitting.

    The first term uses the Debye model and the further 11 terms use the Lorentz model---note that we need 12 different modes in total to describe the frequency dependence of water's dielectric function.
    
\end{itemize}

    
\subsection{The Kramers-Kronig Relations}
\textit{State and discuss the Kramers-Kronig relations for the dielectric function.}
\begin{itemize}
    \item The Kramers-Kronig relations relate the real and imaginary components of the dielectric function $ \epsilon $. In other words, if one knows one component of $ \epsilon $, one can then use the Kramers-Kronig relations to find the other component.

    \item Without derivation, the Kramers-Kronig relations read:
    \begin{align*}
        & \Re(\epsilon) = 1 + \frac{2}{\pi} \mathcal{P} \int_{0}^{\infty} \frac{\omega' \Im \big[ \ln(\epsilon(\omega')) \big]}{\omega'^{2} - \omega^{2}} \diff \omega'\\
        & \Im(\epsilon) = - \frac{2\omega}{\pi} \mathcal{P} \int_{0}^{\infty} \frac{\omega' \Re \big[ \ln(\epsilon(\omega'))  \big]- 1}{\omega'^{2} - \omega^{2}} \diff \omega'.
    \end{align*}
    Note that the real and imaginary components of $ \epsilon $ are coupled, and that each component is given in terms of an integral of the other component.

    The symbol $ \mathcal{P} $ denotes the integral's Cauchy principle value and is defined as 
    \begin{equation*}
        \mathcal{P}\int_{-\infty}^{\infty} \frac{g(\omega')}{\omega' - \omega} \diff \omega' = \lim_{\epsilon \to 0} \left[ \int_{-\infty}^{\omega-\epsilon} \frac{g(\omega')}{\omega'-\omega}\diff \omega' + \int_{\omega+\epsilon}^{\infty} \frac{g(\omega')}{\omega'-\omega}\diff \omega' \right].
    \end{equation*}
    Put simply, the principle value is a way to avoid the pole in the left-hand integral at $ \omega' = \omega $. We note in passing that the Kramers-Kronig relations are derived from the definition of the Cauchy principle value, followed by Hilbert transforms, followed by the Plemelj equations, but the derivation is beyond the scope of this course.

\end{itemize}

    
\subsection{The Dielectric Function and Dissipation of Electric Energy}
\textit{State and derive the expression for the dissipation of electric field energy in matter in terms of the dielectric function in the frequency domain.}

\begin{itemize}

    \item As long as a material has nonzero $ \Im(\epsilon) $, then electromagnetic energy dissipates when passing through the material.

    \item The change in electric field energy associated with an electric field $ \E $ in a material with dielectric function $ \epsilon(\omega) $ between two distance times $ t_{1,2} \to \pm \infty $ is
    \begin{equation*}
        \Delta W_{\text{E}} = \frac{\ee}{2\pi} \int \omega \Im \epsilon(\omega) \diff \omega \iiint_{V} \big |\E(\r, \omega)\big |^{2} \dr.
    \end{equation*}

    Interpretation: If $ \Im(\epsilon) \neq 0 $, then electromagnetic energy in a material dissipates with time, and the dissipation is proportional to $ \Im(\epsilon) $.

    \item We will consider only the electric component of electromagnetic energy density. Neglecting the magnetic component is generally acceptable because $ \epsilon $ tends to vary with frequency across multiple order of magnitude, while $ \mu $ tends to be of order one. As a result, only $ \epsilon $ and thus electric energy vary appreciably with frequency.

\end{itemize}

\textbf{Derivation: The Dielectric Function and Energy Dissipation}
\begin{itemize}

    % \item We begin our analysis by writing the complex number $ \epsilon $ in terms of amplitude and phase, in which case the relationship $ \D = \epsilon \ee \E $ implies that a nonzero imaginary component $ \Im \epsilon \neq 0 $ results in the $ \E $ and $ \D $ fields falling out of phase in matter.

    \item Assuming a linear constitutive relation, the EM power density in material is
    \begin{equation*}
        \pdv{w}{t} = \E \pdv{\D}{t}.
    \end{equation*}
    We then integrate the above expression to get the change in energy over time:
    \begin{equation*}
        w(2) - w(1) = \int_{t_{1}}^{t_{2}} \pdv{w}{t}\diff t = \int_{t_{1}}^{t_{2}} \E \pdv{\D}{t} \diff t.
    \end{equation*}
    
    \item Next, we Fourier-transform $ \E $ and $ \D $ to the frequency domains to get
    \begin{equation*}
        \E(t) = \frac{1}{2\pi} \int \E(\omega) e^{-i\omega t} \diff \omega \qquad \text{and} \qquad \D(t) = \frac{1}{2\pi} \int \D(\omega) e^{-i\omega t} \diff \omega.
    \end{equation*}
    We then substitute these transformations into the energy difference $ w(2) - w(1) $ and evaluate the time derivative of $ \D $, which produces
    \begin{align*}
        w(2) - w(1) = \frac{1}{(2\pi)^{2}}\int_{t_{1}}^{t_{2}} \left[ \int \E(\omega)e^{-i\omega t} \diff \omega \cdot \int -i \omega' \D(\omega')e^{-i\omega't} \diff \omega' \right] \diff t.
    \end{align*}
    Next, we substitute in the linear constitutive relation
    \begin{equation*}
        \D(\omega') = \epsilon(\omega')\ee \E(\omega'),
    \end{equation*}
    which leads to
    \begin{equation*}
        w(2) - w(1) = \frac{\ee}{(2\pi)^{2}} \int \E(\omega) \diff \omega \int  (-i\omega')\epsilon(\omega')\E(\omega') \diff \omega' \int_{t_{1}}^{t_{2}} e^{-i(\omega + \omega')t} \diff t.
    \end{equation*}
    
    \item Next, we send the integration limits of the time integral to $ t_{1, 2} \to \pm \infty $, in which case the time integral becomes the delta function $ 2\pi \delta(\omega + \omega') $, leaving
    \begin{equation*}
        w(2) - w(1) = \frac{\ee}{2\pi} \int \E(\omega) \diff \omega \int  (-i\omega')\epsilon(\omega')\E(\omega') \diff \omega' \delta(\omega + \omega').
    \end{equation*}
    Sketched derivation from here forward: the presence of the delta function means the frequency integrals give non-zerp contributions only when $ \omega = \omega' $. Combining this fact with the complex number identity $ i\epsilon(-\omega) - i \epsilon(\omega) = 2\Im \epsilon(\omega)$ leads to
    \begin{equation*}
        w(2) - w(1) = \frac{\ee}{2\pi} \int \omega \Im \epsilon(\omega) \big |\E(\omega)\big |^{2}  \diff \omega.
    \end{equation*}
    Finally, we integrate over volume to get electric energy from electric energy density, and write the $ \E $ field's position dependence explicitly to get
    \begin{equation*}
        W(2) - W(1) = \frac{\ee}{2\pi} \int \omega \Im \epsilon(\omega) \diff \omega \iiint_{V} \big |\E(\r, \omega)\big |^{2} \dr.
    \end{equation*}
    This is the desired end result for the change in electromagnetic energy as it passes through material over the course of two distance times $ t_{1,2} \to \pm \infty $.

\end{itemize}
    


\newpage
\section{Hamiltonian Formalism for the Electromagnetic Field}

\subsection{The Lagrangian of a Charge Particle in an EM Field} \label{ss:lagrangian}
\textit{State and derive the Lagrange function for a charged particle in an electromagnetic field.}

\begin{itemize}
    \item The Lagrangian function for a particle of charge $ q $ in an electromagnetic field is
    \begin{equation*}
        L(\r, \dot{\vec{r}}, t) = \frac{1}{2}m \dot{\vec{r}}^{2} - q \phi(\r(t)) + q \dot{\vec{r}} \cdot \A(\r, t).
    \end{equation*}
    Interpretation: The Lagrangian $ L $ has three contributions: kinetic energy, coupling of $ q $ and $ \phi $, and coupling of $ q $, $ \dot{\vec{r}} $ and $ \A $. Note also that the Lagrangian is written in terms of field potentials and not the fields themselves.
    
\end{itemize}

\textbf{Derivation: Largrangian of a Charged Particle in an EM Field}
\begin{itemize}
    \item We consider a particle of charge $ q $ in an external electric field $ \E(\r, t) $ and magnetic field $ \B(\r, t) $. The particle experiences the Lorentz force
    \begin{equation*}
        \vec{F} = q(\E + \vec{v}\cross \B).
    \end{equation*}
    We then substitute the Lorentz force into Newton's second law to get
    \begin{equation*}
        m \ddot{\vec{r}} = q\E + q \vec{v} \cross \B = - q\grad \phi - q \pdv{\A}{t} + q \vec{v}\cross \big( \curl \A \big),
    \end{equation*}
    where we have written the electric and magnetic fields in terms of the potentials $ \phi $ and $ \A $ in the second equality.

    \item Next, we apply the curl vector identity
    \begin{equation*}
        \vec{v}\cross \big( \curl \A \big) = \grad ( \vec{v}\cdot \A) - ( \vec{v}\cdot \grad)\A,
    \end{equation*}
    in terms of which Newton's second law becomes
    \begin{equation*}
        m \ddot{\vec{r}} = - q \grad \phi - q \pdv{\A}{t} + q \grad ( \vec{v}\cdot \A) - q( \vec{v} \cdot \grad)\A.
    \end{equation*}
    Finally, we join the time and directional derivative of $ \A $ into a material derivative, i.e. $ q \pdv{\A}{t} + q(\vec{v}\cdot \grad)\A = q \dv{\A}{t}$, which simplifies the above equation to
    \begin{equation*}
        m \ddot{\vec{r}} = -q \grad \phi + q \grad ( \vec{v} \cdot \A) - q \dv{\A}{t}.
    \end{equation*}
    
    \item We will now rewrite the above equation of motion into a form matching the Euler-Lagrange equation, which will reveal the Lagrange function. The first step is
    \begin{equation*}
        \dv{t} \left( m \dot{\vec{r}} + q\A \right) = - \grad ( q \phi - q \vec{v}\cdot \A ) \equiv - \pdv{\r} ( q \phi - q \dot{\vec{r}} \cdot \A ).
    \end{equation*}
    We then further reverse-engineer the time derivative on the left-hand side to get
    \begin{equation*}
        \dv{t} \left[ \pdv{ \dot{\vec{r}}} \left( \frac{1}{2} m \dot{\vec{r}}^{2} + q \A \cdot \dot{\vec{r}} \right) \right] = - \pdv{\r} ( q \phi - q \dot{\vec{r}} \cdot \A ).
    \end{equation*}
    Comparing this expression to the Lagrange equation $ \dv{t} \left(\pdv{L}{\dot{\r}} \right) = \pdv{L}{\r} $ motivates the following definition for the Lagrangian of a particle in an EM field:
    \begin{equation*}
        L(\r(t), \dot{\vec{r}}(t), t) \equiv \frac{1}{2}m \dot{\vec{r}}^{2} - q \phi(\r(t)) + q \dot{\vec{r}} \cdot \A(\r(t), t).
    \end{equation*}
    Note that even though the $ q\phi $ term doesn't explicitly appear in the left-hand side of the Lagrange equation and the $ \frac{1}{2}m \dot{\vec{r}}^{2} $ term doesn't explicitly appear in the right-hand side, these terms would vanish under the derivatives $ \pdv{\dot{\vec{r}}} $ and $ \pdv{\r} $, respectively, which ends up satisfying the Lagrange equations.

    
\end{itemize}

    
\subsection{The Hamiltonian of a Charge Particle in an EM Field} \label{ss:hamilton}
\textit{State and derive the Hamiltonian function for a charged particle in an electromagnetic field.}

\begin{itemize}
    \item The Hamiltonian function for a charged particle in an electromagnetic field is
    \begin{equation*}
        H = \frac{\big[\vec{p}(t) - q \A(\r, t)\big]^{2}}{2m} + q \phi,
    \end{equation*}
    where $ q $ and $ m $ are the particle's charge and mass, respectively. The quantity $ \vec{p} $ is the particle's \textit{canonical} momentum, and is equal to $ \vec{p} = m \dot{\vec{r}} + q \A $.
    
\end{itemize}

\textbf{Derivation: Hamiltonian of a Charged Particle in an EM Field}
\begin{itemize}
    \item We find the Hamiltonian $ H $ for a charged particle in an EM field using the Lagrangian function $ L $ (derived in \hyperref[ss:lagrangian]{\underline{Subsection \ref{ss:lagrangian}}}) together with the general relationship
    \begin{equation*}
        H( \vec{p}, \vec{r}, t) = \dot{\vec{r}} \dot{\vec{p}} - L.
    \end{equation*}
    
    \item First, the appropriate canonical momentum $ \vec{p} $ is
    \begin{equation*}
        \vec{p} = \pdv{L}{ \dot{\vec{r}}} = m \dot{\vec{r}} + q \A,
    \end{equation*}
    while the corresponding time derivative of position is
    \begin{equation*}
        \dot{\vec{r}} = \frac{\vec{p} - q\A}{m}.
    \end{equation*}
    Note that the quantity $ q \A $ corresponds to momentum in the Hamiltonian formalism. 

    \item In terms of $ \vec{p} $, $ \dot{\vec{r}} $ and $ L $, the charged particle's Hamiltonian is then
    \begin{equation*}
        H = \frac{1}{m} ( \vec{p} - q\A) \vec{p} - \frac{1}{2m} ( \vec{p} - q\A)^{2} - q\A \left( \frac{ \vec{p} - q \A}{m} \right) + q \phi.
    \end{equation*}
    Finally, we combine like terms to get the desired expression
    \begin{equation*}
        H = \frac{1}{2m}( \vec{p}(t) - q\A (\r, t))^{2} + q \phi(\r, t).
    \end{equation*}
    
\end{itemize}
    

\subsection{Lagrangian Density and the Riemann-Lorenz Equations} \label{ss:lagrange-density}
\textit{State and derive the complete Lagrangian density function of a continuous charge distribution in an external electromagnetic field. Discuss the associated action, Euler-Lagrange equations, and the resulting Riemann-Lorenz equations.}

\begin{itemize}

    \item The total Lagrangian density $ \L_{\text{EM}} $ of a charge distribution $ \rho(\r, t) $ in an external magnetic field---encoding both the external field and the field arising from the charge distribution itself---is
	\begin{equation*}
		\L_{\text{EM}}(\r, t) = \frac{1}{2}\ee E^{2}(\r, t) - \frac{1}{2\mm}B^{2}(\r, t) - \rho(\r, t)\phi(\r, t) + \j(\r, t) \cdot \A(\r, t).
	\end{equation*}

	\item The action $ S $ associated with the above total electromagnetic Lagrangian is 
	\begin{equation*}
        S \equiv \int L(\r, t) \diff t = \int \left[ \iiint_{V} \L_{\text{EM}}(\phi(\r, t), \A(\r, t))\dr \right] \diff t.
	\end{equation*}
	
    \item The Euler-Lagrange equations for the above action read
	\begin{align*}
		&\dv{t}\left[\pdv{\L}{\left(\pdv{\phi}{t}\right)}\right] + \grad \left[\pdv{\L}{(\grad \phi)}\right] - \pdv{\L}{\phi} = 0\\
		& \dv{t} \left[\pdv{\L}{\left(\pdv{A_{i}}{t}\right)}\right] + \grad \left[\pdv{\L}{(\grad A_{i})}\right] - \pdv{\L}{A_{i}} = 0,
	\end{align*}
    where we have written $ \L_{\text{EM}} \to \L $ for conciseness. 
	
	\item Without proof, the above Euler-Lagrange equations lead to the Riemann-Lorenz equations
	\begin{align*}
		\laplacian \phi - \frac{1}{c^{2}}\pdv[2]{\phi}{t} = -\frac{\rho}{\ee} \eqtext{and} \laplacian A_{i} - \frac{1}{c^{2}} \pdv[2]{A_{i}}{t} = - \mm j_{i}.
	\end{align*}
	These two equations encode the fully time-dependent form of the Maxwell equations in terms of the fundamental field potentials $ \phi $ and $ \A $.
	

\end{itemize}

\textbf{Derivation: Total Lagrangian Density of a Charge Distribution, Part 1}
\begin{itemize}
	
	\item Recall that for a single a particle, the Lagrangian reads
	\begin{equation*}
		L = \frac{1}{2}m\bdot{r}^{2} - q \phi(\r, t) + q \bdot{r}\cdot \A(\r, t) \equiv \frac{1}{2}m \dot{\vec{r}}^{2} + L_{\text{ext}},
	\end{equation*}
    where we have used $ L_{\text{ext}} = - q \phi(\r, t) + q \bdot{r}\cdot \A(\r, t) $ to represent the coupling of the particle to the external electromagnetic field---we'll ignore the kinetic term $ \frac{1}{2}m \dot{\vec{r}}^{2} $. 

    \item We then generalize the field-coupled Lagrangian to a charge distribution $ \rho(\r, t) $ via
	\begin{equation*}
		L_{\text{ext}} = - \iiint_{V}\rho(\r, t) \phi(\r, t) \dr + \iiint_{V}\j(\r, t) \cdot \A(\r, t)\dr,
	\end{equation*}
    where we stress that $ L_{\text{ext}} $ encodes only the external EM field-coupled portion of the total Lagrangian $ L $ (and leaves out the kinetic energy term $ \frac{1}{2}m \dot{\vec{r}}^{2} $).
    
    Next, we introduce the Lagrangian density $ \L_{\text{ext}} $, defined according to
	\begin{equation*}
		\L_{\text{ext}} = - \rho(\r, t) \phi(\r, t) + \j(\r, t)\cdot \A(\r, t).
	\end{equation*}
    Note that $ \L_{\text{ext}} $ encodes coupling with an \textit{external} field---it does not account for electromagnetic field of the charge distribution itself (analogous to the considerations of external versus total electromagnetic field energy in Subsections \ref{ss:total-E-energy} and \ref{ss:total-M-energy}).
	
\end{itemize}

\textbf{Derivation Part 2: Lagrangian Function for the Total EM Field}
\begin{itemize}
	\item We now aim to derive a Lagrangian encoding the ``total'' field---both the external field and the field arising from the charge distribution itself. Interpretation: we consider the charge distribution as a field source in itself, and then allow for the distribution to occur in an additional external field.
	
	Motivatived by this separation into internal and external field terms, we write the total electromagnetic Lagrangian in the form
	\begin{equation*}
		L_{\text{EM}} = \iiint_{V} \L_{\text{EM}}(\r, t) \dr = \iiint_{V} \L_{\text{int}}(\r, t) \dr + \iiint_{V} \L_{\text{ext}}(\r, t)\dr,
	\end{equation*}
    where $ \L_{\text{int}} = - \rho(\r, t) \phi(\r, t) + \j(\r, t)\cdot \A(\r, t) $, derived above, represents the charge distribution's coupling to the external EM field, while $ \L_{\text{int}} $ (internal), which we derive below, represents the contribution of the field generated by the charge distribution itself. Without derivation, the correct expression for $ \L_{\text{int}} $ is
	\begin{equation*}
		\L_{\text{int}}(\r, t) = \frac{1}{2}\ee E^{2}(\r, t) - \frac{1}{2\mm}B^{2}(\r, t),
	\end{equation*}
    where $ E $ and $ B $ are the magntitudes of the total electric and magnetic fields.

	We note, again without proof, that precisely this choice of $ \L_{\text{int}} $ leads to the correct form of Maxwell's equations. Note also that the above expression is precisely electromagnetic energy density contained in a field. In other words, $ \L_{\text{int}} $ qualitatively represents the electromagnetic energy density of the charge distribution's internal field, which is an intuitively appropriate quantity for the $ \L_{\text{int}} $ term.
	
	\item In terms of $ \L_{\text{int}} $ and $ \L_{\text{ext}} $, the total electromagnetic Lagrangian density $ \L_{\text{EM}} $ is then
	\begin{equation*}
		\L_{\text{EM}}(\r, t) = \frac{1}{2}\ee E^{2}(\r, t) - \frac{1}{2\mm}B^{2}(\r, t) - \rho(\r, t)\phi(\r, t) + \j(\r, t) \cdot \A(\r, t),
	\end{equation*}
	where $ \E $ and $ \B $ are defined in terms of the funamental potentials by the relationships $ \E = - \grad \phi - \pdv{\A}{t} $ and $ \B = \curl \A $.
\end{itemize}

\newpage
\section{Introduction to Special Relativity}

\subsection{Lorentz Transformations of the Electrmoagnetic Fields}
\textit{Briefly discuss the Loretnz transformation between two frames of reference moving relative to each other along a mutual $ x $ axis. State and sketch the derivation of the Lorentz transformation of the electric and magnetic field between two frames of reference. Demonstrate the implications of the above transformations on how the quantities $ \E \cdot \B $ and $ E^{2} - c^{2}B^{2} $ transform between frames of reference, and provide a physical interpretation of the results.}

\begin{itemize}
	\item We consider two systems $ S $ and $ S' $, where $ S' $  moves at near-light speed $ v \lesssim c $ in the $ x' $ direction relative to $ S $. System $ S $ contains the fields $ \E $ and $ \B $ and has the coordinates $ (x, y, z, t) $, while system $ S' $ contains the fields $ \E' $ and $ \B' $ and has coordinates $ (x', y', z', t') $.
	
	\item In the theory of special relativity, we transform between the systems $ S $ and $ S' $ using the Lorentz transformations, which read
    \begin{equation*}
        \begin{array}{cc}
            x' = \gamma(x - \beta ct) & \qquad y' = y \\
            ct' = \gamma(ct - \beta x) & \qquad z' = z
        \end{array}
    \end{equation*}
    where $ \beta = v/c $ and $ \gamma = \left( 1 - \beta^{2} \right)^{-1/2} $.
	
	\item The electric and magnetic fields transform between $ S $ and $ S' $ according to
    \begin{equation*}
        \begin{array}{ll}
            E_{x} = E'_{x'} & \qquad B_{x} = B'_{x'}\\
            E_{y} = \gamma (E'_{y'} + v B'_{z'}) & \qquad B_{y} = \gamma \left (B'_{y'} - \frac{v}{c^{2}} E'_{z'}\right)\\
            E_{z} = \gamma (E'_{z'} - vB'_{y'}) & \qquad B_{z} = \gamma \left (B'_{z'} + \frac{v}{c^{2}} E'_{y'}\right ).
        \end{array}
    \end{equation*}
   	Note that the $ x $	components are preserved and the $ y $ and $ z $ components are mixed up, which is opposite the transformations of the $ (x, y, z, t) $ coordinates themselves.
\end{itemize}

\subsubsection{Implications of the EM Field Transformations}
\begin{itemize}
    
    \item The dot product $ \E \cdot \B $ is Lorentz-invariant (i.e. angles are preserved) with respect to the Lorentz transformations of the electromagnetic field.
    \begin{quote}
        \textit{Derivation}: First, we calculate $ \E \cdot \B $, and apply the EM transformations:
        \begin{align*}
            \E \cdot \B &= E_{x}B_{x} + E_{y}B_{y} + E_{z}B_{z} \\
            & = E'_{x'}B'_{x'} + \gamma^{2}\left(1 - \frac{v^{2}}{c^{2}}\right)E'_{y'}B'_{y'} + \gamma^{2}\left(1 - \frac{v^{2}}{c^{2}}\right)E'_{z'}B'_{z'}.
        \end{align*}
        We then use $ \gamma^{2}(1 - \frac{v^{2}}{c^{2}}) = 1 $ which simplifies the above relationship to
        \begin{equation*}
            \E \cdot \B = \E' \cdot \B'
        \end{equation*}
        In other words, the dot product $ \E \cdot \B $ is Lorentz-invariant.
    \end{quote}

    \item The transformations of the EM field preserve conservation of energy, which follows from the invariant relationship
	\begin{equation*}
		 E^{2} - c^{2}B^{2} =  E'^{2} - c^{2}B'^{2}.
	\end{equation*}
    \begin{quote}
        \textit{Derivation}: \textbf{TODO} We write the quantities $ E^{2} $ and $ B^{2} $ in component form and apply the electromagnetic field transformations, which leads to
	\begin{equation*}
		 E^{2} - c^{2}B^{2} =  E'^{2} - c^{2}B'^{2}.
	\end{equation*}
        We then multiply through by $ \ee/2 $, and apply $ \ee \mm = 1/c^{2} $ which transforms the quantities to electomagnetic energy density $ w_{\text{E}} $ and $ w_{\text{M}} $.
    \end{quote}

    \item Summary: In free space, the solutions to the Maxwell equations in $ S' $ have the same form as in $ S $, which follows from preservation of angles under the invariance relationship $ \E \cdot \B = \E' \cdot \B' $. The relative magnitudes of the $ \E' $ and $ \B' $ fields however, change relative to $ \E $ and $ \B $.
\end{itemize}

\subsubsection{(Partial) Derivation: Transformation of the Electric and Magnetic Fields}
\begin{itemize}
	\item We begin by imposing the following condition:
    \begin{quote}
        We assume the Maxwell equations for $ \E $ and $ \B $ in $ S $ have the same form as the Maxwell equations for $ \E' $ and $ \B' $ in $ S' $.
    \end{quote}
    Although this assumption makes intuitive sense and leads to a correct result, it is formally not a trivial assumption, and we do not prove its theoretical framework.
	
    \item For simplicity, we make another assumption: we neglect (or assume an absence of) electromagnetic field sources in the frame $ S' $, which allows use to write the first two Maxwell equations as
    \begin{equation*}
        \divp \E' = 0 \qquad \text{and} \qquad \divp \B' = 0.
    \end{equation*}
    The remaining two Maxwell equations in the frame $ S' $ are
	\begin{equation*}
		\curlp \E' = - \pdv{\B'}{t'}  \eqtext{and} \curlp \B' = \frac{1}{c^{2}}\pdv{\E'}{t'},
	\end{equation*}
    where we have used $ \ee \mm = 1 / c^{2} $. 

    \item As an intermediate step, we write the Maxwell equations in Cartisian components, starting with the equation for $ \curl \B' $, which reads
	\begin{align*}
		& \pdv{E'_{z'}}{y'} - \pdv{E'_{y'}}{z'} = - \pdv{B'_{x'}}{t'} \\ 
		&\pdv{E'_{x'}}{z'} - \pdv{E'_{z'}}{x'} = - \pdv{B'_{y'}}{t'}\\
		&\pdv{E'_{y'}}{x'} - \pdv{E'_{x'}}{y'} = - \pdv{B'_{z'}}{t'}.
	\end{align*}
    Meanwhile, the equation for $ \divp \E' $ reads
	\begin{equation*}
		\pdv{E'_{x'}}{x'} + 	\pdv{E'_{y'}}{y'} + 	\pdv{E'_{z'}}{z'} = 0.
	\end{equation*}
    
    \item According to the Lorentz transformations, the coordinate derivatives transform as
	\begin{equation*}
		\pdv{x'}  = \gamma \left(\pdv{x} + \beta \pdv{(ct)}\right) \eqtext{and} \pdv{(ct')} = \gamma \left(\pdv{(ct)} + \beta \pdv{x}\right).
	\end{equation*}
    Note the presence of plus (and not minus) signs in the transformations, which is related to contravariant vector notation and is beyond the scope of our limited treatment of special relativity. The transformations of the $ y $ and $ z $ derivatives are simpler:
	\begin{equation*}
		\pdv{y'} = \pdv{y} \eqtext{and} \pdv{z'} = \pdv{z}.
	\end{equation*}
    
    \item We then apply the above transformations to the component-form Maxwell equations. The $ \curlp \B' $ equation transforms to
	\begin{align*}
		& \pdv{E'_{z'}}{y} - \pdv{E'_{y'}}{z} = - \gamma v \pdv{B'_{x'}}{x} - \gamma \pdv{B'_{x'}}{t}\\
		& \pdv{E'_{x'}}{z} - \left(\gamma \pdv{E'_{z'}}{x} + \gamma \frac{v}{c^{2}} \pdv{E'_{z'}}{t}\right) = - \gamma v \pdv{B'_{y'}}{x} - \gamma \pdv{B'_{y'}}{t}\\
		& \left(\gamma \pdv{E'_{y'}}{x} + \gamma \frac{v}{c^{2}}\pdv{E'_{y'}}{t}\right) - \pdv{E'_{x'}}{y} = - \gamma v \pdv{B'_{z'}}{x} - \gamma \pdv{B'_{z'}}{t},
	\end{align*}
	while divergence equation $ \divp \E' = 0$ becomes
	\begin{equation*}
		\gamma \pdv{E'_{x'}}{x} + \gamma \frac{v}{c^{2}} \pdv{E'_{x'}}{t} + \pdv{E'_{y'}}{y} + \pdv{E'_{z'}}{t} = 0.
	\end{equation*}
    
    \item We would then perform analogous transformations of the equations $ \divp \B' = 0 $ and $ \curlp \E' = - \pdv{B'}{t'} $. The end result is a system of equations relating the components of the electromagnetic field.
    
    \item The next question is: what combination of primed field quantities (i.e. which combination of components $ E'_{x,y,z} $ and $ B'_{x,y,z} $) should we take so that, when substituted into the above equations, we recover the Maxwell equations in terms of the $ S $ frame field quantities $ \E $ and $ \B $.
	
	Without further proof, the correct transformations are 
    \begin{equation*}
        \begin{array}{ll}
            E_{x} = E'_{x'} & \qquad B_{x} = B'_{x'}\\
            E_{y} = \gamma (E'_{y'} + v B'_{z'}) & \qquad B_{y} = \gamma \left (B'_{y'} - \frac{v}{c^{2}} E'_{z'}\right)\\
            E_{z} = \gamma (E'_{z'} - vB'_{y'}) & \qquad B_{z} = \gamma \left (B'_{z'} + \frac{v}{c^{2}} E'_{y'}\right ),
        \end{array}
    \end{equation*}
    which are the transformations quoted at the begining of the subsection.
\end{itemize}

    
\subsection{Minkowski Space and the Current Density Four-Vector}
\textit{Briefly discuss Minkowski space, the concept of covariant and contravariant vectors, and the generalization of the scalar product to Minkowski space. Discuss conservation of charge in Minkowski space and derive the associated expression for the current density four vector. Show that the magnitude of the current density four vector is invariant under Lorentz transformations.}

\subsubsection{Brief Overview of Minkowski Space}
\begin{itemize}

	\item The fundamental element of Minkowski space is the position four-vector, which we write in the form
	\begin{equation*}
		x_{\mu} = (x, y, z, ct).
	\end{equation*}
    In our convention, Greek letter indices (e.g. $ \mu $) run over all four components, while Latin indices (e.g. $ i $) run over the position components only. The index $ \mu $ occuring in the subscript denotes that $ x_{\mu} $ is a covariant vector. An index above, as in $ x^{\mu} $, represents a contravariant vector and reads
	\begin{equation*}
		x^{\mu} = (x, y, z, -ct),
	\end{equation*}
    which corresponds to the $ (+++-) $ metric convention.

	\item The Lorentz transformation preserves the square of the four-vector:
	\begin{equation*}
		x_{\mu}x^{\mu} = x^{2} + y^{2} + z^{2} - c^{2}t^{2} = x'_{\mu} x'^{\mu}.
	\end{equation*}
    This expression is a generalization of the Euclidean dot product to Minkowski space.

\end{itemize}

\subsubsection{Current Density Four-Vector}

\begin{itemize}
    \item In Minkowski space, the current density four vector is written
    \begin{equation*}
        j_{\mu} = (\j, c \rho),
    \end{equation*}
    where $ \j $ is Euclidean three-dimensional current density and $ \rho $ is charge density.

	\item Since $ j_{\mu} $ is independent of $ \gamma $ and $ \beta $, it is a well-defined four vector and obeys the familiar four-vector transformation rules:
	\begin{equation*}
        \begin{array}{ll}
            j'_{x'} = \gamma (j_{x} - \beta c \rho) & \qquad j'_{y'} = j_{y}\\
            j'_{z'} = j_{z} &  \qquad c \rho' = \gamma(c \rho - \beta j_{x}).
        \end{array}
	\end{equation*}
	
    \item The current density four-vector is Lorentz invariant, i.e. 
	\begin{equation*}
		j_{\mu}j^{\mu} = j_{\mu}'j'^{\mu} = \j \cdot \j - c^{2}\rho^{2}.
	\end{equation*}
    Interpretation: Like for the electromagnetic field components $ \E $ and $ \B $, the relative magnitudes of $ \j $ and $ \rho $ are mixed in the transformation between frames of reference. In other words, the relative composition of the current density four vector in terms of current density $ \j $ and charge density $ \rho $ depends on the system in which the four vector is measured. However, the quantity $ \j \cdot \j - c^{2}\rho^{2} $ is preserved across all systems.

\end{itemize}

\textbf{Derivation: The Current Density Four-Vector}
\begin{itemize}
    \item We begin by requiring conservation of total charge across all systems (e.g. $ S $ and $ S' $), which is written in equation form as
    \begin{equation*}
        q = \iiint_{V} \rho \dr = \iiint_{V'} \rho' \dr'.
    \end{equation*}
    
    \item Working in $ (x, y, z) $ components and applying the Lorentz transformations of the coordinates leads to the expression
    \begin{equation*}
        q = \iiint_{V}\rho \diff x \diff y \diff z = \iiint_{V'}\rho' \diff x' \diff y' \diff z' = \iiint_{V}\rho' \gamma \diff x \diff y \diff z,
    \end{equation*}
    which implies $ \rho' = \rho/\gamma $. In other words, the quantity $ \frac{\rho}{\gamma} $ is Lorentz-invariant, and not simply $ \rho $.
    
	\item Next, we begin with the classical exprssion $ \j = \rho \vec{v} $. Using the velocity four-vector $ u_{\mu} $ and the just-derived expression $ \rho \to \rho/\gamma $, the classical expression generalizes to the four vector
	\begin{equation*}
		j_{\mu} = \frac{\rho}{\gamma} u_{\mu} = \frac{\rho}{\gamma}\left(\gamma \vec{v}, \gamma c\right) = \rho(\vec{v}, c) = (\j, c\rho),
	\end{equation*}
    Note that neither $ \gamma $ nor $ \beta $ occur in the current density four-vector.
	
	
\end{itemize}
    

\subsection{The Electromagnetic Potential Four-Vector}
\textit{Derive the expression for the electromagnetic potential four vector in terms of the Riemman-Lorenz equations and the current density four vector. Show that the magnitude of the electromagnetic potential four vector is invariant under Lorentz transformations.}

\begin{itemize}

    \item The electromagnetic potential four-vector $ A_{\mu} $ is defined as
	\begin{equation*}
		A_{\mu} = \left (\A, \frac{\phi}{c}\right ) \eqtext{and} A^{\mu} = \left (\A, -\frac{\phi}{c}\right ).
	\end{equation*}
    \begin{quote}
        \textit{Derivation}: First, we recall the Riemann-Lorenz equations\footnote{See \hyperref[ss:lagrange-density]{\underline{Subsection \ref{ss:lagrange-density}}}} for $ \A  $ and $ \phi $, which we write in terms of the d'Alembert box operator $ \Box $ as
        \begin{align*}
            & \Box^{2}\phi \equiv \laplacian \phi - \frac{1}{c^{2}}\pdv[2]{\phi}{t} = - \frac{\rho}{\ee}\\
            &\Box^{2} \A \equiv \laplacian \A - \frac{1}{c^{2}} \pdv[2]{\A}{t} = - \mm \j.
        \end{align*}
        We then combine these two equations and introduce the current density four vector $ j_{\mu} = (\j, c\rho) $, which motivates the definition of the electromagnetic potential four vector as
        \begin{equation*}
            A_{\mu} = \left (\A, \frac{\phi}{c}\right ) \eqtext{and} A^{\mu} = \left (\A, -\frac{\phi}{c}\right ).
        \end{equation*}
        % We then subtract the first equation from the second to get
        % \begin{equation*}
        %     \laplacian \A - \laplacian \phi - \frac{1}{c^{2}}\left( \pdv[2]{\A}{t} - \pdv[2]{\phi}{t} \right) = - \mm \j + \frac{\rho}{\ee} = - \mm \j + \mm c^{2} \rho
        % \end{equation*}
    \end{quote}

    \item In terms of the EM potential four vector, the Riemann-Lorenz equations can be written in the Lorentz-invariant form
	\begin{equation*}
		\Box ^{2}A_{\mu} = - \mm j_{\mu} \eqtext{and} \Box^{2}A^{\mu} = - \mm j^{\mu},
	\end{equation*}
    where $ \Box $ denotes the d'Alembert box operator.
	
	\item Since $ A_{\mu} $, like $ j_{\mu} $, is independent of $ \gamma $ and $ \beta $, it is a well-defined four vector and obeys the usual Lorentz transformations:
	\begin{equation*}
        \begin{array}{ll}
            A'_{x'} = \gamma \left(A_{x} - \beta \frac{\phi}{c} \right) & \frac{\phi'}{c} = \gamma \left(\frac{\phi}{c} - \beta A_{x}\right) \\
            A'_{y'} = A_{y} & A'_{z'} = A_{z}
        \end{array}
	\end{equation*}
	
    \item The electromagnetic potential four vector obeys the Lorentz invariance relation
	\begin{equation*}
		A_{\mu}A^{\mu} = A_{\mu}' A'^{\mu} = \A \cdot \A - \frac{\phi^{2}}{c^{2}}.
	\end{equation*}
    Analogously to $ j_{\mu} $, the relative magnitudes of the components $ \A $ and $ \phi $ are mixed in the transformation of $ A_{\mu} $ between different frames of reference, while the invariant quantity $ \A \cdot \A - \frac{\phi^{2}}{c^{2}} $ is preserved.
\end{itemize}
    
\subsection{The Electromagnetic Tensor}
\textit{State the electromagnetic tensor in both component and matrix form and sketch the motivation for its definition.}

\begin{itemize}
	% \item For the purposes of this course, we take covariant to denote a quantitiy that, like $ x_{\mu} $, $ j_{\mu} $ or $ A_{\mu} $, is manifestly invariant under the Lorentz transformations. Our goal in this section is to find a covariant expression for the electromagnetic fields $ \E $ and $ \B $.
	
	\item In matrix form in terms of field components, the electromagnetic field tensor reads
	\begin{equation*}
		\operatorname{F}_{\mu\nu} = 
	\begin{bmatrix}
        0 & B_{z} & - B_{y} & -\frac{E_{x}}{c}\\[1mm]
		- B_{z} & 0 & B_{x} & -\frac{E_{y}}{c}\\[1mm]
		B_{y} & -B_{x} & 0 & -\frac{E_{z}}{c}\\
		\frac{E_{x}}{c} & \frac{E_{y}}{c} & \frac{E_{z}}{c} & 0
	\end{bmatrix}
	\end{equation*}
    Interpretation: In four dimensions, the $ \E $ and $ \B $ fields become a single unified electromagnetic field, written in terms of a tensor.

	Without formal proof, we note that this tensor is Lorentz-invariant, although this should make intuitive sense---it is constructed from the invariant quantity $ A_{\mu} $.

    \item In component form, in terms of the EM potential four vector $ A_{\mu} $, the EM tensor is
	\begin{equation*}
        \operatorname{F}_{\mu \nu} = \pdv{A_{\nu}}{x^{\mu}} - \pdv{A_{\mu}}{x^{\nu}} = \partial_{\mu}A_{\nu} - \partial_{\nu}A_{\mu}.
	\end{equation*}

	\item As a side note, the covariant action associated with the EM field tensor is
	\begin{equation*}
		S = \frac{1}{c}\int \left[- \frac{1}{4}\operatorname{F}^{\mu\nu}\operatorname{F}_{\mu\nu} + A^{\mu}j_{\mu}\right]\diff^{4}x_{\lambda}.
	\end{equation*}
	This action is the basis for quantizing the electromagnetic field in more advanced physics.

\end{itemize}

\textbf{Sketched Motivation for the EM Tensor's Definition}
\begin{itemize}
	\item We begin with the expressions for $ \E $ and $ \B $ in terms of their potentials, i.e.:
	\begin{equation*}
		\B = \curl \A \eqtext{and} \E = - \grad \phi - \pdv{\A}{t}.
	\end{equation*}
    
    \item Next, we introduce a four vector derivative, which we define according to
	\begin{equation*}
		\pdv{x_{\mu}} = \left(\pdv{\r}, \pdv{(ct)} \right).
	\end{equation*}
    Interpretation: This derivative is basically a generalization of the gradient operator to Minkowski space.
    
    \item In terms of $ \A = \big( \A, \tfrac{\phi}{c} \big) $ and $ \pdv{x_{\mu}} $, we now have contravariant four-vector expressions for all of the terms in the right hand side of the equations for $ \E $ and $ \B $. With these four vector quantities in mind, we then define the antisymmetric EM field tensor
	\begin{equation*}
        \operatorname{F}_{\mu \nu} = \pdv{A_{\nu}}{x^{\mu}} - \pdv{A_{\mu}}{x^{\nu}} = \partial_{\mu}A_{\nu} - \partial_{\nu}A_{\mu}.
	\end{equation*}
	
	
	
\end{itemize}

    

\end{document}
